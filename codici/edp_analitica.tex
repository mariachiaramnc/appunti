\documentclass{article}
%%%%%%%%%%%%%%%%%%%%%%%%%%%%%%%%%%%%%%%%%%%%%%%%%%%%%%%%%%%%%%%%%%%%%%%%%%%%%%%%%%%%%%%%%%%%%%%%%%%%%%%%%%%%%%%%%%%%%%%%%%%%%%%%%%%%%%%%%%%%%%%%%%%%%%%%%%%%%%%%%%%%%%%%%%%%%%%%%%%%%%%%%%%%%%%%%%%%%%%%%%%%%%%%%%%%%%%%%%%%%%%%%%%%%%%%%%%%%%%%%%%%%%%%%%%%
\usepackage{amssymb}
\usepackage{amsfonts}
\usepackage{amsmath}

\setcounter{MaxMatrixCols}{10}


\textwidth=6.9in
\oddsidemargin=-0.3in
\topmargin=-0.5in
\textheight=9.0in
\linespread{1.8}

\input{preamble}
\usepackage{amsfonts}
\usepackage{amsmath}


\setcounter{MaxMatrixCols}{10}


\title{Metodi analitici per le equazioni alle derivate parziali\footnote{mariachiara.menicucci@mail.polimi.it per segnalare errori, richiedere il codice LaTeX ecc.}}


\begin{document}

\maketitle
Questi appunti sono stati presi durante le lezioni dell'insegnamento "Metodi analitici per le E. D. P." tenuto dal prof. Bramanti durante l'A. A. 2023/24. Non sono stati revisionati da alcun docente (potrebbero contenere errori, in forma e in sostanza, di qualsiasi tipo) e non sono in alcun modo sostitutivi della frequentazione delle lezioni (in particolare non ho riportato nessuno dei grafici e immagini che sono stati mostrati a lezione). 
\newpage
\tableofcontents

\newpage

\section{Introduzione}

Nelle equazioni differenziali ordinarie l'incognita \`{e} una funzione di
una variabile; questo tipo di equazioni (anche quando si considerano pi\`{u}
equazioni, cio\`{e} sistemi differenziali, e aumenta dunque il numero di
gradi di libert\`{a}) sono intrinsecamente inadatte a gestire la fisica del
continuo, per le quali sono molto pi\`{u} adeguate le equazioni alle
derivate parziali, dove l'incognita \`{e} una funzione di pi\`{u} variabili.
Esse hanno solitamente la forma $F\left( x,u,Du,...,D^{k}u\right) =0$, dove $%
u$ \`{e} la funzione incognita, $Du$ \`{e} il gradiente di $u$, $D^{2}u$ 
\`{e} la matrice hessiana, ecc.

Si fanno alcuni esempi di famose equazioni alle derivate parziali.

\begin{description}
\item[i] Equazione della corda vibrante (D'Alembert, 1752): $\frac{\partial
^{2}u}{\partial t^{2}}=c^{2}\frac{\partial ^{2}u}{\partial x^{2}}$, dove $%
z=u\left( x,t\right) $ \`{e} il grafico della corda vibrante al tempo $t$.
L'equazione generalizzata al caso $n$-dimensionale (detta equazione della
membrana vibrante) \`{e} $\frac{\partial ^{2}u}{\partial t^{2}}=c^{2}\Delta
u $, dove $\Delta u=\sum_{i=1}^{n}\frac{\partial ^{2}u}{\partial x_{i}^{2}}$.

\item[ii] Equazione di Laplace: $\Delta u=0$. Spesso $u$ ha il significato
fisico di potenziale (gravitazionale o elettrostatico).

\item[iii] Equazione del calore (Fourier): $\frac{\partial u}{\partial t}%
-c\Delta u=f$. $u$ ha il significato fisico di temperatura in un mezzo
continuo, $f$ di sorgente o pozzo di calore.

\item[iv] Equazione del trasporto: $\frac{\partial u}{\partial t}+c\frac{%
\partial u}{\partial x}=0$.

\item[v] Equazione della piastra vibrante: $u_{tt}-\Delta ^{2}u=0$.
\end{description}

Ci sono poi le equazioni di Navier-Stokes (in fluidodinamica), di Maxwell
(in elettromagnetismo), di Schroedinger (in fisica quantistica).

I modelli fisici non sono l'unico motivo di studio delle EDP: esse appaiono
anche in altre aree della matematica, ad esempio l'analisi complessa (le
condizioni di Cauchy-Riemann sono un sistema di EDP), la geometria
differenziale (una nota equazione \`{e} quella delle superfici minime, $%
\func{div}\left( \frac{\nabla u}{\sqrt{1+\left\vert \left\vert \nabla
u\right\vert \right\vert ^{2}}}\right) =0$), la probabilit\`{a} e i processi
stocastici.

Le EDP possono essere classificate da vari punti di vista, elencati di
seguito.

\begin{description}
\item[i] ordine: il massimo ordine di derivazione che compare
nell'equazione. Le EDP che studieremo saranno perlopi\`{u} del primo o del
second'ordine. Le equazioni di Cauchy-Riemann e del trasporto sono del
prim'ordine; quelle di Laplace, del calore, delle onde, di Schroedinger sono
del secondo. L'equazione della piastra vibrante e [?] sono di ordine
superiore al secondo.

\item[ii] l'essere equazioni scalari oppure sistemi di equazioni.

\item[iii] linearit\`{a}: se esiste un operatore lineare $L$ tale che
l'equazione pu\`{o} essere scritta nella forma $Lu=f$, allora l'equazione si
dice lineare. Le equazioni di Laplace, delle onde, del calore, di
Schroedinger, del trasporto, di Cauchy-Riemann sono lineari; quella delle
superfici minime non lo \`{e}. Ovviamente, quanto pi\`{u} si fanno ipotesi
semplificative nella scrittura del modello fisico, tanto pi\`{u} sono
gestibili le equazioni che ne risultano, e quindi spesso lineari: il
realismo fisico e la trattabilit\`{a} analitica sono obiettivi antagonisti.

\item[iv] l'essere stazionarie oppure evolutive: si dicono evolutive quando
una delle variabili \`{e} il tempo. Queste si dividono a loro volta in
equazioni di diffusione e ondulatorie.

\item Si vedr\`{a} poi una classificazione in equazioni ellittiche (che sono
stazionarie e generalizzano l'equazione di Laplace), paraboliche (che sono
evolutive e generalizzano l'equazione del calore), iperboliche (che sono
evolutive e generalizzano l'equazione delle onde).
\end{description}

Una grande differenza tra le EDO e le EDP \`{e} che, mentre le prime vengono
studiate su intervalli di $%
%TCIMACRO{\U{211d} }%
%BeginExpansion
\mathbb{R}
%EndExpansion
$, nel secondo caso le possibili scelte del dominio di studio sono molto pi%
\`{u} varie, per cui la geometria del dominio contribuisce in modo
significativo alla complessit\`{a} del problema.

Nel corso della storia ci sono stati tre approcci allo studio delle EDP: il
Settecento e l'Ottocento sono stati l'epoca del virtuosismo analitico, con
cui si sono cercate soluzioni esplicite per equazioni semplici in domini
semplici; alla fine dell'Ottocento ha iniziato a essere sviluppata una
teoria pi\`{u} generale sulle EDP e sulle propriet\`{a} di esistenza, unicit%
\`{a} e stabilit\`{a} delle soluzioni\footnote{%
se tutte e tre queste condizioni sono verificate il problema si dice ben
posto}, cio\`{e} il punto di vista dell'analisi funzionale (che ha portato
alla nascita dell'integrale di Lebesgue, del concetto di soluzione debole,
distribuzione e spazio di Sobolev); oggi esiste anche il punto di vista
dell'analisi numerica, che si basa in modo essenziale sul metodo degli
elementi finiti.

Per le EDO la sola equazione spesso individua infinite soluzioni, e - sotto
opportune ipotesi - aggiungendo una condizione iniziale si ottiene un
problema di Cauchy per cui vale esistenza e unicit\`{a} della soluzione. E'
facile vedere che per le EDP la questione \`{e} pi\`{u} complicata: la
semplicissima equazione $\frac{\partial u}{\partial x}\left( x,y\right) =0$
ha le infinite soluzioni $u\left( x,y\right) =f\left( y\right) $ con $f$
qualsiasi, quindi per avere un'unica soluzione occorre assegnare condizioni
pi\`{u} complesse di un unico valore iniziale. Tipicamente si aggiungono
condizioni di $u$ o di qualche sua derivata lungo una curva o una
superficie: si assegnano funzioni come condizioni iniziali, qualora ci sia
anche il tempo come variabile (nel qual caso il problema si dice \textit{ai
valori iniziali}), o condizioni al bordo (il problema si dice \textit{al
contorno}).

\section{Domini}

Si dovranno fare delle scelte sugli insiemi da considerare come domini delle
funzioni incognite di EDP, cos\`{\i} che sia possibile applicare teoremi di
varia natura che permettono di trattare le EDP.

Considerando $u:\Omega \rightarrow 
%TCIMACRO{\U{211d} }%
%BeginExpansion
\mathbb{R}
%EndExpansion
^{n}$ funzione incognita, in primo luogo si dovr\`{a} avere $\Omega $
aperto, altrimenti non sono neanche ben definite le derivate parziali. In
tal caso, spesso si richieder\`{a} $u\in C^{1}\left( \bar{\Omega}\right) $,
il che significa\footnote{%
in generale derivabilit\`{a} e derivate sono definite per funzioni aventi
come dominio un insieme aperto!} che $u\in C^{1}\left( \Omega \right) $ e
che le derivate $\frac{\partial u}{\partial x_{i}}$ possono essere
prolungate per continuit\`{a} su tutto $\bar{\Omega}$.

Si dice dominio un insieme $\Omega $ aperto e connesso.

\begin{enumerate}
\item Uno dei motivi per cui si richiede la connessione \`{e} che se $\Omega 
$ \`{e} un dominio, $u\in C^{1}\left( \Omega \right) $ e $\nabla u=0$ in $%
\Omega $, allora $u$ \`{e} costante in $\Omega $. Questo in generale non 
\`{e} vero se $\Omega $ non \`{e} connesso (si pensi a una funzione costante
a tratti).
\end{enumerate}

Spesso si fa inoltre una richiesta di regolarit\`{a}.

\textbf{Def} Dato $\Omega \subseteq 
%TCIMACRO{\U{211d} }%
%BeginExpansion
\mathbb{R}
%EndExpansion
^{n}$ aperto, $\Omega $ si dice dominio di classe $C^{1}$ se (i) $\forall $ $%
p_{0}\in \partial \Omega $ $\exists $ $B_{r}\left( p_{0}\right) $, esiste un
sistema di riferimento centrato in $p_{0}$ - le coordinate rispetto al quale
si scrivono come $\mathbf{y}=\left( \mathbf{y}^{\prime },y_{n}\right) $ - ed
esiste $\phi :\Delta _{r}\left( p_{0}\right) \subseteq 
%TCIMACRO{\U{211d} }%
%BeginExpansion
\mathbb{R}
%EndExpansion
^{n-1}\rightarrow 
%TCIMACRO{\U{211d} }%
%BeginExpansion
\mathbb{R}
%EndExpansion
,\phi \in C^{1}\left( \Delta _{r}\left( p_{0}\right) \right) $ con $\phi
\left( \mathbf{y}^{\prime }\right) =y_{n}$, tale che $\partial \Omega \cap
B_{r}\left( p_{0}\right) =\left\{ \left( \mathbf{y}^{\prime },y_{n}\right)
:y^{\prime }\in \Delta _{r}\left( p_{0}\right) ,y_{n}=\phi \left( \mathbf{y}%
^{\prime }\right) \right\} =G_{\phi }$; (ii) $\Omega \cap B_{r}\left(
p_{0}\right) \subseteq \left\{ \left( \mathbf{y}^{\prime },y_{n}\right)
:\phi \left( \mathbf{y}^{\prime }\right) <y_{n}\right\} $.

$\Delta _{r}$ \`{e} la proiezione di $B_{r}\left( p_{0}\right) $ su $%
%TCIMACRO{\U{211d} }%
%BeginExpansion
\mathbb{R}
%EndExpansion
^{n-1}$. (i) significa che la frontiera di $\Omega $ pu\`{o} essere
localmente rappresentata come grafico di una funzione $C^{1}$; (iii)
significa che, fissato $p_{0}$ e $\phi $ che rappresenta localmente $%
\partial \Omega $, il dominio $\Omega $ $\phi $ sta "tutto dalla stessa
parte" rispetto al grafico di $\phi $.

\begin{enumerate}
\item Un cerchio in $%
%TCIMACRO{\U{211d} }%
%BeginExpansion
\mathbb{R}
%EndExpansion
^{2}$ privato di un segmento non \`{e} un dominio $C^{1}$: non \`{e}
soddisfatta (ii).
\end{enumerate}

Se nella definizione sopra si sostituisce la richiesta di $\phi \in
C^{1}\left( \Delta _{r}\left( p_{0}\right) \right) $ con $\phi $
lipschitziana in $\Delta _{r}\left( p_{0}\right) $ si ottiene la definizione
di dominio lipschitziano. Un dominio lipschitziano \`{e} meno regolare di un
dominio $C^{1}$, perch\'{e} il grafico di $\phi $ pu\`{o} avere anche punti
angolosi.

Se $\Omega $ \`{e} un dominio $C^{1}$ o lipschitziano, si pu\`{o} definire
localmente la sua misura di superficie $d\sigma =\sqrt{1+\left\vert
\left\vert \nabla \phi \left( \mathbf{y}^{\prime }\right) \right\vert
\right\vert ^{2}}d\mathbf{y}^{\prime }$. Questo permette di calcolare
integrali di superficie su $\Omega $.

\textbf{Teorema della divergenza}%
\begin{eqnarray*}
\text{Hp} &\text{: }&\Omega \subseteq 
%TCIMACRO{\U{211d} }%
%BeginExpansion
\mathbb{R}
%EndExpansion
^{n}\text{ dominio limitato e lipschitziano, }\mathbf{F}:\bar{\Omega}%
\rightarrow 
%TCIMACRO{\U{211d} }%
%BeginExpansion
\mathbb{R}
%EndExpansion
^{n}\text{, }\mathbf{F}\in C^{1}\left( \bar{\Omega}\right) \\
\text{Ts} &\text{: }&\int_{\Omega }\func{div}\mathbf{F}d\mathbf{x}%
=\int_{\partial \Omega }\left\langle \mathbf{F,n}_{e}\right\rangle d\sigma
\end{eqnarray*}

$\mathbf{n}_{e}$ indica la normale esterna.

Da questo teorema e dalla seguente identit\`{a} tra operatori seguono vari
risultati utili.

\textbf{Prop}%
\begin{eqnarray*}
\text{Hp} &\text{: }&\Omega \subseteq 
%TCIMACRO{\U{211d} }%
%BeginExpansion
\mathbb{R}
%EndExpansion
^{n}\text{ dominio limitato e lipschitziano, }f:\bar{\Omega}\rightarrow 
%TCIMACRO{\U{211d} }%
%BeginExpansion
\mathbb{R}
%EndExpansion
\text{, }\mathbf{G}:\bar{\Omega}\rightarrow 
%TCIMACRO{\U{211d} }%
%BeginExpansion
\mathbb{R}
%EndExpansion
^{n}\text{, }f,\mathbf{G}\in C^{1}\left( \bar{\Omega}\right) \\
\text{Ts} &\text{: }&\func{div}\left( f\mathbf{G}\right) =f\func{div}\mathbf{%
G}+\left\langle \nabla f,\mathbf{G}\right\rangle
\end{eqnarray*}

\textbf{Identit\`{a} di Green}%
\begin{eqnarray*}
\text{(1) Hp}\text{: } &&\Omega \subseteq 
%TCIMACRO{\U{211d} }%
%BeginExpansion
\mathbb{R}
%EndExpansion
^{n}\text{ dominio limitato e lipschitziano, }u,v:\bar{\Omega}\rightarrow 
%TCIMACRO{\U{211d} }%
%BeginExpansion
\mathbb{R}
%EndExpansion
\text{, }u\in C^{2}\left( \bar{\Omega}\right) ,v\in C^{1}\left( \bar{\Omega}%
\right) \\
\text{Ts}\text{: } &&\int_{\Omega }\left\langle \nabla u,\nabla
v\right\rangle d\mathbf{x}+\int_{\Omega }v\Delta ud\mathbf{x}=\int_{\partial
\Omega }v\frac{\partial u}{\partial n}d\sigma \\
\text{(2) Hp} &\text{:}&\text{ }\Omega \subseteq 
%TCIMACRO{\U{211d} }%
%BeginExpansion
\mathbb{R}
%EndExpansion
^{n}\text{ dominio limitato e lipschitziano, }u,v:\bar{\Omega}\rightarrow 
%TCIMACRO{\U{211d} }%
%BeginExpansion
\mathbb{R}
%EndExpansion
\text{, }u,v\in C^{2}\left( \bar{\Omega}\right) \\
\text{Ts} &\text{: }&\int_{\Omega }\left( u\Delta v-v\Delta u\right) d%
\mathbf{x}=\int_{\partial \Omega }\left( u\frac{\partial v}{\partial n}-v%
\frac{\partial u}{\partial n}\right) d\sigma
\end{eqnarray*}

Nella tesi di (1) $u$ e $v$ hanno ruoli asimmetrici.

\textbf{Dim} (1) Si applica la proposizione con $f=v$, $\mathbf{G}=\nabla u$%
: si ha $\func{div}\left( v\nabla u\right) =v\func{div}\nabla u+\left\langle
\nabla v,\nabla u\right\rangle =v\Delta u+\left\langle \nabla u,\nabla
v\right\rangle $. Tutte le funzioni coinvolte sono continue fino al bordo e
dunque integrabili su $\Omega $: si ha quindi $\int_{\Omega }\func{div}%
\left( v\nabla u\right) =\int_{\Omega }\left( v\Delta u+\left\langle \nabla
u,\nabla v\right\rangle \right) d\mathbf{x}$, cio\`{e}, con il teorema della
divergenza (che si pu\`{o} applicare perch\'{e} $\nabla u$ \`{e} $C^{1}$), $%
\int_{\Omega }\left\langle v\nabla u,\mathbf{n}_{e}\right\rangle d\mathbf{x}%
=\int_{\Omega }\left( v\Delta u+\left\langle \nabla u,\nabla v\right\rangle
\right) d\mathbf{x}$. $\left\langle v\nabla u,\mathbf{n}_{e}\right\rangle
=v\left\langle \nabla u,\mathbf{n}_{e}\right\rangle =v\frac{\partial u}{%
\partial \mathbf{n}}$, per definizione di derivata direzionale, quindi si 
\`{e} ottenuta la tesi.

(2) Poich\'{e} $u$ e $v$ sono entrambe $C^{2}$, vale (1) sia per $\left(
u,v\right) $ che per $\left( v,u\right) $, dunque si ha $\int_{\Omega
}\left\langle \nabla u,\nabla v\right\rangle d\mathbf{x}+\int_{\Omega
}v\Delta ud\mathbf{x}=\int_{\partial \Omega }v\frac{\partial u}{\partial n}%
d\sigma $ e $\int_{\Omega }\left\langle \nabla u,\nabla v\right\rangle d%
\mathbf{x}+\int_{\Omega }u\Delta vd\mathbf{x}=\int_{\partial \Omega }u\frac{%
\partial v}{\partial n}d\sigma $. Sottraendo membro a membro si ottiene la
tesi. $\blacksquare $

\section{Equazione di Laplace e di Poisson}

\textbf{Modelli fisici} L'equazione di Poisson, o equazione del potenziale, 
\`{e} un'equazione fondamentale della fisica matematica, che nasce da un
modello fisico di base dell'elettromagnetismo. E' noto infatti che, dato $%
\mathbf{E}$ campo elettrostatico generato da una distribuzione di cariche, $%
\forall $ $B_{r}\left( \mathbf{x}_{0}\right) $ vale $\int_{\partial
B_{r}\left( \mathbf{x}_{0}\right) }\left\langle \mathbf{E,n}%
_{e}\right\rangle d\sigma =4k\pi q_{tot}\left( B_{r}\left( \mathbf{x}%
_{0}\right) \right) $ (teorema di Gauss). Supponendo che la distribuzione di
cariche sia volumetrica con densit\`{a} $\rho :%
%TCIMACRO{\U{211d} }%
%BeginExpansion
\mathbb{R}
%EndExpansion
^{3}\rightarrow \lbrack 0,+\infty )$, il teorema di Gauss - usando anche il
teorema della divergenza - diventa $\int_{B_{r}\left( \mathbf{x}_{0}\right) }%
\func{div}\mathbf{E}d\mathbf{x}=4k\pi \int_{B_{r}\left( \mathbf{x}%
_{0}\right) }\rho \left( \mathbf{x}\right) d\mathbf{x}$. Poich\'{e}
l'uguaglianza vale $\forall $ $B_{r}\left( \mathbf{x}_{0}\right) $, se le
funzioni integrande sono continue si ha l'uguaglianza tra le integrande%
\footnote{%
si dimostra, $\forall $ $\mathbf{x}_{0}$, dividendo per $\left\vert
B_{r}\left( \mathbf{x}_{0}\right) \right\vert $, usando il teorema della
media e facendo tendere $r$ a $0$} in tutto $B_{r}\left( \mathbf{x}%
_{0}\right) $. Aggiungendo il fatto che $\mathbf{E}$ \`{e} conservativo, cio%
\`{e} $\exists $ $V:%
%TCIMACRO{\U{211d} }%
%BeginExpansion
\mathbb{R}
%EndExpansion
^{3}\rightarrow 
%TCIMACRO{\U{211d} }%
%BeginExpansion
\mathbb{R}
%EndExpansion
:\mathbf{E}=\nabla V$, si ottiene l'equazione $\Delta V=4k\pi \rho $. Se si
studia l'analogo modello per il campo gravitazionale si ottiene la stessa
equazione, a meno di un segno, con il potenziale gravitazionale.

Dunque l'equazione che si studia, detta equazione di Poisson, \`{e} 
\begin{equation*}
\Delta u=f
\end{equation*}

che nel caso particolare di $f$ nulla si dice equazione di Laplace. Le
soluzioni dell'equazione di Laplace si dicono funzioni armoniche.

Un altro modello fisico che porta all'equazione di Poisson si ha in
fluidodinamica. Se $\mathbf{v}$ \`{e} il campo di velocit\`{a} di un fluido
incomprimibile (quindi $\func{div}\mathbf{v=0}$) con moto non vorticoso ($rot%
\mathbf{v=0}$), almeno localmente esiste un potenziale di velocit\`{a}, cio%
\`{e} $\psi :\mathbf{v}=\nabla \psi $. L'incomprimibilit\`{a} d\`{a} quindi $%
\Delta \psi =0$: il potenziale di velocit\`{a} del moto non vorticoso di un
fluido incomprimibile \`{e} una funzione armonica.

Inoltre l'equazione di Poisson pu\`{o} essere vista come equazione che si
ottiene in stato stazionario da due equazioni evolutive.

\begin{description}
\item[a] L'equazione del calore in stato stazionario (cio\`{e} quando \`{e}
stato raggiunto l'equilibrio termico e la temperatura $u$ non varia pi\`{u})
diventa - dato che $\frac{\partial u}{\partial t}=0$ - $\Delta u=\frac{-f}{c}
$. Nel caso di assenza di pozzi o sorgenti di calore, si ottiene l'equazione
di Laplace: dunque ogni soluzione dell'equazione di Laplace ha anche il
significato fisico di temperatura di un mezzo continuo all'equilibrio
termico in assenza di sorgenti.

\item[b] L'equazione della membrana vibrante all'equilibrio, cio\`{e} quando
la membrana ha smesso di vibrare, diventa - dato che $\frac{\partial u}{%
\partial t}=0$ - $\Delta u=0$. Peraltro immaginandosi fisicamente la
membrana all'equilibrio si nota anche che il la funzione $u$, il cui grafico 
\`{e} la membrana, non pu\`{o} avere punti di estremo all'interno del
dominio, ma solo punti di sella. In effetti in due dimensioni l'equazione 
\`{e} $u_{xx}+u_{yy}=0$, quindi se $u$ come funzione della sola $x$ ha un
punto di massimo $\mathbf{x}^{\ast }$ e vale $u_{xx}\left( \mathbf{x}^{\ast
}\right) >0$, allora $u_{yy}<0$, cio\`{e} $u$ nella direzione $y$ ha un
punto di minimo: cio\`{e} il punto non \`{e} n\'{e} di massimo n\'{e} di
minimo, ma di sella.
\end{description}

Un'altra motivazione per studiare l'equazione di Laplace \`{e} di tipo
matematico. Infatti, se $f\left( z\right) =u\left( x,y\right) +iv\left(
x,y\right) $ \`{e} olomorfa in $%
%TCIMACRO{\U{2102} }%
%BeginExpansion
\mathbb{C}
%EndExpansion
$, allora\footnote{%
Questo deriva dalle condizioni di Cauchy-Riemann: $\frac{\partial u}{%
\partial x}=\frac{\partial v}{\partial y}$, $\frac{\partial u}{\partial y}=-%
\frac{\partial v}{\partial x}$, e derivando la prima rispetto a $x$ e la
seconda rispetto a $y$ e sommando membro a membro si ottiene, per il teorema
di Schwarz, $\Delta u=0$. Analogamente si ricava $\Delta v=0$.} $u$ e $v$
sono $C^{2}$ e armoniche, cio\`{e} $\Delta u=\Delta v=0$ in $%
%TCIMACRO{\U{211d} }%
%BeginExpansion
\mathbb{R}
%EndExpansion
^{2}$. Viceversa, si pu\`{o} mostrare che per ogni funzione armonica $u$
esiste un'altra funzione armonica $v$, detta armonica coniugata di $u$, tale
che $u+iv$ \`{e} olomorfa.

In questo modo possiamo facilmente trovare delle funzioni armoniche.

\begin{enumerate}
\item $e^{z}$ \`{e} una funzione olomorfa, quindi $u\left( x,y\right)
=e^{x}\cos y$ e $v\left( x,y\right) =e^{x}\sin y$ sono funzioni armoniche.

\item $f\left( z\right) =z^{n}$ \`{e} olomorfa, quindi $\func{Re}\left(
x+iy\right) ^{n}=\func{Re}\rho ^{n}e^{in\theta }=\rho ^{n}\cos n\theta
=:u_{n}$ e $\func{Im}\rho ^{n}e^{in\theta }=\rho ^{n}\sin n\theta =:v_{n}$
sono armoniche e si dicono armoniche elementari del piano.

Se $n=3$, $\left( x+iy\right) ^{3}=x^{3}-iy^{3}+3ix^{2}y-3xy^{2}$ ha $%
u_{3}\left( x,y\right) =x^{3}-3xy^{2},v_{3}\left( x,y\right) =3x^{2}y-y^{3}$%
. Valgono dunque le identit\`{a} $x^{3}-3xy^{2}=\rho ^{3}\cos 3\theta $ e $%
3x^{2}y-y^{3}=\rho ^{3}\sin 3\theta $, che sarebbe molto laborioso
verificare in altro modo.
\end{enumerate}

\subsection{Problemi al contorno per l'equazione di Poisson}

L'equazioni di Poisson \`{e} stazionaria, quindi non ha alcun senso porre
condizioni iniziali: non esiste un istante iniziale. Si possono invece porre
condizioni al contorno di diversi tipi, che danno origine ad altrettanti
problemi differenziali.

\begin{enumerate}
\item \textbf{Condizione di Dirichlet} Si pone la condizione $u=g$ in $%
\partial \Omega $. Se si pensa alla membrana vibrante all'equilibrio, questo
ha il significato fisico di "fissare" il perimetro della membrana.

\item \textbf{Condizione di Neumann} Si pone la condizione $\frac{\partial u%
}{\partial n}=g$ in $\partial \Omega $ ($\frac{\partial u}{\partial n}$ \`{e}
la derivata normale, ed \`{e} un'informazione indipendente dalla derivata
tangenziale $\frac{\partial u}{\partial t}$ e da $u$). Dal punto di vista
fisico, $\frac{\partial u}{\partial n}$ in $\partial \Omega $ ha il
significato fisico di flusso di calore (che infatti dipende dalla variazione
di calore tra le due zone) attraverso il bordo: se \`{e} nulla, non c'\`{e}
flusso di calore, cio\`{e} il corpo \`{e} termicamente isolato.

\item \textbf{Problema misto} Si partiziona $\partial \Omega $ in $\Sigma
_{1},\Sigma _{2}$ e si pone la condizione di Dirichlet su $\Sigma _{1}$ e di
Neumann su $\Sigma _{2}$, cio\`{e} $\left\{ 
\begin{array}{c}
u=g_{1}\text{ in }\Sigma _{1} \\ 
\frac{\partial u}{\partial n}=g_{2}\text{ in }\Sigma _{2}%
\end{array}%
\right. $.

\item \textbf{Condizione di Robin} Si pone la condizione $\frac{\partial u}{%
\partial n}+\alpha u=g$ in $\partial \Omega $, con $\alpha >0$. Questa ha
origine dalla seguente situazione: se un corpo ha temperatura $u$ e
l'ambiente esterno ha temperatura $u_{0}$, il flusso di calore al bordo del
corpo \`{e} proporzionale al salto termico $u-u_{0}$, cio\`{e} $\frac{%
\partial u}{\partial n}=\alpha \left( u_{0}-u\right) $: se $u_{0}>u$, il
calore passa al corpo, quindi $u$ e cresce e dunque $\frac{\partial u}{%
\partial n},\alpha >0$. Nel caso $\alpha =0$ si ritrova la condizione di
Neumann.
\end{enumerate}

\textbf{Teo (unicit\`{a} della soluzione)}%
\begin{gather*}
\text{Hp: }\Omega \subseteq 
%TCIMACRO{\U{211d} }%
%BeginExpansion
\mathbb{R}
%EndExpansion
^{n}\text{ \`{e} un dominio limitato e lipschitziano} \\
\text{Ts: nella classe di funzioni }C^{2}\left( \Omega \right) \cap
C^{1}\left( \bar{\Omega}\right) \text{, la soluzione per il problema di
Dirichlet, misto, di } \\
\text{Robin, \`{e} unica se esiste; per il problema di Neumann \`{e} unica a
meno di una costante additiva}
\end{gather*}

Il problema di Neumann \`{e} $\left\{ 
\begin{array}{c}
\Delta u=f \\ 
\frac{\partial u}{\partial n}=g%
\end{array}%
\right. $: $u$ compare solo attraverso le sue derivate, quindi evidentemente
se $u$ \`{e} soluzione anche $u+c$ lo \`{e}.

\textbf{Dim} Siano $u_{1},u_{2}\in C^{2}\left( \Omega \right) \cap
C^{1}\left( \bar{\Omega}\right) $ due soluzioni dello stesso problema (di
uno dei quattro tipi): si mostra che coincidono (per Neumann, si mostra che
coincidono a meno di una costant). $u=u_{1}-u_{2}\in C^{2}\left( \Omega
\right) \cap C^{1}\left( \bar{\Omega}\right) $ e, per linearit\`{a} del
laplaciano, $\Delta u=\Delta u_{1}-\Delta u_{2}=f-f=0$; inoltre $u$ al bordo
soddisfa la stessa condizione di $u_{1}$ e $u_{2}$ ma con $g=0$. Quindi $u$
risolve il problema omogeneo.

Uso la prima identit\`{a} di Green $\int_{\Omega }\left\langle \nabla
u,\nabla v\right\rangle d\mathbf{x}+\int_{\Omega }v\Delta ud\mathbf{x}%
=\int_{\partial \Omega }v\frac{\partial u}{\partial n}d\sigma $ con $u=v$,
ma occorre $u\in C^{2}\left( \bar{\Omega}\right) $, che \`{e} una richiesta
pi\`{u} forte dell'ipotesi del teorema. Per ora supponiamo che $u\in
C^{2}\left( \bar{\Omega}\right) $, e poi vedremo che questa ipotesi pu\`{o}
essere rilassata. Si ha quindi $\int_{\Omega }\left\vert \left\vert \nabla
u\right\vert \right\vert ^{2}d\mathbf{x}+\int_{\Omega }u\Delta ud\mathbf{x}%
=\int_{\partial \Omega }u\frac{\partial u}{\partial n}d\sigma $, cio\`{e} $%
\int_{\Omega }\left\vert \left\vert \nabla u\right\vert \right\vert ^{2}d%
\mathbf{x}=\int_{\partial \Omega }u\frac{\partial u}{\partial n}d\sigma $
(cf. parti!)

Se il problema \`{e} di Dirichlet, $u=0$ in $\partial \Omega $, quindi il
lato destro \`{e} nullo e dunque anche il lato sinistro; lo stesso vale se
il problema \`{e} di Neumann (perch\'{e} $\frac{\partial u}{\partial n}=0$).
Se il problema \`{e} misto, $\int_{\partial \Omega }u\frac{\partial u}{%
\partial n}d\sigma =\int_{\Sigma _{1}}u\frac{\partial u}{\partial n}d\sigma
+\int_{\Sigma _{2}}u\frac{\partial u}{\partial n}d\sigma =0$. Se invece il
problema \`{e} di Robin, si ha $\frac{\partial u}{\partial n}+\alpha u=0$ in 
$\partial \Omega $, quindi $\int_{\partial \Omega }u\frac{\partial u}{%
\partial n}d\sigma =\int_{\partial \Omega }-\alpha u^{2}d\sigma
=\int_{\Omega }\left\vert \left\vert \nabla u\right\vert \right\vert ^{2}d%
\mathbf{x}$, e poich\'{e} l'ultimo integrale \`{e} sicuramente nonnegativo ($%
\alpha >0$) mentre il penultimo \`{e} sicuramente nonpositivo, l'unica
possibilit\`{a} \`{e} che sia nullo.

Dunque per tutti i problemi si ha $\int_{\Omega }\left\vert \left\vert
\nabla u\right\vert \right\vert ^{2}d\mathbf{x}=0$. Ma poich\'{e} $u\in
C^{1}\left( \bar{\Omega}\right) $ e la norma \`{e} continua, $\left\vert
\left\vert \nabla u\right\vert \right\vert ^{2}$ \`{e} una funzione continua
e nonnegativa e dunque $\left\vert \left\vert \nabla u\right\vert
\right\vert =0$ in $\Omega $, per cui $u$ \`{e} costante in $\Omega $
(essendo $\Omega $ connesso) e anche in $\bar{\Omega}$ per continuit\`{a}.
La tesi \`{e} dimostrata per il problema di Neumann. Se il problema \`{e} di
Dirichlet, $u=0$ in $\partial \Omega $, quindi la costante \`{e} $0$, e si
ha la tesi, e analogamente si conclude se il problema \`{e} misto; se il
problema \`{e} di Robin, $\frac{\partial u}{\partial n}+\alpha u=0$ in $%
\partial \Omega $, ma $\frac{\partial u}{\partial n}=0$ perch\'{e} $u$ \`{e}
costante, quindi $u=0$ in $\partial \Omega $ e la costante \`{e} $0$.

Ora occorre mostrare che l'ipotesi $u\in C^{2}\left( \bar{\Omega}\right) $
in realt\`{a} non \`{e} necessaria. Suppongo solo che $u\in C^{2}\left(
\Omega \right) \cap C^{1}\left( \bar{\Omega}\right) $. Considero $\Omega
_{k}\subseteq \Omega $: in $\Omega _{k}$ $u$ \`{e} $C^{2}$ fino al bordo.
Allora prendo $\left\{ \Omega _{k}\right\} _{k=1}^{+\infty }$ successione di
domini tale che $\Omega _{k}\subseteq \Omega _{k+1}$ $\forall $ $k$ e $%
\bigcup_{i=1}^{+\infty }\Omega _{i}=\Omega $, con $\bar{\Omega}_{k}\subseteq
\Omega $ $\forall $ $k$: applicando la prima identit\`{a} di Green a $\Omega
_{k}$ si ottiene $\int_{\Omega _{k}}\left\vert \left\vert \nabla
u\right\vert \right\vert ^{2}d\mathbf{x}=\int_{\partial \Omega _{k}}u\frac{%
\partial u}{\partial n}d\sigma $, che \`{e} un'uguaglianza tra integrali ben
definiti anche se $u\in C^{2}\left( \Omega \right) \cap C^{1}\left( \bar{%
\Omega}\right) $. Allora si pu\`{o} calcolare il limite per $k\rightarrow
+\infty $ e si ottiene $\int_{\Omega }\left\vert \left\vert \nabla
u\right\vert \right\vert ^{2}d\mathbf{x}=\int_{\partial \Omega }u\frac{%
\partial u}{\partial n}d\sigma $, da cui segue la tesi per tutti i
ragionamenti fatti sopra. $\blacksquare $

\textbf{Prop (condizione di compatibilit\`{a} per il problema di Neumann)}%
\begin{gather*}
\text{Hp}\text{: }u\in C^{2}\left( \Omega \right) \cap C^{1}\left( \bar{%
\Omega}\right) \text{ risolve il problema di Neumann }\left\{ 
\begin{array}{c}
\Delta u=f\text{ in }\Omega \\ 
\frac{\partial u}{\partial n}=g\text{ in }\partial \Omega%
\end{array}%
\right. \text{, }f=0 \\
\text{Ts}\text{: }\int_{\partial \Omega }gd\sigma =0
\end{gather*}

Quindi, se $u$ \`{e} una soluzione per il problema, deve valere
necessariamente una condizione su $g$ (detta condizione di compatibilit\`{a}
sul dato al bordo). Se $g$ non la soddisfa, allora il problema non ha
soluzione\footnote{%
Questo ricorda ci\`{o} che accade per i sistemi lineari: se non \`{e} vero
che $r\left( A|\mathbf{b}\right) =r\left( A\right) $ per $A$ a scala (che 
\`{e} una condizione sui dati), allora il sistema non ha soluzione.}.

\textbf{Dim} Poich\'{e} $\Delta u=0$, $\int_{\Omega }\Delta ud\mathbf{x}%
=0=\int_{\Omega }\func{div}\left( \nabla u\right) d\mathbf{x}$, cio\`{e} $%
\int_{\partial \Omega }\left\langle \nabla u,n\right\rangle d\sigma
=\int_{\partial \Omega }\frac{\partial u}{\partial n}d\sigma =0$ per il
teorema della divergenza. $\blacksquare $

Il significato fisico della condizione su $g$ si pu\`{o} capire con
l'equazione del calore: ogni soluzione dell'equazione di Laplace ha anche il
significato fisico di temperatura di un mezzo continuo all'equilibrio
termico in assenza di sorgenti, mentre $g$ \`{e} il flusso di calore
attraverso il bordo e $\int_{\partial \Omega }gd\sigma $ \`{e} il calore
attraverso il bordo. Questo non pu\`{o} che essere nullo, per un fatto di
bilancio energetico: se ci fosse del calore entrante, o uscente, il mezzo
non sarebbe all'equilibrio.

\subsection{Problema di Dirichlet per il laplaciano sul cerchio}

Ora si sceglie $\Omega $: si cerca di risolvere il problema di Dirichlet per
l'equazione di Laplace nel caso particolare di $\Omega $ cerchio
bidimensionale di raggio $r$. Il problema \`{e} 
\begin{equation*}
\left\{ 
\begin{array}{c}
\frac{\partial ^{2}u}{\partial x^{2}}+\frac{\partial ^{2}u}{\partial y^{2}}=0%
\text{ se }x^{2}+y^{2}<r^{2} \\ 
u=g\text{ se }x^{2}+y^{2}=r^{2}%
\end{array}%
\right.
\end{equation*}

Essendo il dominio un cerchio, \`{e} naturale riformulare il problema in
coordinate polari $\left( \rho ,\theta \right) $: usando la formula per il
laplaciano in coordinate polari si ottiene%
\begin{equation*}
\left\{ 
\begin{array}{c}
\frac{\partial ^{2}u}{\partial \rho ^{2}}+\frac{1}{\rho }\frac{\partial u}{%
\partial \rho }+\frac{1}{\rho ^{2}}\frac{\partial ^{2}u}{\partial \theta ^{2}%
}=0\text{ se }\rho <r,\theta \in \lbrack 0,2\pi ) \\ 
u\left( r,\theta \right) =f\left( \theta \right)%
\end{array}%
\right.
\end{equation*}

Per risolvere l'equazione uso la tecnica di separazione delle variabili:
suppongo esistano $R,\Theta :u\left( \rho ,\theta \right) =R\left( \rho
\right) \Theta \left( \theta \right) $ e cerco la soluzione. Se la trovo,
non serve fare altro, dato che la soluzione \`{e} unica.

Allora, poich\'{e} $\frac{\partial u}{\partial \rho }=R^{\prime }\Theta $ e $%
\frac{\partial u}{\partial \theta }=R\Theta ^{\prime }$, la prima equazione
diventa $R^{\prime \prime }\Theta +\frac{1}{\rho }R^{\prime }\Theta +\frac{1%
}{\rho ^{2}}R\Theta ^{\prime \prime }=0$. Ora voglio separare le variabili:
dividendo per $R\Theta $ si ottiene $\frac{R^{\prime \prime }}{R}+\frac{1}{%
\rho }\frac{R^{\prime }}{R}+\frac{1}{\rho ^{2}}\frac{\Theta ^{\prime \prime }%
}{\Theta }=0$, e moltiplicando per $\rho ^{2}$ si ha $\rho ^{2}\frac{%
R^{\prime \prime }}{R}+\rho \frac{R^{\prime }}{R}=-\frac{\Theta ^{\prime
\prime }}{\Theta }$. Si \`{e} scritto che una funzione della sola $\rho $ 
\`{e} uguale a una funzione della sola $\theta $: questo pu\`{o} essere vero 
$\forall $ $\rho <r,\theta \in \lbrack 0,2\pi )$ se e solo se entrambe le
funzioni sono costanti. Dunque l'equazione si pu\`{o} riscrivere con il
sistema%
\begin{equation*}
\left\{ 
\begin{array}{c}
\rho ^{2}R^{\prime \prime }+\rho R^{\prime }-\lambda R=0\text{ se }\rho <r
\\ 
\Theta ^{\prime \prime }+\lambda \Theta =0\text{ se }\theta \in \lbrack
0,2\pi )%
\end{array}%
\right.
\end{equation*}

con $\lambda $ costante da determinare. Parto dalla seconda: affinch\'{e} $u$
sia continua e derivabile $\Theta $ dev'essere $2\pi $-periodica (e cos%
\`{\i} anche la sua derivata prima e seconda, dovendo $u$ essere $%
C^{2}\left( \Omega \right) $), per il significato geometrico delle
coordinate polari: quindi l'unica soluzione accettabile\footnote{%
per $\lambda <0$ si ha $\Theta \left( \theta \right) =c_{1}e^{\sqrt{\lambda }%
\theta }+c_{2}e^{-\sqrt{\lambda }\theta }$, che implica $c_{1}=c_{2}=0$ per
avere periodicit\`{a}; per $\lambda =0$ si ha $\Theta \left( \theta \right)
=c_{1}\theta +c_{2}$, che implica $c_{1}=0$ per avere periodicit\`{a}. Tutte
le soluzioni costanti si ritrovano comunque considerando solo la soluzione
trigonometrica.} \`{e} quella trigonometrica che si ottiene per $\lambda
\geq 0$, con argomento delle funzioni trigonometriche pari a $\sqrt{\lambda }%
t$, il cui periodo \`{e} quindi $\frac{2\pi }{\sqrt{\lambda }}$. Affinch\'{e}
esse siano effettivamente $2\pi $-periodiche $\frac{2\pi }{\sqrt{\lambda }}$
dev'essere del tipo $\frac{2\pi }{n}$, da cui $\lambda =n^{2}$. Allora la
soluzione \`{e} $\Theta _{n}\left( \theta \right) =a_{n}\cos n\theta
+b_{n}\sin n\theta $.

L'equazione in $\rho $ \`{e} $\rho ^{2}R^{\prime \prime }+\rho R^{\prime
}-n^{2}R=0$, lineare omogenea del second'ordine a coefficienti non costanti,
ed \`{e} un'equazione di Eulero\footnote{%
Si chiama equazione di Eulero un'equazione del tipo $x^{n}u^{\left( n\right)
}+a_{n-1}x^{n-1}u^{\left( n-1\right) }+...+a_{1}xu^{\prime }+a_{0}u=g\left(
x\right) $. Nel caso particolare di $g=0$ si possono cercare due soluzioni
del tipo $u\left( x\right) =x^{\alpha }$, con $\alpha $ da determinare.},
per cui \`{e} noto che la soluzione \`{e} $R\left( \rho \right) =\rho
^{\alpha }$ con $\alpha $ da determinare. $\rho ^{2}\alpha \left( \alpha
-1\right) \rho ^{\alpha -2}+\rho \alpha \rho ^{\alpha -1}-n^{2}\rho ^{\alpha
}=0\Longleftrightarrow \rho ^{\alpha }\left( \alpha \left( \alpha -1\right)
+\alpha -n^{2}\right) =0$, da cui $\alpha ^{2}=n^{2}$, cio\`{e} $\alpha =\pm
n$. Quindi l'integrale generale \`{e} $R\left( \rho \right) =c_{1}\rho
^{n}+c_{2}\rho ^{-n}$, con $n>0$. Se invece $n=0$ l'equazione \`{e} $\rho
^{2}R^{\prime \prime }+\rho R^{\prime }=0$, cio\`{e} $\rho v^{\prime }+v=0$,
da cui $v=c\frac{1}{\rho }$, cio\`{e} $R=c\log \rho $; allora $R\left( \rho
\right) =c_{1}\log \rho +c_{2}$.

Dunque si ha $u\left( \rho ,\theta \right) =\left( c_{n}\rho ^{n}+d_{n}\rho
^{-n}\right) \left( a_{n}\cos n\theta +b_{n}\sin n\theta \right) $ per $n>0$
e $u\left( \rho ,\theta \right) =a_{0}\left( c_{0}\log \rho +d_{0}\right) $
per $n=0$. Si stanno cercando soluzioni ben definite nell'origine, per cui
necessariamente $d_{n}=c_{0}=0$.

Allora si \`{e} determinata una successione di soluzioni a variabili
separate:%
\begin{eqnarray*}
u_{n}\left( \rho ,\theta \right) &=&\left( a_{n}\cos n\theta +b_{n}\sin
n\theta \right) \rho ^{n} \\
u_{0}\left( \rho ,\theta \right) &=&a_{0}
\end{eqnarray*}

Si noti che si sono ritrovate anche le soluzioni costanti.

\begin{enumerate}
\item Si pu\`{o} risolvere con la stessa tecnica il problema di Dirichlet
sulla corona circolare.
\end{enumerate}

Occorre ora imporre le condizioni al contorno. Noto che, essendo $u_{n}$
soluzione $\forall $ $n$, anche ogni somma finita delle $u_{n}$ risolve
l'equazione (perch\'{e} \`{e} lineare). E' naturale spingersi fino alla
serie delle $u_{n}$ e provare a imporre su essa la condizione al contorno:
non c'\`{e} speranza che questa sia soddisfatta se $u$ \`{e} una somma
finita delle $u_{n}$ (a meno che $f$ sia esattamente una combinazione
lineare di seni e coseni). $u\left( \rho ,\theta \right)
:=a_{0}+\sum_{n=1}^{+\infty }\rho ^{n}\left( a_{n}\cos n\theta +b_{n}\sin
n\theta \right) $: voglio $u\left( r,\theta \right) =f\left( \theta \right) $%
, cio\`{e} $a_{0}+\sum_{n=1}^{+\infty }r^{n}\left( a_{n}\cos n\theta
+b_{n}\sin n\theta \right) =f\left( \theta \right) $ $\forall $ $\theta \in
\lbrack 0,2\pi )$. Sembra proprio la serie di Fourier di $f$. Supponendo che 
$f$ si possa sviluppare in serie di Fourier, si ha dunque $f\left( \theta
\right) =\frac{\alpha _{0}}{2}+\sum_{n=1}^{+\infty }\left( \alpha _{n}\cos
n\theta +\beta _{n}\sin n\theta \right) $, con $\alpha _{n}=\frac{1}{\pi }%
\int_{0}^{2\pi }f\left( \theta \right) \cos n\theta d\theta $, $\beta _{n}=%
\frac{1}{\pi }\int_{0}^{2\pi }f\left( \theta \right) \sin n\theta d\theta $.
Si impone $a_{0}=\frac{\alpha _{0}}{2},a_{n}r^{n}=\alpha
_{n},b_{n}r^{n}=\beta _{n}$: si \`{e} quindi individuata la soluzione
particolare, scegliendo $a_{n}$ e $b_{n}$ grazie al dato $f\left( \theta
\right) $.

La candidata soluzione \`{e} 
\begin{equation*}
u\left( \rho ,\theta \right) :=\frac{\alpha _{0}}{2}+\sum_{n=1}^{+\infty
}\left( \frac{\rho }{r}\right) ^{n}\left( \alpha _{n}\cos n\theta +\beta
_{n}\sin n\theta \right)
\end{equation*}

ricordando che $\alpha _{n}$ e $\beta _{n}$ sono coefficienti di Fourier di $%
f$. Noto che gli addendi elementari sono proprio le armoniche elementari del
piano.

Occorre in realt\`{a} fare un'analisi critica di questa candidata soluzione,
visto che essa scaturisce da procedimenti non del tutti rigorosi fatti
supponendo che ogni operazione fosse lecita. Ora si deve fare qualche
ipotesi e mostrare rigorosamente che la candidata \`{e} effettivamente
soluzione.

\textbf{Analisi critica della buona posizione e della soddisfazione
dell'equazione} Il seguente teorema fornisce delle ipotesi sotto le quali la
soluzione candidata \`{e} effettivamente una buona soluzione.

\textbf{Teo (regolarit\`{a} della soluzione)}

\begin{gather*}
\text{Hp: }f\in L^{1}\left( \left[ 0,2\pi \right] \right) \\
\text{Ts: (i) }u\text{ \`{e} derivabile infinite volte termine a termine in
ogni cerchio di raggio }\rho \leq \delta \text{ con }\delta <r \\
\text{(ii) }u\in C^{\infty }\left( B_{r}\left( \mathbf{0}\right) \right) \\
\text{(iii) }\Delta u=0\text{ se }\rho <r
\end{gather*}

Ovviamente si vuole che la serie che assegna $u$ converga, cos\`{\i} che $u$
sia ben definita, che si possa derivare termine a termine\footnote{%
in questo modo \`{e} facilmente verificabile che $\Delta u=0$, scambiando
serie e derivata} due volte rispetto a $\theta $ e a $\rho $ (i), e che
risolva l'equazione (iii).

A priori sarebbe naturale chiedere $u$ almeno $C^{2}$, ma la solita ipotesi $%
f\in L^{1}$ produce una regolarit\`{a} molto migliore: in questo senso
l'equazione di Laplace \`{e} regolarizzante.

\textbf{Dim} (i) (ii) Se $f\in L^{1}\left( \left[ 0,2\pi \right] \right) $,
i coefficienti $\alpha _{n}=\frac{1}{\pi }\int_{0}^{2\pi }f\left( \theta
\right) \cos n\theta d\theta $ e $\beta _{n}$ sono ben definiti, dunque il
termine generale della serie che definisce $u$ \`{e} ben definito; vale $%
\left\vert \alpha _{n}\right\vert ,\left\vert \beta _{n}\right\vert \leq 
\frac{1}{\pi }\left\vert \left\vert f\right\vert \right\vert _{L^{1}\left( %
\left[ 0,2\pi \right] \right) }$ Ora si mostra con la definizione che la
serie converge totalmente: $\left\vert \left( \frac{\rho }{r}\right)
^{n}\left( \alpha _{n}\cos n\theta +\beta _{n}\sin n\theta \right)
\right\vert \leq \left( \frac{\rho }{r}\right) ^{n}\left( \left\vert \alpha
_{n}\right\vert +\left\vert \beta _{n}\right\vert \right) \leq \frac{2}{\pi }%
\left( \frac{\rho }{r}\right) ^{n}\left\vert \left\vert f\right\vert
\right\vert _{L^{1}\left( \left[ 0,2\pi \right] \right) }$: poich\'{e} $\rho
\leq \delta $, si ottiene $\left\vert \left( \frac{\rho }{r}\right)
^{n}\left( \alpha _{n}\cos n\theta +\beta _{n}\sin n\theta \right)
\right\vert \leq \frac{2}{\pi }\left( \frac{\delta }{r}\right)
^{n}\left\vert \left\vert f\right\vert \right\vert _{L^{1}\left( \left[
0,2\pi \right] \right) }$. Il lato destro \`{e} il termine generale di una
serie geometrica, quindi convergente: allora la serie iniziale converge
totalmente in ogni cerchio chiuso di raggio $\delta <r$ (questo non implica
la convergenza totale nel cerchio aperto di raggio $r$!).

La serie delle derivate rispetto a $\rho $ ha termine generale $n\frac{1}{r}%
\left( \frac{\rho }{r}\right) ^{n-1}\left( \alpha _{n}\cos n\theta +\beta
_{n}\sin n\theta \right) $, per cui, come sopra, si ottiene la maggiorazione 
$\frac{2}{\pi r}n\left( \frac{\delta }{r}\right) ^{n-1}\left\vert \left\vert
f\right\vert \right\vert _{L^{1}\left( \left[ 0,2\pi \right] \right) }$, che 
\`{e} il termine generale di una serie convergente: quindi la serie delle
derivate converge totalmente in ogni cerchio chiuso, perci\`{o} esiste $%
\frac{\partial u}{\partial \rho }$ ed \`{e} calcolabile derivando termine a
termine in ogni cerchio chiuso (si ha anche la convergenza puntuale della
serie non derivata, dato che questa converge totalmente). Si capisce che il
ragionamento si pu\`{o} iterare alla derivata di qualunque ordine rispetto a 
$\rho $, e lo stesso vale per le derivate rispetto a $\theta $ e le derivate
miste. Se ne conclude che tutte le derivate sono somma di una serie di
potenze che converge totalmente, perci\`{o} ogni ogni derivata \`{e}
calcolabile derivando termine a termine. Inoltre $u\in C^{\infty }\left( 
\bar{B}_{\delta }\left( \mathbf{0}\right) \right) $ $\forall $ $\delta <r$:
ma allora, $\forall $ $\left( \rho ,\theta \right) \in B_{r}\left( \mathbf{0}%
\right) $ $\exists $ $\delta :\left( \rho ,\theta \right) \in \bar{B}%
_{\delta }\left( \mathbf{0}\right) $, e dunque $u\in C^{\infty }\left(
B_{r}\left( \mathbf{0}\right) \right) \footnote{%
Questo ragionamento funziona perch\'{e} la continuit\`{a} \`{e} un concetto
locale; non si pu\`{o} fare per estendere la convergenza totale a tutto il
cerchio aperto, perch\'{e} la convergenza totale \`{e} un concetto globale}$.

(iii) Poich\'{e} la serie si pu\`{o} derivare termine a termine, $\Delta
u=\sum_{n=1}^{+\infty }\left( \frac{\partial ^{2}}{\partial \rho ^{2}}+\frac{%
1}{\rho }\frac{\partial }{\partial \rho }+\frac{1}{\rho ^{2}}\frac{\partial
^{2}}{\partial \theta ^{2}}\right) \left( \frac{\rho }{r}\right) ^{n}\left(
\alpha _{n}\cos n\theta +\beta _{n}\sin n\theta \right) =\sum_{n=1}^{+\infty
}\frac{1}{r^{n}}\left( \alpha _{n}\Delta \left( \rho ^{n}\cos n\theta
\right) +\beta _{n}\Delta \left( \rho ^{n}\sin n\theta \right) \right) =0$
perch\'{e} ogni addendo \`{e} nullo, dato che $\rho ^{n}\cos n\theta $ e $%
\rho ^{n}\sin n\theta $ sono armoniche. $\blacksquare $

Riassumendo, se $f\in L^{1}\left( \left[ 0,2\pi \right] \right) $ la $u$
sopra \`{e} ben definita, risolve l'equazione ed \`{e} $C^{\infty }\left(
B_{r}\left( \mathbf{0}\right) \right) $.

\textbf{Analisi critica della soddisfazione della condizione al bordo} Si
deve poi valutare se $u$ soddisfa la condizione al bordo $u\left( r,\theta
\right) =f\left( \theta \right) $, precisando in quale senso la soddisfa.

\textbf{Teo (condizione al contorno classica)}
\begin{eqnarray*}
\text{Hp}\text{: } &&f:\left[ 0,2\pi \right] \rightarrow 
%TCIMACRO{\U{211d} }%
%BeginExpansion
\mathbb{R}
%EndExpansion
\text{ \`{e} continua e regolare a tratti, }f\left( 0\right) =f\left( 2\pi
\right) \\
\text{Ts}\text{: } &&u\text{ \`{e} continua fino al bordo del cerchio e in
particolare }\lim_{\left( \rho ,\theta \right) \rightarrow \left( r,\theta
_{0}\right) }u\left( \rho ,\theta \right) =f\left( \theta _{0}\right) \text{ 
}\forall \text{ }\theta _{0}\in \lbrack 0,2\pi )
\end{eqnarray*}

Le richieste di continuit\`{a} e $f\left( 0\right) =f\left( 2\pi \right) $
devono essere viste come "continuit\`{a} sulla circonferenza". Si noti che
se $f$ \`{e} continua \`{e} anche $L^{1}\left( \left[ 0,2\pi \right] \right) 
$, quindi vale anche il teorema sopra sulla regolarit\`{a} di $u$. Dunque,
con le ipotesi di questo teorema, $u$ \`{e} \textit{soluzione classica} del
problema di Dirichlet.

\textbf{Dim} Se $f:\left[ 0,2\pi \right] \rightarrow 
%TCIMACRO{\U{211d} }%
%BeginExpansion
\mathbb{R}
%EndExpansion
$ \`{e} continua, regolare a tratti e $f\left( 0\right) =f\left( 2\pi
\right) $, allora $\sum_{n=1}^{+\infty }\left( \left\vert \alpha
_{n}\right\vert +\left\vert \beta _{n}\right\vert \right) <+\infty $, cio%
\`{e} la serie di Fourier di $f$ converge totalmente: allora la serie che
assegna $u$ converge totalmente nel cerchio chiuso $\bar{B}_{r}\left( 
\mathbf{0}\right) $ (non c'\`{e} pi\`{u} bisogno di ricorrere all'argomento
con $\delta $ usato sopra: si maggiora $\rho $ con $r$). Quindi $u$ \`{e}
continua nel cerchio chiuso, in particolare sul bordo, e $u\left( r,\theta
\right) =f\left( \theta \right) $. $\blacksquare $ senso classico: serie
converge uniformente al dato al bordo

Si noti che prima di questo teorema si \`{e} ottenuta la convergenza totale
in ogni cerchio chiuso, ma non si sapeva niente della convergenza sul bordo.
Nei casi in cui questo teorema si applica, si dice che il dato al bordo \`{e}
assunto con continuit\`{a}, o che \`{e} assunto in senso classico. In realt%
\`{a} si pu\`{o} mostrare che basta che $f$ sia continua; spesso la tecnica
di separazione delle variabili porta a ipotesi eccessive.

Se invece $f$ non \`{e} regolare a tratti ed eventualmente neanche continua, 
\`{e} vero che $\lim_{\rho \rightarrow r^{-}}u\left( \rho ,\theta \right)
=f\left( \theta \right) $ in un qualche senso? S\`{\i}: si pu\`{o} ancora
dar senso alla condizione al contorno grazie alla teoria delle serie di
Fourier in $L^{2}$.

\textbf{Teo (dato al bordo }$L^{2}$\textbf{)}%
\begin{eqnarray*}
\text{Hp}\text{: } &&f\in L^{2}\left( \left[ 0,\pi \right] \right) \\
\text{Ts}\text{: } &&\left\vert \left\vert u\left( \rho ,\cdot \right)
-f\right\vert \right\vert _{L^{2}\left( 0,2\pi \right) }\rightarrow 0\text{
per }\rho \rightarrow r^{-}
\end{eqnarray*}

Si noti che se $f\in L^{2}\left( \left[ 0,\pi \right] \right) $ allora \`{e}
anche $L^{1}\left( \left[ 0,\pi \right] \right) $, e quindi si applica
ancora il teorema di regolarit\`{a} della soluzione.

\textbf{Dim} Poich\'{e} $f$ \`{e} $L^{2}$, $\sum_{n=1}^{+\infty }\left(
\alpha _{n}^{2}+\beta _{n}^{2}\right) <+\infty $ (che \`{e} una condizione pi%
\`{u} debole di quella che garantisce la convergenza totale!). $u\left( \rho
,\theta \right) =\frac{\alpha _{0}}{2}+\sum_{n=1}^{+\infty }\left( \frac{%
\rho }{r}\right) ^{n}\left( \alpha _{n}\cos n\theta +\beta _{n}\sin n\theta
\right) $ $\forall $ $\rho <r$, $\forall $ $\theta \in \lbrack 0,2\pi )$,
punto per punto; inoltre $%
f\left( \theta \right) =^{L^{2}}\frac{\alpha _{0}}{2}+\sum_{n=1}^{+\infty
}\left( \alpha _{n}\cos n\theta +\beta _{n}\sin n\theta \right) $. Vale $%
u\left( \rho ,\theta \right) -f\left( \theta \right) =\sum_{n=1}^{+\infty
}\left( \left( \frac{\rho }{r}\right) ^{n}-1\right) \left( \alpha _{n}\cos
n\theta +\beta _{n}\sin n\theta \right) $. Il lato sinistro \`{e} $L^{2}$
perch\'{e} $u$ lo \`{e} in quanto continua. Allora, per il teorema di
Pitagora in spazi di Hilbert, $\left\vert \left\vert u\left( \rho ,\theta
\right) -f\left( \theta \right) \right\vert \right\vert
_{L^{2}}^{2}=\int_{0}^{2\pi }\left\vert u\left( \rho ,\theta \right)
-f\left( \theta \right) \right\vert ^{2}=\sum_{n=1}^{+\infty }\left\vert
\left\vert \left( \left( \frac{\rho }{r}\right) ^{n}-1\right) \left( \alpha
_{n}\cos n\theta +\beta _{n}\sin n\theta \right) \right\vert \right\vert
^{2}=\sum_{n=1}^{+\infty }\left( \left( \frac{\rho }{r}\right) ^{n}-1\right)
^{2}\pi \left( \alpha _{n}^{2}+\beta _{n}^{2}\right) $. Per calcolare il
limite per $\rho \rightarrow r^{-}$ vorrei scambiare limite e serie: e
questo \`{e} lecito perch\'{e}, vedendo il lato destro come serie di
funzioni nella variabile $\rho $, si ha $\left\vert \left( \left( \frac{\rho 
}{r}\right) ^{n}-1\right) ^{2}\pi \left( \alpha _{n}^{2}+\beta
_{n}^{2}\right) \right\vert \leq \pi \left( \alpha _{n}^{2}+\beta
_{n}^{2}\right) $, la cui serie converge, e dunque la convergenza della
serie di funzioni \`{e} totale. Allora $\lim_{\rho \rightarrow
r^{-}}\left\vert \left\vert u\left( \rho ,\theta \right) -f\left( \theta
\right) \right\vert \right\vert _{L^{2}}^{2}=\sum_{n=1}^{+\infty }\lim_{\rho
\rightarrow r^{-}}\left( \left( \frac{\rho }{r}\right) ^{n}-1\right) ^{2}\pi
\left( \alpha _{n}^{2}+\beta _{n}^{2}\right) =0$. $\blacksquare $

Quando il teorema si applica si dice che il dato al bordo \`{e} assunto in
senso $L^{2}$.

Si noti che in tutta questa trattazione non si sono esibite ipotesi sotto le
quali $u$ rientri nella classe di funzioni per cui vale il teorema di unicit%
\`{a} (non si \`{e} dimostrato che $u$ \`{e} $C^{1}$ \textit{fino al bordo}%
), per cui ad ora non si pu\`{o} affermare che la soluzione trovata sia
unica.

\subsection{Problema di Neumann per il laplaciano sul cerchio}

Il problema \`{e}%
\begin{equation*}
\left\{ 
\begin{array}{c}
\Delta u=0\text{ se }\rho <r,\theta \in \lbrack 0,2\pi ) \\ 
\frac{\partial u}{\partial \rho }\left( r,\theta \right) =f\left( \theta
\right)%
\end{array}%
\right.
\end{equation*}

Infatti la direzione normale \`{e} proprio quella radiale.

Usando di nuovo la separazione delle variabili, si cercano ancora soluzioni
del tipo $u\left( \rho ,\theta \right) =a_{0}+\sum_{n=1}^{+\infty }\rho
^{n}\left( a_{n}\cos n\theta +b_{n}\sin n\theta \right) $ con $a_{n},b_{n}$
da determinare in base alla condizione al bordo.

Supponendo che la serie si possa derivare termine a termine, $\frac{\partial
u}{\partial \rho }=\sum_{n=1}^{+\infty }n\rho ^{n-1}\left( a_{n}\cos n\theta
+b_{n}\sin n\theta \right) $: imponendo la condizione si ottiene $%
\sum_{n=1}^{+\infty }nr^{n-1}\left( a_{n}\cos n\theta +b_{n}\sin n\theta
\right) =f\left( \theta \right) $. Lo sviluppo di $f$ in serie di Fourier
(supponendo che sia sviluppabile) \`{e} $\frac{A_{0}}{2}+\sum_{n=1}^{+\infty
}\left( A_{n}\cos n\theta +B_{n}\sin n\theta \right) $. Imponendo $%
A_{0}=0,nr^{n-1}a_{n}=A_{n},nr^{n-1}b_{n}=B_{n}$ si ottiene $a_{n}=\frac{%
A_{n}}{nr^{n-1}},b_{n}=\frac{B_{n}}{nr^{n-1}}$, mentre $a_{0}$ resta
indeterminato, e ci\`{o} \`{e} perfettamente coerente col fatto che per il
problema di Neumann la soluzione \`{e} unica a meno di una costante. Quindi
la candidata soluzione \`{e} 
\begin{equation*}
u\left( \rho ,\theta \right) =a_{0}+r\sum_{n=1}^{+\infty }\frac{1}{n}\left( 
\frac{\rho }{r}\right) ^{n}\left( A_{n}\cos n\theta +B_{n}\sin n\theta
\right)
\end{equation*}

\textbf{Analisi critica} E' ancora vero che se $f\in L^{1}\left( \left[
0,2\pi \right] \right) $ allora la convergenza della serie che definisce $u$ 
\`{e} totale in ogni cerchio di raggio $\delta <r$, e questo vale anche per
tutte le derivate: in tal caso quindi $u\in C^{\infty }\left( B_{r}\left( 
\mathbf{0}\right) \right) $ e $\Delta u=0$. E' anche lecita l'operazione
compiuta di derivazione termine a termine.

Affinch\'{e} il dato al bordo sia assunto in senso classico occorre che $%
\frac{\partial u}{\partial \rho }$ sia continua fino al bordo, e per dire ci%
\`{o} la convergenza totale della serie delle derivate $\sum_{n=1}^{+\infty
}\left( \frac{\rho }{r}\right) ^{n-1}\left( A_{n}\cos n\theta +B_{n}\sin
n\theta \right) $ nel cerchio chiuso \`{e} sufficiente: questa vale se $%
\sum_{n=1}^{+\infty }\left( \left\vert A_{n}\right\vert +\left\vert
B_{n}\right\vert \right) <+\infty $, il che segue dalle stesse condizioni di
regolarit\`{a} su $f$ viste sopra. In tal caso dunque $\frac{\partial u}{%
\partial \rho }\in C^{0}\left( \bar{B}_{r}\left( \mathbf{0}\right) \right) $
e la condizione al contorno \`{e} soddisfatta in senso classico.

\subsection{Rappresentazione integrale della soluzione del problema di
Dirichlet sul cerchio}

Le ipotesi che si sono viste affinch\'{e} $u$ sia soluzione classica del
problema di Dirichlet sono $f$ continua e regolare a tratti, $f\left(
0\right) =f\left( 2\pi \right) $. Esiste un altro modo di risolvere il
problema che permetta di indebolire queste ipotesi?

La strategia \`{e} riscrivere la soluzione in maniera da ottenere una
formulazione integrale, che sia ben posta anche con ipotesi meno forti%
\footnote{%
Le formule risolutive integrali in genere richiedono a $f$ meno regolarit%
\`{a} delle formule risolutive per serie.}.

Si ha $u\left( \rho ,\theta \right) =\frac{a_{0}}{2}+\sum_{n=1}^{+\infty
}\left( \frac{\rho }{r}\right) ^{n}\left( a_{n}\cos n\theta +b_{n}\sin
n\theta \right) $, con $a_{n}=\frac{1}{\pi }\int_{0}^{2\pi }f\left( s\right)
\cos nsds,b_{n}=\frac{1}{\pi }\int_{0}^{2\pi }f\left( s\right) \sin nsds$.
Sostituendo in $u$ si ha $u\left( \rho ,\theta \right) =\frac{1}{2\pi }%
\int_{0}^{2\pi }f\left( s\right) ds+\frac{1}{\pi }\sum_{n=1}^{+\infty
}\left( \frac{\rho }{r}\right) ^{n}\left( \int_{0}^{2\pi }f\left( s\right)
\cos nsds\cos n\theta +\int_{0}^{2\pi }f\left( s\right) \sin nsds\sin
n\theta \right) $, e la seconda parentesi si pu\`{o} riscrivere con la
formula di sottrazione del coseno, per cui $u\left( \rho ,\theta \right) =%
\frac{1}{2\pi }\int_{0}^{2\pi }f\left( s\right) ds+\frac{1}{\pi }%
\sum_{n=1}^{+\infty }\left( \frac{\rho }{r}\right) ^{n}\left( \int_{0}^{2\pi
}f\left( s\right) \cos n\left( \theta -s\right) ds\right) $. Scambio serie e
integrale (pi\`{u} tardi si vedranno le ipotesi sotto cui l'operazione \`{e}
lecita): $u\left( \rho ,\theta \right) =\frac{1}{2\pi }\int_{0}^{2\pi
}f\left( s\right) ds+\frac{1}{\pi }\int_{0}^{2\pi }\left[ f\left( s\right)
\sum_{n=1}^{+\infty }\left( \frac{\rho }{r}\right) ^{n}\cos n\left( \theta
-s\right) \right] ds=\frac{1}{\pi }\int_{0}^{2\pi }f\left( s\right) \left[ 
\frac{1}{2}+\sum_{n=1}^{+\infty }\left( \frac{\rho }{r}\right) ^{n}\cos
n\left( s-\theta \right) \right] ds$. Si definisce $k\left( \rho ,\theta
-s\right) :=\frac{1}{2}+\sum_{n=1}^{+\infty }\left( \frac{\rho }{r}\right)
^{n}\cos n\left( s-\theta \right) $. In questo caso sappiamo calcolare la
somma della serie: si nota che $\left( \frac{\rho }{r}\right) ^{n}\cos
n\left( \theta -s\right) =\func{Re}\left( \frac{\rho }{r}\right)
^{n}e^{in\left( \theta -s\right) }=\func{Re}\left( \frac{\rho }{r}e^{i\left(
\theta -s\right) }\right) ^{n}$. Sostituendo e scambiando parte reale e
serie si trova $k\left( \rho ,\theta -s\right) =\frac{1}{2}+\func{Re}%
\sum_{n=1}^{+\infty }\left( \frac{\rho }{r}e^{i\left( \theta -s\right)
}\right) ^{n}=\frac{1}{2}+\func{Re}\left( \frac{1}{1-\frac{\rho }{r}%
e^{i\left( \theta -s\right) }}-1\right) =\frac{1}{2}\frac{r^{2}-\rho ^{2}}{%
r^{2}-2r\rho \cos \left( \theta -s\right) +\rho ^{2}}$. Definendo $K\left(
\rho ,\theta -s\right) :=\frac{k\left( \rho ,\theta -s\right) }{\pi }$, si
ottiene infine%
\begin{equation*}
u\left( \rho ,\theta \right) =\int_{0}^{2\pi }f\left( s\right) K\left( \rho
,\theta -s\right) ds=\frac{1}{2\pi }\int_{0}^{2\pi }f\left( s\right) \frac{%
r^{2}-\rho ^{2}}{r^{2}-2r\rho \cos \left( \theta -s\right) +\rho ^{2}}ds
\end{equation*}

che pu\`{o} essere visto come un integrale di convoluzione, ma fatto sulla
circonferenza. E' una formula di rappresentazione di gran lunga pi\`{u}
snella di quella che usa le serie, e peraltro richiede di calcolare un solo
integrale. $K\left( \rho ,\theta -s\right) $ si dice nucleo integrale di
Poisson.

Lo scambio fatto sopra tra serie e integrale \`{e} lecito se la serie di
termine generale $f_{n}\left( s\right) =\left( \frac{\rho }{r}\right)
^{n}f\left( s\right) \cos ns$ (l'argomento del coseno sarebbe $n\left(
s-\theta \right) $, ma una traslazione di $\theta $ \`{e} irrilevante)
converge totalmente. Questo \`{e} vero se $f$ \`{e} limitata (oltre che $%
L^{1}\left( \left[ 0,2\pi \right] \right) $): infatti in tal caso $%
\left\vert f_{n}\left( s\right) \right\vert \leq \sup \left\vert
f\right\vert \left( \frac{\rho }{r}\right) ^{n}$, termine generale di una
serie convergente se $\rho <r$. Quindi, se $f$ \`{e} $L^{1}\left( \left[
0,2\pi \right] \right) $ e inoltre limitata, la $u$ data dalla formula
risolutiva per serie coincide con quella data dalla formula di
rappresentazione integrale. Da questo segue immediatamente che $u\in
C^{\infty }\left( B_{r}\left( \mathbf{0}\right) \right) ,\Delta u=0$.

Il seguente teorema stabilisce sotto quali ipotesi $u$ assume il dato al
bordo in senso classico.

\textbf{Teo (condizione al contorno classica)}
\begin{gather*}
\text{Hp}\text{: }f:\left[ 0,2\pi \right] \rightarrow 
%TCIMACRO{\U{211d} }%
%BeginExpansion
\mathbb{R}
%EndExpansion
\text{ \`{e} continua, }f\left( 0\right) =f\left( 2\pi \right) \\
\text{Ts}\text{: }u\text{ assegnata della formula integrale di Poisson \`{e}
continua fino al bordo } \\
\text{del cerchio e in particolare }\lim_{\left( \rho ,\theta \right)
\rightarrow \left( r^{-},\theta _{0}\right) }u\left( \rho ,\theta \right)
=f\left( \theta _{0}\right) \text{ }\forall \text{ }\theta _{0}\in \lbrack
0,2\pi )
\end{gather*}

Quindi, se $f$ soddisfa queste ipotesi (che implicano $f$ integrabile e
limitata), $u$ \`{e} soluzione classica. Rispetto alle ipotesi viste con la
formula assegnata per serie, si \`{e} rimossa l'ipotesi di regolarit\`{a} a
tratti.

\textbf{Dim} In primo luogo si osservano alcune propriet\`{a} del nucleo di
Poisson. (1) Poich\'{e} $K\left( \rho ,\theta \right) =\frac{1}{2\pi }\frac{%
r^{2}-\rho ^{2}}{r^{2}-2r\rho \cos \left( \theta \right) +\rho ^{2}}$, il
numeratore \`{e} positivo $\forall $ $\rho <r$, mentre $r^{2}-2r\rho \cos
\left( \theta -s\right) +\rho ^{2}\geq r^{2}-2r\rho +\rho ^{2}=\left( r-\rho
\right) ^{2}>0$ se $\rho <r$ (se poi $\rho \leq \delta <r$, il denominatore 
\`{e} $\geq \left( r-\rho \right) ^{2}>0$, cio\`{e} discosto da $0$), quindi 
$K\left( \rho ,\theta \right) >0$ se $\rho <r,\theta \in \left[ 0,2\pi %
\right] $.

(2) Inoltre $\int_{0}^{2\pi }K\left( \rho ,\theta \right) d\theta =1$ $%
\forall $ $\rho <r$. Infatti, se $f\in L^{1}\cap L^{\infty }$, la $u$ data
dalla formula risolutiva per serie coincide con quella data dalla formula di
rappresentazione integrale; inoltre, se $f$ \`{e} continua e regolare a
tratti, $f\left( 0\right) =f\left( 2\pi \right) $, allora $u$ data dalla
formula risolutiva per serie assume il dato al bordo in senso classico. $f=1$
soddisfa tutte queste ipotesi, quindi $u=1$ risolve il problema di
Dirichlet, e inoltre $u\left( \rho ,\theta \right) =\int_{0}^{2\pi }K\left(
\rho ,\theta \right) d\theta $ $\forall $ $\rho <r,\theta \in \left[ 0,2\pi %
\right] $.

Ora si dimostra (1) che poich\'{e} $\forall $ $\theta _{0}\in \left[ 0,2\pi %
\right] $ $f\left( \theta \right) $ \`{e} continua in $\theta _{0}$, allora $%
\lim_{\rho \rightarrow r^{-}}u\left( \rho ,\theta _{0}\right) =f\left(
\theta _{0}\right) $ (convergenza radiale). Essendo $f$ continua in tutto $%
\left[ 0,2\pi \right] $ e quindi ivi uniformemente continua, la suddetta
convergenza di $u\left( \rho ,\theta \right) $ a $f\left( \theta \right) $ 
\`{e} uniforme rispetto a $\theta $ (cio\`{e}, fissato $\varepsilon $, il $%
\delta $ non dipende da $\theta $). (2) Se ora si sa che $f$ \`{e} continua
su tutto il bordo, da (1) segue che $\lim_{\left( \rho ,\theta \right)
\rightarrow \left( r^{-},\theta _{0}\right) }u\left( \rho ,\theta \right)
=f\left( \theta _{0}\right) $. Infatti $\left\vert u\left( \rho ,\theta
\right) -f\left( \theta _{0}\right) \right\vert \leq \left\vert u\left( \rho
,\theta \right) -f\left( \theta \right) \right\vert +\left\vert f\left(
\theta \right) -f\left( \theta _{0}\right) \right\vert $: ma per continuit%
\`{a} di $f$ $\forall $ $\varepsilon >0$ $\exists $ $\delta _{2}=\delta
_{2}\left( \varepsilon \right) >0:\left\vert \theta -\theta _{0}\right\vert
<\delta _{2}\Longrightarrow \left\vert f\left( \theta \right) -f\left(
\theta _{0}\right) \right\vert <\varepsilon $, mentre per (1) $\exists $ $%
\delta _{1}=\delta _{1}\left( \varepsilon \right) >0:\left\vert \rho
-r\right\vert <\delta _{1}\Longrightarrow \left\vert u\left( \rho ,\theta
\right) -f\left( \theta \right) \right\vert <\varepsilon $ $\forall $ $%
\theta \in \left[ 0,2\pi \right] $, quindi $\left\vert u\left( \rho ,\theta
\right) -f\left( \theta _{0}\right) \right\vert $ tende a $0$ se $\rho
\rightarrow r^{-},\theta \rightarrow \theta _{0}$.

Si dimostra (1). Sia $\rho <r$. $u\left( \rho ,\theta _{0}\right) -f\left(
\theta _{0}\right) =\int_{0}^{2\pi }f\left( s\right) K\left( \rho ,\theta
_{0}-s\right) ds-f\left( \theta _{0}\right) =\int_{0}^{2\pi }K\left( \rho
,\theta _{0}-s\right) \left( f\left( s\right) -f\left( \theta _{0}\right)
\right) ds$ per la propriet\`{a} (2) del nucleo; allora $\left\vert u\left(
\rho ,\theta _{0}\right) -f\left( \theta _{0}\right) \right\vert \leq
\int_{0}^{2\pi }K\left( \rho ,\theta _{0}-s\right) \left\vert f\left(
s\right) -f\left( \theta _{0}\right) \right\vert ds=\int_{\substack{ s\in %
\left[ 0,2\pi \right]  \\ \left\vert s-\theta _{0}\right\vert >\delta }}%
id+\int_{\substack{ s\in \left[ 0,2\pi \right]  \\ \left\vert s-\theta
_{0}\right\vert <\delta }}id=A_{\delta }+B_{\delta }$, dove $\delta >0$ \`{e}
ancora da definire. Poich\'{e} $f$ \`{e} continua in $\theta _{0}$, fissato $%
\varepsilon >0$ $\exists $ $\delta >0:\left\vert s-\theta _{0}\right\vert
<\delta \Longrightarrow \left\vert f\left( s\right) -f\left( \theta
_{0}\right) \right\vert <\varepsilon $. Scelgo proprio questo $\delta $ (che
in realt\`{a} \`{e} indipendente da $\theta _{0}$, perch\'{e} $f$ \`{e}
continua nel compatto $\left[ 0,2\pi \right] $ e quindi uniformemente
continua!): si ha in tal caso $B_{\delta }<\varepsilon \int_{\substack{ s\in %
\left[ 0,2\pi \right]  \\ \left\vert s-\theta _{0}\right\vert <\delta }}%
K\left( \rho ,\theta _{0}-s\right) ds\leq ^{K>0}\varepsilon \int_{s\in \left[
0,2\pi \right] }K\left( \rho ,\theta _{0}-s\right) ds=\footnote{%
con un cambio di variabile si vede facilmente che la traslazione nella
seconda variabile preserva la propriet\`{a} di integrare a $1$}\varepsilon $ 
$\forall $ $\rho <r$; l'ultima disuguaglianza \`{e} valida grazie alla
positivit\`{a} del nucleo. Invece $A_{\delta }=\int_{\substack{ s\in \left[
0,2\pi \right]  \\ \left\vert s-\theta _{0}\right\vert >\delta }}K\left(
\rho ,\theta _{0}-s\right) \left\vert f\left( s\right) -f\left( \theta
_{0}\right) \right\vert ds\leq 2\max \left\vert f\right\vert \int_{\substack{
s\in \left[ 0,2\pi \right]  \\ \left\vert s-\theta _{0}\right\vert >\delta }}%
K\left( \rho ,\theta _{0}-s\right) ds$. Poich\'{e} $\left\vert s-\theta
_{0}\right\vert >\delta $, non si possono sfruttare propriet\`{a} di $f$:
occorre usare la definizione di $K$. $K\left( \rho ,\theta _{0}-s\right) =%
\frac{1}{2\pi }\frac{r^{2}-\rho ^{2}}{r^{2}-2r\rho \cos \left( \theta
_{0}-s\right) +\rho ^{2}}$: se $\left\vert s-\theta _{0}\right\vert >\delta $%
, $\cos \left( \theta _{0}-s\right) <\cos \delta $ (se $\delta >\frac{\pi }{2%
}$ la disuguaglianza si inverte??), quindi $r^{2}-2r\rho \cos \left( \theta
_{0}-s\right) +\rho ^{2}\geq r^{2}-2r\rho \cos \delta +\rho ^{2}$. Allora $%
K\left( \rho ,\theta _{0}-s\right) \leq \frac{1}{2\pi }\frac{r^{2}-\rho ^{2}%
}{r^{2}-2r\rho \cos \delta +\rho ^{2}}$: se $\rho \rightarrow r^{-}$, il
denominatore tende a $2r^{2}\left( 1-\cos \delta \right) >0$, il numeratore
tende a $0$, quindi $A_{\delta }+B_{\delta }\rightarrow 0$ e $\left\vert
u\left( \rho ,\theta _{0}\right) -f\left( \theta _{0}\right) \right\vert
\rightarrow 0$ per $\rho \rightarrow r^{-}$. $\blacksquare $

Di conseguenza, richiedendo solo $f:\left[ 0,2\pi \right] \rightarrow 
%TCIMACRO{\U{211d} }%
%BeginExpansion
\mathbb{R}
%EndExpansion
$ continua e $f\left( 0\right) =f\left( 2\pi \right) $, la $u$ assegnata
dalla formula integrale di Poisson \`{e} l'unica soluzione classica del
problema di Dirichlet, ed \`{e} $C^{\infty }\left( B_{r}\left( \mathbf{0}%
\right) \right) $.

Si potrebbe tuttavia seguire un percorso logico diverso e dimostrare
direttamente dalla formula di rappresentazione integrale che la $u$ cos\`{\i}
assegnata \`{e} $C^{\infty }$ e $\Delta u=0$. Per fare ci\`{o} occorre
calcolare le derivate $u_{\rho \rho },u_{\theta \theta },u_{\rho }$: come
calcolare e. g. $\frac{\partial }{\partial \rho }u\left( \rho ,\theta
\right) =\frac{\partial }{\partial \rho }\int_{0}^{2\pi }f\left( s\right)
K\left( \rho ,\theta -s\right) ds$? E, prima ancora, come fare a stabilire
la continuit\`{a} della $u$ come funzione di $\rho $? Questo \`{e} un caso
del problema generale di stabilire continuit\`{a} e derivabilit\`{a} di un
integrale dipendente da un parametro.

\textbf{Continuit\`{a} e derivabilit\`{a} di un integrale dipendente da un
parametro} Il problema in forma generale \`{e} il seguente: vogliamo capire
sotto quali ipotesi $u\left( x\right) :=\int_{\Omega }k\left( x,\mathbf{y}%
\right) d\mathbf{y}$, con $\mathbf{y}\subseteq \Omega \subseteq 
%TCIMACRO{\U{211d} }%
%BeginExpansion
\mathbb{R}
%EndExpansion
^{n}$ e $x\in \left( a,b\right) $, \`{e} continua e sotto quali derivabile.
I seguenti teoremi rispondono a queste domande.

\textbf{Teo (continuit\`{a})}%
\begin{gather*}
\text{Hp: }u\left( x\right) :=\int_{\Omega }k\left( x,\mathbf{y}\right) d%
\mathbf{y}\text{, }\mathbf{y}\subseteq \Omega \subseteq 
%TCIMACRO{\U{211d} }%
%BeginExpansion
\mathbb{R}
%EndExpansion
^{n}\text{, }x\in \left( a,b\right) \text{; }\forall \text{ }x\in \left(
a,b\right) \text{ }k\left( x,\cdot \right) \in L^{1}\left( \Omega \right) 
\text{;} \\
x_{0}\in \left( a,b\right) \text{; per quasi ogni }\mathbf{y}\in \Omega 
\text{ la funzione }x\mapsto k\left( x,y\right) \text{ \`{e} continua in }%
x_{0}\in \left( a,b\right) \text{;} \\
\exists \text{ }B_{\delta }\left( x_{0}\right) ,g\in L^{1}\left( \Omega
\right) :\left\vert k\left( x,\mathbf{y}\right) \right\vert \leq g\left( 
\mathbf{y}\right) \text{ }\forall \text{ }x\in B_{\delta }\left(
x_{0}\right) \text{, per quasi ogni }\mathbf{y}\in \Omega \\
\text{Ts: }u\text{ \`{e} continua in }x_{0}\text{ e }\lim_{x\rightarrow
x_{0}}\int_{\Omega }k\left( x,\mathbf{y}\right) d\mathbf{y=}\int_{\Omega
}\left( \lim_{x\rightarrow x_{0}}k\left( x,\mathbf{y}\right) \right) d%
\mathbf{y}
\end{gather*}

Queste ipotesi permettono lo scambio tra limite e integrale, come \`{e}
naturale sperare.

\textbf{Teo (derivabilit\`{a})}%
\begin{gather*}
\text{Hp: }u\left( x\right) :=\int_{\Omega }k\left( x,\mathbf{y}\right) d%
\mathbf{y}\text{, }\mathbf{y}\subseteq \Omega \subseteq 
%TCIMACRO{\U{211d} }%
%BeginExpansion
\mathbb{R}
%EndExpansion
^{n}\text{, }x\in \left( a,b\right) \text{; }\forall \text{ }x\in \left(
a,b\right) \text{ }k\left( x,\cdot \right) \in L^{1}\left( \Omega \right) 
\text{; }x_{0}\in \left( a,b\right) \text{; } \\
\exists \text{ }B_{\delta }\left( x_{0}\right) ,g\in L^{1}\left( \Omega
\right) :\exists \text{ }\frac{\partial k}{\partial x}\left( x,\mathbf{y}%
\right) \text{ e }\left\vert \frac{\partial k\left( x,\mathbf{y}\right) }{%
\partial x}\right\vert \leq g\left( \mathbf{y}\right) \text{ }\forall \text{ 
}x\in B_{\delta }\left( x_{0}\right) \text{, per quasi ogni }\mathbf{y}\in
\Omega \\
\text{Ts: }\exists \text{ }u^{\prime }\left( x_{0}\right) =\int_{\Omega }%
\frac{\partial }{\partial x}k\left( x,\mathbf{y}\right) d\mathbf{y}
\end{gather*}

Queste ipotesi permettono lo scambio tra derivata e integrale, come \`{e}
naturale sperare.

Entrambi i teoremi si basano sul teorema di convergenza dominata, come si pu%
\`{o} ravvisare dalle ipotesi.

Da questi teoremi seguono alcuni risultati sulla trasformata di Fourier che
saranno usati in seguito. Il primo teorema permette di dedurre che se $f\in
L^{1}\left( 
%TCIMACRO{\U{211d} }%
%BeginExpansion
\mathbb{R}
%EndExpansion
^{n}\right) $ la trasformata di Fourier di $f$ \`{e} una funzione continua
di $\mathbf{\xi }$; il secondo che se $f\in L^{1}\left( 
%TCIMACRO{\U{211d} }%
%BeginExpansion
\mathbb{R}
%EndExpansion
\right) ,xf\in L^{1}\left( 
%TCIMACRO{\U{211d} }%
%BeginExpansion
\mathbb{R}
%EndExpansion
\right) $ allora la trasformata \`{e} derivabile (in tal caso $g=xf$).
Ancora pi\`{u} in generale, se $\exists $ $k:\left( 1+\left\vert
x\right\vert ^{k}\right) f\in L^{1}\left( 
%TCIMACRO{\U{211d} }%
%BeginExpansion
\mathbb{R}
%EndExpansion
^{n}\right) $, allora $f\in C^{k}\left( 
%TCIMACRO{\U{211d} }%
%BeginExpansion
\mathbb{R}
%EndExpansion
^{n}\right) $. Inoltre, se si calcola la trasformata di $f^{\prime }$
integrando per parti, si vede che se $f\in L^{1}\left( 
%TCIMACRO{\U{211d} }%
%BeginExpansion
\mathbb{R}
%EndExpansion
^{n}\right) \cap C_{\ast }^{0}\left( 
%TCIMACRO{\U{211d} }%
%BeginExpansion
\mathbb{R}
%EndExpansion
^{n}\right) $ e $\exists $ q.o. $\frac{\partial f}{\partial x_{j}}\in
L^{1}\left( 
%TCIMACRO{\U{211d} }%
%BeginExpansion
\mathbb{R}
%EndExpansion
^{n}\right) $, allora $\mathcal{F}\left( \frac{\partial f}{\partial x_{j}}%
\right) =2\pi i\xi _{j}\hat{f}\left( \mathbf{\xi }\right) $. Estendendo le
ipotesi alla derivata seconda, si trova $\mathcal{F}\left( \frac{\partial
^{2}f}{\partial x_{j}^{2}}\right) =\left( 2\pi i\right) ^{2}\xi _{j}^{2}\hat{%
f}\left( \mathbf{\xi }\right) $. Dunque $\mathcal{F}\left( \Delta f\right) =%
\mathcal{F}\left( \frac{\partial ^{2}f}{\partial x^{2}}+\frac{\partial ^{2}f%
}{\partial y^{2}}\right) =-4\pi ^{2}\xi _{1}^{2}\hat{f}\left( \mathbf{\xi }%
\right) -4\pi ^{2}\xi _{2}^{2}\hat{f}\left( \mathbf{\xi }\right) =-4\pi
^{2}\left\vert \left\vert \mathbf{\xi }\right\vert \right\vert ^{2}\hat{f}%
\left( \mathbf{\xi }\right) $.

Si calcola infine $\mathcal{F}\left( e^{-a\left\vert x\right\vert }\right)
=\int_{%
%TCIMACRO{\U{211d} }%
%BeginExpansion
\mathbb{R}
%EndExpansion
}e^{-a\left\vert x\right\vert }e^{-2\pi i\xi x}dx=\int_{%
%TCIMACRO{\U{211d} }%
%BeginExpansion
\mathbb{R}
%EndExpansion
}e^{-a\left\vert x\right\vert }\cos \left( -2\pi \xi x\right) dx+i\int_{%
%TCIMACRO{\U{211d} }%
%BeginExpansion
\mathbb{R}
%EndExpansion
}e^{-a\left\vert x\right\vert }\sin \left( -2\pi \xi x\right) dx$, ma il
secondo addendo \`{e} nullo in quanto integrale di una funzione dispari.
Allora $\mathcal{F}\left( e^{-a\left\vert x\right\vert }\right) =\int_{%
%TCIMACRO{\U{211d} }%
%BeginExpansion
\mathbb{R}
%EndExpansion
}e^{-a\left\vert x\right\vert }\cos \left( -2\pi \xi x\right)
dx=2\int_{0}^{+\infty }\func{Re}\left( e^{-a\left\vert x\right\vert
}e^{-2\pi i\xi x}\right) dx=2\func{Re}\left( \int_{0}^{+\infty
}e^{-ax}e^{-2\pi i\xi x}dx\right) =$ $2\func{Re}\left[ \frac{-1}{a+2\pi i\xi 
}e^{-\left( a+2\pi i\xi \right) x}\right] _{0}^{+\infty }=2\func{Re}\frac{1}{%
a+2\pi i\xi }=2\func{Re}\left( \frac{a-2\pi i\xi }{a^{2}+4\pi ^{2}\xi ^{2}}%
\right) =\frac{2a}{a^{2}+4\pi ^{2}\xi ^{2}}$.

Si ricorda, infine, il teorema di inversione: se $f,\hat{f}\in L^{1}$, vale $%
f\left( \mathbf{x}\right) =\int_{%
%TCIMACRO{\U{211d} }%
%BeginExpansion
\mathbb{R}
%EndExpansion
^{n}}\hat{f}\left( \mathbf{\xi }\right) e^{2\pi i\left\langle \mathbf{x,\xi }%
\right\rangle }d\mathbf{\xi }$, per cui $\mathcal{F}\left( \mathcal{F}\left(
f\right) \right) =f\left( -\mathbf{x}\right) $.

\subsection{Problema di Dirichlet per il laplaciano sul semipiano}

Si cerca di risolvere il problema di Dirichlet per l'equazione di Laplace
nel caso particolare di $\Omega =\left\{ \left( x,y\right) \in 
%TCIMACRO{\U{211d} }%
%BeginExpansion
\mathbb{R}
%EndExpansion
^{2}:x\in 
%TCIMACRO{\U{211d} }%
%BeginExpansion
\mathbb{R}
%EndExpansion
,y>0\right\} $. Il problema \`{e} 
\begin{equation*}
\left\{ 
\begin{array}{c}
\Delta u=0\text{ se }x\in 
%TCIMACRO{\U{211d} }%
%BeginExpansion
\mathbb{R}
%EndExpansion
,y>0 \\ 
u\left( x,0\right) =f\left( x\right) \text{ se }x\in 
%TCIMACRO{\U{211d}}%
%BeginExpansion
\mathbb{R}%
%EndExpansion
\end{array}%
\right.
\end{equation*}

Prima o poi si dovr\`{a} imporre qualche condizione sul comportamento di $u$
all'infinito (che consideriamo far parte del "bordo"), altrimenti il
problema sarebbe sottodeterminato.

Considero $u$ funzione della sola $x$ e la trasformo secondo Fourier%
\footnote{%
non si potrebbe fare rispetto a $y$, dato che questa non varia in $%
%TCIMACRO{\U{211d} }%
%BeginExpansion
\mathbb{R}
%EndExpansion
$}: si ottiene $u\left( \xi ,y\right) =\int_{%
%TCIMACRO{\U{211d} }%
%BeginExpansion
\mathbb{R}
%EndExpansion
}u\left( x,y\right) e^{-2\pi i\xi x}dx$. Allora la trasformata rispetto a $x$
di $\Delta u$ \`{e} $\mathcal{F}_{x}\left( u_{xx}+u_{yy}\right) =-4\pi
^{2}\xi ^{2}\hat{u}\left( \xi ,y\right) +\int_{%
%TCIMACRO{\U{211d} }%
%BeginExpansion
\mathbb{R}
%EndExpansion
}u_{yy}\left( x,y\right) e^{-2\pi i\xi x}dx=-4\pi ^{2}\xi ^{2}\hat{u}\left(
\xi ,y\right) +\hat{u}_{yy}\left( \xi ,y\right) $, ricordando che $y$ \`{e}
un parametro e scambiando derivata e integrale. Imponendo $\Delta u=0$ si ottiene
l'equazione differenziale ordinaria lineare del second'ordine a coefficienti
costanti, nell'incognita $\hat{u}\left( \xi ,y\right) $ e nella variabile $y$%
,%
\begin{equation*}
-4\pi ^{2}\xi ^{2}\hat{u}\left( \xi ,y\right) +\hat{u}_{yy}\left( \xi
,y\right) =0
\end{equation*}

E' noto che la soluzione \`{e} del tipo $\hat{u}\left( \xi ,y\right)
=c_{1}\left( \xi \right) e^{2\pi \left\vert \xi \right\vert y}+c_{2}\left(
\xi \right) e^{-2\pi \left\vert \xi \right\vert y}$; i coefficienti della
combinazione sono costanti in $y$, variabile in cui si risolve l'EDO, quindi 
\`{e} opportuno esplicitare la dipendenza da $\xi $.

Si impone ora la condizione che la soluzione $u$ tenda a $0$ per $%
y\rightarrow +\infty $, il che implica $\lim_{y\rightarrow +\infty }\hat{u}%
\left( \xi ,y\right) =0$ $\forall $ $\xi $ per il teorema di continuit\`{a}.
Quindi dev'essere $c_{1}\left( \xi \right) =0$ $\forall $ $\xi $, da cui $%
\hat{u}\left( \xi ,y\right) =c_{2}\left( \xi \right) e^{-2\pi \left\vert \xi
\right\vert y}$.

Per dedurre quale condizione al bordo si deve imporre sulla trasformata di $%
u $, si trasforma la condizione al bordo e si ottiene $\hat{u}\left( \xi
,0\right) =\hat{f}\left( \xi \right) $. Ma dalla formula sopra si ha $\hat{u}%
\left( \xi ,0\right) =c_{2}\left( \xi \right) $, per cui $c_{2}\left( \xi
\right) =\hat{f}\left( \xi \right) $.

La formula esplicita per la trasformata della soluzione \`{e} dunque $\hat{u}%
\left( \xi ,y\right) =\hat{f}\left( \xi \right) e^{-2\pi \left\vert \xi
\right\vert y}$: occorre antitrasformarla. La strategia naturale (grazie
all'iniettivit\`{a} della trasformata) \`{e} antitrasformare $e^{-2\pi
\left\vert \xi \right\vert y}$ e usare al contrario la regola di
trasformazione di un prodotto di convoluzione. Poich\'{e} $\mathcal{F}\left( 
\mathcal{F}\left( f\right) \right) =f\left( -x\right) $ ed \`{e} noto che $%
\mathcal{F}\left( e^{-a\left\vert x\right\vert }\right) =\frac{2a}{%
a^{2}+4\pi ^{2}\xi ^{2}}$, si ha che $\mathcal{F}\left( \mathcal{F}\left(
e^{-a\left\vert x\right\vert }\right) \right) =\mathcal{F}\left( \frac{2a}{%
a^{2}+4\pi ^{2}\xi ^{2}}\right) =e^{-a\left\vert x\right\vert }$.
L'antitrasformata che dobbiamo trovare si ha dunque ponendo $a=2\pi y$: vale 
$\mathcal{F}\left( \frac{4\pi y}{4\pi ^{2}y^{2}+4\pi ^{2}x^{2}}\right)
=e^{-2\pi y\left\vert \xi \right\vert }$, per cui l'antitrasformata della
soluzione \`{e}, definito $k_{y}\left( x\right) =\frac{y}{\pi \left(
y^{2}+x^{2}\right) }$ nucleo di Poisson nel semipiano,%
\begin{equation*}
u\left( x,y\right) =f\ast k_{y}=\int_{%
%TCIMACRO{\U{211d} }%
%BeginExpansion
\mathbb{R}
%EndExpansion
}f\left( z\right) k_{y}\left( x-z\right) dz
\end{equation*}

che \`{e} detta formula integrale di Poisson nel semipiano.

\textbf{Analisi critica} Richiediamo che $u$ si possa derivare rispetto a $x$
e $y$ scambiando derivata e integrale (in modo che si possa verificare
facilmente che $\Delta u=0$). Per il teorema di derivabilit\`{a} occorre che 
$f\left( z\right) \frac{\partial }{\partial x\left( y\right) }k_{y}\left(
x-z\right) $ abbia una dominante integrabile: se la derivata \`{e} limitata, 
\`{e} sufficiente che $f$ sia integrabile. Ma per la forma di $k_{y}$ le sue
derivate di ogni ordine sono limitate in $x$ e $y$ se $y\geq \delta >0$:
dunque in ogni semipiano di questo tipo si ha $\frac{\partial ^{\alpha }}{%
\partial x^{\alpha }}k_{y}\left( x-z\right) f\left( z\right) \leq
c\left\vert f\left( z\right) \right\vert $, e perci\`{o} $u$ \`{e} $%
C^{\infty }$ nel semipiano (non fino al bordo!) e si possono scambiare
derivata e integrale.

Si ha quindi $\Delta u\left( x,y\right) =\Delta \left( \int_{%
%TCIMACRO{\U{211d} }%
%BeginExpansion
\mathbb{R}
%EndExpansion
}f\left( z\right) k_{y}\left( x-z\right) dz\right) =\int_{%
%TCIMACRO{\U{211d} }%
%BeginExpansion
\mathbb{R}
%EndExpansion
}\Delta \left[ f\left( z\right) k_{y}\left( x-z\right) \right] dz$. Si
potrebbero calcolare con pazienza tutte le derivate, ma si nota che $%
k_{y}\left( x\right) =\frac{y}{\pi \left( y^{2}+x^{2}\right) }=\func{Im}%
\left( \frac{-1}{\pi z}\right) $, e $\frac{-1}{\pi z}$ \`{e} olomorfa in $%
%TCIMACRO{\U{2102} }%
%BeginExpansion
\mathbb{C}
%EndExpansion
\backslash \left\{ 0\right\} $, dunque la sua parte immaginaria \`{e}
armonica in $%
%TCIMACRO{\U{211d} }%
%BeginExpansion
\mathbb{R}
%EndExpansion
^{2}\backslash \left\{ \left( 0,0\right) \right\} ,y>0$ e ivi soddisfa $%
\Delta u=0$.

Si \`{e} quindi ottenuto che se $f\in L^{1}\left( 
%TCIMACRO{\U{211d} }%
%BeginExpansion
\mathbb{R}
%EndExpansion
\right) $ allora $u\in C^{\infty }$ e $\Delta u=0$.

Per determinare in che modo \`{e} assunta la condizione al bordo occorre una
parentesi sui nuclei regolarizzanti.

\subsubsection{Nuclei regolarizzanti}

Sia $\phi :%
%TCIMACRO{\U{211d} }%
%BeginExpansion
\mathbb{R}
%EndExpansion
^{n}\rightarrow 
%TCIMACRO{\U{211d} }%
%BeginExpansion
\mathbb{R}
%EndExpansion
,\phi \in L^{1}\left( 
%TCIMACRO{\U{211d} }%
%BeginExpansion
\mathbb{R}
%EndExpansion
^{n}\right) ,\phi \geq 0:\int_{%
%TCIMACRO{\U{211d} }%
%BeginExpansion
\mathbb{R}
%EndExpansion
^{n}}\phi \left( \mathbf{x}\right) d\mathbf{x}=1$: $\phi $ siffatta si dice
funzione madre.

\begin{enumerate}
\item $\phi \left( \mathbf{x}\right) =e^{-\pi \left\vert \mathbf{x}%
\right\vert ^{2}}$ soddisfa le ipotesi sopra e si dice nucleo di Gauss in $%
%TCIMACRO{\U{211d} }%
%BeginExpansion
\mathbb{R}
%EndExpansion
^{n}$.

\item $\phi \left( x\right) =\frac{1}{\pi }\frac{1}{x^{2}+1}$ soddisfa le
ipotesi sopra\footnote{%
si pu\`{o} verificare che \`{e} integrabile con il calcolo esplicito, usando
l'arcotangente} e si dice nucleo di Poisson.
\end{enumerate}

A partire da $\phi $ si costruisce una successione $\left\{ \phi
_{\varepsilon }\left( \mathbf{x}\right) \right\} _{\varepsilon >0},\phi
_{\varepsilon }\left( \mathbf{x}\right) :=\frac{1}{\varepsilon ^{n}}\phi
\left( \frac{\mathbf{x}}{\varepsilon }\right) $. Vale $\int_{%
%TCIMACRO{\U{211d} }%
%BeginExpansion
\mathbb{R}
%EndExpansion
^{n}}\phi _{\varepsilon }\left( \mathbf{x}\right) d\mathbf{x}=\int_{%
%TCIMACRO{\U{211d} }%
%BeginExpansion
\mathbb{R}
%EndExpansion
^{n}}\varepsilon ^{n}\frac{1}{\varepsilon ^{n}}\phi \left( \mathbf{y}\right)
d\mathbf{y}=1$.

$\phi _{\varepsilon }$ si dice approssimazione dell'identit\`{a}, o nucleo
regolarizzante, o mollificatore; $\left\{ \phi _{\varepsilon }\right\}
_{\varepsilon >0}$ si dice famiglia di approssimazioni dell'identit\`{a}. Si
noti che l'effetto di dividere per $\varepsilon $, qualora $\varepsilon $
sia molto vicino a $0$, \`{e} dilatare verticalmente il grafico di $\phi $,
mentre l'effetto di dividere per $\varepsilon $ \`{e} comprimere
orizzontalmente il grafico di $\phi $. Visivamente, al tendere di $%
\varepsilon $ a $0$ (da verde a rosso) si osserva una campana che si
restringe in ampiezza e ha un massimo sempre pi\`{u} alto.\FRAME{dtbpFX}{%
3.5829in}{2.3886in}{0pt}{}{}{Plot}{\special{language "Scientific Word";type
"MAPLEPLOT";width 3.5829in;height 2.3886in;depth 0pt;display
"USEDEF";plot_snapshots TRUE;mustRecompute FALSE;lastEngine "MuPAD";xmin
"-2";xmax "2";xviewmin "-2";xviewmax "2";yviewmin "0";yviewmax
"2";viewset"XY";rangeset"X";plottype 4;axesFont "Times New
Roman,12,0000000000,useDefault,normal";numpoints 100;plotstyle
"patch";axesstyle "normal";axestips FALSE;xis \TEXUX{x};var1name
\TEXUX{$x$};function \TEXUX{$\frac{1}{0.8}e^{-\pi \left\vert
\frac{x}{0.8}\right\vert ^{2}}$};linecolor "green";linestyle 1;pointstyle
"point";linethickness 1;lineAttributes "Solid";var1range
"-2,2";num-x-gridlines 100;curveColor "[flat::RGB:0x00008000]";curveStyle
"Line";function \TEXUX{$\frac{1}{0.5}e^{-\pi \left\vert
\frac{x}{0.5}\right\vert ^{2}}$};linecolor "blue";linestyle 1;pointstyle
"point";linethickness 1;lineAttributes "Solid";var1range
"-2,2";num-x-gridlines 100;curveColor "[flat::RGB:0x000000ff]";curveStyle
"Line";VCamFile 'S9KA1Q09.xvz';valid_file "T";tempfilename
'S9KA1Q02.wmf';tempfile-properties "XPR";}}

Data $f\in L^{1}\left( 
%TCIMACRO{\U{211d} }%
%BeginExpansion
\mathbb{R}
%EndExpansion
^{n}\right) $, si definisce ora $f_{\varepsilon }\left( \mathbf{x}\right)
:=\left( f\ast \phi _{\varepsilon }\right) \left( \mathbf{x}\right) =\int_{%
%TCIMACRO{\U{211d} }%
%BeginExpansion
\mathbb{R}
%EndExpansion
^{n}}f\left( \mathbf{y}\right) \phi _{\varepsilon }\left( \mathbf{x-y}%
\right) d\mathbf{y}$: questa operazione deve essere vista come una media di $%
f$ pesata sulle $\phi _{\varepsilon }$ (dato che queste integrano a $1$),
dove i valori di $f$ per $\mathbf{x}$ prossimo a $0$ sono pesati tanto pi%
\`{u} quanto pi\`{u} $\varepsilon $ \`{e} piccolo.

\textbf{Teo}%
\begin{gather*}
\text{Hp: }\left\{ \phi _{\varepsilon }\right\} _{\varepsilon >0}\text{ \`{e}
una famiglia di approssimazioni dell'identit\`{a}, }\exists \text{ }p\in
\lbrack 1,+\infty ):f\in L^{p}\left( 
%TCIMACRO{\U{211d} }%
%BeginExpansion
\mathbb{R}
%EndExpansion
^{n}\right) \\
\text{Ts: (i) }f_{\varepsilon }\text{ converge a }f\text{ in }L^{p}\left( 
%TCIMACRO{\U{211d} }%
%BeginExpansion
\mathbb{R}
%EndExpansion
^{n}\right) \text{ per }\varepsilon \rightarrow 0^{+} \\
\text{(ii) se inoltre }f\in L^{1}\left( 
%TCIMACRO{\U{211d} }%
%BeginExpansion
\mathbb{R}
%EndExpansion
^{n}\right) \cap C_{\ast }^{0}\left( 
%TCIMACRO{\U{211d} }%
%BeginExpansion
\mathbb{R}
%EndExpansion
^{n}\right) \text{, }f_{\varepsilon }\text{ converge a }f\text{ anche
uniformemente per }\varepsilon \rightarrow 0^{+} \\
\text{(iii) se inoltre }\phi \in C^{\infty }\left( 
%TCIMACRO{\U{211d} }%
%BeginExpansion
\mathbb{R}
%EndExpansion
^{n}\right) \text{, allora }f_{\varepsilon }\in C^{\infty }\left( 
%TCIMACRO{\U{211d} }%
%BeginExpansion
\mathbb{R}
%EndExpansion
^{n}\right) \text{ e }f_{\varepsilon }\text{ converge a }f\text{ in }%
L^{p}\left( 
%TCIMACRO{\U{211d} }%
%BeginExpansion
\mathbb{R}
%EndExpansion
^{n}\right) \text{ per }\varepsilon \rightarrow 0^{+}
\end{gather*}

( $f_{\varepsilon }$, se $f\in L^{p}$, \`{e} in $L^{p}$ per il teorema di
Young). In (iii), la prima tesi \`{e} una conseguenza del teorema di
regolarit\`{a} della convoluzione. (iii) mostra che \`{e} possibile
approssimare una funzione che \`{e} solo $L^{p}$ con funzioni $C^{\infty }$.

Si noti che se $f$ non \`{e} continua, (ii) \textit{non pu\`{o} essere vera}
in $L^{\infty }$, perch\'{e} la convergenza uniforme preserva la continuit%
\`{a}; \`{e} dunque naturale escludere $p=\infty $ nelle ipotesi.

Se $\phi \left( x\right) =\frac{1}{\pi \left( 1+x^{2}\right) }$, $\phi
_{\varepsilon }\left( x\right) =\frac{1}{\varepsilon }\frac{1}{\pi \left(
1+\left( \frac{x}{\varepsilon }\right) ^{2}\right) }=\frac{\varepsilon }{\pi
\left( x^{2}+\varepsilon ^{2}\right) }$: dunque, ponendo $y=\varepsilon $,
si ha che $\phi _{\varepsilon }\left( x\right) $ \`{e} proprio il nucleo di
Poisson $k_{y}\left( x\right) $, perci\`{o} $\left\{ k_{y}\left( x\right)
\right\} _{y>0}$ \`{e} una famiglia di nuclei regolarizzanti.

Ne segue che, se $f\in L^{1}\left( 
%TCIMACRO{\U{211d} }%
%BeginExpansion
\mathbb{R}
%EndExpansion
\right) $, $u\left( x,y\right) =f\ast k_{y}=f_{y}\left( x\right) $ converge
in $L^{1}$ a $f$ per $y\rightarrow 0^{+}$, cio\`{e} $\left\vert \left\vert
u\left( \cdot ,y\right) -f\right\vert \right\vert _{L^{1}}\rightarrow 0$ per 
$y\rightarrow 0^{+}$: il dato al bordo \`{e} assunto in senso $L^{1}$.

Se inoltre $f\in L^{1}\cap C_{\ast }^{0}$, $u\left( x,y\right) =f\ast k_{y}$
converge a $f$ anche uniformemente per $y\rightarrow 0^{+}$, dunque $%
\lim_{y\rightarrow 0^{+}}\sup_{x\in 
%TCIMACRO{\U{211d} }%
%BeginExpansion
\mathbb{R}
%EndExpansion
}\left\vert u\left( x,y\right) -f\left( x\right) \right\vert =0$ e $u$ \`{e}
soluzione classica.

\begin{enumerate}
\item Considero il problema di Dirichlet nel semipiano $\left\{ 
\begin{array}{c}
\Delta u=0\text{ se }x\in 
%TCIMACRO{\U{211d} }%
%BeginExpansion
\mathbb{R}
%EndExpansion
,y>0 \\ 
u\left( x,0\right) =I_{\left[ -1,1\right] }\left( x\right) \text{ se }x\in 
%TCIMACRO{\U{211d}}%
%BeginExpansion
\mathbb{R}%
%EndExpansion
\end{array}%
\right. $. E' noto che $u\left( x,y\right) =\int_{%
%TCIMACRO{\U{211d} }%
%BeginExpansion
\mathbb{R}
%EndExpansion
}f\left( z\right) k_{y}\left( x-z\right) dz=\int_{-1}^{1}\frac{y}{\pi \left(
y^{2}+\left( x-z\right) ^{2}\right) }dz=\frac{1}{\pi }\left[ -\arctan \left( 
\frac{x-z}{y}\right) \right] _{z=-1}^{z=1}=\frac{1}{\pi }\left( \arctan 
\frac{x+1}{y}-\arctan \frac{x-1}{y}\right) $.

Poich\'{e} $f\in L^{1}$, il dato al bordo \`{e} assunto in senso $L^{1}$;
nei punti in cui $f$ \`{e} continua, \`{e} assunto anche in senso classico.
Infatti, fissato $x\in 
%TCIMACRO{\U{211d} }%
%BeginExpansion
\mathbb{R}
%EndExpansion
$, $\lim_{y\rightarrow 0^{+}}\frac{1}{\pi }\left( \arctan \frac{x+1}{y}%
-\arctan \frac{x-1}{y}\right) =1$ se $x\in \left( -1,1\right) $, $0$ se $%
x<-1,x>1$; se invece $x=1$, $\lim_{y\rightarrow 0^{+}}\frac{1}{\pi }\left(
\arctan \frac{x+1}{y}-\arctan \frac{x-1}{y}\right) =\frac{1}{2}$ e se $x=-1$%
, $\lim_{y\rightarrow 0^{+}}\frac{1}{\pi }\left( \arctan \frac{x+1}{y}%
-\arctan \frac{x-1}{y}\right) =\frac{1}{2}$.
\end{enumerate}

Si noti che in questo caso non si pu\`{o} usare il teorema di unicit\`{a}
per affermare che la soluzione del problema di Dirichlet nel semipiano \`{e}
unica, perch\'{e} tale teorema richiede di lavorare su dominii limitati e
lipschitizani, mentre il piano \`{e} un dominio illimitato. In effetti, in
generale la soluzione del problema di Dirichlet nel semipiano non \`{e}
unica.

\begin{enumerate}
\item $\left\{ 
\begin{array}{c}
\Delta u=0\text{ se }x\in 
%TCIMACRO{\U{211d} }%
%BeginExpansion
\mathbb{R}
%EndExpansion
,y>0 \\ 
u\left( x,0\right) =0\text{ se }x\in 
%TCIMACRO{\U{211d}}%
%BeginExpansion
\mathbb{R}%
%EndExpansion
\end{array}%
\right. $ ha soluzione $u=0$, ma anche $u\left( x,y\right) =e^{x}\sin y=%
\func{Im}e^{z}$, dato che $e^{z}$ \`{e} una funzione armonica.
\end{enumerate}

Tuttavia si pu\`{o} dimostrare che la soluzione \`{e} unica nella classe
delle funzioni limitate (infatti $e^{x}\sin y$ non lo \`{e}).

\subsection{Equazione di Poisson in $%
%TCIMACRO{\U{211d} }%
%BeginExpansion
\mathbb{R}
%EndExpansion
^{n}$}

Si vuole risolvere $\Delta u=f$ in $%
%TCIMACRO{\U{211d} }%
%BeginExpansion
\mathbb{R}
%EndExpansion
^{n}$. In questo caso la trasformata di Fourier non aiuta..

L'idea \`{e} invece trovare una soluzione fondamentale. Per motivi fisici 
\`{e} naturale studiare $\Delta u=-\delta $, dove $\delta $ rappresenta la
densit\`{a} di carica di una carica puntiforme nell'origine. E' quindi
naturale che il potenziale elettrostatico sia a simmetria radiale. Si
cercano allora le funzioni radiali armoniche per $\mathbf{x\neq 0}$.

Ovviamente serve sapere scrivere il laplaciano per funzioni radiali.

\textbf{Prop (laplaciano per funzioni radiali)}%
\begin{eqnarray*}
\text{Hp}\text{: } &&u:%
%TCIMACRO{\U{211d} }%
%BeginExpansion
\mathbb{R}
%EndExpansion
^{n}\rightarrow 
%TCIMACRO{\U{211d} }%
%BeginExpansion
\mathbb{R}
%EndExpansion
,u\left( \mathbf{x}\right) =f\left( \left\vert \left\vert \mathbf{x}%
\right\vert \right\vert \right) =f\left( \rho \right) \\
\text{Ts} &\text{:}&\text{ }\Delta u=f^{\prime \prime }\left( \rho \right) +%
\frac{n-1}{\rho }f^{\prime }\left( \rho \right)
\end{eqnarray*}

\textbf{Dim} Vale $\Delta u=\sum_{i=1}^{n}\frac{\partial ^{2}u}{\partial
x_{i}^{2}}$. $\frac{\partial u}{\partial x_{i}}=\frac{\partial f}{\partial
\rho }\frac{\partial \rho }{\partial x_{i}}=f^{\prime }\left( \rho \right)
\rho _{x_{i}}$, quindi $\frac{\partial ^{2}u}{\partial x_{i}^{2}}=f^{\prime
\prime }\left( \rho \right) \rho _{x_{i}}^{2}+f^{\prime }\left( \rho \right)
\rho _{x_{i}x_{i}}$. Allora $\Delta u=f^{\prime \prime }\left( \rho \right)
\sum_{i=1}^{n}\rho _{x_{i}}^{2}+f^{\prime }\left( \rho \right)
\sum_{i=1}^{n}\rho _{x_{i}x_{i}}$. Poich\'{e} $\rho =\rho \left( \mathbf{x}%
\right) =\sqrt{\sum_{j=1}^{n}x_{j}^{2}}$, $\rho _{x_{i}}=\frac{x_{i}}{\sqrt{%
\sum_{j=1}^{n}x_{j}^{2}}}=\frac{x_{i}}{\rho }$ e $\rho _{x_{i}x_{i}}=\frac{%
\sqrt{\sum_{j=1}^{n}x_{j}^{2}}-x_{i}\frac{x_{i}}{\sqrt{%
\sum_{j=1}^{n}x_{j}^{2}}}}{\sum_{j=1}^{n}x_{j}^{2}}=\frac{\rho ^{2}-x_{i}^{2}%
}{\rho ^{3}}$, per cui $\Delta u=f^{\prime \prime }\left( \rho \right)
+f^{\prime }\left( \rho \right) +f^{\prime }\left( \rho \right) \frac{n-1}{%
\rho }$. $\blacksquare $

E' noto che in coordinate polari $\Delta u=\left( \frac{\partial ^{2}}{%
\partial \rho ^{2}}+\frac{1}{\rho }\frac{\partial }{\partial \rho }+\frac{%
\partial ^{2}}{\partial \theta ^{2}}\right) u$, quindi se $n=2$ e $u=u\left(
\rho \right) $ \`{e} una funzione radiale, $\Delta u=\left( \frac{\partial
^{2}}{\partial \rho ^{2}}+\frac{1}{\rho }\frac{\partial }{\partial \rho }%
\right) u$, in accordo con la tesi sopra.

Allora per trovare le funzioni radiali armoniche per $\mathbf{x\neq 0}$
basta risolvere l'EDO $f^{\prime \prime }\left( \rho \right) +\frac{n-1}{%
\rho }f^{\prime }\left( \rho \right) =0$ per $\rho >0$: ponendo $v\left(
\rho \right) =f^{\prime }\left( \rho \right) $ si ottiene l'equazione $%
v^{\prime }\left( \rho \right) +\frac{n-1}{\rho }v\left( \rho \right) =0$,
da cui - separando le variabili - $v\left( \rho \right) =\frac{c}{\rho ^{n-1}%
}$. Allora si ottiene%
\begin{equation*}
f\left( \rho \right) =\left\{ 
\begin{array}{c}
c_{1}\ln \rho +c_{2}\text{ se }n=2 \\ 
c_{1}\frac{1}{\rho ^{n-2}}+c_{2}\text{ se }n>2%
\end{array}%
\right.
\end{equation*}

Queste sono tutte e sole le funzioni armoniche radiali in $%
%TCIMACRO{\U{211d} }%
%BeginExpansion
\mathbb{R}
%EndExpansion
^{n}\backslash \left\{ \mathbf{0}\right\} $. Dato che comunque il potenziale 
\`{e} definito a meno di una costante additiva, si pu\`{o} porre $c_{2}=0$. 
\begin{equation*}
u\left( \mathbf{x}\right) =\left\{ 
\begin{array}{c}
c\ln \left\vert \left\vert \mathbf{x}\right\vert \right\vert \text{ se }n=2
\\ 
c\frac{1}{\left\vert \left\vert \mathbf{x}\right\vert \right\vert ^{n-2}}%
\text{ se }n>2%
\end{array}%
\right.
\end{equation*}

E' evidente tuttavia che questa $u$ non potr\`{a} essere una soluzione in
senso classico: la soluzione di $\Delta u=-\delta $, essendo $\delta $ una
distribuzione, non potr\`{a} che essere una distribuzione. Occorre quindi
valutare se esiste $c$ tale che la $u$ definita sopra risolva l'equazione in
senso distribuzionale, cio\`{e} se il laplaciano della distribuzione%
\footnote{%
Si ricorda che $D_{x_{i}}\left( T\right) :=-T\left( \frac{\partial \phi }{%
\partial x_{i}}\right) $, quindi $D_{x_{i}x_{i}}\left( T\right)
=D_{x_{i}}\left( D_{x_{i}}T\right) =-D_{x_{i}}T\left( \frac{\partial \phi }{%
\partial x_{i}}\right) =T\left( \frac{\partial ^{2}\phi }{\partial x_{i}^{2}}%
\right) $. Allora $\Delta T:=\sum_{i=1}^{n}D_{x_{i}x_{i}}\left( T\right)
=T\left( \sum_{i=1}^{n}\frac{\partial ^{2}\phi }{\partial x_{i}^{2}}\right)
=T\left( \Delta \phi \right) $.} associata a $u$ \`{e} proprio $-\delta $
(si noti che tale distribuzione \`{e} ben definita perch\'{e} $u\left( 
\mathbf{x}\right) =c\frac{1}{\left\vert \left\vert \mathbf{x}\right\vert
\right\vert ^{n-2}}$ \`{e} $L_{loc}^{1}$, come si mostrer\`{a} pi\`{u}
avanti): questo significa chiedere che $\Delta T_{u}=-\delta $, cio\`{e} $%
T_{u}\left( \Delta \phi \right) =-\phi \left( 0\right) $ $\forall $ $\phi
\in \mathcal{D}\left( \Omega \right) $. Si cerca perci\`{o} $c:\int_{%
%TCIMACRO{\U{211d} }%
%BeginExpansion
\mathbb{R}
%EndExpansion
^{n}}u\left( \mathbf{x}\right) \Delta \phi \left( \mathbf{x}\right) d\mathbf{%
x}=-\phi \left( 0\right) $ $\forall $ $\phi \in \mathcal{D}\left( \Omega
\right) $.

\textbf{Integrali di funzioni radiali} Essendo $u$ una funzione radiale, 
\`{e} utile ricordare come si calcolano gli integrali di funzioni siffatte.
Ogni punto in $%
%TCIMACRO{\U{211d} }%
%BeginExpansion
\mathbb{R}
%EndExpansion
^{n}$ \`{e} individuato univocamente dalle sue coordinate cartesiane $\left(
x_{1},...,x_{n}\right) $ oppure dalle sue coordinate sferiche $\left( \rho
,\phi _{1},...,\phi _{n-1}\right) $, dove le ultime $n-1$ coordinate sono
angoli. La trasformazione da un sistema di coordinate all'altro \`{e} $%
x_{i}=\rho \omega _{i}\left( \phi _{1},...,\phi _{n-1}\right) $, con $\omega
_{i}:%
%TCIMACRO{\U{211d} }%
%BeginExpansion
\mathbb{R}
%EndExpansion
^{n-1}\rightarrow 
%TCIMACRO{\U{211d} }%
%BeginExpansion
\mathbb{R}
%EndExpansion
$ e $\rho =\sqrt{\sum_{i=1}^{n}x_{i}^{2}}$; l'elemento di volume, che d\`{a}
conto della contrazione o dilatazione dei volumi che si ha passando da un
sistema all'altro, \`{e} $d\mathbf{x}=dx_{1}...dx_{n}=\rho ^{n-1}\omega
\left( \phi _{1},...,\phi _{n}\right) d\rho d\phi _{1}...d\phi _{n}$, dove $%
\omega :%
%TCIMACRO{\U{211d} }%
%BeginExpansion
\mathbb{R}
%EndExpansion
^{n-1}\rightarrow 
%TCIMACRO{\U{211d} }%
%BeginExpansion
\mathbb{R}
%EndExpansion
$ \`{e} il prodotto di funzioni seno o coseno dei vari angoli.

\begin{enumerate}
\item In $%
%TCIMACRO{\U{211d} }%
%BeginExpansion
\mathbb{R}
%EndExpansion
^{3}$ $d\mathbf{x}=\rho ^{2}\sin \phi d\theta d\phi $.
\end{enumerate}

Perci\`{o}, calcolando gli integrali con il cambio di variabile da
coordinate cartesiane a coordinate sferiche, si ha $\int_{B_{R}\left( 
\mathbf{0}\right) }f\left( \mathbf{x}\right) d\mathbf{x}=\int_{0}^{R}\int_{%
\Sigma _{1}}f\left( \rho ,\phi _{1},...,\phi _{n-1}\right) \rho ^{n-1}\omega
\left( \phi _{1},...,\phi _{n}\right) d\rho d\phi _{1}...d\phi _{n-1}$, dove 
$\Sigma _{1}$ \`{e} l'insieme in cui variano le coordinate angolari (in $%
%TCIMACRO{\U{211d} }%
%BeginExpansion
\mathbb{R}
%EndExpansion
^{2}$ $\Sigma _{1}=\left[ 0,2\pi \right] $, in $%
%TCIMACRO{\U{211d} }%
%BeginExpansion
\mathbb{R}
%EndExpansion
^{3}$ $\Sigma _{1}=\left[ 0,2\pi \right] \times \left[ 0,\pi \right] $).
L'ultimo integrale si pu\`{o} riscrivere, decomponendo in componente radiale
e componente superficiale, come $\int_{0}^{R}\left( \int_{\Sigma
_{1}}fd\sigma _{\rho }\right) d\rho $, dove $d\sigma _{\rho }$ indica
l'elemento d'area della superficie sferica di raggio $\rho $: vale quindi $%
d\sigma _{\rho }=\rho ^{n-1}\omega \left( \phi _{1},...,\phi _{n}\right)
d\phi _{1}...d\phi _{n-1}=\rho ^{n-1}d\sigma _{1}$, dove $d\sigma _{1}$
indica l'elemento d'area della sfera di raggio $1$ in $%
%TCIMACRO{\U{211d} }%
%BeginExpansion
\mathbb{R}
%EndExpansion
^{n}$.

Se $f$ \`{e} radiale, $\int_{B_{R}\left( \mathbf{0}\right) }f\left( \mathbf{x%
}\right) d\mathbf{x=}\int_{0}^{R}\left( \int_{\Sigma _{1}}f\rho
^{n-1}d\sigma _{1}\right) d\rho =\int_{0}^{R}f\left( \rho \right) \rho
^{n-1}d\rho \int_{\Sigma _{1}}d\sigma _{1}$. Il numero $\int_{\Sigma
_{1}}d\sigma _{1}$, misura della superficie della sfera di raggio $1$ in $%
%TCIMACRO{\U{211d} }%
%BeginExpansion
\mathbb{R}
%EndExpansion
^{n}$, prende il nome di $\omega _{n}$ ($\omega _{2}=2\pi ,\omega _{3}=4\pi $%
). Quindi%
\begin{equation*}
\int_{B_{R}\left( \mathbf{0}\right) }f\left( \mathbf{x}\right) d\mathbf{x}%
=\omega _{n}\int_{0}^{R}f\left( \rho \right) \rho ^{n-1}d\rho
\end{equation*}

\begin{enumerate}
\item In particolare se $f=1$ si ottiene che $\left\vert B_{R}\left( \mathbf{%
0}\right) \right\vert =\omega _{n}\int_{0}^{R}\rho ^{n-1}d\rho =\frac{\omega
_{n}}{n}R^{n}$, mentre $\left\vert \partial B_{r}\left( \mathbf{0}\right)
\right\vert =\omega _{n}R^{n-1}$ (che \`{e} la derivata in $R$ della formula
del volume!).

\item Affinch\'{e} sia ben definita la distribuzione associata a $u$, ha
senso chiedersi per quali $\alpha $ esiste finito $\int_{B_{R}\left( \mathbf{%
0}\right) }\frac{1}{\left\vert \left\vert \mathbf{x}\right\vert \right\vert
^{\alpha }}d\mathbf{x}$. Poich\'{e} $\int_{B_{R}\left( \mathbf{0}\right) }%
\frac{1}{\left\vert \left\vert \mathbf{x}\right\vert \right\vert ^{\alpha }}d%
\mathbf{x}=\omega _{n}\int_{0}^{R}f\left( \rho \right) \rho ^{n-1}d\rho
=\omega _{n}\int_{0}^{R}\frac{1}{\rho ^{\alpha }}\rho ^{n-1}d\rho $, che 
\`{e} finito se e solo se $\alpha <n$. Quindi $u\left( \mathbf{x}\right) =c%
\frac{1}{\left\vert \left\vert \mathbf{x}\right\vert \right\vert ^{n-2}}$ 
\`{e} $L_{loc}^{1}\left( 
%TCIMACRO{\U{211d} }%
%BeginExpansion
\mathbb{R}
%EndExpansion
^{n}\right) $, cos\`{\i} come $\frac{1}{\left\vert \left\vert \mathbf{x}%
\right\vert \right\vert ^{n-1}}$, mentre non lo \`{e} $\frac{1}{\left\vert
\left\vert \mathbf{x}\right\vert \right\vert ^{n}}$.
\end{enumerate}

Si pu\`{o} ora mostrare che la $u$ sopra definita \`{e} soluzione
distribuzionale dell'equazione di Poisson.

\textbf{Teo (soluzione fondamentale dell'equazione di Poisson)}%
\begin{eqnarray*}
\text{Hp}\text{: } &&u:%
%TCIMACRO{\U{211d} }%
%BeginExpansion
\mathbb{R}
%EndExpansion
^{n}\rightarrow 
%TCIMACRO{\U{211d} }%
%BeginExpansion
\mathbb{R}
%EndExpansion
,\int_{%
%TCIMACRO{\U{211d} }%
%BeginExpansion
\mathbb{R}
%EndExpansion
^{n}}u\left( \mathbf{x}\right) \Delta \phi \left( \mathbf{x}\right) d\mathbf{%
x}=-\phi \left( 0\right) \text{ }\forall \text{ }\phi \in \mathcal{D}\left( 
%TCIMACRO{\U{211d} }%
%BeginExpansion
\mathbb{R}
%EndExpansion
^{n}\right) \\
\text{Ts}\text{: } &&\Gamma \left( \mathbf{x}\right) =\left\{ 
\begin{array}{c}
\frac{-1}{2\pi }\ln \left\vert \left\vert \mathbf{x}\right\vert \right\vert 
\text{ se }n=2 \\ 
\frac{1}{\omega _{n}\left( n-2\right) }\frac{1}{\left\vert \left\vert 
\mathbf{x}\right\vert \right\vert ^{n-2}}\text{ se }n>2%
\end{array}%
\right. \text{ risolve l'equazione, cio\`{e} }\Delta \Gamma =-\delta _{0}
\end{eqnarray*}

Si noti che $\Gamma $ \`{e} la $u$ ricavata sopra, in cui per\`{o} sono
state determinate le costanti. La tesi pu\`{o} essere espressa pi\`{u}
snellamente dicendo che $\Delta T_{\Gamma }=-\delta $, cio\`{e} $\Gamma $
risolve l'equazione in senso distribuzionale. $\Gamma $ si dice quindi
soluzione fondamentale dell'operatore differenziale $-\Delta $. Si noti che $%
\Gamma $ \`{e} localmente integrabile, quindi l'integrale nell'ipotesi \`{e}
ben definito quando $u=\Gamma $; ogni derivata prima di $\Gamma $ va a $%
+\infty $ come $\frac{1}{\left\vert \left\vert \mathbf{x}\right\vert
\right\vert ^{n-1}}$, che \`{e} localmente integrabile, mentre ogni derivata
seconda come $\frac{1}{\left\vert \left\vert \mathbf{x}\right\vert
\right\vert ^{n}}$, che \textit{non \`{e} localmente integrabile} (anche se
il laplaciano di $\Gamma $, la somma di tutte le derivate seconde, d\`{a} $0$
tranne nell'origine): non \`{e} banale il modo di vederla come distribuzione.

\textbf{Dim} Si dimostra solo il caso $n>2$, e si dimostrer\`{a} una tesi pi%
\`{u} forte, cio\`{e} che $\int_{%
%TCIMACRO{\U{211d} }%
%BeginExpansion
\mathbb{R}
%EndExpansion
^{n}}\Gamma \left( \mathbf{x}\right) \Delta \phi \left( \mathbf{x}\right) d%
\mathbf{x}=-\phi \left( 0\right) $ $\forall $ $\phi \in C_{0}^{2}\left( 
%TCIMACRO{\U{211d} }%
%BeginExpansion
\mathbb{R}
%EndExpansion
^{n}\right) $. A priori non si conosce $\omega _{n}$: si calcola il lato
sinistro dell'uguaglianza per $\gamma \left( \mathbf{x}\right) =\frac{1}{%
\left\vert \left\vert \mathbf{x}\right\vert \right\vert ^{n-2}}$, e si trover%
\`{a} poi l'opportuna costante di normalizzazione.

Si vuole quindi calcolare $\int_{%
%TCIMACRO{\U{211d} }%
%BeginExpansion
\mathbb{R}
%EndExpansion
^{n}}\gamma \left( \mathbf{x}\right) \Delta \phi \left( \mathbf{x}\right) d%
\mathbf{x}$ con $\phi \in C_{0}^{2}\left( 
%TCIMACRO{\U{211d} }%
%BeginExpansion
\mathbb{R}
%EndExpansion
^{n}\right) $; dato che le $\phi $ e le derivate fino alla seconda sono a
supporto compatto, $\forall $ $\phi \in C_{0}^{2}\left( 
%TCIMACRO{\U{211d} }%
%BeginExpansion
\mathbb{R}
%EndExpansion
^{n}\right) $ $\exists $ $B_{r}\left( \mathbf{0}\right) :supp\left( \phi
\right) \subset B_{r}\left( \mathbf{0}\right) $, quindi $\int_{%
%TCIMACRO{\U{211d} }%
%BeginExpansion
\mathbb{R}
%EndExpansion
^{n}}\gamma \left( \mathbf{x}\right) \Delta \phi \left( \mathbf{x}\right) d%
\mathbf{x=}\int_{B_{r}\left( \mathbf{0}\right) }\gamma \left( \mathbf{x}%
\right) \Delta \phi \left( \mathbf{x}\right) d\mathbf{x}$. Ora \`{e}
naturale voler usare la seconda identit\`{a} di Green, ma $\gamma $ non \`{e}
$C^{2}$ in $\Omega =B_{r}\left( \mathbf{0}\right) $: in $0$ non \`{e}
definita! Allora si considera $\varepsilon <r$ e si scrive $%
\int_{B_{r}\left( \mathbf{0}\right) }\gamma \left( \mathbf{x}\right) \Delta
\phi \left( \mathbf{x}\right) d\mathbf{x=}\int_{B_{\varepsilon }\left( 
\mathbf{0}\right) }\gamma \left( \mathbf{x}\right) \Delta \phi \left( 
\mathbf{x}\right) d\mathbf{x+}\int_{B_{r}\left( \mathbf{0}\right) \backslash
B_{\varepsilon }\left( \mathbf{0}\right) }\gamma \left( \mathbf{x}\right)
\Delta \phi \left( \mathbf{x}\right) d\mathbf{x}=I_{\varepsilon
}+II_{\varepsilon }$.

Dico che $I_{\varepsilon }\rightarrow 0$ se $\varepsilon \rightarrow 0^{+}$:
intuitivamente, perch\'{e} $\gamma \left( \mathbf{x}\right) \Delta \phi
\left( \mathbf{x}\right) $ \`{e} una funzione integrabile su $B_{r}\left( 
\mathbf{0}\right) $. Infatti $\left\vert \int_{B_{\varepsilon }\left( 
\mathbf{0}\right) }\gamma \left( \mathbf{x}\right) \Delta \phi \left( 
\mathbf{x}\right) d\mathbf{x}\right\vert \leq \int_{B_{\varepsilon }\left( 
\mathbf{0}\right) }\frac{1}{\left\vert \left\vert \mathbf{x}\right\vert
\right\vert ^{n-2}}\left\vert \Delta \phi \left( \mathbf{x}\right)
\right\vert d\mathbf{x\leq }\int_{B_{\varepsilon }\left( \mathbf{0}\right) }%
\frac{1}{\left\vert \left\vert \mathbf{x}\right\vert \right\vert ^{n-2}}%
\left\vert \left\vert \Delta \phi \right\vert \right\vert _{C^{0}\left( 
%TCIMACRO{\U{211d} }%
%BeginExpansion
\mathbb{R}
%EndExpansion
^{n}\right) }d\mathbf{x}$, dato che $\Delta \phi $ \`{e} $C^{0}$ e a
supporto compatto. Allora $\left\vert I_{\varepsilon }\right\vert \leq
\left\vert \left\vert \Delta \phi \right\vert \right\vert _{C^{0}\left( 
%TCIMACRO{\U{211d} }%
%BeginExpansion
\mathbb{R}
%EndExpansion
^{n}\right) }\int_{B_{\varepsilon }\left( \mathbf{0}\right) }\frac{1}{%
\left\vert \left\vert \mathbf{x}\right\vert \right\vert ^{n-2}}d\mathbf{x}%
=\left\vert \left\vert \Delta \phi \right\vert \right\vert _{C^{0}\left( 
%TCIMACRO{\U{211d} }%
%BeginExpansion
\mathbb{R}
%EndExpansion
^{n}\right) }\omega _{n}\int_{0}^{\varepsilon }\frac{\rho ^{n-1}}{\rho ^{n-2}%
}d\rho =\left\vert \left\vert \Delta \phi \right\vert \right\vert
_{C^{0}\left( 
%TCIMACRO{\U{211d} }%
%BeginExpansion
\mathbb{R}
%EndExpansion
^{n}\right) }\omega _{n}\frac{1}{2}\varepsilon ^{2}\rightarrow ^{\varepsilon
\rightarrow 0^{+}}0$.

Dunque $\int_{B_{r}\left( \mathbf{0}\right) }\gamma \left( \mathbf{x}\right)
\Delta \phi \left( \mathbf{x}\right) d\mathbf{x}=\lim_{\varepsilon
\rightarrow 0^{+}}\int_{B_{r}\left( \mathbf{0}\right) }\gamma \left( \mathbf{%
x}\right) \Delta \phi \left( \mathbf{x}\right) d\mathbf{x}=\lim_{\varepsilon
\rightarrow 0^{+}}\int_{B_{r}\left( \mathbf{0}\right) \backslash
B_{\varepsilon }\left( \mathbf{0}\right) }\gamma \left( \mathbf{x}\right)
\Delta \phi \left( \mathbf{x}\right) d\mathbf{x}$. Per calcolare
quest'ultimo si usa la seconda identit\`{a} di Green, dato che in questo
dominio $\gamma $ e $\phi $ sono $C^{2}$ fino al bordo: ponendo $\Omega
_{\varepsilon }=B_{r}\left( \mathbf{0}\right) \backslash B_{\varepsilon
}\left( \mathbf{0}\right) $, $\int_{\Omega _{\varepsilon }}\gamma \left( 
\mathbf{x}\right) \Delta \phi \left( \mathbf{x}\right) d\mathbf{x}%
=\int_{\Omega _{\varepsilon }}\Delta \gamma \left( \mathbf{x}\right) \phi
\left( \mathbf{x}\right) d\mathbf{x+}\int_{\partial \Omega _{\varepsilon
}}\left( \gamma \frac{\partial \phi }{\partial n}-\phi \frac{\partial \gamma 
}{\partial n}\right) \left( \sigma \right) d\sigma =\int_{\partial \Omega
_{\varepsilon }}\left( \gamma \frac{\partial \phi }{\partial n}-\phi \frac{%
\partial \gamma }{\partial n}\right) \left( \sigma \right) d\sigma $, perch%
\'{e} $\Delta \gamma =0$: $\gamma $ \`{e} stata trovata proprio come
funzione armonica radiale in $%
%TCIMACRO{\U{211d} }%
%BeginExpansion
\mathbb{R}
%EndExpansion
^{n}\backslash \left\{ \mathbf{0}\right\} $. $\partial \Omega _{\varepsilon
} $ \`{e} il bordo di una corona circolare: sul bordo esterno $\phi $ (e
quindi anche ogni sua derivata) \`{e} nulla "gi\`{a} da un po'" perch\'{e}
si \`{e} fuori dal suo supporto, quindi rimane solo l'integrale sul bordo
interno $\int_{\partial B_{\varepsilon }\left( \mathbf{0}\right) }\left(
\gamma \frac{\partial \phi }{\partial n_{i}}-\phi \frac{\partial \gamma }{%
\partial n_{i}}\right) \left( \sigma \right) d\sigma =\int_{\partial
B_{\varepsilon }\left( \mathbf{0}\right) }\left( \gamma \frac{\partial \phi 
}{\partial n_{i}}\right) \left( \sigma \right) d\sigma -\int_{\partial
B_{\varepsilon }\left( \mathbf{0}\right) }\left( \phi \frac{\partial \gamma 
}{\partial n_{i}}\right) \left( \sigma \right) d\sigma =A_{\varepsilon
}+B_{\varepsilon }$.

Si intuisce che il primo addendo (proprio come accaduto a $I_{\varepsilon }$%
) tende a $0$ per $\varepsilon \rightarrow 0^{+}$, perch\'{e} - ragionando
intuitivamente con gli ordini di grandezza - $\partial B_{\varepsilon
}\left( \mathbf{0}\right) $ pesa come $\varepsilon ^{n-1}$, mentre $\gamma $
tende a $\infty $ come $\frac{1}{\left\vert \left\vert \mathbf{x}\right\vert
\right\vert ^{n-2}}=\frac{1}{\varepsilon ^{n-2}}$, quindi rimane un $%
\varepsilon $ al numeratore; invece nel secondo ci dovrebbe essere una
compensazione, perch\'{e} la derivata di $\gamma $ tende a $\infty $ come $%
\frac{1}{\left\vert \left\vert \mathbf{x}\right\vert \right\vert ^{n-1}}=%
\frac{1}{\varepsilon ^{n-1}}$, che si semplifica con $\varepsilon ^{n-1}$.
Infatti vale $\left\vert A_{\varepsilon }\right\vert =\left\vert
\int_{\partial B_{\varepsilon }\left( \mathbf{0}\right) }\left( \frac{1}{%
\left\vert \left\vert \sigma \right\vert \right\vert ^{n-2}}\frac{\partial
\phi }{\partial n_{i}}\right) \left( \sigma \right) d\sigma \right\vert =%
\frac{1}{\varepsilon ^{n-2}}\left\vert \int_{\partial B_{\varepsilon }\left( 
\mathbf{0}\right) }\frac{\partial \phi }{\partial n_{i}}\left( \sigma
\right) d\sigma \right\vert \leq \frac{1}{\varepsilon ^{n-2}}\int_{\partial
B_{\varepsilon }\left( \mathbf{0}\right) }\left\vert \left\vert \nabla \phi
\left( \sigma \right) \right\vert \right\vert d\sigma \leq \frac{1}{%
\varepsilon ^{n-2}}\left\vert \left\vert \nabla \phi \right\vert \right\vert
_{C^{0}}\omega _{n}\varepsilon ^{n-1}=\varepsilon \omega _{n}\left\vert
\left\vert \nabla \phi \right\vert \right\vert _{C^{0}}$, che \`{e}
infinitesimo per $\varepsilon \rightarrow 0^{+}$. I passaggi seguono dal
fatto che $\left\vert \frac{\partial \phi }{\partial n_{i}}\right\vert
=\left\vert \left\langle \nabla \phi ,n_{i}\right\rangle \right\vert \leq
\left\vert \left\vert \nabla \phi \right\vert \right\vert $ per
Cauchy-Schwarz e dalla formula della misura della superficie di una sfera di
raggio $R=\varepsilon $.

Invece $B_{\varepsilon }=-\int_{\partial B_{\varepsilon }\left( \mathbf{0}%
\right) }\phi \left( \sigma \right) \left\langle \nabla \gamma \left( \sigma
\right) ,n_{i}\right\rangle d\sigma $: poich\'{e} $\gamma \left( \mathbf{x}%
\right) =\frac{1}{\left\vert \left\vert \mathbf{x}\right\vert \right\vert
^{n-2}}$, $\frac{\partial \gamma }{\partial x_{i}}=2x_{i}\frac{2-n}{2}\frac{1%
}{\left\vert \left\vert \mathbf{x}\right\vert \right\vert ^{n}}$, perci\`{o} 
$\nabla \gamma =\frac{2-n}{\left\vert \left\vert \mathbf{x}\right\vert
\right\vert ^{n}}\mathbf{x}$, mentre il versore normale esterno a un cerchio 
\`{e} $n_{e}=\frac{\mathbf{x}}{\left\vert \left\vert \mathbf{x}\right\vert
\right\vert }$. Allora $\left\langle \nabla \gamma \left( \sigma \right)
,n_{e}\right\rangle =\left\langle \frac{2-n}{\left\vert \left\vert \mathbf{x}%
\right\vert \right\vert ^{n}}\mathbf{x},\frac{\mathbf{x}}{\left\vert
\left\vert \mathbf{x}\right\vert \right\vert }\right\rangle =\frac{2-n}{%
\left\vert \left\vert \mathbf{x}\right\vert \right\vert ^{n-1}}$, che su $%
\partial B_{\varepsilon }\left( \mathbf{0}\right) $ vale $\frac{2-n}{%
\varepsilon ^{n-1}}$, e $B_{\varepsilon }=\frac{2-n}{\varepsilon ^{n-1}}%
\int_{\partial B_{\varepsilon }\left( \mathbf{0}\right) }\phi \left( \sigma
\right) d\sigma =\omega _{n}\left( 2-n\right) \frac{\int_{\partial
B_{\varepsilon }\left( \mathbf{0}\right) }\phi \left( \sigma \right) d\sigma 
}{\omega _{n}\varepsilon ^{n-1}}=\omega _{n}\left( 2-n\right) \phi \left( 
\mathbf{x}^{\ast }\right) $, dove l'ultimo passaggio segue dal teorema della
media: essendo $\mathbf{x}^{\ast }\in \partial B_{\varepsilon }\left( 
\mathbf{0}\right) $, se $\varepsilon \rightarrow 0^{+}$ il lato destro tende
a $\omega _{n}\left( 2-n\right) \phi \left( \mathbf{0}\right) $.

Si \`{e} ottenuto dunque $\int_{B_{r}\left( \mathbf{0}\right) }\gamma \left( 
\mathbf{x}\right) \Delta \phi \left( \mathbf{x}\right) d\mathbf{x}%
=\lim_{\varepsilon \rightarrow 0^{+}}B_{\varepsilon }=\omega _{n}\left(
2-n\right) \phi \left( 0\right) $. Allora, se si prende $\Gamma \left( 
\mathbf{x}\right) :=\frac{\gamma \left( \mathbf{x}\right) }{\omega
_{n}\left( n-2\right) }$, l'integrale calcolato \`{e} proprio $-\phi \left(
0\right) $ e si ha la tesi, per cui $\Delta \Gamma =-\delta $. $\blacksquare 
$

Tutta questa fatica per trovare la soluzione fondamentale non \`{e} fatta
senza motivo. Se si conosce la soluzione fondamentale, infatti, \`{e} molto
facile risolvere l'equazione $\Delta u=f$.

\textbf{Corollario (soluzione dell'equazione di Poisson)}%
\begin{eqnarray*}
\text{Hp}\text{: } &&f\in C_{0}^{2}\left( 
%TCIMACRO{\U{211d} }%
%BeginExpansion
\mathbb{R}
%EndExpansion
^{n}\right) \\
\text{Ts}\text{: } &&u=-\Gamma \ast f\text{ risolve }\Delta u=f\text{ in }%
%TCIMACRO{\U{211d} }%
%BeginExpansion
\mathbb{R}
%EndExpansion
^{n}
\end{eqnarray*}

Dunque $u\left( \mathbf{x}\right) =-\int_{%
%TCIMACRO{\U{211d} }%
%BeginExpansion
\mathbb{R}
%EndExpansion
^{n}}\Gamma \left( \mathbf{x-y}\right) f\left( \mathbf{y}\right) d\mathbf{y}$%
. Per quanto l'ipotesi su $f$ sembri strana, nella dimostrazione si vede che
essa gioca di fatto il ruolo di una funzione test. La $u$ \`{e} ben definita
perch\'{e} $\Gamma $ \`{e} localmente integrabile e $f$ ha supporto
compatto, quindi su tale supporto si ha il prodotto di una funzione
integrabile e una limitata.

\textbf{Dim} Per ora scriviamo, pur senza sapere se il passaggio \`{e}
lecito, $\Delta u=-\int_{%
%TCIMACRO{\U{211d} }%
%BeginExpansion
\mathbb{R}
%EndExpansion
^{n}}\left( \Delta _{\mathbf{x}}f\left( \mathbf{x-y}\right) \right) \Gamma
\left( \mathbf{y}\right) d\mathbf{y}$.

Poich\'{e} $f\in C_{0}^{2}\left( 
%TCIMACRO{\U{211d} }%
%BeginExpansion
\mathbb{R}
%EndExpansion
^{n}\right) $, esiste $B_{r}\left( \mathbf{0}\right) :supp\left( f\right)
\subseteq B_{r}\left( \mathbf{0}\right) $, cio\`{e} se $\mathbf{x}\in
supp\left( f\right) $ allora $\left\vert \left\vert \mathbf{x}\right\vert
\right\vert <r$. Si pu\`{o} quindi affermare che $\forall $ $\mathbf{x}\in
B_{r}\left( \mathbf{0}\right) $ $supp\left( f\left( \mathbf{x}-\cdot \right)
\right) \subseteq B_{2r}\left( \mathbf{0}\right) $, cio\`{e} se $\mathbf{y}%
\in supp\left( f\left( \mathbf{x}-\cdot \right) \right) $, allora $%
\left\vert \left\vert \mathbf{y}\right\vert \right\vert <2r$. Infatti $%
\mathbf{y}\in supp\left( f\left( \mathbf{x}-\cdot \right) \right)
\Longleftrightarrow \mathbf{x-y}\in supp\left( f\right) $, il che implica $%
\left\vert \left\vert \mathbf{x-y}\right\vert \right\vert <r$; ma poich\'{e} 
$\left\vert \left\vert \mathbf{x-y}\right\vert \right\vert \geq \left\vert
\left\vert \left\vert \mathbf{x}\right\vert \right\vert -\left\vert
\left\vert \mathbf{y}\right\vert \right\vert \right\vert \geq \left\vert
\left\vert \mathbf{x}\right\vert \right\vert -\left\vert \left\vert \mathbf{y%
}\right\vert \right\vert $, si ha $\left\vert \left\vert \mathbf{y}%
\right\vert \right\vert <r+\left\vert \left\vert \mathbf{x}\right\vert
\right\vert <2r$.

Quindi $\left\vert \Delta _{\mathbf{x}}f\left( \mathbf{x-y}\right)
\right\vert \leq c$ $\forall $ $\mathbf{y}\in B_{2r}\left( \mathbf{0}\right) 
$ per Weierstrass: $\Delta _{\mathbf{x}}f\left( \mathbf{x-y}\right) $ \`{e}
limitato nella variabile $\mathbf{y}$. Dunque, poich\'{e} $\Gamma \in
L^{1}\left( B_{2r}\left( \mathbf{0}\right) \right) $, in quanto $L_{loc}^{1}$%
, sono verificate le ipotesi del teorema di derivabilit\`{a} ed \`{e} lecito
lo scambio tra laplaciano e integrale. Definendo $F_{\mathbf{x}}\left( 
\mathbf{y}\right) :=f\left( \mathbf{x-y}\right) $, si osserva che $\Delta _{%
\mathbf{x}}f\left( \mathbf{x-y}\right) =\Delta _{\mathbf{y}}f\left( \mathbf{%
x-y}\right) $ (si hanno due segni meno che si annullano): quindi $\Delta
u=-\int_{%
%TCIMACRO{\U{211d} }%
%BeginExpansion
\mathbb{R}
%EndExpansion
^{n}}\left( \Delta _{\mathbf{x}}f\left( \mathbf{x-y}\right) \right) \Gamma
\left( \mathbf{y}\right) d\mathbf{y}=-\int_{%
%TCIMACRO{\U{211d} }%
%BeginExpansion
\mathbb{R}
%EndExpansion
^{n}}\left( \Delta _{\mathbf{y}}F_{\mathbf{x}}\left( \mathbf{y}\right)
\right) \Gamma \left( \mathbf{y}\right) d\mathbf{y}=F_{\mathbf{x}}\left( 
\mathbf{0}\right) $ per il teorema precedente, dove $\phi \left( \mathbf{y}%
\right) =F_{\mathbf{x}}\left( \mathbf{y}\right) $, e quindi $\Delta
u=f\left( \mathbf{x}\right) $. $\blacksquare $

Con il formalismo delle distribuzioni, che per\`{o} noi non conosciamo bene
e dunque non usiamo, la dimostrazione \`{e} molto pi\`{u} veloce, perch\'{e} 
$\Delta u=\Delta \left( -\Gamma \ast f\right) =-\Delta \Gamma \ast f=\delta
\ast f=f$.

Per ora abbiamo trovato una soluzione, ma \`{e} evidente che ne esistono
infinite: sommando a $u$ qualsiasi funzione armonica si trova un'altra
soluzione.

\subsection{Equazione di Poisson in un dominio $\Omega $}

E' il momento di imporre una condizione al bordo: ci occupiamo del problema
di Dirichlet%
\begin{equation*}
\left\{ 
\begin{array}{c}
\Delta u=f\text{ in }\Omega \\ 
u=g\text{ in }\partial \Omega%
\end{array}%
\right.
\end{equation*}

con $\Omega \subseteq 
%TCIMACRO{\U{211d} }%
%BeginExpansion
\mathbb{R}
%EndExpansion
^{n}$ dominio. Per studiare il problema occorre un concetto analogo a quello
di soluzione fondamentale, ma riferito a un dominio $\Omega $.

\textbf{Def} Dato $\Omega \subseteq 
%TCIMACRO{\U{211d} }%
%BeginExpansion
\mathbb{R}
%EndExpansion
^{n}$ dominio, si dice - qualora esista - funzione di Green per $\Omega $
una funzione $G\left( \mathbf{x,y}\right) $ tale che $\forall $ $\mathbf{x}%
\in \Omega $ vale $\Delta _{\mathbf{y}}G\left( \mathbf{x,y}\right) =-\delta
_{\mathbf{x}}$ $\forall $ $\mathbf{y}\in \Omega $ e $G\left( \mathbf{x,y}%
\right) =0$ $\forall $ $\mathbf{y}\in \partial \Omega $.

$G$ rappresenta il potenziale elettrostatico generato da una carica
puntiforme collocata in $\mathbf{x}$ (detto polo), e messo a terra in $%
\partial \Omega $: la variabile spaziale che indica il punto in cui si
calcola il potenziale \`{e} $\mathbf{y}$.

E' naturale aspettarsi che $G\left( \mathbf{x,y}\right) \rightarrow +\infty $
se $\mathbf{y\rightarrow x}$. Si dimostra che (1) se $\exists $ $G$, allora $%
G\left( \mathbf{x,y}\right) >0$ $\forall $ $\mathbf{x,y}\in \Omega ,\mathbf{%
y\neq x}$ (positivit\`{a}) e (2) se $\exists $ $G$, allora $G\left( \mathbf{%
x,y}\right) =G\left( \mathbf{y,x}\right) $ $\forall $ $\mathbf{x,y}\in
\Omega $.

Perch\'{e} $G$ dovrebbe servire a risolvere il problema di Dirichlet? La
giustificazione non sar\`{a} del tutto rigorosa. Supponiamo che $\Omega $
sia un dominio limitato e lipschitziano, che $\exists $ $G\left( \mathbf{x,y}%
\right) $ per $\Omega $ e che $u\in C^{2}\left( \bar{\Omega}\right) $
funzione qualsiasi. La regolarit\`{a} di $G$ che \`{e} ragionevole
aspettarsi, pensando a com'\`{e} definita, \`{e} che $G$, vista come
funzione della sola $\mathbf{y}$, sia $C^{2}\left( \Omega \backslash \left\{ 
\mathbf{x}\right\} \right) $ (nel polo tender\`{a} a $+\infty $), $%
C^{0}\left( \bar{\Omega}\backslash \left\{ \mathbf{x}\right\} \right) $
(visto che sul bordo deve valere $0$), $L^{1}\left( \Omega \right) $ (visto
che \`{e} almeno localmente integrabile). Allora applichiamo, solo
formalmente visto che non siamo certi di alcuna ipotesi, la seconda identit%
\`{a} di Green a $u$ e a $G$ come funzione di $\mathbf{y}$.

Si ha $\int_{\Omega }u\left( \mathbf{y}\right) \Delta _{\mathbf{y}}G\left( 
\mathbf{x,y}\right) d\mathbf{y}=\int_{\Omega }G\left( \mathbf{x,y}\right)
\Delta u\left( \mathbf{y}\right) d\mathbf{y}+\int_{\partial \Omega }u\left(
\sigma \right) \left\langle \nabla _{\mathbf{y}}G\left( \mathbf{x,y}\right)
,n\right\rangle d\sigma -\int_{\partial \Omega }G\left( \mathbf{x},\sigma
\right) \left\langle \nabla u,n\right\rangle d\sigma $. Il terzo addendo a
destra \`{e} nullo perch\'{e} $G$ \`{e} nulla sul bordo; inoltre il lato
sinistro \`{e} $\int_{\Omega }u\left( \mathbf{y}\right) \Delta _{\mathbf{y}%
}G\left( \mathbf{x,y}\right) d\mathbf{y}=-\int_{\Omega }u\left( \mathbf{y}%
\right) \delta _{\mathbf{x}}d\mathbf{y=}-u\left( \mathbf{x}\right) $, dove
la scrittura della delta dentro l'integrale \`{e} da considerarsi abuso di
notazione. Allora si ha $u\left( \mathbf{x}\right) =-\int_{\Omega }G\left( 
\mathbf{x,y}\right) \Delta u\left( \mathbf{y}\right) d\mathbf{y}%
-\int_{\partial \Omega }u\left( \sigma \right) \left\langle \nabla _{\mathbf{%
y}}G\left( \mathbf{x,y}\right) ,n\right\rangle d\sigma $ (formula di
integrale per parti due volte con $\delta $ circa 1 derivata seconda di $G$%
!!). Pensando al problema $\left\{ 
\begin{array}{c}
\Delta u=f\text{ in }\Omega \\ 
u=g\text{ in }\partial \Omega%
\end{array}%
\right. $, la formula scritta suggerisce che una soluzione possa essere,
sotto opportune ipotesi, della forma%
\begin{equation}
u\left( \mathbf{x}\right) =-\int_{\Omega }G\left( \mathbf{x,y}\right)
f\left( \mathbf{y}\right) d\mathbf{y}-\int_{\partial \Omega }g\left( \sigma
\right) \left\langle \nabla _{\mathbf{y}}G\left( \mathbf{x,y}\right)
,n\right\rangle d\sigma  \label{1}
\end{equation}

Se fosse vera, con la sola difficolt\`{a} di determinare la funzione di
Green si potrebbe poi calcolare facilmente la soluzione del problema per
qualunque dato.\ref{1}

Il seguente teorema rende rigorose le speranze enunciate finora.

\textbf{Teo (regolarit\`{a} di }$G$ \textbf{e rappresentazione di }$u$)%
\textbf{\ } 
\begin{gather*}
\text{(1) Hp: }\Omega \subseteq 
%TCIMACRO{\U{211d} }%
%BeginExpansion
\mathbb{R}
%EndExpansion
^{n}\text{ \`{e} un dominio limitato e lipschitziano} \\
\text{Ts: }\exists \text{ }G\left( \mathbf{x,y}\right) \text{ funzione di
Green per }\Omega \text{, }G\left( \mathbf{x},\cdot \right) \in C^{2}\left(
\Omega \backslash \left\{ \mathbf{x}\right\} \right) \cap C^{0}\left( \bar{%
\Omega}\backslash \left\{ \mathbf{x}\right\} \right) \cap L^{1}\left( \Omega
\right) \text{ } \\
\forall \text{ }\mathbf{x}\in \Omega \text{, }G\left( \mathbf{x,y}\right) 
\text{ \`{e} continua in }\Omega ^{2}\text{ se }\mathbf{x\neq y} \\
\text{(2) Hp: }\Omega \subseteq 
%TCIMACRO{\U{211d} }%
%BeginExpansion
\mathbb{R}
%EndExpansion
^{n}\text{ \`{e} un dominio limitato e lipschitziano, }u\in C^{2}\left( \bar{%
\Omega}\right) \text{ } \\
\text{Ts: vale }u\left( \mathbf{x}\right) =-\int_{\Omega }G\left( \mathbf{x,y%
}\right) \Delta u\left( \mathbf{y}\right) d\mathbf{y}-\int_{\partial \Omega
}u\left( \sigma \right) \left\langle \nabla _{\mathbf{y}}G\left( \mathbf{x}%
,\sigma \right) ,n\right\rangle d\sigma
\end{gather*}

(1) conferma le speranze sulla regolarit\`{a} di $G$, data una certa
regolarit\`{a} del dominio. (2) mostra che se $u\in C^{2}\left( \bar{\Omega}%
\right) $ (che in effetti sarebbe un'ipotesi della seconda identit\`{a} di
Green) i passaggi fatti sopra sono leciti. (2) \`{e} una formula di
rappresentazione della funzione $u$ sul dominio $\Omega $ mediante la
funzione di Green.

\textbf{Corollario}%
\begin{gather*}
\text{Hp: }\Omega \subseteq 
%TCIMACRO{\U{211d} }%
%BeginExpansion
\mathbb{R}
%EndExpansion
^{n}\text{ \`{e} un dominio limitato e lipschitziano, }u:\Omega \rightarrow 
%TCIMACRO{\U{211d} }%
%BeginExpansion
\mathbb{R}
%EndExpansion
,u=1 \\
\text{Ts: }\int_{\partial \Omega }\frac{-\partial G}{\partial n}\left( 
\mathbf{x},\sigma \right) d\sigma =1
\end{gather*}

\textbf{Dim} E' un'applicazione immediata del teorema (2) sopra con $u=1$. $%
\blacksquare $

\textbf{Def} Dato $\Omega \subseteq 
%TCIMACRO{\U{211d} }%
%BeginExpansion
\mathbb{R}
%EndExpansion
^{n}$ dominio limitato e lipschitziano con funzione di Green $G$, si dice
nucleo di Poisson per $\Omega $ la funzione $P\left( \mathbf{x},\sigma
\right) :=\frac{-\partial G}{\partial n}\left( \mathbf{x},\sigma \right) $.

Si ricorda che $\frac{\partial G}{\partial n}\left( \mathbf{x},\sigma
\right) =\left\langle \nabla _{\mathbf{y}}G,n\right\rangle $: $\mathbf{x}$
non \`{e} considerarsi una variabile, ma fissata una volta per tutte!

Dunque $\int_{\partial \Omega }P\left( \mathbf{x},\sigma \right) d\sigma =1$%
, e inoltre $P$ \`{e} una funzione armonica nel suo primo argomento: $\Delta
_{\mathbf{x}}\left( \frac{\partial G}{\partial n}\left( \mathbf{x},\sigma
\right) \right) =\Delta _{\mathbf{x}}\left( \left\langle \nabla _{\mathbf{y}%
}G,n\right\rangle \right) =\left\langle \nabla _{\mathbf{y}}\left( \Delta _{%
\mathbf{x}}G\left( \mathbf{x},\sigma \right) \right) ,n\right\rangle $ (ma $%
n $ \`{e} indipendente da $\mathbf{x}$ perch\'{e} $n$ normale a $\partial
\Omega $ dove sto integrando in $d\mathbf{y}$?? lo scambio degli operatori
differenziali \`{e} lecito per Schwartz, essendo $G$ $C^{2}$), ma per
simmetria di $G$ vale $\Delta _{\mathbf{x}}G=-\delta _{\sigma }$, che \`{e}
nullo (perch\'{e}, considerando $\delta _{\sigma }\left( \mathbf{x}\right)
=\left\{ 
\begin{array}{c}
1\text{ se }\mathbf{x=\sigma } \\ 
0\text{ altrimenti}%
\end{array}%
\right. $, si ha che la variabile $\sigma $ sta su $\partial \Omega $,
mentre $\mathbf{x}$ varia in $\Omega $). Quindi $\Delta _{\mathbf{x}}P\left( 
\mathbf{x},\sigma \right) =0$ $\forall $ $\mathbf{x}\in \Omega $.

Ora il problema \`{e} passare dalla formula di (2) alla candidata formula
risolutiva: sotto che ipotesi tale formula d\`{a} una soluzione del
problema? Quale regolarit\`{a} ha tale soluzione? In particolare, ci
chiediamo quali ipotesi su $f$ e $g$ tale formula d\`{a} una soluzione
classica del problema. Prendiamo $f=0$, perch\'{e} le ipotesi su $f$
sarebbero molto delicate (come in effetti si \`{e} visto in un precedente
corollario in cui \`{e} stata necessaria l'ipotesi $f\in C_{0}^{2}\left(
\Omega \right) $, decisamente troppo forte per un dato).

\textbf{Teo}%
\begin{eqnarray*}
\text{Hp}\text{: } &&\Omega \subseteq 
%TCIMACRO{\U{211d} }%
%BeginExpansion
\mathbb{R}
%EndExpansion
^{n}\text{ \`{e} un dominio limitato e lipschitziano, }P\left( \mathbf{x}%
,\sigma \right) \text{ nucleo di Poisson per }\Omega \text{, }g\in
C^{0}\left( \partial \Omega \right) \\
\text{Ts}\text{: } &&u\left( \mathbf{x}\right) =\int_{\partial \Omega
}P\left( \mathbf{x},\sigma \right) g\left( \sigma \right) d\sigma \text{ 
\`{e} soluzione classica di }\left\{ 
\begin{array}{c}
\Delta u=0\text{ in }\Omega \\ 
u=g\text{ in }\partial \Omega%
\end{array}%
\right.
\end{eqnarray*}

La formula che assegna $u$ \`{e} proprio la candidata che si \`{e} trovata
sopra, dove si \`{e} sostituito $f=0$ e la definizione di nucleo di Poisson.
Per ora non sappiamo niente sull'unicit\`{a}.

E' dunque ovvio che per poter risolvere il problema \`{e} necessario
calcolare $P$, e perci\`{o} conoscere la funzione di Green.

\textbf{Costruzione della funzione di Green} Esiste una teoria generale, ma
noi ci occupiamo solo di alcuni casi particolarmente semplici.

Considero il semispazio: $%
%TCIMACRO{\U{211d} }%
%BeginExpansion
\mathbb{R}
%EndExpansion
_{+}^{n}:=\left\{ \left( \mathbf{x}^{\prime },y\right) :\mathbf{x}^{\prime
}\in 
%TCIMACRO{\U{211d} }%
%BeginExpansion
\mathbb{R}
%EndExpansion
^{n-1},y>0\right\} $. Dato che $\Delta \Gamma =-\delta $, l'idea \`{e}
costruire $G\left( \mathbf{X,Y}\right) $ a partire della soluzione
fondamentale $\Gamma \left( \mathbf{X-Y}\right) $ con il cosiddetto metodo
di riflessione. Si vuole $\forall $ $\mathbf{X}\in \Omega $ $\Delta _{%
\mathbf{Y}}G\left( \mathbf{X,Y}\right) =-\delta _{\mathbf{X}}$ in $\Omega $
e $G\left( \mathbf{X,Y}\right) =0$ se $\mathbf{Y}\in \partial \Omega $.
Sappiamo gi\`{a} che vale $\Delta _{\mathbf{Y}}\Gamma \left( \mathbf{X-Y}%
\right) =-\delta _{\mathbf{X}}$ (si \`{e} gi\`{a} dimostrato che $\Delta _{%
\mathbf{X}}\Gamma \left( \mathbf{X}\right) =-\delta _{\mathbf{0}}$: se si
trasla tutto nella dimostrazione si ottiene questo risultato), ma non
sappiamo cosa accade sul bordo.

Fissata $\mathbf{Y}\in \partial \Omega $, considero $\mathbf{X}=\left( 
\mathbf{x}^{\prime },y\right) $ e $\mathbf{X}^{\ast }=\left( \mathbf{x}%
^{\prime },-y\right) $, riflessa di $\mathbf{X}$ rispetto all'asse $y$. Vale 
$\left\vert \left\vert \mathbf{X-Y}\right\vert \right\vert =\left\vert
\left\vert \mathbf{X}^{\ast }\mathbf{-Y}\right\vert \right\vert $: ma poich%
\'{e} $\Gamma \left( \mathbf{X}\right) =\left\{ 
\begin{array}{c}
\frac{-1}{2\pi }\ln \left\vert \left\vert \mathbf{X}\right\vert \right\vert 
\text{ se }n=2 \\ 
\frac{1}{\omega _{n}\left( n-2\right) }\frac{1}{\left\vert \left\vert 
\mathbf{X}\right\vert \right\vert ^{n-2}}\text{ se }n>2%
\end{array}%
\right. $ dipende dai vettori solo attraverso il loro modulo, vale $\Gamma
\left( \mathbf{X-Y}\right) =\Gamma \left( \mathbf{X}^{\ast }\mathbf{-Y}%
\right) $, per cui $\Gamma \left( \mathbf{X-Y}\right) -\Gamma \left( \mathbf{%
X}^{\ast }\mathbf{-Y}\right) =0$ $\forall $ $\mathbf{Y}\in \partial \Omega $%
. Allora si pu\`{o} prendere $G\left( \mathbf{X,Y}\right) =\left\{ 
\begin{array}{c}
-\frac{1}{2\pi }\left( \ln \left\vert \left\vert \mathbf{X-Y}\right\vert
\right\vert -\ln \left\vert \left\vert \mathbf{X}^{\ast }\mathbf{-Y}%
\right\vert \right\vert \right) \text{ se }n=2 \\ 
\frac{1}{\omega _{n}\left( n-2\right) }\left( \frac{1}{\left\vert \left\vert 
\mathbf{X-Y}\right\vert \right\vert ^{n-2}}-\frac{1}{\left\vert \left\vert 
\mathbf{X}^{\ast }-\mathbf{Y}\right\vert \right\vert ^{n-2}}\right) \text{
se }n>2%
\end{array}%
\right. $, che in effetti \`{e} nulla se $\mathbf{Y}\in \partial \Omega $ e
il cui laplaciano in $\mathbf{Y}$ vale $-\delta _{\mathbf{X}}$ $\forall $ $%
\mathbf{X,Y}\in \Omega $ (teoricamente varrebbe $-\delta _{\mathbf{X}%
}+\delta _{\mathbf{X}^{\ast }}$, ma non ci pu\`{o} essere densit\`{a} di
carica in $\mathbf{X}^{\ast }$...?).

Si pu\`{o} calcolare il nucleo di Poisson per $n>2$: poich\'{e} la derivata
normale al bordo di $\Omega $, essendo $\Omega $ il semispazio superiore, 
\`{e} l'opposto della derivata rispetto a $y$, si ha - ponendo $\mathbf{x}%
=\left( \mathbf{x}^{\prime },x_{n}\right) $ - $P\left( \mathbf{x},\sigma
\right) =\frac{-\partial G}{\partial n}\left( \mathbf{x},\sigma \right) =%
\frac{\partial }{\partial y_{n}}G\left( \mathbf{x,y}\right) |_{\mathbf{y}%
=\left( \sigma ,0\right) }=$

$\frac{\partial }{\partial y_{n}}\frac{1}{\omega _{n}\left( n-2\right) }%
\left( \frac{1}{\left( \left( x_{1}^{\prime }-y_{1}\right) ^{2}+...+\left(
x_{n}-y_{n}\right) ^{2}\right) ^{\frac{n-2}{2}}}-\frac{1}{\left( \left(
x_{1}^{\prime }-y_{1}\right) ^{2}+...+\left( -x_{n}-y_{n}\right) ^{2}\right)
^{\frac{n-2}{2}}}\right) |_{\mathbf{y}=\left( \sigma ,0\right) }=$

$\frac{-2\left( x_{n}-y_{n}\right) }{\omega _{n}\left( n-2\right) }\frac{2-n%
}{2}\frac{1}{\left( \left( x_{1}^{\prime }-y_{1}\right) ^{2}+...+\left(
x_{n}-y_{n}\right) ^{2}\right) ^{\frac{n}{2}}}|_{\mathbf{y}=\left( \sigma
,0\right) }-\frac{1}{\omega _{n}\left( n-2\right) }\left( -2\right) \left(
-x_{n}-y_{n}\right) \frac{2-n}{2}\frac{1}{\left( \left( x_{1}^{\prime
}-y_{1}\right) ^{2}+...+\left( -x_{n}-y_{n}\right) ^{2}\right) ^{\frac{n}{2}}%
}|_{\mathbf{y}=\left( \sigma ,0\right) }=$ $\frac{2x_{n}}{\omega _{n}\left(
\left( x_{1}^{\prime }-y_{1}\right) ^{2}+...+\left( x_{n-1}^{\prime
}-y_{n-1}\right) ^{2}+\left( -x_{n}\right) ^{2}\right) ^{\frac{n}{2}}}|_{|_{%
\mathbf{y}=\left( \sigma ,0\right) }}=\frac{2x_{n}}{\omega _{n}\left(
\left\vert \left\vert \mathbf{x}^{\prime }\mathbf{-}\sigma \right\vert
\right\vert ^{2}+x_{n}^{2}\right) ^{\frac{n}{2}}}$, con $\mathbf{y}=\left(
\sigma ,0\right) $.

Se invece $n=2$ $G\left( \mathbf{x,y}\right) =-\frac{1}{4\pi }\left( \ln
\left\vert \left\vert \mathbf{x-y}\right\vert \right\vert ^{2}-\ln
\left\vert \left\vert \mathbf{x}^{\ast }\mathbf{-y}\right\vert \right\vert
^{2}\right) =-\frac{1}{4\pi }\left( \ln \left( \left( x_{1}-y_{1}\right)
^{2}+\left( x_{2}-y_{2}\right) ^{2}\right) -\ln \left( \left(
x_{1}-y_{1}\right) ^{2}+\left( -x_{2}-y_{2}\right) ^{2}\right) \right) $, e
si ha $P\left( \mathbf{x},\sigma \right) =\frac{\partial G}{\partial y_{2}}%
\left( \mathbf{x},\sigma \right) =\frac{x_{2}}{\pi \left( \left(
x_{1}-\sigma \right) ^{2}+x_{2}^{2}\right) }$: si ritrova il nucleo di
Poisson sul semipiano $k_{y}\left( x-z\right) =\frac{y}{\pi \left( \left(
x-z\right) ^{2}+y^{2}\right) }$, dove $z$ gioca il ruolo di $\sigma $ e $y$
gioca il ruolo di $x_{n}$: si noti che in dimensione $2$ l'integrale di
superficie sul bordo di $\Omega $ si riduce a un integrale semplice su $%
%TCIMACRO{\U{211d} }%
%BeginExpansion
\mathbb{R}
%EndExpansion
$, bordo del semipiano. Inoltre in senso stretto il nucleo non \`{e} $%
k_{y}\left( x\right) $, ma $k_{y}\left( x-z\right) $: contiene gi\`{a} la
traslazione che in dimensione $2$ \`{e} una convoluzione, e in questo \`{e}
coerente con il fatto che il risultato sopra (in cui non figurano
convoluzioni) \`{e} una (poderosa!) generalizzazione del risultato visto per
il semipiano. \footnote{%
In realt\`{a} si sarebbe potuto usare il metodo della trasformata di
Fourier, applicato per trovare il nucleo di Poisson sul semipiano, anche in $%
%TCIMACRO{\U{211d} }%
%BeginExpansion
\mathbb{R}
%EndExpansion
^{n}$: tuttavia le trasformate notevoli che si sono utilizzate diventano
molto meno elementari in dimensione qualsiasi, e richiedono tecniche molto pi%
\`{u} avanzate.}.

Considero ora la sfera $B_{R}\left( \mathbf{0}\right) $. Si riesce a
scrivere la funzione di Green col metodo delle immagini, che per\`{o} \`{e}
piuttosto complicato. In ogni caso, il nucleo di Poisson che si ottiene \`{e}
$P\left( \mathbf{x},\sigma \right) =\frac{R^{2}-\left\vert \left\vert 
\mathbf{x}\right\vert \right\vert ^{2}}{\omega _{n}R\left\vert \left\vert 
\mathbf{x}-\sigma \right\vert \right\vert ^{2}}$, che, di nuovo, \`{e} una
generalizzazione del $K\left( \rho ,\theta -s\right) $ gi\`{a} visto per il
problema di Dirichlet sul cerchio. Il nucleo per una sfera di centro $%
\mathbf{x}_{0}$ \`{e} invece $P\left( \mathbf{x},\sigma \right) =\frac{%
R^{2}-\left\vert \left\vert \mathbf{x-x}_{0}\right\vert \right\vert ^{2}}{%
\omega _{n}R\left\vert \left\vert \mathbf{x}-\sigma \right\vert \right\vert
^{2}}$.

Il seguente teorema particolarizza il teorema x al caso della sfera, facendo
uso del nucleo appena trovato.

\textbf{Teo}%
\begin{eqnarray*}
\text{Hp}\text{: } &&P\left( \mathbf{x},\sigma \right) =\frac{%
R^{2}-\left\vert \left\vert \mathbf{x-x}_{0}\right\vert \right\vert ^{2}}{%
\omega _{n}R\left\vert \left\vert \mathbf{x}-\sigma \right\vert \right\vert
^{2}}\text{ nucleo di Poisson per }B_{R}\left( \mathbf{x}_{0}\right) \text{, 
}g\in C^{0}\left( \partial B_{R}\left( \mathbf{x}_{0}\right) \right) \\
\text{Ts}\text{: } &&u\left( \mathbf{x}\right) =\frac{R^{2}-\left\vert
\left\vert \mathbf{x-x}_{0}\right\vert \right\vert ^{2}}{\omega _{n}R}%
\int_{\partial \Omega }\frac{g\left( \sigma \right) }{\left\vert \left\vert 
\mathbf{x}-\sigma \right\vert \right\vert ^{2}}d\sigma \text{ \`{e}
soluzione di }\left\{ 
\begin{array}{c}
\Delta u=0\text{ in }B_{R}\left( \mathbf{x}_{0}\right) \\ 
u=g\text{ in }\partial B_{R}\left( \mathbf{x}_{0}\right)%
\end{array}%
\right.
\end{eqnarray*}

Quindi la soluzione al problema di Dirichlet sulla sfera in dimensione 
\textit{qualsiasi} \`{e} completamente determinata.

\textbf{Dim (cenni)} Per mostrare che $u$ assume il dato al bordo con
continuit\`{a} si sfrutta il fatto che $P\left( \mathbf{x},\sigma \right) >0$
$\forall $ $\mathbf{x}\in B_{R}\left( \mathbf{x}_{0}\right) ,\sigma \in
\partial B_{R}\left( \mathbf{x}_{0}\right) $ e le due propriet\`{a} mostrate
sopra, che valgono per qualsiasi nucleo di Poisson: si fa una dimostrazione
simile a quella gi\`{a} vista sul cerchio e si mostra che $u\left( \mathbf{x}%
\right) \rightarrow f\left( \sigma \right) $ se $\mathbf{x}\rightarrow
\sigma $.

Per mostrare che $\Delta u=0$ in $\Omega $ si calcola $\Delta _{\mathbf{x}%
}\left( \int_{\partial \Omega }P\left( \mathbf{x},\sigma \right) g\left(
\sigma \right) d\sigma \right) =\int_{\partial \Omega }\Delta _{\mathbf{x}%
}P\left( \mathbf{x},\sigma \right) g\left( \sigma \right) d\sigma =0$. Lo
scambio tra laplaciano e integrale \`{e} lecito per il teorema di derivabilit%
\`{a} perch\'{e} $\mathbf{x}\in \Omega $ e osservando l'espressione di $P$
si nota che le sue derivate di qualsiasi ordine sono limitate per $\mathbf{x}%
\in B_{r}\left( \mathbf{x}_{0}\right) $ con $r<R$, poich\'{e} $\sigma $
varia in $\partial \Omega $ e $\mathbf{x}\in \Omega $: dunque la funzione
integranda \`{e} maggiorabile da $kg\left( \sigma \right) $, che \`{e}
integrabile su $\partial \Omega $ in quanto ivi continua. Allora $\Delta u=0$
e $u\in C^{\infty }\left( B_{R}\left( \mathbf{x}_{0}\right) \right) $. $%
\blacksquare $

\subsection{Propriet\`{a} generali delle funzioni armoniche}

\textbf{Teo (media integrale per funzioni armoniche)}%
\begin{gather*}
\text{Hp: }\Omega \subseteq 
%TCIMACRO{\U{211d} }%
%BeginExpansion
\mathbb{R}
%EndExpansion
^{n}\text{ aperto, }u\in C^{2}\left( \Omega \right) ,\Delta u=0\text{, }%
B_{R}\left( \mathbf{x}_{0}\right) :\bar{B}_{R}\left( \mathbf{x}_{0}\right)
\subseteq \Omega \\
\text{Ts: (i) }u\left( \mathbf{x}_{0}\right) =\frac{1}{\left\vert
B_{R}\left( \mathbf{x}_{0}\right) \right\vert _{n}}\int_{B_{R}\left( \mathbf{%
x}_{0}\right) }u\left( \mathbf{y}\right) d\mathbf{y} \\
\text{(ii) }u\left( \mathbf{x}_{0}\right) =\frac{1}{area\left( \partial
B_{R}\left( \mathbf{x}_{0}\right) \right) }\int_{\partial B_{R}\left( 
\mathbf{x}_{0}\right) }u\left( \sigma \right) d\sigma
\end{gather*}

Si ricorda che $area\left( \partial B_{R}\left( \mathbf{x}_{0}\right)
\right) =\omega _{n}R^{n-1}$ e $\left\vert B_{R}\left( \mathbf{x}_{0}\right)
\right\vert _{n}=\frac{\omega _{n}}{n}R^{n}$.

\textbf{Dim} (ii) Dato $\bar{B}_{R}\left( \mathbf{x}_{0}\right) \subseteq
\Omega $, considero, per $r<R$, $\phi \left( r\right) =\frac{1}{\omega
_{n}r^{n-1}}\int_{\partial B_{r}\left( \mathbf{x}_{0}\right) }u\left( \sigma
\right) d\sigma $, e mostro che \`{e} costante. Per fare la derivata in $r$
pi\`{u} facilmente si fa un cambio di variabile che faccia sparire $r$ dal
dominio d'integrazione: $\sigma =\mathbf{x}_{0}+rs$, cos\`{\i} che $\phi
\left( r\right) =\frac{1}{\omega _{n}r^{n-1}}\int_{\partial B_{1}\left( 
\mathbf{x}_{0}\right) }r^{n-1}u\left( \mathbf{x}_{0}+rs\right) ds=\frac{1}{%
\omega _{n}}\int_{\partial B_{1}\left( \mathbf{x}_{0}\right) }u\left( 
\mathbf{x}_{0}+rs\right) ds$. Allora $\phi ^{\prime }\left( r\right) =\frac{1%
}{\omega _{n}}\int_{\partial B_{1}\left( \mathbf{x}_{0}\right) }\frac{%
\partial }{\partial r}u\left( \mathbf{x}_{0}+rs\right) ds=\frac{1}{\omega
_{n}}\int_{\partial B_{1}\left( \mathbf{x}_{0}\right) }\left\langle \nabla
u\left( \mathbf{x}_{0}+rs\right) ,\mathbf{s}\right\rangle ds=\frac{1}{\omega
_{n}}\int_{\partial B_{1}\left( \mathbf{x}_{0}\right) }\frac{\partial u}{%
\partial n}\left( \mathbf{x}_{0}+rs\right) ds$: lo scambio iniziale tra
derivata e integrale \`{e} lecito perch\'{e} $\left\vert \left\langle \nabla
u\left( \mathbf{x}_{0}+rs\right) ,\mathbf{s}\right\rangle \right\vert \leq
\left\vert \left\vert \nabla u\left( \mathbf{x}_{0}+rs\right) \right\vert
\right\vert $, che \`{e} una funzione limitata in $s$ perch\'{e} $u$ \`{e} $%
C^{2}$, e le costanti sono integrabili sui compatti. Allora si ha, per il
teorema della divergenza e cambiando di nuovo variabili all'indietro, $\phi
^{\prime }\left( r\right) =\frac{1}{r^{n-1}\omega _{n}}\int_{B_{r}\left( 
\mathbf{x}_{0}\right) }\func{div}\nabla u\left( \mathbf{x}\right) d\mathbf{x=%
}\frac{1}{r^{n-1}\omega _{n}}\int_{B_{r}\left( \mathbf{x}_{0}\right) }\Delta
u\left( \mathbf{x}\right) d\mathbf{x}=0$. Dunque $\phi ^{\prime }\left(
r\right) =0$ $\forall $ $r<R$ e $\phi \left( r\right) =k$. Allora $\phi
\left( R\right) =\lim_{r\rightarrow 0^{+}}\phi \left( r\right)
=\lim_{r\rightarrow 0^{+}}\frac{1}{\omega _{n}r^{n-1}}\int_{\partial
B_{r}\left( \mathbf{x}_{0}\right) }u\left( \sigma \right) d\sigma
=\lim_{r\rightarrow 0^{+}}u\left( \mathbf{x}^{\ast }\right) =u\left( \mathbf{%
x}_{0}\right) $ per il teorema della media standard, dato che $\mathbf{x}%
^{\ast }\in B_{r}\left( \mathbf{x}_{0}\right) $. D'altra parte $\phi \left( R\right) =%
\frac{1}{area\left( \partial B_{R}\left( \mathbf{x}_{0}\right) \right) }%
\int_{\partial B_{R}\left( \mathbf{x}_{0}\right) }u\left( \sigma \right)
d\sigma $, per cui si ha la tesi.

(iii) E' noto dall'integrazione in coordinate sferiche che $%
\int_{B_{R}\left( \mathbf{x}_{0}\right) }u\left( \mathbf{y}\right) d\mathbf{%
y=}\int_{0}^{R}\left( \int_{\partial B_{\rho }\left( \mathbf{x}_{0}\right)
}u\left( \sigma \right) d\sigma _{\rho }\right) d\rho =\int_{0}^{R}u\left( 
\mathbf{x}_{0}\right) \omega _{n}\rho ^{n-1}d\rho $, per il risultato appena
dimostrato. Si ottiene quindi $\int_{B_{R}\left( \mathbf{x}_{0}\right)
}u\left( \mathbf{y}\right) d\mathbf{y}=u\left( \mathbf{x}_{0}\right) \omega
_{n}\frac{R^{n}}{n}$. $\blacksquare $

Quando si \`{e} risolto il problema di Dirichlet per il laplaciano sul
cerchio, usando la formula per serie, si \`{e} ottenuta la soluzione $%
u\left( \rho ,\theta \right) :=\frac{\alpha _{0}}{2}+\sum_{n=1}^{+\infty
}\left( \frac{\rho }{r}\right) ^{n}\left( \alpha _{n}\cos n\theta +\beta
_{n}\sin n\theta \right) $: $u\left( \mathbf{x}_{0}\right) $ si scrive come $%
u\left( 0,\theta \right) =\frac{\alpha _{0}}{2}=$ $\frac{1}{2\pi }%
\int_{0}^{2\pi }f\left( \theta \right) d\theta $, che \`{e} proprio la media
integrale di $u$ su $\partial B_{R}\left( \mathbf{x}_{0}\right) $, dato che $%
u$ sul bordo vale $f$.

\textbf{Teo (principio del massimo)}%
\begin{gather*}
\text{Hp: }\Omega \subseteq 
%TCIMACRO{\U{211d} }%
%BeginExpansion
\mathbb{R}
%EndExpansion
^{n}\text{ dominio, }u\in C^{0}\left( \Omega \right) \text{ soddisfa la
propriet\`{a} di media in }\Omega \text{,} \\
u\text{ ha massimo o minimo globale in }\Omega \\
\text{Ts: }u\text{ \`{e} costante}
\end{gather*}

Questo intuitivamente \`{e} naturale: si pensi ad esempio alla membrana
vibrante. Si noti che si sono rafforzate le ipotesi su $\Omega $; inoltre,
se $u$ soddisfa le ipotesi del teorema sopra allora soddisfa anche quelle di
questo.

\textbf{Dim} Dato $\mathbf{x}_{0}$ punto di massimo globale per $u$ in $%
\Omega $ e $B_{R}\left( \mathbf{x}_{0}\right) :\bar{B}_{R}\left( \mathbf{x}%
_{0}\right) \subseteq \Omega $, suppongo per assurdo che $u$ non sia
costante in $B_{R}\left( \mathbf{x}_{0}\right) $, cio\`{e} che $\exists $ $%
\mathbf{x}^{\ast }\in B_{R}\left( \mathbf{x}_{0}\right) $ tale che $u\left( 
\mathbf{x}^{\ast }\right) <u\left( \mathbf{x}_{0}\right) $. Allora per
permanenza del segno e continuit\`{a} $\exists $ $B_{r}\left( \mathbf{x}%
^{\ast }\right) \subseteq B_{R}\left( \mathbf{x}_{0}\right) :u\left( \mathbf{%
x}\right) <u\left( \mathbf{x}_{0}\right) $ se $\mathbf{x}\in B_{r}\left( 
\mathbf{x}^{\ast }\right) $. Applicando la propriet\`{a} della media a $%
B_{R}\left( \mathbf{x}_{0}\right) $ si ha $u\left( \mathbf{x}_{0}\right) =%
\frac{1}{\left\vert B_{R}\left( \mathbf{x}_{0}\right) \right\vert _{n}}%
\left( \int_{B_{R}\left( \mathbf{x}_{0}\right) \backslash B_{r}\left( 
\mathbf{x}_{0}\right) }u\left( \mathbf{y}\right) d\mathbf{y+}%
\int_{B_{r}\left( \mathbf{x}_{0}\right) }u\left( \mathbf{y}\right) d\mathbf{y%
}\right) $: il primo addendo \`{e} minore o uguale di $u\left( \mathbf{x}%
_{0}\right) \left\vert B_{R}\left( \mathbf{x}_{0}\right) \backslash
B_{r}\left( \mathbf{x}_{0}\right) \right\vert _{n}$, il secondo \`{e} minore
stretto di $u\left( \mathbf{x}_{0}\right) \left\vert B_{r}\left( \mathbf{x}%
_{0}\right) \right\vert _{n}$. Quindi $u\left( \mathbf{x}_{0}\right) <\frac{1%
}{\left\vert B_{R}\left( \mathbf{x}^{\ast }\right) \right\vert _{n}}u\left( 
\mathbf{x}_{0}\right) \left( \left\vert B_{R}\left( \mathbf{x}_{0}\right)
\backslash B_{r}\left( \mathbf{x}_{0}\right) \right\vert _{n}+\left\vert
B_{r}\left( \mathbf{x}_{0}\right) \right\vert _{n}\right) =u\left( \mathbf{x}%
_{0}\right) $, che \`{e} assurdo. Allora $u$ \`{e} costante in $B_{R}\left( 
\mathbf{x}_{0}\right) $.

Si pu\`{o} allora considerare un altro $B_{R^{\prime }}\left( \mathbf{x}%
_{0}\right) $ che abbia intersezione non nulla con $B_{R}\left( \mathbf{x}%
_{0}\right) $ e si ripete il ragionamento: si conclude che in entrambi gli
intorni $u$ ha il valore costante $u\left( \mathbf{x}_{0}\right) $, e si
procede cos\`{\i} invadendo $\Omega $ di sfera in sfera (questo \`{e}
possibile perch\'{e} $\Omega $ \`{e} connesso!). Dunque $u$ \`{e} costante
in $\Omega $. $\blacksquare $

\textbf{Corollario 1 (principio di massimo forte)}%
\begin{gather*}
\text{Hp: }\Omega \subseteq 
%TCIMACRO{\U{211d} }%
%BeginExpansion
\mathbb{R}
%EndExpansion
^{n}\text{ dominio limitato, }u\in C^{0}\left( \bar{\Omega}\right) \text{
soddisfa la propriet\`{a} di media in }\Omega \\
\text{Ts: }u\text{ \`{e} costante, oppure assume
massimo e minimo globale solo su }\partial \Omega
\end{gather*}

Si noti che l'ipotesi di continuit\`{a} su un compatto implica esistenza di
massimo e minimo globale in $\bar{\Omega}$.

\textbf{Dim} Ci sono due casi: o esiste un punto di massimo interno a $%
\Omega $, oppure tutti i punti di massimo sono in $\partial \Omega $. Nel
primo caso, per il teorema sopra $u$ \`{e} costante in $\Omega $, e dunque
anche in $\bar{\Omega}$ per continuit\`{a}. Nel secondo, tutti i punti di
massimo sono in $\partial \Omega $. $\blacksquare $

\textbf{Corollario\ 2}%
\begin{gather*}
\text{Hp: }\Omega \subseteq 
%TCIMACRO{\U{211d} }%
%BeginExpansion
\mathbb{R}
%EndExpansion
^{n}\text{ dominio limitato, }u\in C^{2}\left( \Omega \right) \cap
C^{0}\left( \bar{\Omega}\right) \text{, }\Delta u=0\text{ in }\Omega \\
\text{Ts: }\max_{\mathbf{x}\in \bar{\Omega}}\left\vert u\left( \mathbf{x}%
\right) \right\vert =\max_{\mathbf{x}\in \partial \Omega }\left\vert u\left( 
\mathbf{x}\right) \right\vert
\end{gather*}

\textbf{Dim} E' una conseguenza immediata del precedente corollario. $%
\blacksquare $

\textbf{Corollario 3 (unicit\`{a} della soluzione del problema di Dirichlet)}%
\begin{gather*}
\text{Hp: }\Omega \subseteq 
%TCIMACRO{\U{211d} }%
%BeginExpansion
\mathbb{R}
%EndExpansion
^{n}\text{ dominio limitato, }\left\{ 
\begin{array}{c}
\Delta u=f\text{ in }\Omega \\ 
u=g\text{ in }\partial \Omega%
\end{array}%
\right. \\
\text{Ts: la soluzione del problema in }C^{2}\left( \Omega \right) \cap
C^{0}\left( \bar{\Omega}\right) \text{, se esiste, \`{e} unica}
\end{gather*}

\textbf{Dim} Siano $u_{1},u_{2}\in C^{2}\left( \Omega \right) \cap
C^{0}\left( \bar{\Omega}\right) $ soluzioni del problema. $u:=u_{1}-u_{2}$
appartiene alla stessa classe e soddisfa $\left\{ 
\begin{array}{c}
\Delta u=0\text{ in }\Omega \\ 
u=0\text{ in }\partial \Omega%
\end{array}%
\right. $. Per il corollario sopra, $\max_{\mathbf{x}\in \bar{\Omega}%
}\left\vert u\left( \mathbf{x}\right) \right\vert =\max_{\mathbf{x}\in
\partial \Omega }\left\vert u\left( \mathbf{x}\right) \right\vert =0$,
quindi il massimo modulo di $u$ \`{e} $0$, cio\`{e} $u$ \`{e} nulla in tutto 
$\bar{\Omega}$ e $u_{1}=u_{2}$. $\blacksquare $

Ne segue immediatamente il

\textbf{Corollario 4 (unicit\`{a} della soluzione del problema di Dirichlet
per il laplaciano sul cerchio)}%
\begin{gather*}
\text{Hp: }\left\{ 
\begin{array}{c}
\Delta u=0\text{ in }B_{R}\left( \mathbf{x}_{0}\right) \\ 
u=g\text{ in }\partial B_{R}\left( \mathbf{x}_{0}\right)%
\end{array}%
\right. \\
\text{Ts: la soluzione del problema esiste, \`{e} unica, \`{e} assegnata } \\
\text{dalla formula integrale di Poisson ed \`{e} }C^{\infty }\left(
B_{R}\left( \mathbf{x}_{0}\right) \right)
\end{gather*}

% (non \`{e} unica in $C^{2}\left( \Omega \right) \cap C^{0}\left( \bar{\Omega}%
% \right) $?)

Tutte le propriet\`{a} della tesi tranne l'unicit\`{a} erano gi\`{a} note.

Vale inoltre, per il problema di Dirichlet per il laplaciano, una propriet%
\`{a} di dipendenza continua delle soluzioni dai dati, spesso detta stabilit%
\`{a}.

\textbf{Teo (stabilit\`{a} delle soluzioni del problema di Dirichlet)}%
\begin{eqnarray*}
\text{Hp} &\text{: }&\Omega \subseteq 
%TCIMACRO{\U{211d} }%
%BeginExpansion
\mathbb{R}
%EndExpansion
^{n}\text{ dominio limitato, }u_{1},u_{2}\in C^{2}\left( \Omega \right) \cap
C^{0}\left( \bar{\Omega}\right) \text{, }\left\{ 
\begin{array}{c}
\Delta u_{1}=0\text{ in }\Omega \\ 
u_{1}=g_{1}\text{ su }\partial \Omega%
\end{array}%
\right. ,\left\{ 
\begin{array}{c}
\Delta u_{2}=0\text{ in }\Omega \\ 
u_{2}=g_{2}\text{ su }\partial \Omega%
\end{array}%
\right. \\
\text{Ts} &\text{: }&\max_{\bar{\Omega}}\left\vert u_{1}-u_{2}\right\vert
=\max_{\bar{\Omega}}\left\vert g_{1}-g_{2}\right\vert
\end{eqnarray*}

In questo caso nel problema di Dirichlet non figura l'equazione di Poisson,
ma quella di Laplace: infatti il corollario al principio del massimo si
applica a funzioni armoniche.

La tesi significa che una piccola perturbazione sui dati $g_{1},g_{2}$ si
traduce in una piccola perturbazione delle soluzioni.

\textbf{Dim} $u=u_{1}-u_{2}\in C^{2}\left( \Omega \right) \cap C^{0}\left( 
\bar{\Omega}\right) $, e $\left\{ 
\begin{array}{c}
\Delta u=0\text{ in }\Omega \\ 
u=g_{1}-g_{2}\text{ su }\partial \Omega%
\end{array}%
\right. $: a $u$ si applica il corollario $2$, per cui $\max_{\bar{\Omega}%
}\left\vert u\right\vert =\max_{\partial \Omega }\left\vert u\right\vert
=\max_{\partial \Omega }\left\vert g_{1}-g_{2}\right\vert $. $\blacksquare $

\textbf{Teo (regolarit\`{a} delle funzioni armoniche)}%
\begin{eqnarray*}
\text{Hp}\text{: } &&\Omega \subseteq 
%TCIMACRO{\U{211d} }%
%BeginExpansion
\mathbb{R}
%EndExpansion
^{n}\text{ aperto, }u\in C^{2}\left( \Omega \right) \text{, }\Delta u=0\text{
in }\Omega \\
\text{Ts}\text{: } &&u\in C^{\infty }\left( \Omega \right)
\end{eqnarray*}

\textbf{Dim} Considero $B_{R}\left( \mathbf{x}_{0}\right) :\bar{B}_{R}\left( 
\mathbf{x}_{0}\right) \subseteq \Omega $: in tale palla $u\in C^{2}$.
Considero il problema di Dirichlet $\left\{ 
\begin{array}{c}
\Delta v=0\text{ in }B_{R}\left( \mathbf{x}_{0}\right) \\ 
v=u\text{ su }\partial B_{R}\left( \mathbf{x}_{0}\right)%
\end{array}%
\right. $. Poich\'{e} $u$ \`{e} armonica, $u$ \`{e} certamente una
soluzione; ma poich\'{e} per il corollario 4 la soluzione esiste unica (\`{e}
assegnata dalla formula di Poisson sulla sfera) ed \`{e} $C^{\infty }$, $v=u$
\`{e} l'unica soluzione, ed \`{e} $C^{\infty }\left( B_{R}\left( \mathbf{x}%
_{0}\right) \right) $. Essendo $B_{R}\left( \mathbf{x}_{0}\right) $
arbitrario, $u\in C^{\infty }\left( \Omega \right) $. $\blacksquare $

\textbf{Teorema inverso della media}%
\begin{eqnarray*}
\text{Hp} &\text{: }&\Omega \subseteq 
%TCIMACRO{\U{211d} }%
%BeginExpansion
\mathbb{R}
%EndExpansion
^{n}\text{ aperto, }u\in C^{0}\left( \Omega \right) \text{ soddisfa la
propriet\`{a} di media in }\Omega \\
\text{Ts} &\text{: }&u\in C^{2}\left( \Omega \right) \text{ e }\Delta u=0%
\text{ in }\Omega
\end{eqnarray*}

Questo teorema stabilisce che vale anche l'implicazione inversa rispetto al
teorema della media gi\`{a} visto: tale propriet\`{a} \`{e} quindi una
caratterizzazione delle funzioni armoniche.

\textbf{Dim} E' noto che $\forall $ $B_{R}\left( \mathbf{x}_{0}\right) :\bar{%
B}_{R}\left( \mathbf{x}_{0}\right) \subseteq \Omega $ vale la propriet\`{a}
di media. Considero il problema di Dirichlet $\left\{ 
\begin{array}{c}
\Delta v=0\text{ in }B_{R}\left( \mathbf{x}_{0}\right) \\ 
v=u\text{ su }\partial B_{R}\left( \mathbf{x}_{0}\right)%
\end{array}%
\right. $, la cui soluzione $v$ esiste unica: $v$ \`{e} armonica e $C^{2}$,
quindi soddisfa la propriet\`{a} di media. Allora anche $v-u$ soddisfa la
propriet\`{a} di media, e dunque $\max_{B_{R}\left( \mathbf{x}_{0}\right)
}\left\vert v-u\right\vert =\max_{\partial B_{R}\left( \mathbf{x}_{0}\right)
}\left\vert v-u\right\vert =0$, cio\`{e} $u=v$ in $B_{R}\left( \mathbf{x}%
_{0}\right) $: dunque $u$ \`{e} armonica in $B_{R}\left( \mathbf{x}%
_{0}\right) $ e $C^{2}\left( B_{R}\left( \mathbf{x}_{0}\right) \right) $.
Poich\'{e} $B_{R}\left( \mathbf{x}_{0}\right) $ \`{e} arbitrario, si ha che $%
u\in C^{2}\left( \Omega \right) $ e $\Delta u=0$ in $\Omega $. $\blacksquare 
$

\section{Equazione di diffusione del calore}

\textbf{Modello fisico} Si pone $u\left( \mathbf{x},t\right) $ temperatura
di un corpo continuo $\Omega \subseteq 
%TCIMACRO{\U{211d} }%
%BeginExpansion
\mathbb{R}
%EndExpansion
^{3}$ al tempo $t$, con densit\`{a} di massa $\rho \left( \mathbf{x}%
,t\right) $; $e\left( \mathbf{x},t\right) $ densit\`{a} di energia termica,
che \`{e} pari a $c\left( \mathbf{x},t\right) u\left( \mathbf{x},t\right) $; 
$\mathbf{q}\left( \mathbf{x},t\right) $ densit\`{a} di corrente termica, di
modo che il flusso di $\mathbf{q}$ attraverso una superficie $\Sigma $ sia
la quantit\`{a} di calore che attraversa $\Sigma $ nell'unit\`{a} di tempo; $%
r\left( \mathbf{x},t\right) $ tasso istantaneo di calore prodotto o
sottratto in $\mathbf{x}$ all'istante $t$, per unit\`{a} di volume (dovuto
alla presenza di una sorgente o un pozzo).

Voglio scrivere un bilancio energetico in $B_{R}\left( \mathbf{x}_{0}\right)
:\bar{B}_{R}\left( \mathbf{x}_{0}\right) \subseteq \Omega $. In tale intorno
come varia nel tempo l'energia termica? L'energia totale interna a $%
B_{R}\left( \mathbf{x}_{0}\right) $ \`{e} $\int_{B_{R}\left( \mathbf{x}%
_{0}\right) }e\left( \mathbf{x},t\right) \rho \left( \mathbf{x},t\right) d%
\mathbf{x}=\int_{B_{R}\left( \mathbf{x}_{0}\right) }cu\rho \left( \mathbf{x}%
,t\right) d\mathbf{x}$; il suo tasso istantaneo di variazione \`{e} -
ipotizzando che lo scambio sia lecito - $\frac{d}{dt}\int_{B_{R}\left( 
\mathbf{x}_{0}\right) }e\left( \mathbf{x},t\right) \rho \left( \mathbf{x}%
,t\right) d\mathbf{x=}\int_{B_{R}\left( \mathbf{x}_{0}\right) }\frac{%
\partial }{\partial t}cu\rho \left( \mathbf{x},t\right) d\mathbf{x}$. Tale
tasso istantaneo \`{e} da considerarsi derivante da due fenomeni: la
produzione/sottrazione di calore ($A$), dovuta a sorgenti o pozzi, e il
calore entrante/uscente in $B_{R}\left( \mathbf{x}_{0}\right) $ ($B$). Vale $%
A=\int_{B_{R}\left( \mathbf{x}_{0}\right) }r\left( \mathbf{x},t\right) \rho
\left( \mathbf{x},t\right) d\mathbf{x}$, mentre $B=-\int_{\partial
B_{R}\left( \mathbf{x}_{0}\right) }\left\langle \mathbf{q}\left( \mathbf{x}%
,t\right) ,\mathbf{n}_{e}\right\rangle d\sigma =-\int_{B_{R}\left( \mathbf{x}%
_{0}\right) }\func{div}\mathbf{q}\left( \mathbf{x},t\right) d\mathbf{x}$.
Allora si ottiene l'equazione 
\begin{equation*}
\int_{B_{R}\left( \mathbf{x}_{0}\right) }%
\frac{\partial }{\partial t}cu\rho \left( \mathbf{x},t\right) d\mathbf{x=}%
\int_{B_{R}\left( \mathbf{x}_{0}\right) }r\left( \mathbf{x},t\right) \rho
\left( \mathbf{x},t\right) d\mathbf{x-}\int_{B_{R}\left( \mathbf{x}%
_{0}\right) }\func{div}\mathbf{q}\left( \mathbf{x},t\right) d\mathbf{x}
\end{equation*}
cio%
\`{e} $\int_{B_{R}\left( \mathbf{x}_{0}\right) }\left[ \frac{\partial }{%
\partial t}cu\rho \left( \mathbf{x},t\right) \mathbf{-}r\left( \mathbf{x}%
,t\right) \rho \left( \mathbf{x},t\right) \mathbf{+}\func{div}\mathbf{q}%
\left( \mathbf{x},t\right) \right] d\mathbf{x}$. Usando il teorema della
media in $B_{R}\left( \mathbf{x}_{0}\right) $ come gi\`{a} visto, se la
funzione integranda \`{e} continua si ha $\frac{\partial }{\partial t}cu\rho
\left( \mathbf{x},t\right) \mathbf{-}r\left( \mathbf{x},t\right) \rho \left( 
\mathbf{x},t\right) \mathbf{+}\func{div}\mathbf{q}\left( \mathbf{x},t\right)
=0$.

Si scrive ora esplicitamente $\mathbf{q}$: la corrente termica \`{e} dovuta
alla conduzione\footnote{%
meccanismo di trasmissione di calore che avviene a livello microscopico a
causa delle oscillazioni delle molecole}, anche detta diffusione, e (nei
fluidi) alla convezione\footnote{%
meccanismo di trasporto del calore dovuto al moto d'insieme del fluido, che
si sposta}, anche detta deriva.

Il termine di diffusione \`{e} proporzionale alla differenza di temperatura,
per cui $\mathbf{q}=-k\nabla u$ (il segno meno d\`{a} conto del fatto che il
calore fluisce dai corpi caldi ai corpi freddi). In realt\`{a}, se il mezzo
non \`{e} isotropo, potrebbe non essere $\mathbf{q}//\nabla u$, ma c'\`{e}
comunque un angolo acuto tra $\mathbf{q}$ e $\nabla u$: dunque, nel caso pi%
\`{u} generale, $\mathbf{q}=-A\left( \mathbf{x}\right) \nabla u\left( 
\mathbf{x}\right) $, con $A$ definita positiva. Questo termine \`{e} detto
anche di diffusione.

Il termine di convezione \`{e} invece, se $\mathbf{v}$ \`{e} la velocit\`{a}
del fluido, $\mathbf{q}=c_{1}\mathbf{v}u$. Questo termine \`{e} anche detto
di trasporto, o di deriva.

Si ottiene allora nella sua forma pi\`{u} generale l'equazione di diffusione
nell'incognita $u$%
\begin{equation*}
\frac{\partial cu\rho }{\partial t}\mathbf{+}\func{div}\left( -A\nabla
u+c_{1}\mathbf{v}u\right) =r\rho
\end{equation*}

Nel caso pi\`{u} semplice in cui tutti i coefficienti siano costanti e il
mezzo sia omogeneo e isotropo (quindi $c,c_{1},k$ costanti, $\rho $ costante
nel tempo, $\mathbf{v}$ costante nello spazio) si ottiene, dividendo per $%
c\rho $, $\frac{\partial u}{\partial t}-\frac{k}{c\rho }\Delta u+\frac{c_{1}%
}{c\rho }\left\langle \nabla u,\mathbf{v}\right\rangle =\frac{r}{c}$.
Rinominando i coefficienti $\frac{k}{c\rho }=:D$, detto coefficiente di
diffusione, $\frac{c_{1}\mathbf{v}}{c\rho }=:\mathbf{b}$ e ponendo $f\left( 
\mathbf{x},t\right) :=\frac{r}{c}$, si ottiene l'equazione%
\begin{equation*}
\frac{\partial u}{\partial t}-D\Delta u+\left\langle \nabla u,\mathbf{b}%
\right\rangle =f
\end{equation*}

dove il secondo addendo \`{e} il termine di diffusione, il terzo quello di
trasporto, il lato destro \`{e} di sorgente.

A volte con equazione del calore si intende, in senso pi\`{u} restrittivo,
l'equazione $\frac{\partial u}{\partial t}-D\Delta u=f$, spesso con sorgente
nulla.

Tale modello descrive anche la diffusione di altri tipi, ad esempio di una
sostanza disciolta in un'altra: in tal caso $u$ ha il significato fisico di
concentrazione. L'equazione diventa $\frac{\partial u}{\partial t}-D\Delta
u+\left\langle \nabla u,\mathbf{b}\right\rangle +\gamma u=f$, dove il
termine $\gamma u$ (con $\gamma >0$) rappresenta la decrescita di
concentrazione dovuta a reazioni chimiche.

\subsection{Problemi al contorno e ai valori iniziali per l'equazione del
calore}

Noi studieremo l'equazione del calore standard\footnote{%
alla fine del corso vedremo il caso stazionario dell'equazione pi\`{u}
generale: $-\func{div}\left( A\left( x\right) \nabla u\right) +\left\langle 
\mathbf{b},\nabla u\right\rangle +\gamma u=f$} 
\begin{equation*}
\frac{\partial u}{\partial t}-D\Delta u=f
\end{equation*}

che talora si scrive come $Hu=f$, dove $H$ \`{e} un operatore differenziale
lineare, detto operatore del calore, tale che $Hu:=\frac{\partial u}{%
\partial t}-D\Delta u$. $u=u\left( \mathbf{x},t\right) $, $\mathbf{x}\in
\Omega \subseteq 
%TCIMACRO{\U{211d} }%
%BeginExpansion
\mathbb{R}
%EndExpansion
^{n}$, $D>0$.

L'equazione \`{e} del prim'ordine in $t$: ha senso assegnare $u$ all'istante 
$0$. Ne nasce un problema detto di Cauchy globale:%
\begin{equation*}
\left\{ 
\begin{array}{c}
Hu=f\text{ con }\mathbf{x}\in 
%TCIMACRO{\U{211d} }%
%BeginExpansion
\mathbb{R}
%EndExpansion
^{n},t>0 \\ 
u\left( \mathbf{x},0\right) =g\left( \mathbf{x}\right) \text{ con }\mathbf{x}%
\in 
%TCIMACRO{\U{211d} }%
%BeginExpansion
\mathbb{R}
%EndExpansion
^{n}%
\end{array}%
\right.
\end{equation*}

Se invece $\mathbf{x}$ varia in un insieme $\Omega $ e $t$ in $\left(
0,T\right) $, si definisce $Q_{T}:=\Omega \times \left( 0,T\right) $, detto
cilindro\footnote{%
con riferimento al caso di $\Omega $ cerchio}. In tal caso si possono
mettere delle condizioni al contorno anche su $\Omega $: si ottiene quindi
il problema%
\begin{equation*}
\left\{ 
\begin{array}{c}
Hu=f\text{ in }Q_{T} \\ 
u\left( \mathbf{x},0\right) =g\left( \mathbf{x}\right) \text{ con }\mathbf{x}%
\in \Omega \\ 
\text{condizione al bordo con }\mathbf{x}\in \partial \Omega ,t\in \left(
0,T\right)%
\end{array}%
\right.
\end{equation*}

dove la condizione al bordo pu\`{o} essere $u=h$ (condizione di Dirichlet), $%
\frac{\partial u}{\partial n}=h$ (di Neumann), mista, $\frac{\partial u}{%
\partial n}+\gamma u=h$ (di Robin). Ne nasceranno il problema di
Cauchy-Dirichlet, Cauchy-Neumann, ecc.

Si noti che le condizioni sopra sono poste non su tutto il bordo di $Q_{T}$,
ma solo sulla sua base e sulla sua superficie laterale: non sul "coperchio",
dato che non ha senso imporre la temperatura finale. Si dice allora
frontiera parabolica\footnote{%
questo nome acquisir\`{a} un senso quando si studieranno le equazioni
paraboliche} il sottinsieme di $\partial Q_{T}$ che \`{e} effettivamente
soggetto a condizioni: $\partial _{p}Q_{T}:=\left( \bar{\Omega}\times
\left\{ 0\right\} \right) \cup \left( \partial \Omega \times \left[ 0,T%
\right] \right) $.

Tale definizione permette una notevole sintesi del problema di
Cauchy-Dirichlet: $\left\{ 
\begin{array}{c}
Hu=f\text{ in }Q_{T} \\ 
u=g\text{ su }\partial _{p}Q_{T}%
\end{array}%
\right. $.

Osserviamo - in realt\`{a} questo \`{e} valido in generale per EDP lineari a
coefficienti costanti - che l'equazione del calore in $%
%TCIMACRO{\U{211d} }%
%BeginExpansion
\mathbb{R}
%EndExpansion
^{n}$ \`{e} invariante\footnote{\tiny{parola scelta male?}} per traslazioni spazio
temporali: cio\`{e}, fissato $\left( \mathbf{x}_{0},t_{0}\right) $, $H\left[
u\left( \mathbf{x+x}_{0},t+t_{0}\right) \right] =\left( Hu\right) \left( 
\mathbf{x+x}_{0},t+t_{0}\right) $. Inoltre \`{e} invariante per riflessioni
spaziali: $H\left[ u\left( -\mathbf{x},t\right) \right] =\left( Hu\right)
\left( -\mathbf{x},t\right) $. Non c'\`{e} invarianza per riflessioni
temporali (coerentemente col fatto che la termodinamica \`{e} in genere
molto sensibile alla "freccia del tempo", a differenza di altre aree della
fisica): se $Hu=\frac{\partial u}{\partial t}-D\Delta u=0$ e $v\left( 
\mathbf{x},t\right) :=u\left( \mathbf{x},-t\right) $, non necessariamente $%
Hv=0$; vale per\`{o} $\frac{\partial v}{\partial t}+D\Delta v=0$, che \`{e}
detta equazione del calore all'indietro.

\subsection{Propriet\`{a} generali dell'equazione del calore}

Proprio come visto per le funzioni armoniche, anche per l'equazione del
calore dimostrare princ\`{\i}pi di massimo porta facilmente a risultati di
unicit\`{a}.

Si fanno prima alcune osservazioni intuitive sulle soluzioni dell'equazione
del calore e su dove vengono assunti punti di massimo e minimo. Data $%
u:u_{t}-D\Delta u=0$ e fissato $t_{0}$, si osserva il grafico di $u\left( 
\mathbf{x},t_{0}\right) $ al variare di $\mathbf{x}$: se in un punto di
massimo $\mathbf{x}^{\ast }$ si ha $\Delta u\left( \mathbf{x}^{\ast
},t_{0}\right) <0$, si avr\`{a} anche $u_{t}\left( \mathbf{x}^{\ast
},t_{0}\right) <0$; se in un punto di minimo $\mathbf{x}^{\ast }$ si ha $%
\Delta u\left( \mathbf{x}^{\ast },t_{0}\right) >0$, si avr\`{a} anche $%
u_{t}\left( \mathbf{x}^{\ast },t_{0}\right) >0$. Quindi, se si considera il
grafico di $u\left( \mathbf{x},t_{0}\right) $ al crescere di $t_{0}$,
si nota che tale grafico tende a "spianarsi" rispetto a quello iniziale: i
massimi si abbassano, i minimi si sollevano (questo, fisicamente, \`{e}
l'effetto della diffusione: gli estremi vengono smussati).

Considerando l'equazione del calore nel cilindro $Q_{T}$ e $u:Hu=0$ in $%
Q_{T} $, dove ci si aspetta che vengano assunti massimi e minimi globali?
Per quanto detto sopra, non ci si aspetta che possano esserci punti di
massimo o minimo interni a $Q_{T}$: aspettando un istante, si vedrebbe la
temperatura scendere o salire. Per lo stesso motivo, non ci si aspetta che
ci siano punti di massimo o di minimo nell'istante finale $T$ (sul
"coperchio" del cilindro). Rimane dunque la frontiera parabolica $%
\partial _{p}Q_{T}$.

\textbf{Teo (principio del massimo per l'equazione del calore)}
\begin{gather*}
\text{Hp: }\Omega \subseteq 
%TCIMACRO{\U{211d} }%
%BeginExpansion
\mathbb{R}
%EndExpansion
^{n}\text{ dominio limitato, }u\in C^{2,1}\left( Q_{T}\right) \cap
C^{0}\left( \bar{Q}_{T}\right) \\
\text{Ts: (i) se }Hu\leq 0\text{ in }Q_{T}\text{, allora il massimo globale
di }u\text{ in }\bar{Q}_{T}\text{ \`{e} assunto su }\partial _{p}Q_{T} \\
\text{(ii) se }Hu\geq 0\text{ in }Q_{T}\text{, allora il minimo globale di }u%
\text{ in }\bar{Q}_{T}\text{ \`{e} assunto su }\partial _{p}Q_{T} \\
\text{(iii) se }Hu=0\text{ in }Q_{T}\text{, allora }\max_{\bar{Q}%
_{T}}\left\vert u\right\vert =\max_{\partial _{p}Q_{T}}\left\vert
u\right\vert
\end{gather*}

sarebbe meglio dire che \`{e} assunto \textit{solo} su??

Le ipotesi sullo spazio funzionale di appartenenza di $u$ sono quelle
naturali affinch\'{e} l'equazione del calore sia ben definita e i termini
che vi appaiono siano continui (e quelli naturali per ambientare il problema
di Cauchy-Dirichlet).

Massimo e minimo globali esistono per Weierstrass, essendo $u\in C^{0}\left( 
\bar{Q}_{T}\right) $.

\textbf{Dim} Se vale (i), allora si ottiene banalmente (ii) applicando (i) a 
$-u$: \`{e} sufficiente dunque mostrare (i).

L'idea della dimostrazione \`{e} procedere per assurdo: se $\left( \mathbf{x}%
_{0}\mathbf{,}t_{0}\right) $ \`{e} un punto di massimo in $Q_{T}$, per il
teorema di Fermat vale $\frac{\partial u}{\partial t}\left( \mathbf{x}%
_{0},t_{0}\right) =0$, e inoltre $\Delta u\left( \mathbf{x}_{0},t_{0}\right)
\leq 0$, ma allora si avrebbe $Hu\geq 0$; se si riuscisse ad avere la
disuguaglianza stretta si troverebbe un assurdo. Sul "coperchio" del
cilindro invece non si possono sfruttare risultati sulle derivate.

$\forall $ $\varepsilon >0$ pongo\footnote{%
il $-\varepsilon t$ serve proprio ad avere una disuguaglianza stretta quando
si applica l'operatore del calore} $v_{\varepsilon }:=u-\varepsilon t$: $%
v_{\varepsilon }\in C^{2,1}\left( Q_{T}\right) \cap C^{0}\left( \bar{Q}%
_{T}\right) $ e studio $v_{\varepsilon }$ su $Q_{T-\varepsilon }$: mostro
che $v_{\varepsilon }$ avr\`{a} massimo globale in $\bar{Q}_{T-\varepsilon }$%
, e in particolare in $\partial _{p}Q_{T-\varepsilon }$. Per assurdo, non
sia cos\`{\i}: (1) $\exists $ $\left( \mathbf{x}_{0},t_{0}\right) \in
Q_{T-\varepsilon }$ punto di massimo per $v_{\varepsilon }$. Allora $%
Hv_{\varepsilon }\left( \mathbf{x}_{0},t_{0}\right) =Hu-\varepsilon <0$ perch%
\'{e} per ipotesi $Hu\leq 0$; ma $\frac{\partial v_{\varepsilon }}{\partial t%
}\left( \mathbf{x}_{0},t_{0}\right) =0$ per Fermat, e per concavit\`{a} $%
\Delta v_{\varepsilon }\left( \mathbf{x}_{0},t_{0}\right) \leq 0$, dal che
si ottiene che $Hv_{\varepsilon }\left( \mathbf{x}_{0},t_{0}\right) \geq 0$,
assurdo.

(2) suppongo, di nuovo per assurdo, che esista $\left( \mathbf{x}%
_{0},t_{0}\right) \in \Omega \times \left\{ T-\varepsilon \right\} $ punto
di massimo per $v_{\varepsilon }$. Allora per concavit\`{a} $\Delta
v_{\varepsilon }\left( \mathbf{x}_{0},t_{0}\right) \leq 0$, mentre $\frac{%
\partial v_{\varepsilon }}{\partial t}\left( \mathbf{x}_{0},t_{0}\right)
\geq 0$, per cui $Hv_{\varepsilon }\left( \mathbf{x}_{0},t_{0}\right) \geq 0$%
, assurdo perch\'{e} $Hv_{\varepsilon }\left( \mathbf{x}_{0},t_{0}\right)
=Hu-\varepsilon <0$.

Questo significa che $\max_{Q_{T-\varepsilon }}v_{\varepsilon
}=\max_{\partial _{p}Q_{T-\varepsilon }}v_{\varepsilon }$. Se $\varepsilon
\rightarrow 0^{+}$, $v_{\varepsilon }\rightarrow u$ uniformemente\footnote{%
la successione \`{e} indicizzata su $\varepsilon $ reale positivo che tende
a $0^{+}$: \`{e} equivalente a considerare la successione indicizzata su $n$
con $\varepsilon =a_{n}:a_{n}\rightarrow ^{n\rightarrow +\infty }0^{+}$, $%
a_{n}$ qualsiasi.
\par
$\left\vert v_{\varepsilon }-u\right\vert =\varepsilon t$: la convergenza
uniforme \`{e} ovvia} e quindi $\max_{\bar{Q}_{T}}u=\max_{\partial
_{p}Q_{T}}u$.

(iii) \`{e} una conseguenza immediata di (i) e (ii). $\blacksquare $

La tesi che ci interessa in particolare \`{e} la terza.

\textbf{Corollario (unicit\`{a} della soluzione del problema di
Cauchy-Dirichlet)}

\begin{gather*}
\text{Hp: }\Omega \subseteq 
%TCIMACRO{\U{211d} }%
%BeginExpansion
\mathbb{R}
%EndExpansion
^{n}\text{ dominio limitato, }\left\{ 
\begin{array}{c}
Hu=f\text{ in }Q_{T} \\ 
u=g\text{ su }\partial _{p}Q_{T}%
\end{array}%
\right. \\
\text{Ts: la soluzione del problema nella classe }C^{2,1}\left( Q_{T}\right)
\cap C^{0}\left( \bar{Q}_{T}\right) \text{, se esiste, \`{e} unica}
\end{gather*}

\textbf{Dim} Siano $u_{1},u_{2}$ soluzioni del problema nella classe $%
C^{2,1}\left( Q_{T}\right) \cap C^{0}\left( \bar{Q}_{T}\right) $. Allora $%
u:=u_{1}-u_{2}\in C^{2,1}\left( Q_{T}\right) \cap C^{0}\left( \bar{Q}%
_{T}\right) $ e soddisfa $\left\{ 
\begin{array}{c}
Hu=0\text{ in }Q_{T} \\ 
u=0\text{ su }\partial _{p}Q_{T}%
\end{array}%
\right. $: per il principio del massimo, poich\'{e} $Hu=0$, $\max_{\bar{Q}%
_{T}}\left\vert u\right\vert =\max_{\partial _{p}Q_{T}}\left\vert
u\right\vert =0$, per cui $u_{1}=u_{2}$. $\mathbf{\blacksquare }$

\textbf{Corollario (teorema di confronto)}%
\begin{gather*}
\text{Hp: }\Omega \subseteq 
%TCIMACRO{\U{211d} }%
%BeginExpansion
\mathbb{R}
%EndExpansion
^{n}\text{ dominio limitato, }u\in C^{2,1}\left( Q_{T}\right) \cap
C^{0}\left( \bar{Q}_{T}\right) \\
\text{Ts: (i) se }Hu\leq 0\text{ in }Q_{T}\text{ e }u\leq 0\text{ su }%
\partial _{p}Q_{T}\text{, allora }u\leq 0\text{ in }Q_{T} \\
\text{(ii) se }Hu\geq 0\text{ in }Q_{T}\text{ e }u\geq 0\text{ su }\partial
_{p}Q_{T}\text{, allora }u\geq 0\text{ in }Q_{T}
\end{gather*}

\textbf{Dim} E' una conseguenza immediata delle tesi (i),(ii) del principio
del massimo. $\blacksquare $

\textbf{Teo (stabilit\`{a} del problema di Cauchy-Dirichlet)}%
\begin{gather*}
\text{Hp: }\Omega \subseteq 
%TCIMACRO{\U{211d} }%
%BeginExpansion
\mathbb{R}
%EndExpansion
^{n}\text{ dominio limitato, }u\in C^{2,1}\left( Q_{T}\right) \cap
C^{0}\left( \bar{Q}_{T}\right) \text{ risolve}\left\{ 
\begin{array}{c}
Hu=f\text{ in }Q_{T} \\ 
u=g\text{ su }\partial _{p}Q_{T}%
\end{array}%
\right. \text{, } \\
\text{Ts: }\max_{\bar{Q}_{T}}\left\vert u\right\vert \leq T\max_{\bar{Q}%
_{T}}\left\vert f\right\vert +\max_{\partial _{p}Q_{T}}\left\vert
g\right\vert
\end{gather*}

La tesi significa che una piccola perturbazione sui dati $g,f$ si traduce in
una piccola perturbazione della soluzione, per tempi \textit{finiti} (se $%
T\rightarrow +\infty $, la stima non ha pi\`{u} alcuna utilit\`{a}).

\textbf{Dim} L'idea \`{e} applicare il teorema del confronto a due funzioni
ben scelte. Considero $v_{1}=u+t\max_{\bar{Q}_{T}}\left\vert f\right\vert
+\max_{\partial _{p}Q_{T}}\left\vert g\right\vert $: $Hv_{1}=Hu+\max_{\bar{Q}%
_{T}}\left\vert f\right\vert =f+\max_{\bar{Q}_{T}}\left\vert f\right\vert
\geq 0$ in $\bar{Q}_{T}$. Allora si pu\`{o} provare ad applicare il teorema
del confronto a $v_{1}$: poich\'{e} su $\partial _{p}Q_{T}$ $v_{1}=g+t\max_{%
\bar{Q}_{T}}\left\vert f\right\vert +\max_{\partial _{p}Q_{T}}\left\vert
g\right\vert \geq g+\max_{\partial _{p}Q_{T}}\left\vert g\right\vert \geq 0$%
, si ha $v_{1}\geq 0$ in tutto $Q_{T}$.

Considero $v_{2}=-u+t\max_{\bar{Q}_{T}}\left\vert f\right\vert
+\max_{\partial _{p}Q_{T}}\left\vert g\right\vert $: $Hv_{2}=-Hu+\max_{\bar{Q%
}_{T}}\left\vert f\right\vert =-f+\max_{\bar{Q}_{T}}\left\vert f\right\vert
\geq 0$ in $\bar{Q}_{T}$. Su $\partial _{p}Q_{T}$ $v_{2}=-g+t\max_{\bar{Q}%
_{T}}\left\vert f\right\vert +\max_{\partial _{p}Q_{T}}\left\vert
g\right\vert \geq -g+\max_{\partial _{p}Q_{T}}\left\vert g\right\vert \geq 0$%
, quindi per confronto $v_{2}\geq 0$ in tutto $Q_{T}$.

Allora $\pm u+t\max_{\bar{Q}_{T}}\left\vert f\right\vert +\max_{\partial
_{p}Q_{T}}\left\vert g\right\vert \geq 0$, cio\`{e} $\left\vert u\right\vert
\leq t\max_{\bar{Q}_{T}}\left\vert f\right\vert +\max_{\partial
_{p}Q_{T}}\left\vert g\right\vert $ $\forall $ $t\in \left( 0,T\right)
,\forall $ $\mathbf{x}\in \Omega $, per cui - maggiorando $t$ con $T$ - si
ha la tesi. $%
\blacksquare $

Si noti che tutti questi risultati non necessitano di alcuna regolarit\`{a}
della frontiera parabolica.

\subsection{Problema di Cauchy-Dirichlet sul segmento}

Ora si sceglie $\Omega $: si cerca di risolvere il problema di Dirichlet per
l'equazione del calore omogenea nel caso particolare di $\Omega =\left[ 0,L%
\right] $ segmento e $t>0$ (cio\`{e} diffusione del calore in una sbarra
omogenea). Il problema \`{e} 
\begin{equation*}
\left\{ 
\begin{array}{c}
\frac{\partial u}{\partial t}-D\frac{\partial ^{2}u}{\partial x^{2}}=0\text{
in }\left( 0,L\right) \times \left( 0,+\infty \right) \\ 
u\left( x,0\right) =u_{0}\left( x\right) \text{ se }x\in \left( 0,L\right)
\\ 
u\left( 0,t\right) =u\left( L,t\right) =0\text{ }\forall \text{ }t>0%
\end{array}%
\right.
\end{equation*}

Il significato fisico della terza condizione \`{e} che gli estremi della
sbarra sono termostatati.
Fisicamente, ci si aspetta che i grafici $\left( x,u\left( x,t\right)
\right) $ all'aumentare di $t$ si facciano sempre pi\`{u} "bassi" e si
schiaccino sull'asse $x$.

Per risolvere l'equazione uso la tecnica di separazione delle variabili:
suppongo esistano $X,T:u\left( x,t\right) =X\left( x\right) T\left( t\right) 
$ e cerco la soluzione. Se la trovo nella classe per cui vale l'unicit\`{a},
non serve fare altro.

L'equazione si riscrive quindi come $XT^{\prime }-DX^{\prime \prime }T=0$ in 
$\left( 0,L\right) \times \left( 0,+\infty \right) $, che, dividendo per $%
DXT $, diventa $\frac{T^{\prime }}{TD}=\frac{X^{\prime \prime }}{X}$ in $%
\left( 0,L\right) \times \left( 0,+\infty \right) $. Questo pu\`{o} essere
vero se e solo se entrambi i lati sono costanti: si ottengono quindi due
sottoproblemi 
\begin{equation*}
\left\{ 
\begin{array}{c}
\text{(1) }T^{\prime }\left( t\right) =\lambda DT\left( t\right) \text{ in }%
\left( 0,+\infty \right) \\ 
\text{(2) }X^{\prime \prime }\left( x\right) =\lambda X\left( x\right) \text{
in }\left( 0,L\right) \\ 
X\left( 0\right) =X\left( L\right) =0%
\end{array}%
\right.
\end{equation*}

dove nel secondo si sono gi\`{a} imposte le condizioni agli estremi e $%
\lambda \in 
%TCIMACRO{\U{211d} }%
%BeginExpansion
\mathbb{R}
%EndExpansion
$ \`{e} da determinare.

Cerco una soluzione $X$ non identicamente nulla del secondo problema. Se $%
\lambda >0$, le soluzioni sono della forma $X\left( x\right) =c_{1}e^{\sqrt{%
\lambda }x}+c_{2}e^{-\sqrt{\lambda }x}$: $X\left( 0\right) =X\left( L\right)
=0$ implica $\left\{ 
\begin{array}{c}
c_{1}+c_{2}=0 \\ 
c_{1}e^{\sqrt{\lambda }L}+c_{2}e^{-\sqrt{\lambda }L}=0%
\end{array}%
\right. $, che se $\lambda >0$ ha la sola soluzione $c_{1}=c_{2}=0$, che non 
\`{e} accettabile. Se $\lambda =0$, le soluzioni sono della forma $X\left(
x\right) =c_{1}x+c_{2}$: $X\left( 0\right) =X\left( L\right) =0$ implica $%
c_{1}=c_{2}=0$, che non \`{e} accettabile. Se $\lambda <0$, le soluzioni
sono della forma $X\left( x\right) =c_{1}\cos \left( \sqrt{-\lambda }%
x\right) +c_{2}\sin \left( \sqrt{-\lambda }x\right) $: $X\left( 0\right) =0$
implica $c_{1}=0$, $X\left( L\right) =0$ implica - escludendo la soluzione
nulla - $\sqrt{-\lambda }L=n\pi $, da cui $\lambda =\lambda _{n}=-\frac{%
n^{2}\pi ^{2}}{L^{2}}$, con $n=1,2,...$.

Si \`{e} quindi determinata una successione di soluzioni $X_{n}\left(
t\right) =\sin \frac{n\pi x}{L}$ (per ora tralasciando costanti
moltiplicative). La soluzione del primo problema \`{e} invece - a meno di
costanti - $T\left( t\right) =e^{\frac{-n^{2}\pi ^{2}}{L^{2}}Dt}$. Dunque la
soluzione a variabili separate determinata \`{e} 
\begin{equation*}
u_{n}\left( x,t\right) =e^{\frac{-n^{2}\pi ^{2}}{L^{2}}Dt}\sin \frac{n\pi x}{%
L}\text{, }n=1,2,...
\end{equation*}

Ogni $u_{n}$, cos\`{\i} come ogni combinazione lineare finita delle $u_{n}$ $%
\sum_{k=1}^{n}c_{k}u_{k}$, risolve l'equazione differenziale e soddisfa le
condizioni al bordo; ma limitandosi alle combinazioni finite non c'\`{e}
speranza di soddisfare la condizione iniziale. Si considera allora la serie $%
\sum_{n=1}^{+\infty }c_{n}u_{n}\left( x,t\right) =\sum_{n=1}^{+\infty
}c_{n}e^{\frac{-n^{2}\pi ^{2}}{L^{2}}Dt}\sin \frac{n\pi x}{L}$.

Imponendo la condizione iniziale $u\left( x,0\right) =u_{0}\left( x\right) $
si ottiene $\sum_{n=1}^{+\infty }c_{n}\sin \frac{n\pi x}{L}=u_{0}\left(
x\right) $ $\forall $ $x\in \left( 0,L\right) $. Si noti che questa non pu%
\`{o} essere la serie di Fourier di $u_{0}$ in $\left[ 0,L\right] $: il
sistema trigonometrico di Fourier in $\left[ 0,L\right] $ comprende funzioni
del tipo $\sin n\omega x$ con $\omega =\frac{2\pi }{T}=\frac{2\pi }{L}$, e
inoltre \`{e} una serie di soli seni sebbene a priori $u_{0}$ non sia
dispari. Tuttavia, $\sin \frac{n\pi x}{L}$ \`{e} una funzione del sistema
trigonometrico adattato a $\left[ -L,L\right] $. Allora la cosa naturale 
\`{e} prolungare $u_{0}$ per disparit\`{a} in $\left[ -L,L\right] $, cos%
\`{\i} che $u_{0}^{\ast }\left( x\right) =\left\{ 
\begin{array}{c}
u_{0}\left( x\right) \text{ se }x\in \left[ 0,L\right] \\ 
-u_{0}\left( -x\right) \text{ se }x\in \left[ -L,0\right]%
\end{array}%
\right. $: questa funzione \`{e} dispari, perci\`{o} il suo sviluppo di
Fourier sar\`{a} $u_{0}^{\ast }\left( x\right) =\sum_{n=1}^{+\infty
}b_{n}\sin \frac{n\pi x}{L}$, dove $b_{n}=\frac{1}{L}\int_{-L}^{L}u_{0}^{%
\ast }\left( x\right) \sin \frac{n\pi x}{L}dx=\frac{2}{L}\int_{0}^{L}u_{0}%
\left( x\right) \sin \frac{n\pi x}{L}dx$. Vale quindi $u_{0}\left( x\right)
=\sum_{n=1}^{+\infty }b_{n}\sin \frac{n\pi x}{L}$ in $\left[ 0,L\right] $.
Dunque la candidata soluzione \`{e} 
\begin{equation*}
u\left( x,t\right) =\sum_{n=1}^{+\infty }b_{n}e^{\frac{-n^{2}\pi ^{2}}{L^{2}}%
Dt}\sin \frac{n\pi x}{L}
\end{equation*}

\textbf{Analisi critica} Occorre mostrare che la soluzione candidata risolve
effettivamente l'equazione, ha qualche regolarit\`{a} e assume in un qualche
senso il dato al bordo.

\textbf{Teo (regolarit\`{a} della soluzione)}%
\begin{gather*}
\text{Hp: }u_{0}\in L^{1}\left( \left( 0,L\right) \right) \\
\text{Ts: (i) }\left\{ b_{n}\right\} \text{ \`{e} una successione limitata}
\\
\text{(ii) la serie che assegna }u\text{ converge totalmente in ogni regione
del tipo }\left[ 0,L\right] \times \lbrack \delta ,+\infty )\text{ }\forall 
\text{ }\delta >0\text{; lo } \\
\text{stesso vale per le serie derivate termine a termine rispetto a }x\text{
o }t\text{ un numero qualsiasi di volte} \\
\text{(iii) }u\text{ assegnata dalla serie \`{e} }C^{\infty }\left( \left[
0,L\right] \times \left( 0,+\infty \right) \right) \text{, risolve
l'equazione differenziale e soddisfa le} \\
\text{condizioni agli estremi}
\end{gather*}

\textbf{Dim} (i) $b_{n}=\frac{2}{L}\int_{0}^{L}u_{0}\left( x\right) \sin 
\frac{n\pi x}{L}dx$: $\left\vert b_{n}\right\vert \leq \frac{2}{L}%
\int_{0}^{L}\left\vert u_{0}\left( x\right) \right\vert dx=\frac{2}{L}%
\left\vert \left\vert u_{0}\right\vert \right\vert _{L^{1}\left( 0,L\right)
} $.

(ii) Il termine generale della serie \`{e}, in valore assoluto, $\left\vert
b_{n}e^{\frac{-n^{2}\pi ^{2}}{L^{2}}Dt}\sin \frac{n\pi x}{L}\right\vert $:
se $t\geq \delta $, questo \`{e} maggiorato - dato che l'esponenziale \`{e}
decrescente - da $\left\vert b_{n}\right\vert e^{\frac{-n^{2}\pi ^{2}}{L^{2}}%
D\delta }\leq \frac{2}{L}\left\vert \left\vert u_{0}\right\vert \right\vert
_{L^{1}\left( 0,L\right) }e^{\frac{-n^{2}\pi ^{2}}{L^{2}}D\delta }$, termine
generale di una serie convergente: quindi la serie che assegna $u$ converge
totalmente in ogni regione del tipo $\left[ 0,L\right] \times \lbrack \delta
,+\infty )$. Questo implica che $\forall $ $\delta >0$ $u\in C^{0}\left( %
\left[ 0,L\right] \times \lbrack \delta ,+\infty )\right) $, per cui $u\in
C^{0}\left( \left[ 0,L\right] \times \left( 0,+\infty \right) \right) $.

Ogni derivata rispetto a $x$ del termine generale $b_{n}e^{\frac{-n^{2}\pi
^{2}}{L^{2}}Dt}\sin \frac{n\pi x}{L}$ fa apparire un coefficiente
moltiplicativo $n$, ogni derivata rispetto a $t$ un coefficiente $n^{2}$; ma
la serie di termine generale $\frac{2}{L}\left\vert \left\vert
u_{0}\right\vert \right\vert _{L^{1}\left( 0,L\right) }n^{k}e^{\frac{%
-n^{2}\pi ^{2}}{L^{2}}D\delta }$ comunque converge $\forall $ $k$, quindi la
serie di qualsiasi derivata converge totalmente in $\left[ 0,L\right] \times
\lbrack \delta ,+\infty )$, la $u$ \`{e} derivabile infinite volte e la
derivata si pu\`{o} calcolare derivando la serie termine a termine.

(iii) Per quanto detto sopra $u\in C^{\infty }\left( \left[ 0,L\right]
\times \lbrack \delta ,+\infty )\right) $ $\forall $ $\delta $, per cui $%
u\in C^{\infty }\left( \left[ 0,L\right] \times \left( 0,+\infty \right)
\right) $. Inoltre, poich\'{e} la serie si pu\`{o} derivare termine a
termine, $H\left( \sum_{n=1}^{+\infty }b_{n}e^{\frac{-n^{2}\pi ^{2}}{L^{2}}%
Dt}\sin \frac{n\pi x}{L}\right) =\sum_{n=1}^{+\infty }H\left( b_{n}e^{\frac{%
-n^{2}\pi ^{2}}{L^{2}}Dt}\sin \frac{n\pi x}{L}\right) =0$, perch\'{e} ogni
addendo \`{e} stato trovato proprio risolvendo l'equazione.

Si noti che inoltre $u$ \`{e} continua fino al bordo in $x$, quindi il dato
al bordo \`{e} assunto con continuit\`{a}. $\blacksquare $

Spesso si dice che l'equazione del calore \`{e} regolarizzante: un dato solo
integrabile \`{e} sufficiente per avere una soluzione $C^{\infty }$.

Ha senso chiedersi qual \`{e} il comportamento della soluzione per tempi
lunghi. Sotto le ipotesi del teorema sopra la serie che definisce $u$
converge totalmente in $\left[ 0,L\right] \times \lbrack 1,+\infty )$:
quindi \`{e} lecito lo scambio di limite e serie e $\lim_{t\rightarrow
+\infty }\sum_{n=1}^{+\infty }b_{n}e^{\frac{-n^{2}\pi ^{2}}{L^{2}}Dt}\sin 
\frac{n\pi x}{L}=\sum_{n=1}^{+\infty }\lim_{t\rightarrow +\infty }b_{n}e^{%
\frac{-n^{2}\pi ^{2}}{L^{2}}Dt}\sin \frac{n\pi x}{L}=0$ $\forall $ $x\in %
\left[ 0,L\right] $. La velocit\`{a} di convergenza \`{e} dell'ordine di $%
\frac{D}{L^{2}}$, il che \`{e} coerente con l'intuito fisico: maggiore \`{e}
la conducibilit\`{a} e minore la lunghezza del segmento, meno tempo impiega
la temperatura ad andare a $0$ in tutto il segmento.

Finora non si \`{e} parlato della condizione iniziale: sotto quali ipotesi 
\`{e} assunta con continuit\`{a}? $u$ dev'essere continua in $\left[ 0,L%
\right] \times \lbrack 0,+\infty )$, e per ci\`{o} occorre che $%
\sum_{n=1}^{+\infty }b_{n}e^{\frac{-n^{2}\pi ^{2}}{L^{2}}Dt}\sin \frac{n\pi x%
}{L}$ converga totalmente in $\left[ 0,L\right] \times \lbrack 0,+\infty )$,
quindi fino a $t=0$. In $t=0$ non si pu\`{o} sfruttare l'esponenziale: \`{e}
sufficiente l'ipotesi che $\sum_{n=1}^{+\infty }\left\vert b_{n}\right\vert $
converga. Per questo \`{e} sufficiente che $u_{0}^{\ast }$ sia continua in $%
\left[ -L,L\right] $, che $u_{0}^{\ast }\left( -L\right) =u_{0}^{\ast
}\left( L\right) $ e $u_{0}^{\ast }$ sia regolare a tratti in $\left[ -L,L%
\right] $ (queste ipotesi implicano che la serie di Fourier di $u_{0}^{\ast
} $ converga totalmente). Essendo $u_{0}^{\ast }$ la riflessa dispari di $%
u_{0} $, tali ipotesi sono equivalenti alle seguenti richieste su $u$: $%
u_{0} $ continua in $\left[ 0,L\right] $ e $u_{0}\left( 0\right) =0$, $%
u_{0}\left( L\right) =0$ e $u_{0}$ regolare a tratti in $\left[ 0,L\right] $.

Questo dimostra il seguente

\textbf{Teo (condizione iniziale classica)}%
\begin{gather*}
\text{Hp}\text{: }u_{0}\in C^{0}\left( \left[ 0,L\right] \right)
,u_{0}\left( 0\right) =u_{0}\left( L\right) =0,u_{0}\text{ regolare a tratti
in }\left[ 0,L\right] \\
\text{Ts}\text{: }\sum_{n=1}^{+\infty }b_{n}e^{\frac{-n^{2}\pi ^{2}}{L^{2}}%
Dt}\sin \frac{n\pi x}{L}\text{ converge totalmente in }\left[ 0,L\right]
\times \lbrack 0,+\infty )\text{,} \\
u\in C^{0}\left( [0,L]\times \lbrack 0,+\infty \right) )\text{ e la
condizione iniziale \`{e} assunta in senso classico}
\end{gather*}

La regolarit\`{a} a tratti \`{e} un'ipotesi forte! L'ultima affermazione
della tesi, come al solito, significa che $\lim_{t\rightarrow 0^{+}}u\left(
x,t\right) =u_{0}\left( x\right) $ uniformemente rispetto a $x$.

% In tal caso la soluzione trovata \`{e} effettivamente unica perch\'{e} $u$
% rientra nella classe di funzioni cui si applica il teorema di unicit\`{a}?
% ma il dominio non \`{e} limitato... e nel caso $L^{2}$?

Come gi\`{a} visto per il laplaciano sul cerchio, se invece $u_{0}$ non \`{e}
regolare a tratti ed eventualmente neanche continua, si pu\`{o} ancora dare
senso alla condizione iniziale con la convergenza $L^{2}$.

\textbf{Teo (condizione iniziale in senso }$L^{2}$\textbf{)}%
\begin{eqnarray*}
\text{Hp}\text{: } &&u_{0}\in L^{2}\left( 0,L\right) \\
\text{Ts} &\text{:}&\text{ }\left\vert \left\vert u\left( \cdot ,t\right)
-u_{0}\right\vert \right\vert _{L^{2}\left( 0,L\right) }\rightarrow
^{t\rightarrow 0^{+}}0
\end{eqnarray*}

Il dato iniziale in tal caso \`{e} assunto in senso $L^{2}$. Si noti che con
tale ipotesi vale anche il teorema sopra sulla regolarit\`{a} della
soluzione.

\textbf{Dim} Poich\'{e} $u_{0}\in L^{2}\left( 0,L\right) $, anche la sua
riflessa dispari $u_{0}^{\ast }\in L^{2}\left( -L,L\right) $. Allora $%
\sum_{n=1}^{+\infty }b_{n}^{2}<+\infty $. E' noto che $u_{0}\left( x\right)
=\sum_{n=1}^{+\infty }b_{n}\sin \frac{n\pi x}{L}$, $u\left( x,t\right)
=\sum_{n=1}^{+\infty }b_{n}e^{\frac{-n^{2}\pi ^{2}}{L^{2}}Dt}\sin \frac{n\pi
x}{L}$, quindi ancora in senso $L^{2}$ si ha $%
u_{0}\left( x\right) -u\left( x,t\right) =\sum_{n=1}^{+\infty }b_{n}\sin 
\frac{n\pi x}{L}\left( 1-e^{\frac{-n^{2}\pi ^{2}}{L^{2}}Dt}\right) $. Per il
teorema di Pitagora $\left\vert \left\vert u\left( \cdot ,t\right)
-u_{0}\right\vert \right\vert _{L^{2}\left( 0,L\right)
}^{2}=\sum_{n=1}^{+\infty }b_{n}^{2}\left( 1-e^{\frac{-n^{2}\pi ^{2}}{L^{2}}%
Dt}\right) ^{2}\left\vert \left\vert \sin \frac{n\pi x}{L}\right\vert
\right\vert _{L^{2}\left( 0,L\right) }^{2}=\sum_{n=1}^{+\infty
}b_{n}^{2}\left( 1-e^{\frac{-n^{2}\pi ^{2}}{L^{2}}Dt}\right) ^{2}\frac{L}{2}$
$\forall $ $t>0$. Ponendo $g_{n}\left( t\right) :=b_{n}^{2}\frac{L}{2}\left(
1-e^{\frac{-n^{2}\pi ^{2}}{L^{2}}Dt}\right) ^{2}$, si ha $\left\vert
g_{n}\left( t\right) \right\vert \leq \frac{L}{2}b_{n}^{2}$, che \`{e} il
termine generale di una serie convergente: quindi $\sum_{n=1}^{+\infty
}g_{n}\left( t\right) $ converge totalmente e si possono scambiare limite e
integrale, cio\`{e} $\lim_{t\rightarrow 0^{+}}\left\vert \left\vert u\left(
\cdot ,t\right) -u_{0}\right\vert \right\vert _{L^{2}\left( 0,L\right)
}=\sum_{n=1}^{+\infty }\lim_{t\rightarrow 0^{+}}b_{n}^{2}\left( 1-e^{\frac{%
-n^{2}\pi ^{2}}{L^{2}}Dt}\right) ^{2}\frac{L}{2}=0$. $\blacksquare $

\begin{enumerate}
\item Con tecniche analoghe a quelle usate si pu\`{o} mostrare che il
problema di Cauchy per l'equazione del calore retrograda \`{e} malposto: la
soluzione non dipende con continuit\`{a} dalla condizione iniziale. Ad
esempio, se $\left\{ 
\begin{array}{c}
u_{t}+u_{xx}=0\text{ in }\left( 0,\pi \right) \times \left( 0,T\right) \\ 
u\left( x,0\right) =\frac{\sin nx}{n}\text{ se }x\in \left( 0,\pi \right) \\ 
u\left( 0,t\right) =u\left( \pi ,t\right) =0\text{ }\forall \text{ }t>0%
\end{array}%
\right. $, per separazione di variabili e notando che l'estesa dispari di $%
u_{0}\left( x\right) $ \`{e} gi\`{a} sviluppata in serie di Fourier con $%
b_{n}=\frac{1}{n}$, si ottiene che la soluzione \`{e} $u\left( x,t\right) =%
\frac{1}{n}e^{\frac{n^{2}\pi ^{2}Dt}{L^{2}}}\sin \frac{n\pi x}{L}=\frac{1}{n}%
e^{n^{2}t}\sin nx$. In questo caso non \`{e} vero che il massimo modulo di $%
u $ pu\`{o} essere maggiorato dal massimo modulo di $u_{0}$: infatti $%
\sup_{x\in \left( 0,\pi \right) }\frac{\sin nx}{n}=\frac{1}{n}\rightarrow
^{n\rightarrow +\infty }0$, mentre $\forall $ $T>0$ $\sup_{\substack{ x\in
\left( 0,\pi \right)  \\ t\in \left( 0,T\right) }}\frac{1}{n}e^{\frac{%
n^{2}\pi ^{2}Dt}{L^{2}}}\sin \frac{n\pi x}{L}=\frac{1}{n}e^{n^{2}T}%
\rightarrow +\infty $ per $n\rightarrow +\infty $.
\end{enumerate}

\subsection{Problema di Cauchy-Neumann sul segmento}

Il problema \`{e}%
\begin{equation*}
\left\{ 
\begin{array}{c}
\frac{\partial u}{\partial t}-D\frac{\partial ^{2}u}{\partial x^{2}}=0\text{
in }\left( 0,L\right) \times \left( 0,+\infty \right) \\ 
u\left( x,0\right) =u_{0}\left( x\right) \text{ se }x\in \left( 0,L\right)
\\ 
\frac{\partial u}{\partial x}\left( 0,t\right) =\frac{\partial u}{\partial x}%
\left( L,t\right) =0\text{ }\forall \text{ }t>0%
\end{array}%
\right.
\end{equation*}

Infatti la direzione normale \`{e} proprio lungo $x$. %(normale uscente??
% quindi ci sarebbe un meno a sx?)

Usando di nuovo la separazione delle variabili, si cercano ancora soluzioni
del tipo $u\left( x,t\right) =X\left( x\right) T\left( t\right) $. Si
ottengono i sottoproblemi 
\begin{equation*}
\left\{ 
\begin{array}{c}
\text{(1) }T^{\prime }\left( t\right) =\lambda DT\left( t\right) \text{ in }%
\left( 0,+\infty \right) \\ 
\text{(2) }X^{\prime \prime }\left( x\right) =\lambda X\left( x\right) \text{
in }\left( 0,L\right) \\ 
X^{\prime }\left( 0\right) =X^{\prime }\left( L\right) =0%
\end{array}%
\right.
\end{equation*}

dove nel secondo si \`{e} gi\`{a} imposta la condizione di Neumann, che
stavolta riguarda le derivate, e $\lambda \in 
%TCIMACRO{\U{211d} }%
%BeginExpansion
\mathbb{R}
%EndExpansion
$ \`{e} da determinare.

Cerco una soluzione $X$ non identicamente nulla del secondo problema. Se $%
\lambda >0$, le soluzioni sono della forma $X\left( x\right) =c_{1}e^{\sqrt{%
\lambda }x}+c_{2}e^{-\sqrt{\lambda }x}$: $X^{\prime }\left( 0\right)
=X^{\prime }\left( L\right) =0$ implica $\left\{ 
\begin{array}{c}
c_{1}-c_{2}=0 \\ 
c_{1}\sqrt{\lambda }e^{\sqrt{\lambda }L}-c_{2}\sqrt{\lambda }e^{-\sqrt{%
\lambda }L}=0%
\end{array}%
\right. $, che ha la sola soluzione $c_{1}=c_{2}=0$, che non \`{e}
accettabile. Se $\lambda =0$, le soluzioni sono della forma $X\left(
x\right) =c_{1}x+c_{2}$: $X^{\prime }\left( 0\right) =X^{\prime }\left(
L\right) =0$ implica $c_{1}=0$, quindi si ottiene la soluzione $X\left(
x\right) =c_{2}$. Se $\lambda <0$, le soluzioni sono della forma $X\left(
x\right) =c_{1}\cos \left( \sqrt{-\lambda }x\right) +c_{2}\sin \left( \sqrt{%
-\lambda }x\right) $: $X^{\prime }\left( 0\right) =0$ implica $c_{2}=0$, $%
X^{\prime }\left( L\right) =0$ implica - escludendo la soluzione nulla - $%
\sqrt{-\lambda }L=n\pi $, da cui $\lambda =\lambda _{n}=-\frac{n^{2}\pi ^{2}%
}{L^{2}}$, con $n=1,2,...$.

Si \`{e} quindi determinata una successione di soluzioni $X_{n}\left(
t\right) =\cos \frac{n\pi x}{L}$ (per ora tralasciando costanti
moltiplicative). La soluzione del primo problema \`{e} invece - a meno di
costanti - $T\left( t\right) =e^{\frac{-n^{2}\pi ^{2}}{L^{2}}Dt}$. Dunque la
soluzione a variabili separate determinata \`{e} 
\begin{equation*}
u_{n}\left( x,t\right) =e^{\frac{-n^{2}\pi ^{2}}{L^{2}}Dt}\cos \frac{n\pi x}{%
L}\text{, }n=1,2,...
\end{equation*}

Ogni $u_{n}$, cos\`{\i} come ogni combinazione lineare finita pi\`{u} una
costante $c_{0}+$ $\sum_{k=1}^{n}c_{k}u_{k}$, risolve l'equazione
differenziale e soddisfa le condizioni al bordo. Si considera come al solito
la serie $c_{0}+\sum_{n=1}^{+\infty }c_{n}u_{n}\left( x,t\right)
=c_{0}+\sum_{n=1}^{+\infty }c_{n}e^{\frac{-n^{2}\pi ^{2}}{L^{2}}Dt}\cos 
\frac{n\pi x}{L}$.

Imponendo la condizione iniziale $u\left( x,0\right) =u_{0}\left( x\right) $
si ottiene $c_{0}+\sum_{n=1}^{+\infty }c_{n}\cos \frac{n\pi x}{L}%
=u_{0}\left( x\right) $ $\forall $ $x\in \left( 0,L\right) $. Si noti che
questa non pu\`{o} essere la serie di Fourier di $u_{0}$ in $\left[ 0,L%
\right] $: il sistema trigonometrico di Fourier in $\left[ 0,L\right] $
comprende funzioni del tipo $\cos n\omega x$ con $\omega =\frac{2\pi }{T}=%
\frac{2\pi }{L}$, e inoltre \`{e} una serie di soli coseni sebbene a priori $%
u_{0}$ non sia pari. Tuttavia, $\cos \frac{n\pi x}{L}$ \`{e} una funzione
del sistema trigonometrico adattato a $\left[ -L,L\right] $. Allora la cosa
naturale \`{e} prolungare $u_{0}$ per parit\`{a} in $\left[ -L,L\right] $,
cos\`{\i} che $u_{0}^{\ast }\left( x\right) =\left\{ 
\begin{array}{c}
u_{0}\left( x\right) \text{ se }x\in \left[ 0,L\right] \\ 
u_{0}\left( -x\right) \text{ se }x\in \left[ -L,0\right]%
\end{array}%
\right. $: questa funzione \`{e} pari, perci\`{o} il suo sviluppo di Fourier
sar\`{a} $u_{0}^{\ast }\left( x\right) =\frac{a_{0}}{2}+\sum_{n=1}^{+\infty
}a_{n}\cos \frac{n\pi x}{L}$, dove $a_{n}=\frac{1}{L}\int_{-L}^{L}u_{0}^{%
\ast }\left( x\right) \cos \frac{n\pi x}{L}dx=\frac{2}{L}\int_{0}^{L}u_{0}%
\left( x\right) \cos \frac{n\pi x}{L}dx$. Vale quindi $u_{0}\left( x\right) =%
\frac{a_{0}}{2}+\sum_{n=1}^{+\infty }a_{n}\sin \frac{n\pi x}{L}$ in $\left[
0,L\right] $. Dunque la candidata soluzione \`{e} 
\begin{equation*}
u\left( x,t\right) =\frac{a_{0}}{2}+\sum_{n=1}^{+\infty }a_{n}e^{\frac{%
-n^{2}\pi ^{2}}{L^{2}}Dt}\cos \frac{n\pi x}{L}
\end{equation*}

Si noti che, a differenza del problema di Neumann per il laplaciano, in
questo caso aggiungendo una costante arbitraria non si ottiene un'altra
soluzione: la condizione iniziale non varebbe pi\`{u}.

\textbf{Analisi critica} Occorre mostrare che la soluzione candidata risolve
effettivamente l'equazione, ha qualche regolarit\`{a} e assume in un qualche
senso il dato al bordo.

\textbf{Teo (regolarit\`{a} della soluzione)}%
\begin{gather*}
\text{Hp: }u_{0}\in L^{1}\left( \left( 0,L\right) \right) \\
\text{Ts: (i) }\left\{ a_{n}\right\} \text{ \`{e} una successione limitata}
\\
\text{(ii) la serie che assegna }u\text{ converge totalmente in ogni regione
del tipo }\left[ 0,L\right] \times \lbrack \delta ,+\infty )\text{ }\forall 
\text{ }\delta >0\text{; lo } \\
\text{stesso vale per le serie derivate termine a termine rispetto a }x\text{
o }t\text{ un numero qualsiasi di volte} \\
\text{(iii) }u\text{ assegnata dalla serie \`{e} }C^{\infty }\left( \left[
0,L\right] \times \left( 0,+\infty \right) \right) \text{, risolve
l'equazione differenziale e soddisfa le} \\
\text{condizioni agli estremi}
\end{gather*}

Il teorema \`{e} identico a quello gi\`{a} visto per il problema di
Cauchy-Dirichlet.

\textbf{Dim} E' identica a quella vista per il problema di Cauchy-Dirichlet. 
$\blacksquare $

Ha senso chiedersi qual \`{e} il comportamento della soluzione per tempi
lunghi. Sotto le ipotesi del teorema sopra la serie che definisce $u$
converge totalmente in $\left[ 0,L\right] \times \lbrack 1,+\infty )$:
quindi \`{e} lecito lo scambio di limite e serie e $\lim_{t\rightarrow
+\infty }\left( \frac{a_{0}}{2}+\sum_{n=1}^{+\infty }a_{n}e^{\frac{-n^{2}\pi
^{2}}{L^{2}}Dt}\cos \frac{n\pi x}{L}\right) =\sum_{n=1}^{+\infty
}\lim_{t\rightarrow +\infty }\left( \frac{a_{0}}{2}+a_{n}e^{\frac{-n^{2}\pi
^{2}}{L^{2}}Dt}\cos \frac{n\pi x}{L}\right) =\frac{a_{0}}{2}$ $\forall $ $%
x\in \left[ 0,L\right] $ (cio\`{e} uniformemente rispetto a $x$): $\frac{%
a_{0}}{2}=\frac{1}{L}\int_{0}^{L}u_{0}\left( x\right) dx$ \`{e} la media
integrale della temperatura iniziale. Dunque per tempi lunghi la temperatura 
\`{e} costante.

In che senso e sotto quali ipotesi \`{e} assunta la condizione iniziale?

\textbf{Teo (condizione iniziale in senso }$L^{2}$\textbf{)}%
\begin{eqnarray*}
\text{Hp}\text{: } &&u_{0}\in L^{2}\left( 0,L\right) \\
\text{Ts} &\text{:}&\text{ }\left\vert \left\vert u\left( \cdot ,t\right)
-u_{0}\right\vert \right\vert _{L^{2}\left( 0,L\right) }\rightarrow
^{t\rightarrow 0^{+}}0
\end{eqnarray*}

Il teorema \`{e} identico a quello gi\`{a} visto per il problema di
Cauchy-Dirichlet. Il dato iniziale in tal caso \`{e} assunto in senso $L^{2}$%
. Si noti che con tale ipotesi vale anche il teorema sopra sulla regolarit%
\`{a} della soluzione.

\textbf{Dim} Poich\'{e} $u_{0}\in L^{2}\left( 0,L\right) $, anche la sua
riflessa pari $u_{0}^{\ast }\in L^{2}\left( -L,L\right) $. Allora $%
\sum_{n=1}^{+\infty }a_{n}^{2}<+\infty $. E' noto che $u_{0}\left( x\right)
=\sum_{n=1}^{+\infty }a_{n}\cos \frac{n\pi x}{L}$, $u\left( x,t\right)
=\sum_{n=1}^{+\infty }a_{n}e^{\frac{-n^{2}\pi ^{2}}{L^{2}}Dt}\cos \frac{n\pi
x}{L}$, quindi ancora in senso $L^{2}$ si ha $u_{0}\left( x\right) -u\left(
x,t\right) =\sum_{n=1}^{+\infty }a_{n}\cos \frac{n\pi x}{L}\left( 1-e^{\frac{%
-n^{2}\pi ^{2}}{L^{2}}Dt}\right) $. Per il teorema di Pitagora $\left\vert
\left\vert u\left( \cdot ,t\right) -u_{0}\right\vert \right\vert
_{L^{2}\left( 0,L\right) }^{2}=\sum_{n=1}^{+\infty }a_{n}^{2}\left( 1-e^{%
\frac{-n^{2}\pi ^{2}}{L^{2}}Dt}\right) ^{2}\left\vert \left\vert \cos \frac{%
n\pi x}{L}\right\vert \right\vert _{L^{2}\left( 0,L\right)
}^{2}=\sum_{n=1}^{+\infty }a_{n}^{2}\left( 1-e^{\frac{-n^{2}\pi ^{2}}{L^{2}}%
Dt}\right) ^{2}\frac{L}{2}$ $\forall $ $t>0$. Ponendo $g_{n}\left( t\right)
:=a_{n}^{2}\frac{L}{2}\left( 1-e^{\frac{-n^{2}\pi ^{2}}{L^{2}}Dt}\right)
^{2} $, si ha $\left\vert g_{n}\left( t\right) \right\vert \leq \frac{L}{2}%
a_{n}^{2}$, che \`{e} il termine generale di una serie convergente: quindi $%
\sum_{n=1}^{+\infty }g_{n}\left( t\right) $ converge totalmente e si possono
scambiare limite e integrale, cio\`{e} $\lim_{t\rightarrow 0^{+}}\left\vert
\left\vert u\left( \cdot ,t\right) -u_{0}\right\vert \right\vert
_{L^{2}\left( 0,L\right) }=\sum_{n=1}^{+\infty }\lim_{t\rightarrow
0^{+}}a_{n}^{2}\left( 1-e^{\frac{-n^{2}\pi ^{2}}{L^{2}}Dt}\right) ^{2}\frac{L%
}{2}=0$. $\blacksquare $

Sotto quali ipotesi la condizione iniziale \`{e} invece assunta con continuit%
\`{a}? $u$ dev'essere continua in $\left[ 0,L\right] \times \lbrack
0,+\infty )$, e per ci\`{o} occorre che $\sum_{n=1}^{+\infty }a_{n}e^{\frac{%
-n^{2}\pi ^{2}}{L^{2}}Dt}\cos \frac{n\pi x}{L}$ converga totalmente in $%
\left[ 0,L\right] \times \lbrack 0,+\infty )$, quindi fino a $t=0$. In $t=0$
non si pu\`{o} sfruttare l'esponenziale: \`{e} sufficiente l'ipotesi che $%
\sum_{n=1}^{+\infty }\left\vert a_{n}\right\vert $ converga. Per questo \`{e}
sufficiente che $u_{0}^{\ast }$ sia continua in $\left[ -L,L\right] $, che $%
u_{0}^{\ast }\left( -L\right) =u_{0}^{\ast }\left( L\right) $ e $u_{0}^{\ast
}$ sia regolare a tratti in $\left[ -L,L\right] $ (queste ipotesi implicano
che la serie di Fourier di $u_{0}^{\ast }$ converga totalmente). Essendo $%
u_{0}^{\ast }$ la riflessa pari di $u_{0}$, tali ipotesi sono equivalenti
alle seguenti richieste su $u$: $u_{0}$ continua in $\left[ 0,L\right] $ e $%
u_{0}$ regolare a tratti in $\left[ 0,L\right] $.

Questo dimostra il seguente

\textbf{Teo (condizione iniziale classica)}%
\begin{gather*}
\text{Hp}\text{: }u_{0}\in C^{0}\left( \left[ 0,L\right] \right) \text{, }%
u_{0}\text{ regolare a tratti in }\left[ 0,L\right] \\
\text{Ts}\text{: }\sum_{n=1}^{+\infty }a_{n}e^{\frac{-n^{2}\pi ^{2}}{L^{2}}%
Dt}\cos \frac{n\pi x}{L}\text{ converge totalmente in }\left[ 0,L\right]
\times \lbrack 0,+\infty )\text{,} \\
u\in C^{0}\left( [0,L]\times \lbrack 0,+\infty \right) )\text{ }\text{ e la
condizione iniziale \`{e} assunta in senso classico}
\end{gather*}

La regolarit\`{a} a tratti \`{e} un'ipotesi forte! L'ultima affermazione
della tesi, come al solito, significa che $\lim_{t\rightarrow 0^{+}}u\left(
x,t\right) =u_{0}\left( x\right) $ uniformemente rispetto a $x$.

Si ricorda che per il problema di Cauchy-Neumann non si \`{e} enunciato
alcun teorema di unicit\`{a}, per cui ad ora non \`{e} noto se la soluzione
trovata sia l'unica. Vale il seguente

\textbf{Teo (unicit\`{a} della soluzione del problema di Cauchy-Neumann)}

\begin{gather*}
\text{Hp: }\Omega \subseteq 
%TCIMACRO{\U{211d} }%
%BeginExpansion
\mathbb{R}
%EndExpansion
^{n}\text{ dominio limitato con frontiera regolare\footnote{%
Non specifichiamo che tipo di regolarit\`{a} sia richiesta, ma serve una
frontiera pi\`{u} regolare di $C^{1}$.}, }\left\{ 
\begin{array}{c}
Hu=f\text{ in }Q_{T} \\ 
u\left( \mathbf{x},0\right) =g\left( \mathbf{x}\right) \text{ con }\mathbf{x}%
\in \Omega \\ 
\frac{\partial u}{\partial n}\left( 0,t\right) =\frac{\partial u}{\partial n}%
\left( L,t\right) =h\left( t\right) \text{ }\forall \text{ }t\in \left(
0,T\right)%
\end{array}%
\right. \\
\text{Ts: la soluzione del problema nella classe delle }u\in C^{2,1}\left(
Q_{T}\right) \cap C^{0}\left( \bar{Q}_{T}\right) :u_{x_{i}}\in C^{0}\left( 
\bar{Q}_{T}\right) \text{, se esiste, \`{e} unica}
\end{gather*}

\subsection{Problema di Cauchy globale per l'equazione del calore in $%
%TCIMACRO{\U{211d} }%
%BeginExpansion
\mathbb{R}
%EndExpansion
^{n}$}

\subsubsection{Equazione omogenea}

Il problema\footnote{%
il dominio \`{e} di fatto un semispazio, e questo giustifica la somiglianza
della tecnica usata con quella del problema di dirichlet nel semipiano} \`{e}%
\begin{equation*}
\left\{ 
\begin{array}{c}
\frac{\partial u}{\partial t}-D\Delta u=0\text{ per }\mathbf{x}\in 
%TCIMACRO{\U{211d} }%
%BeginExpansion
\mathbb{R}
%EndExpansion
^{n},t>0 \\ 
u\left( \mathbf{x},0\right) =u_{0}\left( \mathbf{x}\right) \text{ }\forall 
\text{ }\mathbf{x}\in 
%TCIMACRO{\U{211d} }%
%BeginExpansion
\mathbb{R}
%EndExpansion
^{n}%
\end{array}%
\right.
\end{equation*}

Finora non si sono visti risultati di unicit\`{a} per questo tipo di
problema.

\textbf{Teo (unicit\`{a} della soluzione del problema di Cauchy globale)}%
\begin{gather*}
\text{Hp: }\left\{ 
\begin{array}{c}
\frac{\partial u}{\partial t}-D\Delta u=0\text{ per }\mathbf{x}\in 
%TCIMACRO{\U{211d} }%
%BeginExpansion
\mathbb{R}
%EndExpansion
^{n},t>0 \\ 
u\left( \mathbf{x},0\right) =u_{0}\left( \mathbf{x}\right) \text{ }\forall 
\text{ }\mathbf{x}\in 
%TCIMACRO{\U{211d} }%
%BeginExpansion
\mathbb{R}
%EndExpansion
^{n}%
\end{array}%
\right. \text{ \`{e} il problema di Cauchy globale} \\
\text{Ts: nella classe di funzioni }C^{2,1}\left( 
%TCIMACRO{\U{211d} }%
%BeginExpansion
\mathbb{R}
%EndExpansion
^{n}\times \left( 0,+\infty \right) \right) \cap C_{b}^{0}\left( 
%TCIMACRO{\U{211d} }%
%BeginExpansion
\mathbb{R}
%EndExpansion
^{n}\times \lbrack 0,+\infty )\right) \text{ la soluzione, se esiste, \`{e}
unica}
\end{gather*}

Segue dalla validit\`{a} di un principio di massimo per la classe di
funzioni che appare nella tesi, simile a quello visto per il problema di
Cauchy-Dirichlet.

Quando si applica il teorema si ha anche che $u_{0}\in C_{b}^{0}\left( 
%TCIMACRO{\U{211d} }%
%BeginExpansion
\mathbb{R}
%EndExpansion
^{n})\right) $.

Per risolvere il problema, analogamente a quanto fatto per il problema di
Dirichlet nel semipiano, si usa la trasformata di Fourier. Posto $\hat{u}%
\left( \mathbf{\xi },t\right) =\int_{%
%TCIMACRO{\U{211d} }%
%BeginExpansion
\mathbb{R}
%EndExpansion
^{n}}u\left( \mathbf{x},t\right) e^{-2\pi i\left\langle \mathbf{x,\xi }%
\right\rangle }d\mathbf{x}$ e ricordando che $\mathcal{F}\left(
u_{x_{j}x_{j}}\right) =\left( 2\pi i\xi _{j}\right) ^{2}\hat{u}\left( 
\mathbf{\xi },t\right) =-4\pi ^{2}\xi _{j}^{2}\hat{u}\left( \mathbf{\xi }%
,t\right) $, per cui $\mathcal{F}\left( \Delta u\right) =-4\pi
^{2}\left\vert \left\vert \mathbf{\xi }\right\vert \right\vert ^{2}\hat{u}%
\left( \mathbf{\xi },t\right) $. Invece $\mathcal{F}\left( \frac{\partial u}{%
\partial t}\right) =\int_{%
%TCIMACRO{\U{211d} }%
%BeginExpansion
\mathbb{R}
%EndExpansion
^{n}}\frac{\partial }{\partial t}u\left( \mathbf{x},t\right) e^{-2\pi
i\left\langle \mathbf{x,\xi }\right\rangle }d\mathbf{x=}\frac{\partial }{%
\partial t}\hat{u}\left( \mathbf{\xi },t\right) $ (quest'uguaglianza varr%
\`{a} sotto opportune ipotesi).

Allora trasformando il problema (supponendo $u_{0}\in L^{1}\left( 
%TCIMACRO{\U{211d} }%
%BeginExpansion
\mathbb{R}
%EndExpansion
^{n}\right) $) si ottiene%
\begin{equation*}
\left\{ 
\begin{array}{c}
\frac{\partial }{\partial t}\hat{u}\left( \mathbf{\xi },t\right) +4\pi
^{2}D\left\vert \left\vert \mathbf{\xi }\right\vert \right\vert ^{2}\hat{u}%
\left( \mathbf{\xi },t\right) =0\text{ per }\mathbf{\xi }\in 
%TCIMACRO{\U{211d} }%
%BeginExpansion
\mathbb{R}
%EndExpansion
^{n},t>0 \\ 
\hat{u}\left( \mathbf{\xi },0\right) =\hat{u}_{0}\left( \mathbf{\xi }\right)%
\end{array}%
\right.
\end{equation*}

La prima equazione \`{e} un'EDO del prim'ordine nella variabile $t$ con
funzione incognita $\hat{u}\left( \mathbf{\xi },t\right) $; $\mathbf{\xi }$ 
\`{e} un parametro. La soluzione \`{e} $\hat{u}\left( \mathbf{\xi },t\right)
=c\left( \mathbf{\xi }\right) e^{-4\pi ^{2}D\left\vert \left\vert \mathbf{%
\xi }\right\vert \right\vert ^{2}t}$; imponendo la condizione iniziale si
trova $c\left( \mathbf{\xi }\right) =\hat{u}_{0}\left( \mathbf{\xi }\right) $%
. Quindi si \`{e} individuata univocamente la trasformata della soluzione $%
\hat{u}\left( \mathbf{\xi },t\right) =\hat{u}_{0}\left( \mathbf{\xi }\right)
e^{-4\pi ^{2}D\left\vert \left\vert \mathbf{\xi }\right\vert \right\vert
^{2}t}$. A questo punto basta antitrasformare $e^{-4\pi ^{2}D\left\vert
\left\vert \mathbf{\xi }\right\vert \right\vert ^{2}t}$.

[Diamo per noto che $\mathcal{F}\left( e^{-\pi \left\vert \left\vert \mathbf{%
x}\right\vert \right\vert ^{2}}\right) =e^{-\pi \left\vert \left\vert 
\mathbf{\xi }\right\vert \right\vert ^{2}}$; poniamo $f^{a}\left( \mathbf{x}%
\right) :=f\left( a\mathbf{x}\right) ,f_{a}\left( \mathbf{x}\right) =\frac{1%
}{a^{n}}f\left( \frac{\mathbf{x}}{a}\right) $, e ricordiamo che $\mathcal{F}%
\left( f^{a}\left( \mathbf{x}\right) \right) =\left( \hat{f}\right) _{a},%
\mathcal{F}\left( f_{a}\left( \mathbf{x}\right) \right) =\left( \hat{f}%
\right) ^{a}$. Allora si ha $\mathcal{F}\left( e^{-\pi \left\vert \left\vert
a\mathbf{x}\right\vert \right\vert ^{2}}\right) =\frac{1}{a^{n}}e^{-\pi 
\frac{\left\vert \left\vert \mathbf{\xi }\right\vert \right\vert ^{2}}{a^{2}}%
}$, e ponendo $A:=a^{2}\pi $ si ottiene $\mathcal{F}\left( e^{-A\left\vert
\left\vert \mathbf{x}\right\vert \right\vert ^{2}}\right) =\left( \frac{\pi 
}{A}\right) ^{\frac{n}{2}}e^{-\pi ^{2}\frac{\left\vert \left\vert \mathbf{%
\xi }\right\vert \right\vert ^{2}}{A}}$. Dal teorema di inversione si ha
infine $\mathcal{F}\left( \left( \frac{\pi }{A}\right) ^{\frac{n}{2}}e^{-\pi
^{2}\frac{\left\vert \left\vert \mathbf{\xi }\right\vert \right\vert ^{2}}{A}%
}\right) =e^{-A\left\vert \left\vert \mathbf{x}\right\vert \right\vert ^{2}}$%
.]

Ponendo $A=4\pi ^{2}Dt$ si ha che $\mathcal{F}_{\mathbf{x}}\left( \left( 
\frac{1}{4\pi Dt}\right) ^{\frac{n}{2}}e^{-\frac{\left\vert \left\vert 
\mathbf{x}\right\vert \right\vert ^{2}}{4Dt}}\right) =e^{-4\pi
^{2}D\left\vert \left\vert \mathbf{\xi }\right\vert \right\vert ^{2}t}$.
Definiamo $k\left( \mathbf{x},t\right) :=\left( \frac{1}{4\pi Dt}\right) ^{%
\frac{n}{2}}e^{-\frac{\left\vert \left\vert \mathbf{x}\right\vert
\right\vert ^{2}}{4Dt}}$: questa funzione prende il nome di nucleo del
calore.

Allora la candidata soluzione \`{e}%
\begin{equation*}
u\left( \mathbf{x},t\right) =k\left( \mathbf{x},t\right) \ast u_{0}\left( 
\mathbf{x}\right) =\int_{%
%TCIMACRO{\U{211d} }%
%BeginExpansion
\mathbb{R}
%EndExpansion
^{n}}k\left( \mathbf{x-y},t\right) u_{0}\left( \mathbf{y}\right) d\mathbf{y}%
=\int_{%
%TCIMACRO{\U{211d} }%
%BeginExpansion
\mathbb{R}
%EndExpansion
^{n}}\left( \frac{1}{4\pi Dt}\right) ^{\frac{n}{2}}e^{-\frac{\left\vert
\left\vert \mathbf{x-y}\right\vert \right\vert ^{2}}{4Dt}}u_{0}\left( 
\mathbf{y}\right) d\mathbf{y}
\end{equation*}

\textbf{Analisi critica} Sotto quali ipotesi la candidata soluzione risolve
effettivamente il problema?

\textbf{Teo (regolarit\`{a} della soluzione)}%
\begin{gather*}
\text{Hp: }u_{0}\in L^{1}\left( 
%TCIMACRO{\U{211d} }%
%BeginExpansion
\mathbb{R}
%EndExpansion
^{n}\right) \\
\text{Ts: (i) }u\text{ assegnata dalla formula integrale \`{e} }C^{\infty
}\left( 
%TCIMACRO{\U{211d} }%
%BeginExpansion
\mathbb{R}
%EndExpansion
^{n}\times \left( 0,+\infty \right) \right) \\
\text{(ii) }u\text{ assegnata dalla formula integrale pu\`{o} essere
derivata scambiando derivata e integrale }\forall \text{ }t>0 \\
\text{(iii) }u\text{ assegnata dalla formula integrale risolve l'equazione }%
\forall \text{ }t>0
\end{gather*}

\textbf{Dim} (i) (ii) In ogni regione del tipo $%
%TCIMACRO{\U{211d} }%
%BeginExpansion
\mathbb{R}
%EndExpansion
^{n}\times \lbrack \delta ,+\infty )$ la funzione $k\left( \mathbf{x}%
,t\right) $ \`{e} limitata e ha derivate di ogni ordine globalmente limitate
(dato che n\'{e} il denominatore del primo fattore n\'{e} l'esponente del
secondo fattore si annullano), per cui, data $u\left( \mathbf{x},t\right)
=\int_{%
%TCIMACRO{\U{211d} }%
%BeginExpansion
\mathbb{R}
%EndExpansion
^{n}}\left( \frac{1}{4\pi Dt}\right) ^{\frac{n}{2}}e^{-\frac{\left\vert
\left\vert \mathbf{x-y}\right\vert \right\vert ^{2}}{4Dt}}u_{0}\left( 
\mathbf{y}\right) d\mathbf{y}$, la funzione integranda pu\`{o} essere
maggiorata $\forall $ $\mathbf{x}\in 
%TCIMACRO{\U{211d} }%
%BeginExpansion
\mathbb{R}
%EndExpansion
^{n},\forall $ $t\geq \delta >0$ da $\left( \frac{1}{4\pi D\delta }\right) ^{%
\frac{n}{2}}u_{0}\left( \mathbf{y}\right) $, che \`{e} integrabile, dal che
si conclude che $u\in C^{0}\left( 
%TCIMACRO{\U{211d} }%
%BeginExpansion
\mathbb{R}
%EndExpansion
^{n}\times \left( 0,+\infty \right) \right) $; analogamente ogni derivata
rispetto a $x_{i}$, oppure rispetto a $t$, pu\`{o} essere maggiorata da una
funzione di $\mathbf{y}$ integrabile, per cui si conclude che $u\in
C^{\infty }\left( 
%TCIMACRO{\U{211d} }%
%BeginExpansion
\mathbb{R}
%EndExpansion
^{n}\times \left( 0,+\infty \right) \right) $ e si possono scambiare
derivata (di ordine qualsiasi) e integrale.

(iii) Occorre mostrare che $Hu=0$, cio\`{e} - dato che si possono scambiare
l'operatore differenziale $H$ e l'integrale - che $H\left( \left( \frac{1}{%
4\pi Dt}\right) ^{\frac{n}{2}}e^{-\frac{\left\vert \left\vert \mathbf{x-y}%
\right\vert \right\vert ^{2}}{4Dt}}\right) =0$: equivalentemente, che $%
H\left( \frac{1}{t^{n/2}}e^{-\frac{\left\vert \left\vert \mathbf{x}%
\right\vert \right\vert ^{2}}{4Dt}}\right) =0$. Poich\'{e} la funzione
argomento \`{e} radiale, si ha $\left( \frac{\partial }{\partial t}-D\Delta
\right) \left( \frac{1}{t^{n/2}}e^{-\frac{\left\vert \left\vert \mathbf{x}%
\right\vert \right\vert ^{2}}{4Dt}}\right) =\left( \frac{\partial }{\partial
t}-D\left( \frac{\partial ^{2}}{\partial \rho ^{2}}+\frac{n-1}{\rho }\frac{%
\partial }{\partial \rho }\right) \right) \left( \frac{1}{t^{n/2}}e^{-\frac{%
\rho ^{2}}{4Dt}}\right) $, e si ottiene la funzione nulla per $\left( \rho
,t\right) \neq \left( 0,0\right) $. $\blacksquare $

Dalla dimostrazione si vede quindi che anche il nucleo del calore risolve
l'equazione del calore.

Occorre ora chiedersi in che senso \`{e} assunta la condizione iniziale.
Osservando il grafico di $k\left( \mathbf{x},t\right) =\left( \frac{1}{4\pi
Dt}\right) ^{\frac{n}{2}}e^{-\frac{\left\vert \left\vert \mathbf{x}%
\right\vert \right\vert ^{2}}{4Dt}}$ al variare di $t$, si osserva che al
diminuire di $t$ $k$ tende a una delta. Questo suggerisce che sia un nucleo
regolarizzante, cio\`{e} del tipo $\phi _{\varepsilon }\left( \mathbf{x}%
\right) :=\frac{1}{\varepsilon ^{n}}\phi \left( \frac{\mathbf{x}}{%
\varepsilon }\right) $ dove $\phi $ \`{e} una funzione madre da determinare:
nel nostro caso $\varepsilon $ sembra essere $\sqrt{t}$ e $\phi _{\sqrt{t}%
}\left( \mathbf{x}\right) =\left( \frac{1}{\sqrt{t}}\right) ^{n}\frac{1}{%
\left( 4\pi D\right) ^{n/2}}e^{-\frac{\left\vert \left\vert \mathbf{x/}\sqrt{%
t}\right\vert \right\vert ^{2}}{4D}}$, per cui si avrebbe la funzione madre $%
\phi \left( \mathbf{x}\right) =k\left( \mathbf{x}\right) :=k\left( \mathbf{x}%
,1\right) =\frac{1}{\left( 4\pi D\right) ^{n/2}}e^{-\frac{\left\vert
\left\vert \mathbf{x}\right\vert \right\vert ^{2}}{4D}}$. Effettivamente $%
\phi $ \`{e} integrabile e positiva. Si verifica che $\int_{%
%TCIMACRO{\U{211d} }%
%BeginExpansion
\mathbb{R}
%EndExpansion
^{n}}k\left( \mathbf{x}\right) d\mathbf{x}=1$: \`{e} noto che $\mathcal{F}_{%
\mathbf{x}}\left( \left( \frac{1}{4\pi D}\right) ^{\frac{n}{2}}e^{-\frac{%
\left\vert \left\vert \mathbf{x}\right\vert \right\vert ^{2}}{4D}}\right)
=e^{-4\pi ^{2}D\left\vert \left\vert \mathbf{\xi }\right\vert \right\vert
^{2}}$, per cui $\int_{%
%TCIMACRO{\U{211d} }%
%BeginExpansion
\mathbb{R}
%EndExpansion
^{n}}k\left( \mathbf{x}\right) d\mathbf{x}=\hat{k}\left( 0\right) =\mathcal{F%
}_{\mathbf{x}}\left( \left( \frac{1}{4\pi D}\right) ^{\frac{n}{2}}e^{-\frac{%
\left\vert \left\vert \mathbf{x}\right\vert \right\vert ^{2}}{4D}}\right)
\left( 0\right) =1$. Dunque $\left\{ k\left( \mathbf{x},t\right) \right\}
_{t>0}$ \`{e} una famiglia di nuclei regolarizzanti, che d'ora in poi si
indicheranno con $k_{\sqrt{t}}\left( \mathbf{x}\right) $ oppure $\Gamma
_{D}\left( \mathbf{x},t\right) $ (il significato di quest'ultima notazione
sar\`{a} chiarito in seguito).

Poich\'{e} dunque $u\left( \mathbf{x},t\right) =k_{\sqrt{t}}\left( \mathbf{x}%
\right) \ast u_{0}\left( \mathbf{x}\right) $, per sapere in che senso \`{e}
assunta la condizione iniziale si possono usare i teoremi gi\`{a} visti: se $%
u_{0}\in L^{p}\left( 
%TCIMACRO{\U{211d} }%
%BeginExpansion
\mathbb{R}
%EndExpansion
^{n}\right) $, $k_{\sqrt{t}}\ast u_{0}\rightarrow ^{L^{p}}u_{0}$ per $%
t\rightarrow 0^{+}$ (se $p\in \lbrack 1,+\infty )$); se inoltre $u_{0}\in
L^{1}\cap C_{\ast }^{0}$, $k_{\sqrt{t}}\ast u_{0}\rightarrow u_{0}$ per $%
t\rightarrow 0^{+}$ anche uniformemente (per cui il dato al bordo \`{e}
assunto in senso classico). In realt\`{a} si pu\`{o} mostrare che per tale
convergenza uniforme \`{e} sufficiente che $u$ sia uniformemente continua.

Si noti che affinch\'{e} $k_{\sqrt{t}}\ast u_{0}$ sia ben definita \`{e}
sufficiente che $u_{0}\in C_{b}^{0}\left( 
%TCIMACRO{\U{211d} }%
%BeginExpansion
\mathbb{R}
%EndExpansion
^{n}\right) $, non
occorre $u_{0}$ integrabile, e ci\`{o} \`{e} sufficiente per mostrare che la
condizione iniziale \`{e} assunta in senso classico. Il teorema di regolarit%
\`{a} della soluzione ha una dimostrazione che si complica notevolmente se
l'ipotesi \`{e} solo $u_{0}\in C_{b}^{0}\left( 
%TCIMACRO{\U{211d} }%
%BeginExpansion
\mathbb{R}
%EndExpansion
^{n}\right) $ e non $u_{0}\in L^{1}\left( 
%TCIMACRO{\U{211d} }%
%BeginExpansion
\mathbb{R}
%EndExpansion
^{n}\right) $.

Queste osservazioni sono riassunte nel seguente

\textbf{Teo (condizione iniziale)}%
\begin{gather*}
\text{(1) Hp: }u_{0}\in L^{1}\left( 
%TCIMACRO{\U{211d} }%
%BeginExpansion
\mathbb{R}
%EndExpansion
^{n}\right) \text{, }u\left( \mathbf{x},t\right) =k_{\sqrt{t}}\left( \mathbf{%
x}\right) \ast u_{0}\left( \mathbf{x}\right) \\
\text{Ts: (i) }u\in C^{\infty }\left( 
%TCIMACRO{\U{211d} }%
%BeginExpansion
\mathbb{R}
%EndExpansion
^{n}\times \left( 0,+\infty \right) \right) \text{ e risolve l'equazione per 
}t>0 \\
\text{(ii) }\left\vert \left\vert u\left( \cdot ,t\right) -u_{0}\right\vert
\right\vert _{L^{1}\left( 
%TCIMACRO{\U{211d} }%
%BeginExpansion
\mathbb{R}
%EndExpansion
^{n}\right) }\rightarrow 0\text{ per }t\rightarrow 0^{+} \\
\text{(2) Hp: }u_{0}\in L^{1}\left( 
%TCIMACRO{\U{211d} }%
%BeginExpansion
\mathbb{R}
%EndExpansion
^{n}\right) \cap C_{\ast }^{0}\left( 
%TCIMACRO{\U{211d} }%
%BeginExpansion
\mathbb{R}
%EndExpansion
^{n}\right) \text{, }u\left( \mathbf{x},t\right) =k_{\sqrt{t}}\left( \mathbf{%
x}\right) \ast u_{0}\left( \mathbf{x}\right) \\
\text{Ts: }u\left( \cdot ,t\right) \rightarrow u_{0}\text{ uniformemente per 
}t\rightarrow 0^{+}
\end{gather*}

Come si comporta la soluzione per tempi lunghi? Se $u_{0}\in L^{1}\left( 
%TCIMACRO{\U{211d} }%
%BeginExpansion
\mathbb{R}
%EndExpansion
^{n}\right) $, poich\'{e} $u\left( \mathbf{x},t\right) =\frac{1}{\left( 4\pi
Dt\right) ^{n/2}}\int_{%
%TCIMACRO{\U{211d} }%
%BeginExpansion
\mathbb{R}
%EndExpansion
^{n}}e^{-\frac{\left\vert \left\vert \mathbf{x-y}\right\vert \right\vert ^{2}%
}{4Dt}}u_{0}\left( \mathbf{y}\right) d\mathbf{y}$, $\left\vert u\left( 
\mathbf{x},t\right) \right\vert \leq \frac{1}{\left( 4\pi Dt\right) ^{n/2}}%
\left\vert \left\vert u_{0}\right\vert \right\vert _{L^{1}\left( 
%TCIMACRO{\U{211d} }%
%BeginExpansion
\mathbb{R}
%EndExpansion
^{n}\right) }$ $\forall $ $\mathbf{x},t$, e il lato destro tende a $0$ per $%
t\rightarrow +\infty $: $\lim_{t\rightarrow +\infty }\sup_{\mathbf{x}\in 
%TCIMACRO{\U{211d} }%
%BeginExpansion
\mathbb{R}
%EndExpansion
^{n}}\left\vert u\left( \mathbf{x},t\right) \right\vert =0$, cio\`{e} $u$
tende a $0$ uniformemente per $t\rightarrow +\infty $. In questo caso $%
u_{0}\in L^{1}\left( 
%TCIMACRO{\U{211d} }%
%BeginExpansion
\mathbb{R}
%EndExpansion
^{n}\right) $ non \`{e} pi\`{u} un'ipotesi di comodo, come \`{e} stata sopra
per dimostrare che $u$ era soluzione del problema, ecc.: se si chiede solo $%
u_{0}\in C_{b}^{0}\left( 
%TCIMACRO{\U{211d} }%
%BeginExpansion
\mathbb{R}
%EndExpansion
^{n}\right) $, per quanto $u$ sia ancora soluzione del problema, non \`{e} pi%
\`{u} vero che $u$ tende a $0$ per $t\rightarrow +\infty $.

\begin{enumerate}
\item Se $u_{0}\left( \mathbf{x}\right) =1$, $u\left( \mathbf{x},t\right) =1$
risolve il problema (ed \`{e} l'unica soluzione, in quanto appartenente alla
classe per cui vale il teorema di unicit\`{a}), \`{e} continua e limitata ma
non integrabile, e infatti non tende a $0$ per $t\rightarrow +\infty $.
\end{enumerate}

\subsubsection{Equazione del calore non omogenea in $%
%TCIMACRO{\U{211d} }%
%BeginExpansion
\mathbb{R}
%EndExpansion
^{n}$}

Il problema \`{e}%
\begin{equation*}
\left\{ 
\begin{array}{c}
\frac{\partial u}{\partial t}-D\Delta u=f\text{ con }\mathbf{x}\in 
%TCIMACRO{\U{211d} }%
%BeginExpansion
\mathbb{R}
%EndExpansion
^{n},t>0 \\ 
u\left( \mathbf{x},0\right) =u_{0}\left( \mathbf{x}\right) \text{ }\forall 
\text{ }\mathbf{x}\in 
%TCIMACRO{\U{211d} }%
%BeginExpansion
\mathbb{R}
%EndExpansion
^{n}%
\end{array}%
\right.
\end{equation*}

E' quindi presente il termine di sorgente $f=f\left( \mathbf{x},t\right) $:
poich\'{e} si fornisce (o toglie) calore dall'esterno, in generale non c'%
\`{e} motivo per cui la temperatura debba tendere a zero per tempi lunghi.

Essendo l'equazione lineare, si applica il principio di sovrapposizione
degli effetti: la soluzione del problema \`{e} somma delle funzioni $%
u_{1},u_{2}$, dove $u_{1}$ e $u_{2}$ risolvono ciascuna un sottoproblema:%
\begin{equation*}
\text{(1) }\left\{ 
\begin{array}{c}
Hu_{1}=f\text{ con }\mathbf{x}\in 
%TCIMACRO{\U{211d} }%
%BeginExpansion
\mathbb{R}
%EndExpansion
^{n},t>0 \\ 
u_{1}\left( \mathbf{x},0\right) =0\text{ }\forall \text{ }\mathbf{x}\in 
%TCIMACRO{\U{211d} }%
%BeginExpansion
\mathbb{R}
%EndExpansion
^{n}%
\end{array}%
\right. \text{; (2) }\left\{ 
\begin{array}{c}
Hu_{2}=0\text{ con }\mathbf{x}\in 
%TCIMACRO{\U{211d} }%
%BeginExpansion
\mathbb{R}
%EndExpansion
^{n},t>0 \\ 
u_{2}\left( \mathbf{x},0\right) =u_{0}\left( \mathbf{x}\right) \text{ }%
\forall \text{ }\mathbf{x}\in 
%TCIMACRO{\U{211d} }%
%BeginExpansion
\mathbb{R}
%EndExpansion
^{n}%
\end{array}%
\right.
\end{equation*}

In generale, il principio di sovrapposizione permette di dividere il
problema in vari sottoproblemi, in ciascuno dei quali tutti i dati sono
nulli eccetto uno. Si \`{e} gi\`{a} studiato come risolvere (2), quindi
affrontiamo (1).

Per risolvere (1) $\left\{ 
\begin{array}{c}
Hu=f\text{ con }\mathbf{x}\in 
%TCIMACRO{\U{211d} }%
%BeginExpansion
\mathbb{R}
%EndExpansion
^{n},t>0 \\ 
u\left( \mathbf{x},0\right) =0\text{ }\forall \text{ }\mathbf{x}\in 
%TCIMACRO{\U{211d} }%
%BeginExpansion
\mathbb{R}
%EndExpansion
^{n}%
\end{array}%
\right. $ usiamo due metodi indipendenti.

\textbf{Metodo della trasformata} Procede proprio come si \`{e} visto per
l'equazione omogenea. Posto $\hat{u}\left( \mathbf{\xi },t\right) =\int_{%
%TCIMACRO{\U{211d} }%
%BeginExpansion
\mathbb{R}
%EndExpansion
^{n}}u\left( \mathbf{x},t\right) e^{-2\pi i\left\langle \mathbf{x,\xi }%
\right\rangle }d\mathbf{x}$ e ricordando che $\mathcal{F}\left( \Delta
u\right) =-4\pi ^{2}\left\vert \left\vert \mathbf{\xi }\right\vert
\right\vert ^{2}\hat{u}\left( \mathbf{\xi },t\right) $, mentre $\mathcal{F}%
\left( \frac{\partial u}{\partial t}\right) =\int_{%
%TCIMACRO{\U{211d} }%
%BeginExpansion
\mathbb{R}
%EndExpansion
^{n}}\frac{\partial }{\partial t}u\left( \mathbf{x},t\right) e^{-2\pi
i\left\langle \mathbf{x,\xi }\right\rangle }d\mathbf{x=}\frac{\partial }{%
\partial t}\hat{u}\left( \mathbf{\xi },t\right) $ (quest'uguaglianza varr%
\`{a} sotto opportune ipotesi), $\hat{f}\left( \mathbf{\xi },t\right) =\int_{%
%TCIMACRO{\U{211d} }%
%BeginExpansion
\mathbb{R}
%EndExpansion
^{n}}f\left( \mathbf{x},t\right) e^{-2\pi i\left\langle \mathbf{x,\xi }%
\right\rangle }d\mathbf{x}$ (se $f\in L^{1}$), trasformando il problema
(supponendo $u_{0}\in L^{1}\left( 
%TCIMACRO{\U{211d} }%
%BeginExpansion
\mathbb{R}
%EndExpansion
^{n}\right) $) si ottiene%
\begin{equation*}
\left\{ 
\begin{array}{c}
\frac{\partial }{\partial t}\hat{u}\left( \mathbf{\xi },t\right) +4\pi
^{2}D\left\vert \left\vert \mathbf{\xi }\right\vert \right\vert ^{2}\hat{u}%
\left( \mathbf{\xi },t\right) =\hat{f}\left( \xi ,t\right) \text{ per }%
\mathbf{\xi }\in 
%TCIMACRO{\U{211d} }%
%BeginExpansion
\mathbb{R}
%EndExpansion
^{n},t>0 \\ 
\hat{u}\left( \mathbf{\xi },0\right) =0\text{ per }\mathbf{\xi }\in 
%TCIMACRO{\U{211d} }%
%BeginExpansion
\mathbb{R}
%EndExpansion
^{n}%
\end{array}%
\right.
\end{equation*}

L'equazione \`{e} un'EDO lineare del prim'ordine non omogenea nell'incognita 
$\hat{u}\left( \mathbf{\xi },t\right) $ e nella variabile $t$. La soluzione 
\`{e} $\hat{u}\left( \mathbf{\xi },t\right) =e^{-4\pi ^{2}D\left\vert
\left\vert \mathbf{\xi }\right\vert \right\vert ^{2}t}\left( c\left( \mathbf{%
\xi }\right) +\int_{0}^{t}e^{4\pi ^{2}D\left\vert \left\vert \mathbf{\xi }%
\right\vert \right\vert ^{2}s}\hat{f}\left( \xi ,s\right) ds\right) $;
imponendo la condizione iniziale si trova $c\left( \mathbf{\xi }\right) =0$.
La soluzione quindi si pu\`{o} scrivere come $\hat{u}\left( \mathbf{\xi }%
,t\right) =\int_{0}^{t}e^{-4\pi ^{2}D\left\vert \left\vert \mathbf{\xi }%
\right\vert \right\vert ^{2}\left( t-s\right) }\hat{f}\left( \mathbf{\xi }%
,s\right) ds$, che pu\`{o} essere vista come una specie di convoluzione
nella variabile $t$: il primo fattore della funzione integranda \`{e}
proprio la trasformata del nucleo del calore valutata in $t-s$. Quindi la
funzione integranda pu\`{o} essere riscritta come $\mathcal{F}\left( \Gamma
_{D}\left( \cdot ,t-s\right) \ast f\left( \cdot ,s\right) \right) \left( 
\mathbf{\xi }\right) $. Scambiando trasformata e integrale si ottiene $\hat{u%
}\left( \mathbf{\xi },t\right) =\mathcal{F}\left( \int_{0}^{t}\Gamma
_{D}\left( \cdot ,t-s\right) \ast f\left( \cdot ,s\right) ds\right) $.
Antitrasformando ed esplicitando anche la seconda convoluzione si ottiene
infine la candidata soluzione di (1)%
\begin{equation*}
u_{1}\left( \mathbf{x},t\right) =\int_{0}^{t}\left( \int_{%
%TCIMACRO{\U{211d} }%
%BeginExpansion
\mathbb{R}
%EndExpansion
^{n}}\Gamma _{D}\left( \mathbf{x-y},t-s\right) f\left( \mathbf{y},s\right) d%
\mathbf{y}\right) ds
\end{equation*}

Questa formula assomiglia a quella gi\`{a} vista per (2): invece di avere
solo una convoluzione in spazio del nucleo del calore con la condizione
iniziale, si ha una convoluzione in spazio e in tempo (detta convoluzione
finita, cf. trasformata di Laplace).

Usando la soluzione gi\`{a} nota di (2) si ottiene quindi la candidata
soluzione del problema complessivo%
\begin{equation*}
u\left( \mathbf{x},t\right) =\int_{0}^{t}\left( \int_{%
%TCIMACRO{\U{211d} }%
%BeginExpansion
\mathbb{R}
%EndExpansion
^{n}}\Gamma _{D}\left( \mathbf{x-y},t-s\right) f\left( \mathbf{y},s\right) d%
\mathbf{y}\right) ds+\int_{%
%TCIMACRO{\U{211d} }%
%BeginExpansion
\mathbb{R}
%EndExpansion
^{n}}\Gamma _{D}\left( \mathbf{x-y},t\right) u_{0}\left( \mathbf{y}\right) d%
\mathbf{y}
\end{equation*}

Mentre nel secondo addendo, come gi\`{a} visto, \`{e} facile derivare sotto
il segno di integrale per $t>0$ (perch\'{e} se $t$ non si annulla si
ottengono derivate globalmente limitate), nel primo addendo la cosa \`{e}
molto pi\`{u} delicata, perch\'{e} si integra con $s$ che varia da $0$ fino
a $t$, e quindi $\Gamma _{D}$ $\forall $ $t$ viene calcolata anche per tempi
molto piccoli ed \`{e} illimitata. Per ora non facciamo l'analisi critica
della soluzione trovata e studiamo invece il secondo metodo.

\textbf{Metodo di Duhamel }Questo metodo si usa per le equazioni di
evoluzione a coefficienti costanti e serve per costruire soluzioni di
equazioni non omogenee, con dato iniziale nullo, quando si sa risolvere il
problema di Cauchy per le equazioni omogenee.

Supponiamo che la funzione $w\left( \mathbf{x},t,s\right) $, dove $s$ \`{e}
un parametro, sia soluzione del problema di Cauchy 
\begin{equation*}
\left\{ 
\begin{array}{c}
\frac{\partial w}{\partial t}-D\Delta w=0\text{ per }\mathbf{x}\in 
%TCIMACRO{\U{211d} }%
%BeginExpansion
\mathbb{R}
%EndExpansion
^{n},t>s \\ 
w\left( \mathbf{x},s,s\right) =f\left( \mathbf{x},s\right)%
\end{array}%
\right.  
\end{equation*}
Allora $u\left( \mathbf{x},t\right) =\int_{0}^{t}w\left( \mathbf{x%
},t,s\right) ds$ risolve il problema iniziale $\left\{ 
\begin{array}{c}
\frac{\partial u}{\partial t}-D\Delta u=f\text{ per }\mathbf{x}\in 
%TCIMACRO{\U{211d} }%
%BeginExpansion
\mathbb{R}
%EndExpansion
^{n},t>0 \\ 
u\left( \mathbf{x},0\right) =0%
\end{array}%
\right. $. Verifichiamo, solo formalmente, che $u$ cos\`{\i} definita
risolve il problema.

Se $u\left( \mathbf{x},t\right) =\int_{0}^{t}w\left( \mathbf{x},t,s\right)
ds $, $\Delta u\left( \mathbf{x},t\right) =\int_{0}^{t}\Delta w\left( 
\mathbf{x},t,s\right) ds$ (supponendo di poter scambiare laplaciano e
integrale sotto opportune ipotesi), mentre $\frac{\partial u}{\partial t}%
\left( \mathbf{x},t\right) =w\left( \mathbf{x},t,t\right) +\int_{0}^{t}\frac{%
\partial }{\partial t}w\left( \mathbf{x},t,s\right) ds$ (approf): quindi $%
\frac{\partial u}{\partial t}-D\Delta u=w\left( \mathbf{x},t,t\right)
+\int_{0}^{t}\frac{\partial }{\partial t}w\left( \mathbf{x},t,s\right)
ds-D\int_{0}^{t}\Delta w\left( \mathbf{x},t,s\right) ds$, e la somma di
secondo e terzo addendo \`{e} nulla perch\'{e} l'integranda \`{e} nulla, per
le ipotesi su $w$. Allora $\frac{\partial u}{\partial t}-D\Delta u=w\left( 
\mathbf{x},t,t\right) =f\left( \mathbf{x},t\right) $, e inoltre $u\left( 
\mathbf{x},0\right) =0$ dato che $u$ \`{e} definita come integrale.

Nel nostro caso particolare, chi \`{e} $w\left( \mathbf{x},t,s\right) $? E'
sufficiente applicare la formula gi\`{a} vista per la risoluzione del
problema di Cauchy per l'equazione omogenea, con un'unica accortezza:
l'istante iniziale \`{e} $s$. Ma poich\'{e} l'equazione del calore \`{e}
invariante per traslazioni temporali, la soluzione andr\`{a} valutata in $%
t-s $ (istante che si deve annullare nell'istante iniziale $t=s$). Dunque $%
w\left( \mathbf{x},t,s\right) =\int_{%
%TCIMACRO{\U{211d} }%
%BeginExpansion
\mathbb{R}
%EndExpansion
^{n}}\Gamma _{D}\left( \mathbf{x-y},t-s\right) f\left( \mathbf{y},s\right) d%
\mathbf{y}$, e si ha infine%
\begin{equation*}
u_{1}\left( \mathbf{x},t\right) =\int_{0}^{t}\left( \int_{%
%TCIMACRO{\U{211d} }%
%BeginExpansion
\mathbb{R}
%EndExpansion
^{n}}\Gamma _{D}\left( \mathbf{x-y},t-s\right) f\left( \mathbf{y},s\right) d%
\mathbf{y}\right) ds
\end{equation*}

che \`{e} la stessa soluzione trovata col metodo della trasformata di
Fourier.

\paragraph{Analisi critica: soluzione fondamentale}

E' ora il momento di fare un'analisi critica e mostrare rigorosamente se e
in quale senso la candidata soluzione risolve il problema. Per trattare il gi%
\`{a} menzionato problema dovuto al fatto che $s$ non \`{e} discosto da $0$
nell'integrale che definisce $u$ occorre studiare il nucleo del calore da un
punto di vista distribuzionale.

$\Gamma _{D}$ si pu\`{o} vedere come soluzione distribuzionale del problema
di Cauchy $\left\{ 
\begin{array}{c}
H\Gamma _{D}=0\text{ per }t>0,\mathbf{x}\in 
%TCIMACRO{\U{211d} }%
%BeginExpansion
\mathbb{R}
%EndExpansion
^{n} \\ 
\Gamma _{D}\left( \cdot ,0\right) =\delta _{0}\text{ in }%
%TCIMACRO{\U{211d} }%
%BeginExpansion
\mathbb{R}
%EndExpansion
^{n}%
\end{array}%
\right. $. $\Gamma _{D}\left( \cdot ,0\right) =\delta _{0}$ significa che $%
\forall $ $\phi \in \mathcal{D}\left( 
%TCIMACRO{\U{211d} }%
%BeginExpansion
\mathbb{R}
%EndExpansion
^{n}\right) $ $\lim_{t\rightarrow 0^{+}}\int_{%
%TCIMACRO{\U{211d} }%
%BeginExpansion
\mathbb{R}
%EndExpansion
^{n}}\Gamma _{D}\left( \mathbf{y},t\right) \phi \left( \mathbf{y}\right) d%
\mathbf{y}=\phi \left( 0\right) $ (l'uguaglianza non pu\`{o} essere intesa
come uguaglianza vera e propria: anche volendo considerare la distribuzione
associata a $\Gamma _{D}$, essa per $t=0$ non \`{e} ben definita). Sappiamo
gi\`{a} che per $t>0$ $\Gamma _{D}$ risolve l'equazione in senso classico e
anche distribuzionale. La seconda uguaglianza \`{e} vera perch\'{e} sappiamo
che se $u_{0}\in L^{1}\left( 
%TCIMACRO{\U{211d} }%
%BeginExpansion
\mathbb{R}
%EndExpansion
^{n}\right) \cap C_{\ast }^{0}\left( 
%TCIMACRO{\U{211d} }%
%BeginExpansion
\mathbb{R}
%EndExpansion
^{n}\right) $, allora il problema $\left\{ 
\begin{array}{c}
Hu=0\text{ per }t>0,\mathbf{x}\in 
%TCIMACRO{\U{211d} }%
%BeginExpansion
\mathbb{R}
%EndExpansion
^{n} \\ 
u\left( \mathbf{x},0\right) =u_{0}\left( \mathbf{x}\right) \text{ per }%
\mathbf{x}\in 
%TCIMACRO{\U{211d} }%
%BeginExpansion
\mathbb{R}
%EndExpansion
^{n}%
\end{array}%
\right. $ ha soluzione assegnata da $u\left( \mathbf{x},t\right) =\int_{%
%TCIMACRO{\U{211d} }%
%BeginExpansion
\mathbb{R}
%EndExpansion
^{n}}\Gamma _{D}\left( \mathbf{x-y},t\right) u_{0}\left( \mathbf{y}\right) d%
\mathbf{y}$. Ma, data $\phi \in \mathcal{D}\left( 
%TCIMACRO{\U{211d} }%
%BeginExpansion
\mathbb{R}
%EndExpansion
^{n}\right) \subseteq L^{1}\left( 
%TCIMACRO{\U{211d} }%
%BeginExpansion
\mathbb{R}
%EndExpansion
^{n}\right) \cap C_{\ast }^{0}\left( 
%TCIMACRO{\U{211d} }%
%BeginExpansion
\mathbb{R}
%EndExpansion
^{n}\right) $, prendendo $\phi \left( \mathbf{x}\right) =u_{0}\left( \mathbf{%
x}\right) $ si ha che $u\left( \mathbf{x},t\right) =\int_{%
%TCIMACRO{\U{211d} }%
%BeginExpansion
\mathbb{R}
%EndExpansion
^{n}}\Gamma _{D}\left( \mathbf{x-y},t\right) \phi \left( \mathbf{y}\right) d%
\mathbf{y}$, e in particolare $u\left( \mathbf{0},t\right) =\int_{%
%TCIMACRO{\U{211d} }%
%BeginExpansion
\mathbb{R}
%EndExpansion
^{n}}\Gamma _{D}\left( \mathbf{-y},t\right) \phi \left( \mathbf{y}\right) d%
\mathbf{y}$ (che \`{e} uguale a $\int_{%
%TCIMACRO{\U{211d} }%
%BeginExpansion
\mathbb{R}
%EndExpansion
^{n}}\Gamma _{D}\left( \mathbf{y},t\right) \phi \left( \mathbf{y}\right) d%
\mathbf{y}$: si osservi l'espressione analitica di $\Gamma _{D}$): si sa che 
$u\left( \mathbf{x},t\right) $ converge uniformemente a $u_{0}$ per $%
t\rightarrow 0^{+}$, e quindi in particolare se $\mathbf{x=0}$ $%
\lim_{t\rightarrow 0^{+}}\int_{%
%TCIMACRO{\U{211d} }%
%BeginExpansion
\mathbb{R}
%EndExpansion
^{n}}\Gamma _{D}\left( \mathbf{-y},t\right) \phi \left( \mathbf{y}\right) d%
\mathbf{y}=u_{0}\left( \mathbf{0}\right) =\phi \left( \mathbf{0}\right) $.

Quindi il nucleo del calore ha il significato di temperatura che si ha nello
spazio, in assenza di sorgenti esterne, supponendo che il profilo iniziale
della temperatura sia un impulso.

Finora il tempo, nell'espressione analitica di $\Gamma _{D}$, si \`{e}
considerato come parametro: ma per discutere la candidata formula
risolutiva, che \`{e} una sorta di convoluzione in spazio e tempo, occorre
vedere $\Gamma _{D}$ come soluzione fondamentale dell'equazione del calore
in $%
%TCIMACRO{\U{211d} }%
%BeginExpansion
\mathbb{R}
%EndExpansion
^{n+1}$, cio\`{e} tale che $H\Gamma _{D}=\delta _{0}$. Come al solito, se $%
\Gamma _{D}$ \`{e} intesa come distribuzione si ha $H\Gamma _{D}\left( \phi
\right) =\left( \frac{\partial }{\partial t}-D\Delta _{\mathbf{x}}\right)
\Gamma _{D}\left( \phi \right) =\Gamma _{D}\left( -\frac{\partial \phi }{%
\partial t}-D\Delta _{\mathbf{x}}\phi \right) =\int_{%
%TCIMACRO{\U{211d} }%
%BeginExpansion
\mathbb{R}
%EndExpansion
^{n+1}}\Gamma _{D}\left( \mathbf{x},t\right) \left( -\frac{\partial \phi }{%
\partial t}-D\Delta _{\mathbf{x}}\phi \right) \left( \mathbf{x},t\right) d%
\mathbf{x}dt$.

Per far questo naturalmente serve che $\Gamma _{D}$ sia definito in $%
%TCIMACRO{\U{211d} }%
%BeginExpansion
\mathbb{R}
%EndExpansion
^{n+1}$, quindi anche per tempi negativi: si pone $\Gamma _{D}\left( \mathbf{%
x},t\right) =\left\{ 
\begin{array}{c}
0\text{ se }t\leq 0,\mathbf{x}\in 
%TCIMACRO{\U{211d} }%
%BeginExpansion
\mathbb{R}
%EndExpansion
^{n} \\ 
\frac{1}{\left( 4\pi Dt\right) ^{n/2}}e^{-\frac{\left\vert \left\vert 
\mathbf{x}\right\vert \right\vert ^{2}}{4Dt}}\text{ se }t>0,\mathbf{x}\in 
%TCIMACRO{\U{211d} }%
%BeginExpansion
\mathbb{R}
%EndExpansion
^{n}%
\end{array}%
\right. $.

\textbf{Teo (soluzione fondamentale dell'equazione del calore)}%
\begin{eqnarray*}
\text{Hp}\text{: } &&u:%
%TCIMACRO{\U{211d} }%
%BeginExpansion
\mathbb{R}
%EndExpansion
^{n+1}\rightarrow 
%TCIMACRO{\U{211d} }%
%BeginExpansion
\mathbb{R}
%EndExpansion
,\int_{%
%TCIMACRO{\U{211d} }%
%BeginExpansion
\mathbb{R}
%EndExpansion
^{n+1}}u\left( \mathbf{x},t\right) \left( -\frac{\partial \phi }{\partial t}%
-D\Delta _{\mathbf{x}}\phi \right) \left( \mathbf{x},t\right) d\mathbf{x}%
dt=\phi \left( 0\right) \text{ }\forall \text{ }\phi \in \mathcal{D}\left( 
%TCIMACRO{\U{211d} }%
%BeginExpansion
\mathbb{R}
%EndExpansion
^{n+1}\right) \\
\text{Ts}\text{: } &&\Gamma _{D}\left( \mathbf{x},t\right) \text{ risolve
l'equazione, cio\`{e} }H\Gamma _{D}=\delta _{0}
\end{eqnarray*}

\textbf{Dim} Si dimostra una tesi pi\`{u} forte, cio\`{e} che $\int_{%
%TCIMACRO{\U{211d} }%
%BeginExpansion
\mathbb{R}
%EndExpansion
^{n+1}}\Gamma _{D}\left( \mathbf{x},t\right) \left( -\frac{\partial \phi }{%
\partial t}-D\Delta _{\mathbf{x}}\phi \right) \left( \mathbf{x},t\right) d%
\mathbf{x}dt=\phi \left( 0\right) $ $\forall $ $\phi \in C_{0}^{2,1}\left( 
%TCIMACRO{\U{211d} }%
%BeginExpansion
\mathbb{R}
%EndExpansion
^{n+1}\right) $.

Per definizione di $\Gamma _{D}$ $\int_{%
%TCIMACRO{\U{211d} }%
%BeginExpansion
\mathbb{R}
%EndExpansion
^{n+1}}\Gamma _{D}\left( \mathbf{x},t\right) \left( -\frac{\partial \phi }{%
\partial t}-D\Delta _{\mathbf{x}}\phi \right) \left( \mathbf{x},t\right) d%
\mathbf{x}dt=\int_{0}^{+\infty }\int_{%
%TCIMACRO{\U{211d} }%
%BeginExpansion
\mathbb{R}
%EndExpansion
^{n}}\Gamma _{D}\left( \mathbf{x},t\right) \left( -\frac{\partial \phi }{%
\partial t}-D\Delta _{\mathbf{x}}\phi \right) \left( \mathbf{x},t\right) d%
\mathbf{x}dt$. Divido l'integrale in $dt$ in due integrali $%
\int_{0}^{\varepsilon }dt$ e $\int_{\varepsilon }^{+\infty }dt$, che chiamo
rispettivamento $A_{\varepsilon }$ e $B_{\varepsilon }$.

Dico che $A_{\varepsilon }\rightarrow ^{\varepsilon \rightarrow 0^{+}}0$.
Infatti $\left\vert A_{\varepsilon }\right\vert =\left\vert
\int_{0}^{\varepsilon }\int_{%
%TCIMACRO{\U{211d} }%
%BeginExpansion
\mathbb{R}
%EndExpansion
^{n}}\Gamma _{D}\left( \mathbf{x},t\right) \left( -\frac{\partial \phi }{%
\partial t}-D\Delta _{\mathbf{x}}\phi \right) \left( \mathbf{x},t\right) d%
\mathbf{x}dt\right\vert \leq \int_{0}^{\varepsilon }\int_{%
%TCIMACRO{\U{211d} }%
%BeginExpansion
\mathbb{R}
%EndExpansion
^{n}}\Gamma _{D}\left( \mathbf{x},t\right) \left\vert \left( -\frac{\partial
\phi }{\partial t}-D\Delta _{\mathbf{x}}\phi \right) \left( \mathbf{x}%
,t\right) \right\vert d\mathbf{x}dt$: ma per le ipotesi su $\phi $ il
secondo fattore nella funzione integranda \`{e} continuo a supporto compatto
in $%
%TCIMACRO{\U{211d} }%
%BeginExpansion
\mathbb{R}
%EndExpansion
^{n+1}$, per cui, usando anche le propriet\`{a} del nucleo regolarizzante $%
\Gamma _{D}$, $\left\vert A_{\varepsilon }\right\vert \leq \max_{%
%TCIMACRO{\U{211d} }%
%BeginExpansion
\mathbb{R}
%EndExpansion
^{n+1}}\left\vert \frac{\partial \phi }{\partial t}+D\Delta _{\mathbf{x}%
}\phi \right\vert \int_{0}^{\varepsilon }1dt=\varepsilon \max_{%
%TCIMACRO{\U{211d} }%
%BeginExpansion
\mathbb{R}
%EndExpansion
^{n+1}}\left\vert \frac{\partial \phi }{\partial t}+D\Delta _{\mathbf{x}%
}\phi \right\vert \rightarrow ^{\varepsilon \rightarrow 0^{+}}0$. Allora $%
H\Gamma _{D}\left( \phi \right) =\lim_{\varepsilon \rightarrow
0^{+}}B_{\varepsilon }$.

Invece $B_{\varepsilon }=\int_{\varepsilon }^{+\infty }\int_{%
%TCIMACRO{\U{211d} }%
%BeginExpansion
\mathbb{R}
%EndExpansion
^{n}}\Gamma _{D}\left( \mathbf{x},t\right) \left( -\frac{\partial \phi }{%
\partial t}-D\Delta _{\mathbf{x}}\phi \right) \left( \mathbf{x},t\right) d%
\mathbf{x}dt$: poich\'{e} $t$ \`{e} discosto da $0$, si pu\`{o} integrare
per parti nella varabile $t$ e riportare le derivate su $\Gamma _{D}$. Il
primo addendo \`{e} $\int_{\varepsilon }^{+\infty }\int_{%
%TCIMACRO{\U{211d} }%
%BeginExpansion
\mathbb{R}
%EndExpansion
^{n}}\Gamma _{D}\left( \mathbf{x},t\right) \frac{-\partial \phi }{\partial t}%
\left( \mathbf{x},t\right) d\mathbf{x}dt=\int_{%
%TCIMACRO{\U{211d} }%
%BeginExpansion
\mathbb{R}
%EndExpansion
^{n}}\left[ -\Gamma _{D}\left( \mathbf{x},t\right) \phi \left( \mathbf{x}%
,t\right) \right] _{t=\varepsilon }^{t=+\infty }d\mathbf{x}%
+\int_{\varepsilon }^{+\infty }\int_{%
%TCIMACRO{\U{211d} }%
%BeginExpansion
\mathbb{R}
%EndExpansion
^{n}}\frac{\partial }{\partial t}\Gamma _{D}\left( \mathbf{x},t\right) \phi
\left( \mathbf{x},t\right) d\mathbf{x}dt$, cio\`{e} - essendo $\phi $ a
supporto compatto - $\int_{%
%TCIMACRO{\U{211d} }%
%BeginExpansion
\mathbb{R}
%EndExpansion
^{n}}\Gamma _{D}\left( \mathbf{x},\varepsilon \right) \phi \left( \mathbf{x}%
,\varepsilon \right) d\mathbf{x}+\int_{\varepsilon }^{+\infty }\int_{%
%TCIMACRO{\U{211d} }%
%BeginExpansion
\mathbb{R}
%EndExpansion
^{n}}\frac{\partial }{\partial t}\Gamma _{D}\left( \mathbf{x},t\right) \phi
\left( \mathbf{x},t\right) d\mathbf{x}dt$. Il secondo addendo \`{e}
semplicemente $\int_{\varepsilon }^{+\infty }\int_{%
%TCIMACRO{\U{211d} }%
%BeginExpansion
\mathbb{R}
%EndExpansion
^{n}}-D\Delta _{\mathbf{x}}\Gamma _{D}\left( \mathbf{x},t\right) \phi \left( 
\mathbf{x},t\right) d\mathbf{x}dt$.

Allora $B_{\varepsilon }=\int_{\varepsilon }^{+\infty }\int_{%
%TCIMACRO{\U{211d} }%
%BeginExpansion
\mathbb{R}
%EndExpansion
^{n}}H\Gamma _{D}\left( \mathbf{x},t\right) \phi \left( \mathbf{x},t\right) d%
\mathbf{x}dt+\int_{%
%TCIMACRO{\U{211d} }%
%BeginExpansion
\mathbb{R}
%EndExpansion
^{n}}\Gamma _{D}\left( \mathbf{x},\varepsilon \right) \phi \left( \mathbf{x}%
,\varepsilon \right) d\mathbf{x}$: il primo addendo \`{e} nullo perch\'{e} $%
H\Gamma _{D}\left( \mathbf{x},t\right) =0$ $\forall $ $t>0$. Quindi $H\Gamma
_{D}\left( \phi \right) =\lim_{\varepsilon \rightarrow 0^{+}}\int_{%
%TCIMACRO{\U{211d} }%
%BeginExpansion
\mathbb{R}
%EndExpansion
^{n}}\Gamma _{D}\left( \mathbf{x},\varepsilon \right) \phi \left( \mathbf{x}%
,\varepsilon \right) d\mathbf{x}$. Si riscrive l'integrale come $\int_{%
%TCIMACRO{\U{211d} }%
%BeginExpansion
\mathbb{R}
%EndExpansion
^{n}}\Gamma _{D}\left( \mathbf{x},\varepsilon \right) \phi \left( \mathbf{x}%
,0\right) d\mathbf{x}+\int_{%
%TCIMACRO{\U{211d} }%
%BeginExpansion
\mathbb{R}
%EndExpansion
^{n}}\Gamma _{D}\left( \mathbf{x},\varepsilon \right) \left( \phi \left( 
\mathbf{x},\varepsilon \right) -\phi \left( \mathbf{x},0\right) \right) d%
\mathbf{x=}C_{\varepsilon }+D_{\varepsilon }$. Per il teorema di Lagrange $%
\left\vert \phi \left( \mathbf{x},\varepsilon \right) -\phi \left( \mathbf{x}%
,0\right) \right\vert \leq \max_{%
%TCIMACRO{\U{211d} }%
%BeginExpansion
\mathbb{R}
%EndExpansion
^{n+1}}\left\vert \nabla _{%
%TCIMACRO{\U{211d} }%
%BeginExpansion
\mathbb{R}
%EndExpansion
^{n+1}}\phi \right\vert \varepsilon $, quindi $\left\vert D_{\varepsilon
}\right\vert \leq \varepsilon \max_{%
%TCIMACRO{\U{211d} }%
%BeginExpansion
\mathbb{R}
%EndExpansion
^{n+1}}\left\vert \nabla _{%
%TCIMACRO{\U{211d} }%
%BeginExpansion
\mathbb{R}
%EndExpansion
^{n+1}}\phi \right\vert ^{\varepsilon \rightarrow 0^{+}}\rightarrow 0$.

Ma $\int_{%
%TCIMACRO{\U{211d} }%
%BeginExpansion
\mathbb{R}
%EndExpansion
^{n}}\Gamma _{D}\left( \mathbf{x},\varepsilon \right) \phi \left( \mathbf{x}%
,0\right) d\mathbf{x=}\int_{%
%TCIMACRO{\U{211d} }%
%BeginExpansion
\mathbb{R}
%EndExpansion
^{n}}\Gamma _{D}\left( -\mathbf{x},\varepsilon \right) \phi \left( \mathbf{x}%
,0\right) d\mathbf{x=}\int_{%
%TCIMACRO{\U{211d} }%
%BeginExpansion
\mathbb{R}
%EndExpansion
^{n}}\Gamma _{D}\left( \mathbf{-y},\varepsilon \right) \phi \left( \mathbf{y}%
,0\right) d\mathbf{y}$ pu\`{o} essere vista come soluzione classica del
problema di Dirichlet con condizione iniziale $u_{0}\left( \mathbf{x}\right)
=\phi \left( \mathbf{x,0}\right) $ e valutata in $\mathbf{x=0}$: poich\'{e}
questa \`{e} $L^{1}\cap C_{\ast }^{0}$, la soluzione classica converge
uniformemente al dato iniziale per $t=\varepsilon \rightarrow 0^{+}$, e
quindi $\lim_{\varepsilon \rightarrow 0^{+}}\int_{%
%TCIMACRO{\U{211d} }%
%BeginExpansion
\mathbb{R}
%EndExpansion
^{n}}\Gamma _{D}\left( \mathbf{-y},\varepsilon \right) \phi \left( \mathbf{y}%
,0\right) d\mathbf{y}=\phi \left( \mathbf{0}\right) $. $\blacksquare $

L'ultimo passaggio della dimostrazione \`{e} proprio quanto mostrato sopra: $%
\Gamma _{D}\left( \cdot ,t\right) \rightarrow \delta _{0}$ per $%
t=\varepsilon \rightarrow 0^{+}$. Si \`{e} ottenuto quindi che $H\Gamma
_{D}\left( \phi \right) =\lim_{\varepsilon \rightarrow 0^{+}}\int_{%
%TCIMACRO{\U{211d} }%
%BeginExpansion
\mathbb{R}
%EndExpansion
^{n}}\Gamma _{D}\left( \mathbf{x},\varepsilon \right) \phi \left( \mathbf{x}%
,0\right) d\mathbf{x}=\phi \left( \mathbf{0}\right) $.

Ora si pu\`{o} mostrare rigorosamente che la candidata soluzione trovata,
solo formalmente, con il metodo della trasformata di Fourier e il metodo di
Duhamel risolve il problema.

\textbf{Corollario (soluzione dell'equazione del calore non omogenea)}%
\begin{eqnarray*}
\text{Hp}\text{: } &&f\in C_{0}^{2,1}\left( 
%TCIMACRO{\U{211d} }%
%BeginExpansion
\mathbb{R}
%EndExpansion
^{n}\times \lbrack 0,+\infty )\right) \\
\text{Ts}\text{: } &&u\left( \mathbf{x},t\right) =\int_{0}^{t}\int_{%
%TCIMACRO{\U{211d} }%
%BeginExpansion
\mathbb{R}
%EndExpansion
^{n+1}}\Gamma _{D}\left( \mathbf{x-y},t-s\right) f\left( \mathbf{y},s\right)
d\mathbf{y}ds\text{ risolve }\left\{ 
\begin{array}{c}
Hu=f\text{ per }\mathbf{x}\in 
%TCIMACRO{\U{211d} }%
%BeginExpansion
\mathbb{R}
%EndExpansion
^{n},t>0 \\ 
u\left( \mathbf{x},0\right) =0\text{ per }\mathbf{x}\in 
%TCIMACRO{\U{211d} }%
%BeginExpansion
\mathbb{R}
%EndExpansion
^{n}%
\end{array}%
\right.
\end{eqnarray*}

Se la condizione iniziale \`{e} non nulla basta sommare la solita soluzione
dell'omogenea. Le ipotesi su $f$ sono molto forti.

\textbf{Dim} Vogliamo ovviamente applicare il teorema sopra. Con un cambio
di variabili si ha $u\left( \mathbf{x},t\right) =\int_{0}^{t}\int_{%
%TCIMACRO{\U{211d} }%
%BeginExpansion
\mathbb{R}
%EndExpansion
^{n+1}}\Gamma _{D}\left( \mathbf{y},s\right) f\left( \mathbf{x-y},t-s\right)
d\mathbf{y}ds$: $Hu\left( \mathbf{x},t\right) =\int_{%
%TCIMACRO{\U{211d} }%
%BeginExpansion
\mathbb{R}
%EndExpansion
^{n+1}}\Gamma _{D}\left( \mathbf{y},s\right) H_{\mathbf{x},t}\left( f\left( 
\mathbf{x-y},t-s\right) \right) d\mathbf{y}ds$ (lo scambio tra $H$ e
integrale \`{e} lecito perch\'{e} $f$ \`{e} a supporto compatto e quindi $%
\left\vert \Gamma _{D}\left( \mathbf{y},s\right) H_{\mathbf{x},t}f\left( 
\mathbf{x-y},t-s\right) \right\vert \leq k\Gamma _{D}\left( \mathbf{y}%
,s\right) $, che \`{e} integrabile?). Per applicare il teorema occorre per%
\`{o} che le operazioni sulla funzione test $f$ avvengano nelle stesse
variabili in cui si integra: poich\'{e} $\frac{\partial }{\partial t}f\left( 
\mathbf{x-y},t-s\right) =-\frac{\partial }{\partial s}f\left( \mathbf{x-y}%
,t-s\right) $ e $\Delta _{\mathbf{x}}f\left( \mathbf{x-y},t-s\right) =\Delta
_{\mathbf{y}}f\left( \mathbf{x-y},t-s\right) $, si ha $Hu\left( \mathbf{x}%
,t\right) =\int_{%
%TCIMACRO{\U{211d} }%
%BeginExpansion
\mathbb{R}
%EndExpansion
^{n+1}}\Gamma _{D}\left( \mathbf{y},s\right) \left( \frac{-\partial }{%
\partial s}-D\Delta _{\mathbf{y}}\right) \left( f\left( \mathbf{x-y}%
,t-s\right) \right) d\mathbf{y}ds$. Definendo $H^{\ast }=\frac{-\partial }{%
\partial s}-D\Delta _{\mathbf{y}}$ e $F\left( \mathbf{y},s\right) =f\left( 
\mathbf{x-y},t-s\right) $, per il teorema sopra si ha $Hu\left( \mathbf{x}%
,t\right) =\int_{%
%TCIMACRO{\U{211d} }%
%BeginExpansion
\mathbb{R}
%EndExpansion
^{n+1}}\Gamma _{D}\left( \mathbf{y},s\right) H^{\ast }F\left( \mathbf{y}%
,s\right) d\mathbf{y}ds=F\left( \mathbf{0},0\right) =f\left( \mathbf{x}%
,t\right) $, che \`{e} la tesi. $\blacksquare $

\section{Equazione del trasporto lineare in una dimensione}

Nel corso della deduzione fisica dell'equazione del calore si \`{e} arrivati
all'equazione di bilancio $\frac{\partial }{\partial t}\left( cu\rho \right)
+\func{div}\mathbf{q}=r\rho $: d'ora in poi per\`{o} $u$ avr\`{a} il
significato fisico di concentrazione (di una sostanza disciolta in
un'altra), $\rho $ di densit\`{a}, $r$ di tasso istantaneo di
quantit\`{a} di materia immersa, $\mathbf{q}$ di densit\`{a} di
corrente. Al contrario di scrivere $\mathbf{q}$ rintracciandone le due
diverse fonti come si era fatto per l'equazione del calore, si suppone che
non ci sia diffusione, ma solo deriva, per cui $\mathbf{q=q}\left( u\right) $%
. Facendo le solite ipotesi semplificatrici si ricava l'equazione del
trasporto%
\begin{equation*}
\frac{\partial u}{\partial t}+c\func{div}\mathbf{q}\left( u\right) =f
\end{equation*}

In base alla relazione tra $\mathbf{q}$ e $u$ si ottengono vari
sottomodelli, che prendono nomi diversi. A noi interessa il caso di $\mathbf{%
q}\left( u\right) =\mathbf{b}u$: si ottiene l'equazione del trasporto lineare%
\begin{equation*}
\frac{\partial u}{\partial t}+\left\langle \mathbf{v},\nabla _{\mathbf{x}%
}u\right\rangle =f
\end{equation*}

Come al solito il gradiente si suppone fatto rispetto alle coordinate
spaziali. Il lato sinistro \`{e} il termine di deriva, il lato destro quello
di sorgente. Nel caso in cui si voglia considerare anche il termine di
reazione si ha l'equazione $\frac{\partial u}{\partial t}+\left\langle 
\mathbf{v},\nabla _{\mathbf{x}}u\right\rangle +\gamma u=f$ (che non cambia
molto la soluzione), con $\gamma >0$.

\subsection{Equazione del trasporto omogenea}

Considero l'equazione del trasporto omogenea in una dimensione%
\begin{equation*}
\frac{\partial u}{\partial t}+v\frac{\partial u}{\partial x}=0
\end{equation*}

nell'incognita $u=u\left( x,t\right) ,x\in 
%TCIMACRO{\U{211d} }%
%BeginExpansion
\mathbb{R}
%EndExpansion
,t>0$.

Il lato sinistro dell'equazione pu\`{o} essere visto come derivata
direzionale di $u$: se $u\in C^{1}\left( 
%TCIMACRO{\U{211d} }%
%BeginExpansion
\mathbb{R}
%EndExpansion
^{2}\right) $ \`{e} soluzione dell'equazione, allora \`{e} costante lungo le
linee $x-vt=\xi $, $\forall $ $\xi \in 
%TCIMACRO{\U{211d} }%
%BeginExpansion
\mathbb{R}
%EndExpansion
$. Questa osservazione permette di determinare l'integrale generale
dell'equazione.

Sia $\xi \in 
%TCIMACRO{\U{211d} }%
%BeginExpansion
\mathbb{R}
%EndExpansion
$ fissato, $x:x=\xi +vt$ (cio\`{e} su una delle rette menzionate) e $u\in
C^{1}\left( 
%TCIMACRO{\U{211d} }%
%BeginExpansion
\mathbb{R}
%EndExpansion
^{2}\right) $ soluzione dell'equazione. Allora $u\left( x,t\right) =u\left(
\xi +vt,t\right) =:U\left( t\right) $ \`{e} costante in $t$: infatti $%
U^{\prime }\left( t\right) =\left\langle \nabla u\left( \xi +vt\right)
,\left( 
\begin{array}{c}
v \\ 
1%
\end{array}%
\right) \right\rangle =v\frac{\partial u}{\partial x}\left( \xi +vt\right) +%
\frac{\partial u}{\partial t}\left( \xi +vt\right) =0$ poich\'{e} $u$ \`{e}
soluzione. Dunque in particolare $U\left( t\right) =U\left( 0\right) $ $%
\forall $ $t>0$: quindi $\exists $ $g\in C^{1}\left( 
%TCIMACRO{\U{211d} }%
%BeginExpansion
\mathbb{R}
%EndExpansion
\right) :U\left( t\right) =u\left( \xi ,0\right) =:g\left( \xi \right) \in
C^{1}\left( 
%TCIMACRO{\U{211d} }%
%BeginExpansion
\mathbb{R}
%EndExpansion
\right) $, che ha il significato di condizione iniziale.

Allora si pu\`{o} ricavare la soluzione: $u\left( x,t\right) =g\left(
x-vt\right) $ $\forall $ $x,t$. Si \`{e} dimostrato che $\forall $ $u\in
C^{1}\left( 
%TCIMACRO{\U{211d} }%
%BeginExpansion
\mathbb{R}
%EndExpansion
^{2}\right) $ soluzione dell'equazione $\exists $ $g:%
%TCIMACRO{\U{211d} }%
%BeginExpansion
\mathbb{R}
%EndExpansion
\rightarrow 
%TCIMACRO{\U{211d} }%
%BeginExpansion
\mathbb{R}
%EndExpansion
,g\in C^{1}\left( 
%TCIMACRO{\U{211d} }%
%BeginExpansion
\mathbb{R}
%EndExpansion
\right) :u\left( x,t\right) =g\left( x-vt\right) $ $\forall $ $x\in 
%TCIMACRO{\U{211d} }%
%BeginExpansion
\mathbb{R}
%EndExpansion
,t>0$; inoltre $u\left( x,0\right) =g\left( x\right) $ $\forall $ $x\in 
%TCIMACRO{\U{211d} }%
%BeginExpansion
\mathbb{R}
%EndExpansion
$.

Viceversa, se $\forall $ $g\in C^{1}\left( 
%TCIMACRO{\U{211d} }%
%BeginExpansion
\mathbb{R}
%EndExpansion
\right) $ si definisce $u\left( x,t\right) :=g\left( x-vt\right) $, allora $%
u\in C^{1}\left( 
%TCIMACRO{\U{211d} }%
%BeginExpansion
\mathbb{R}
%EndExpansion
^{2}\right) $ e $u$ risolve l'equazione. Infatti $\frac{\partial u}{\partial
t}+v\frac{\partial u}{\partial x}=-vg^{\prime }\left( x-vt\right)
+vg^{\prime }\left( x-vt\right) =0$, e $u\left( x,0\right) =g\left( x\right) 
$ $\forall $ $x\in 
%TCIMACRO{\U{211d} }%
%BeginExpansion
\mathbb{R}
%EndExpansion
$. Si \`{e} dimostrato il seguente

\textbf{Teo (integrale generale dell'equazione lineare del trasporto
omogenea)}%
\begin{gather*}
\text{Hp: }u_{t}+vu_{x}=0 \\
\text{Ts: (i) l'integrale generale dell'equazione \`{e} dato da }u\left(
x,t\right) =g\left( x-vt\right) \text{ al variare di }g:%
%TCIMACRO{\U{211d} }%
%BeginExpansion
\mathbb{R}
%EndExpansion
\rightarrow 
%TCIMACRO{\U{211d} }%
%BeginExpansion
\mathbb{R}
%EndExpansion
\text{ in }C^{1}\left( 
%TCIMACRO{\U{211d} }%
%BeginExpansion
\mathbb{R}
%EndExpansion
\right) \\
\text{(ii) se }g\in C^{1}\left( 
%TCIMACRO{\U{211d} }%
%BeginExpansion
\mathbb{R}
%EndExpansion
\right) \text{ la soluzione del problema di Cauchy }\left\{ 
\begin{array}{c}
u_{t}+vu_{x}=0\text{ per }x\in 
%TCIMACRO{\U{211d} }%
%BeginExpansion
\mathbb{R}
%EndExpansion
,t>0 \\ 
u\left( x,0\right) =g\left( x\right) \text{ per }x\in 
%TCIMACRO{\U{211d}}%
%BeginExpansion
\mathbb{R}%
%EndExpansion
\end{array}%
\right. \text{ esiste, \`{e}} \\
\text{unica nella classe }C^{1}\left( 
%TCIMACRO{\U{211d} }%
%BeginExpansion
\mathbb{R}
%EndExpansion
^{2}\right) \text{ ed \`{e} data da }u\left( x,t\right) =g\left( x-vt\right)
\end{gather*}

Quindi se si ha una condizione iniziale $C^{1}$ si ottiene una soluzione $%
C^{1}$, cio\`{e} la soluzione ha la stessa regolarit\`{a} della condizione
iniziale: l'equazione del trasporto, a differenza dell'equazione del calore,
non regolarizza. Questa \`{e} una conseguenza negativa del fatto che ha
permesso di risolvere l'equazione cos\`{\i} facilmente: la presenza di rette
lungo cui la soluzione $u$ \`{e} costante, di modo che il valore della $u$
all'istante $0$, cio\`{e} l'informazione della condizione iniziale, viaggia
nello spaziotempo lungo queste rette (che prenderanno il nome di rette
caratteristiche). Ovviamente da questo segue che $u$, non essendo altro che
una traslata di $g$ lungo tali rette, non potr\`{a} essere pi\`{u} regolare
di $g$.

Il problema di Cauchy omogeneo in dimensione qualsiasi \`{e} 
\begin{equation*}
\left\{ 
\begin{array}{c}
\frac{\partial u}{\partial t}+\left\langle \mathbf{v},\nabla u\right\rangle
=0 \\ 
u\left( \mathbf{x},0\right) =g\left( \mathbf{x}\right)%
\end{array}%
\right.
\end{equation*}

La situazione \`{e} del tutto analoga all'equazione unidimensionale: si
mostra che $u\left( \mathbf{x},t\right) =g\left( \mathbf{x-v}t\right) $
proprio come sopra.

\subsection{Equazione non omogenea}

Per risolvere il problema non omogeneo si sfrutta come al solito la
sovrapposizione degli effetti e si risolve%
\begin{equation*}
\left\{ 
\begin{array}{c}
\frac{\partial u}{\partial t}+v\frac{\partial u}{\partial x}=f\left(
x,t\right) \\ 
u\left( x,0\right) =0%
\end{array}%
\right. \text{ (1)}
\end{equation*}

con il metodo di Duhamel. Sia $w\left( x,t,s\right) $ soluzione di $\left\{ 
\begin{array}{c}
\frac{\partial w}{\partial t}+v\frac{\partial w}{\partial x}=0\text{ per }%
x\in 
%TCIMACRO{\U{211d} }%
%BeginExpansion
\mathbb{R}
%EndExpansion
,t>s \\ 
w\left( x,s,s\right) =f\left( x,s\right)%
\end{array}%
\right. $: allora $u\left( x,t\right) =\int_{0}^{t}w\left( x,t,s\right) ds$
risolve il problema (1). Formalmente, si ha in effetti $u\left( x,0\right)
=0 $, e $\frac{\partial u}{\partial t}=\int_{0}^{t}\frac{\partial }{\partial
t}w\left( x,t,s\right) ds+w\left( x,t,t\right) =\int_{0}^{t}\frac{\partial }{%
\partial t}w\left( x,t,s\right) ds+f\left( x,t\right) $, mentre $\frac{%
\partial u}{\partial x}=\int_{0}^{t}\frac{\partial }{\partial x}w\left(
x,t,s\right) ds$, per cui $\frac{\partial u}{\partial t}+v\frac{\partial u}{%
\partial x}=f\left( x,t\right) $.

Nel nostro caso $w\left( x,t,s\right) =f\left( x-v\left( t-s\right)
,s\right) $, per cui la candidata soluzione \`{e} $u\left( x,t\right)
=\int_{0}^{t}f\left( x-v\left( t-s\right) ,s\right) ds$. Quali ipotesi fare
su $f$ affinch\'{e} $u$ sia davvero soluzione? Per dimostrare ci\`{o}
agevolmente vorremmo poter scambiare derivata e integrale per scrivere che $%
u_{x}\left( x,t\right) =\int_{0}^{t}f_{x}\left( x-v\left( t-s\right)
,s\right) ds$, quindi \`{e} sufficiente che $f_{x}\in C^{0}\left( 
%TCIMACRO{\U{211d} }%
%BeginExpansion
\mathbb{R}
%EndExpansion
^{2}\right) $ (e ovviamente $f\in C^{0}\left( 
%TCIMACRO{\U{211d} }%
%BeginExpansion
\mathbb{R}
%EndExpansion
^{2}\right) $): in tal caso $f_{x}$ \`{e} limitata in $\left[ 0,t\right] $ e
dunque maggiorata da una costante integrabile. Cos\`{\i} si ha anche $%
u_{t}=f\left( x,t\right) -v\int_{0}^{t}f_{x}\left( x-v\left( t-s\right)
,s\right) ds$, per cui $u_{t}+vu_{x}=f\left( x,t\right)
-v\int_{0}^{t}f_{x}\left( x-v\left( t-s\right) ,s\right)
ds+v\int_{0}^{t}f_{x}\left( x-v\left( t-s\right) ,s\right) ds$. Questo
dimostra il seguente

\textbf{Teo (soluzione dell'equazione unidimensionale non omogenea)}%
\begin{gather*}
\text{Hp: }f\in C^{0}\left( 
%TCIMACRO{\U{211d} }%
%BeginExpansion
\mathbb{R}
%EndExpansion
^{2}\right) ,\exists \text{ }\frac{\partial f}{\partial x}\in C^{0}\left( 
%TCIMACRO{\U{211d} }%
%BeginExpansion
\mathbb{R}
%EndExpansion
^{2}\right) \\
\text{Ts: (i) la soluzione di }\left\{ 
\begin{array}{c}
\frac{\partial u}{\partial t}+v\frac{\partial u}{\partial x}=f\left(
x,t\right) \\ 
u\left( x,0\right) =0%
\end{array}%
\right. \text{ \`{e} data da }u\left( x,t\right) =\int_{0}^{t}f\left(
x-v\left( t-s\right) ,s\right) ds \\
\text{(ii) la soluzione di }\left\{ 
\begin{array}{c}
\frac{\partial u}{\partial t}+v\frac{\partial u}{\partial x}=f\left(
x,t\right) \\ 
u\left( x,0\right) =g\left( x\right)%
\end{array}%
\right. \text{ \`{e} data da }u\left( x,t\right) =g\left( x-vt\right)
+\int_{0}^{t}f\left( x-v\left( t-s\right) ,s\right) ds \\
\text{ed \`{e} unica in }C^{1}\left( 
%TCIMACRO{\U{211d} }%
%BeginExpansion
\mathbb{R}
%EndExpansion
^{2}\right)
\end{gather*}

L'unicit\`{a} \`{e} una conseguenza immediata dell'unicit\`{a} della
soluzione per l'equazione omogenea.

Se in particolare $f\left( x,t\right) =\psi \left( t\right) $ e $g=0$, $%
\int_{0}^{t}f\left( x-v\left( t-s\right) ,s\right) ds=\int_{0}^{t}\psi
\left( s\right) ds=u\left( t\right) $: la soluzione \`{e} indipendente da $x$
e risolve $\frac{\partial u}{\partial t}=\psi \left( t\right) $. Lo stesso
accade se si aggiunge anche il termine di reazione.

Il caso di $f\left( x,t\right) =\phi \left( x\right) $ \`{e} trattato nel
seguente

\textbf{Teo (soluzione del problema di Cauchy con sorgente indipendente da }$%
t$\textbf{)}%
\begin{gather*}
\text{Hp: }f\in C^{0}\left( 
%TCIMACRO{\U{211d} }%
%BeginExpansion
\mathbb{R}
%EndExpansion
^{2}\right) ,\exists \text{ }\frac{\partial f}{\partial x}\in C^{0}\left( 
%TCIMACRO{\U{211d} }%
%BeginExpansion
\mathbb{R}
%EndExpansion
^{2}\right) ,f\left( x,t\right) =\phi \left( x\right) ,\phi \in L^{1}\left( 
%TCIMACRO{\U{211d} }%
%BeginExpansion
\mathbb{R}
%EndExpansion
\right) \cap C_{\ast }^{0}\left( 
%TCIMACRO{\U{211d} }%
%BeginExpansion
\mathbb{R}
%EndExpansion
\right) \\
\text{Ts: la soluzione di }\left\{ 
\begin{array}{c}
\frac{\partial u}{\partial t}+v\frac{\partial u}{\partial x}=\phi \left(
x\right) \\ 
u\left( x,0\right) =0%
\end{array}%
\right. \text{ \`{e} data da }u\left( x,t\right) =\int_{0}^{t}\phi \left(
x-vs\right) ds \\
\text{e }F\left( x\right) =\lim_{t\rightarrow +\infty }u\left( x,t\right) 
\text{ risolve l'equazione stazionaria }vu_{x}=\phi
\end{gather*}

\textbf{Dim} Se $f\left( x,t\right) =\phi \left( x\right) $ e $g=0$, per il
teorema sopra $u\left( x,t\right) =\int_{0}^{t}f\left( x-v\left( t-s\right)
,s\right) ds=\int_{0}^{t}\phi \left( x-v\left( t-s\right) \right) ds=%
\footnote{%
Il passaggio \`{e} giustificato dal fatto che $\int_{0}^{t}h\left(
t-s\right) ds=\int_{t}^{0}-h\left( z\right) dz=\int_{0}^{t}h\left( z\right)
dz$.}\int_{0}^{t}\phi \left( x-vs\right) ds$. $\lim_{t\rightarrow +\infty
}u\left( x,t\right) =\int_{0}^{+\infty }\phi \left( x-vs\right) ds=:F\left(
x\right) $ (ben definita perch\'{e} $\phi \in L^{1}\left( 
%TCIMACRO{\U{211d} }%
%BeginExpansion
\mathbb{R}
%EndExpansion
\right) $): questo limite \`{e} una soluzione del problema stazionario $%
\left\{ 
\begin{array}{c}
v\frac{\partial u}{\partial x}=f\left( x,t\right) \\ 
u\left( x,0\right) =0%
\end{array}%
\right. $. Infatti $F^{\prime }\left( x\right) =\int_{0}^{+\infty }\phi
^{\prime }\left( x-vs\right) ds=^{t=x-vs}\int_{x}^{+\infty }\frac{\phi
^{\prime }\left( t\right) }{-v}dt=\frac{1}{v}\left( \phi \left( x\right)
-\phi \left( +\infty \right) \right) $, e il secondo addendo \`{e} nullo
perch\'{e} $\phi \in C_{\ast }^{0}\left( 
%TCIMACRO{\U{211d} }%
%BeginExpansion
\mathbb{R}
%EndExpansion
\right) $, per cui $vF^{\prime }\left( x\right) =\phi \left( x\right) $. $%
\blacksquare $

\subsection{Equazione con termine di reazione}

Cosa cambia se si considera il problema $\left\{ 
\begin{array}{c}
\frac{\partial u}{\partial t}+v\frac{\partial u}{\partial x}+\gamma
u=f\left( x,t\right) \text{ per }x\in 
%TCIMACRO{\U{211d} }%
%BeginExpansion
\mathbb{R}
%EndExpansion
,t>0 \\ 
u\left( x,0\right) =g\left( x\right) \text{ per }x\in 
%TCIMACRO{\U{211d}}%
%BeginExpansion
\mathbb{R}%
%EndExpansion
\end{array}%
\right. $ con $\gamma >0$? La soluzione cambia di poco perch\'{e} $\gamma u$
rappresenta una piccola perturbazione: \`{e} un'idea che formalizzeremo bene
pi\`{u} avanti, ma in ogni equazione c'\`{e} sempre una parte principale,
che determina il tipo di equazione, e dei termini di ordine inferiore (ad
esempio i termini $u$ quando sono presenti le derivate di $u$) che non
alterano in modo significativo la soluzione. $\gamma u$ rientra in
quest'ultimo caso.

Per risolvere l'equazione suppongo\footnote{%
Si potrebbe anche suppore $u$ costante in $t$ e risolvere supponendo $%
u\left( x,t\right) =e^{-\frac{\gamma }{v}x}w\left( x,t\right) $: facendo
passaggi simili si otterrebbe un risultato analogo} che $u$ sia costante in $%
x$ e risolvo $u_{t}+\gamma u=0$: $u\left( t\right) =ce^{-\gamma t}$. Allora 
\`{e} ragionevole supporre che la soluzione dell'equazione originaria sia
del tipo $u\left( x,t\right) =e^{-\gamma t}w\left( x,t\right) $: dal punto
di vista fisico, la presenza del termine di reazione fa s\`{\i} che la
concentrazione della sostanza tenda a $0$ per $t\rightarrow +\infty $.
Allora riscrivo l'equazione e risolvo in $w$: $u_{t}=e^{-\gamma t}\left(
w_{t}-\gamma w\right) ,u_{x}=e^{-\gamma t}w_{x}$, quindi $\frac{\partial u}{%
\partial t}+v\frac{\partial u}{\partial x}+\gamma u=e^{-\gamma t}\left(
w_{t}-\gamma w\right) +ve^{-\gamma t}w_{x}+\gamma e^{-\gamma t}w=e^{-\gamma
t}\left( w_{t}+vw_{x}\right) $; invece $u\left( x,0\right) =w\left(
x,0\right) $. Si \`{e} quindi ottenuto il problema%
\begin{equation*}
\left\{ 
\begin{array}{c}
w_{t}+vw_{x}=e^{\gamma t}f \\ 
w\left( x,0\right) =g\left( x\right)%
\end{array}%
\right.
\end{equation*}

Si \`{e} quindi trasformato un problema nuovo in un problema noto. La
soluzione \`{e} $w\left( x,t\right) =g\left( x-vt\right)
+\int_{0}^{t}e^{\gamma s}f\left( x-v\left( t-s\right) ,s\right) ds$, per cui
la formula risolutiva del problema originario (trasporto con reazione e
sorgente) \`{e}%
\begin{equation*}
u\left( x,t\right) =e^{-\gamma t}g\left( x-vt\right) +\int_{0}^{t}e^{-\gamma
\left( t-s\right) }f\left( x-v\left( t-s\right) ,s\right) ds
\end{equation*}

Se in particolare $f=0$, la soluzione \`{e} l'onda viaggiante $g$ smorzata
esponenzialmente (trasporto con estinzione progressiva).

Tutto si generalizza al caso $n$-dimensionale $\left\{ 
\begin{array}{c}
\frac{\partial u}{\partial t}+\left\langle \mathbf{v},\nabla u\right\rangle
+\gamma u=f\left( \mathbf{x},t\right) \\ 
u\left( \mathbf{x},0\right) =g\left( \mathbf{x}\right)%
\end{array}%
\right. $: la soluzione \`{e} $u\left( \mathbf{x},t\right) =e^{-\gamma
t}g\left( \mathbf{x-v}t\right) +\int_{0}^{t}e^{-\gamma \left( t-s\right)
}f\left( \mathbf{x-v}\left( t-s\right) ,s\right) ds$, sempre con le stesse
ipotesi.

\subsection{Soluzioni deboli}

Si \`{e} gi\`{a} osservato che le ipotesi da fare sui termini noti del
problema di Cauchy ($g\in C^{1}\left( 
%TCIMACRO{\U{211d} }%
%BeginExpansion
\mathbb{R}
%EndExpansion
\right) ,f\in C^{0}\left( 
%TCIMACRO{\U{211d} }%
%BeginExpansion
\mathbb{R}
%EndExpansion
\times \lbrack 0,+\infty )\right) ,\exists $ $\frac{\partial f}{\partial x}%
\in C^{0}\left( 
%TCIMACRO{\U{211d} }%
%BeginExpansion
\mathbb{R}
%EndExpansion
\times \lbrack 0,+\infty )\right) $) per l'equazione del trasporto sono
piuttosto forti, e l'equazione non \`{e} regolarizzante. E' del tutto
ragionevole chiedersi se ha senso risolvere il problema quando $f$ e $g$
sono meno regolari di quanto richiesto da tali ipotesi. Si noti che il
problema \`{e} diverso da quanto osservato per le ipotesi sovrabbondanti per
il problema di Dirichlet per il laplaciano sul cerchio dovute all'uso della
serie di Fourier: in questo caso si sono fatti passaggi obbligati, e non 
\`{e} possibile che esista una soluzione $u\in C^{2}\left( 
%TCIMACRO{\U{211d} }%
%BeginExpansion
\mathbb{R}
%EndExpansion
\right) $ con $g$ meno regolare di $C^{1}\left( 
%TCIMACRO{\U{211d} }%
%BeginExpansion
\mathbb{R}
%EndExpansion
\right) $. Quello che ci si sta chiedendo \`{e} quindi se si possa \textit{%
indebolire} il concetto di soluzione, dando una nuova definizione di
soluzione di un'equazione differenziale.

L'idea naturale per trovare una nuova definizione \`{e} partire
dall'equazione, trasformarla in un'identit\`{a} integrale e vedere se essa pu%
\`{o} essere ben definita anche con una minore regolarit\`{a} di $f$ e $g$.

Suppongo quindi che $f$ e $g$ siano regolari come sopra e considero $u\in
C^{1}\left( 
%TCIMACRO{\U{211d} }%
%BeginExpansion
\mathbb{R}
%EndExpansion
^{2}\right) $ soluzione del problema 
\begin{equation*}
\left\{ 
\begin{array}{c}
u_{t}+vu_{x}=f\text{ per }x\in 
%TCIMACRO{\U{211d} }%
%BeginExpansion
\mathbb{R}
%EndExpansion
,t>0 \\ 
u\left( x,0\right) =g\text{ per }x\in 
%TCIMACRO{\U{211d}}%
%BeginExpansion
\mathbb{R}%
%EndExpansion
\end{array}%
\right. 
\end{equation*}
Moltiplico entrambi i lati dell'equazione per una funzione test
ben scelta $\phi \in C_{0}^{1}\left( 
%TCIMACRO{\U{211d} }%
%BeginExpansion
\mathbb{R}
%EndExpansion
\times \lbrack 0,+\infty )\right) ,\phi =\phi \left( x,t\right) $ (che
quindi ha supporto contenuto in $%
%TCIMACRO{\U{211d} }%
%BeginExpansion
\mathbb{R}
%EndExpansion
\times \lbrack 0,+\infty )$) e integro in $%
%TCIMACRO{\U{211d} }%
%BeginExpansion
\mathbb{R}
%EndExpansion
\times \lbrack 0,+\infty )$: $\int_{0}^{+\infty }\int_{%
%TCIMACRO{\U{211d} }%
%BeginExpansion
\mathbb{R}
%EndExpansion
}\phi \left( x,t\right) \left( u_{t}+vu_{x}\right) \left( x,t\right)
dxdt=\int_{0}^{+\infty }\int_{%
%TCIMACRO{\U{211d} }%
%BeginExpansion
\mathbb{R}
%EndExpansion
}f\left( x,t\right) \phi \left( x,t\right) dxdt$. Integro per parti per
scaricare le derivate su $\phi $.

Il primo addendo \`{e} $\int_{%
%TCIMACRO{\U{211d} }%
%BeginExpansion
\mathbb{R}
%EndExpansion
}\int_{0}^{+\infty }\phi \left( x,t\right) u_{t}\left( x,t\right) dtdx=\int_{%
%TCIMACRO{\U{211d} }%
%BeginExpansion
\mathbb{R}
%EndExpansion
}\left( \left[ \phi u\right] _{t=0}^{t=+\infty }-\int_{0}^{+\infty }u\left(
x,t\right) \phi _{t}\left( x,t\right) dx\right) dt=-\int_{%
%TCIMACRO{\U{211d} }%
%BeginExpansion
\mathbb{R}
%EndExpansion
}\phi \left( x,0\right) u\left( x,0\right) dx-\int_{0}^{+\infty }\int_{%
%TCIMACRO{\U{211d} }%
%BeginExpansion
\mathbb{R}
%EndExpansion
}u\left( x,t\right) \phi _{t}\left( x,t\right) dxdt$, perch\'{e} $\phi $ 
\`{e} a supporto compatto.

Il secondo addendo \`{e} $\int_{0}^{+\infty }\int_{%
%TCIMACRO{\U{211d} }%
%BeginExpansion
\mathbb{R}
%EndExpansion
}\phi \left( x,t\right) u_{x}\left( x,t\right) dxdt=\int_{0}^{+\infty
}\left( \left[ \phi u\right] _{x=-\infty }^{x=+\infty }-\int_{%
%TCIMACRO{\U{211d} }%
%BeginExpansion
\mathbb{R}
%EndExpansion
}u\left( x,t\right) \phi _{x}\left( x,t\right) dx\right)
dt=-\int_{0}^{+\infty }\int_{%
%TCIMACRO{\U{211d} }%
%BeginExpansion
\mathbb{R}
%EndExpansion
}u\left( x,t\right) \phi _{x}\left( x,t\right) dxdt$, perch\'{e} $\phi $ 
\`{e} a supporto compatto.

Si \`{e} quindi ottenuta l'uguaglianza 
\begin{equation*}
\int_{0}^{+\infty }\int_{%
%TCIMACRO{\U{211d} }%
%BeginExpansion
\mathbb{R}
%EndExpansion
}\left[ u\left( x,t\right) \left( \phi _{t}+v\phi _{x}\right) \left(
x,t\right) +f\left( x,t\right) \phi \left( x,t\right) \right] dxdt+\int_{%
%TCIMACRO{\U{211d} }%
%BeginExpansion
\mathbb{R}
%EndExpansion
}\phi \left( x,0\right) g\left( x\right) dx=0
\end{equation*}

Si \`{e} dimostrato che se $u\in C^{1}\left( 
%TCIMACRO{\U{211d} }%
%BeginExpansion
\mathbb{R}
%EndExpansion
^{2}\right) $ soluzione classica del problema di Cauchy come sopra, allora $%
\forall $ $\phi \in C_{0}^{1}\left( 
%TCIMACRO{\U{211d} }%
%BeginExpansion
\mathbb{R}
%EndExpansion
\times \lbrack 0,+\infty )\right) $ vale tale uguaglianza. Quali sono le
ipotesi minime da fare su $f$, $g$ e $u$ affinch\'{e} essa sia sensata?
Basta che $f,u\in L_{loc}^{1}\left( 
%TCIMACRO{\U{211d} }%
%BeginExpansion
\mathbb{R}
%EndExpansion
\times \lbrack 0,+\infty )\right) ,g\in L_{loc}^{1}\left( 
%TCIMACRO{\U{211d} }%
%BeginExpansion
\mathbb{R}
%EndExpansion
\right) $. Questo giustifica la seguente definizione.

\textbf{Def} Date $f,u\in L_{loc}^{1}\left( 
%TCIMACRO{\U{211d} }%
%BeginExpansion
\mathbb{R}
%EndExpansion
\times \lbrack 0,+\infty )\right) ,g\in L_{loc}^{1}\left( 
%TCIMACRO{\U{211d} }%
%BeginExpansion
\mathbb{R}
%EndExpansion
\right) $, si dice che $u$ \`{e} soluzione debole del problema di Cauchy $%
\left\{ 
\begin{array}{c}
u_{t}+vu_{x}=f\text{ per }x\in 
%TCIMACRO{\U{211d} }%
%BeginExpansion
\mathbb{R}
%EndExpansion
,t>0 \\ 
u\left( x,0\right) =g\text{ per }x\in 
%TCIMACRO{\U{211d}}%
%BeginExpansion
\mathbb{R}%
%EndExpansion
\end{array}%
\right. $ se $\forall $ $\phi \in C_{0}^{1}\left( 
%TCIMACRO{\U{211d} }%
%BeginExpansion
\mathbb{R}
%EndExpansion
\times \lbrack 0,+\infty )\right) $ vale $\int_{0}^{+\infty }\int_{%
%TCIMACRO{\U{211d} }%
%BeginExpansion
\mathbb{R}
%EndExpansion
}\left[ u\left( x,t\right) \left( \phi _{t}+v\phi _{x}\right) \left(
x,t\right) +f\left( x,t\right) \phi \left( x,t\right) \right] dxdt+\int_{%
%TCIMACRO{\U{211d} }%
%BeginExpansion
\mathbb{R}
%EndExpansion
}\phi \left( x,0\right) g\left( x\right) dx=0$.

Si \`{e} effettivamente definito un concetto di soluzione che richiede ai
dati poca regolarit\`{a}. Tutti i passaggi sopra dimostrano quindi che se $u$
\`{e} soluzione classica allora \`{e} anche soluzione debole.

\textbf{Teo (una soluzione classica \`{e} anche soluzione debole)}%
\begin{eqnarray*}
\text{Hp} &\text{: }&g\in C^{1}\left( 
%TCIMACRO{\U{211d} }%
%BeginExpansion
\mathbb{R}
%EndExpansion
\right) ,f\in C^{0}\left( 
%TCIMACRO{\U{211d} }%
%BeginExpansion
\mathbb{R}
%EndExpansion
\times \lbrack 0,+\infty )\right) ,\exists \frac{\partial f}{\partial x}\in
C^{0}\left( 
%TCIMACRO{\U{211d} }%
%BeginExpansion
\mathbb{R}
%EndExpansion
\times \lbrack 0,+\infty )\right) \text{; }u\in C^{1}\left( 
%TCIMACRO{\U{211d} }%
%BeginExpansion
\mathbb{R}
%EndExpansion
\times \lbrack 0,+\infty )\right) \text{ \`{e} } \\
&&\text{soluzione classica di }\left\{ 
\begin{array}{c}
u_{t}+vu_{x}=f\text{ per }x\in 
%TCIMACRO{\U{211d} }%
%BeginExpansion
\mathbb{R}
%EndExpansion
,t>0 \\ 
u\left( x,0\right) =g\text{ per }x\in 
%TCIMACRO{\U{211d}}%
%BeginExpansion
\mathbb{R}%
%EndExpansion
\end{array}%
\right. \\
\text{Ts} &\text{: }&u\text{ \`{e} soluzione debole dello stesso problema}
\end{eqnarray*}

Viceversa, se i dati del problema sono molto regolari, una soluzione debole 
\`{e} anche classica.

\textbf{Teo (con sufficiente regolarit\`{a}, una soluzione debole \`{e}
anche soluzione classica)}%
\begin{eqnarray*}
\text{Hp}\text{: } &&g\in C^{1}\left( 
%TCIMACRO{\U{211d} }%
%BeginExpansion
\mathbb{R}
%EndExpansion
\right) ,f\in C^{0}\left( 
%TCIMACRO{\U{211d} }%
%BeginExpansion
\mathbb{R}
%EndExpansion
\times \lbrack 0,+\infty )\right) \text{; }u\in C^{1}\left( 
%TCIMACRO{\U{211d} }%
%BeginExpansion
\mathbb{R}
%EndExpansion
\times \lbrack 0,+\infty )\right) \text{ \`{e} } \\
&&\text{soluzione debole di }\left\{ 
\begin{array}{c}
u_{t}+vu_{x}=f\text{ per }x\in 
%TCIMACRO{\U{211d} }%
%BeginExpansion
\mathbb{R}
%EndExpansion
,t>0 \\ 
u\left( x,0\right) =g\text{ per }x\in 
%TCIMACRO{\U{211d}}%
%BeginExpansion
\mathbb{R}%
%EndExpansion
\end{array}%
\right. \\
\text{Ts}\text{: } &&u\text{ \`{e} soluzione classica dello stesso problema}
\end{eqnarray*}

Si noti che non occorre l'ipotesi $\frac{\partial f}{\partial x}\in
C^{0}\left( 
%TCIMACRO{\U{211d} }%
%BeginExpansion
\mathbb{R}
%EndExpansion
\times \lbrack 0,+\infty )\right) $.

\textbf{Dim} Applico la definizione di soluzione debole in particolare alle $%
\phi \in C_{0}^{1}\left( 
%TCIMACRO{\U{211d} }%
%BeginExpansion
\mathbb{R}
%EndExpansion
\times (0,+\infty )\right) $: quindi vale 
\begin{equation*}
\int_{0}^{+\infty }\int_{%
%TCIMACRO{\U{211d} }%
%BeginExpansion
\mathbb{R}
%EndExpansion
}\left[ u\left( x,t\right) \left( \phi _{t}+v\phi _{x}\right) \left(
x,t\right) +f\left( x,t\right) \phi \left( x,t\right) \right] dxdt+\int_{%
%TCIMACRO{\U{211d} }%
%BeginExpansion
\mathbb{R}
%EndExpansion
}\phi \left( x,0\right) g\left( x\right) dx=0 \forall \phi \in
C_{0}^{1}\left( 
%TCIMACRO{\U{211d} }%
%BeginExpansion
\mathbb{R}
%EndExpansion
\times (0,+\infty )\right)
\end{equation*}

In primo luogo occorre mostrare che $u_{t}+vu_{x}=f$ per $x\in 
%TCIMACRO{\U{211d} }%
%BeginExpansion
\mathbb{R}
%EndExpansion
,t>0$. Poich\'{e} $\phi \in C_{0}^{1}\left( 
%TCIMACRO{\U{211d} }%
%BeginExpansion
\mathbb{R}
%EndExpansion
\times (0,+\infty )\right) $, $\phi \left( x,0\right) =0$ e si ha $%
\int_{0}^{+\infty }\int_{%
%TCIMACRO{\U{211d} }%
%BeginExpansion
\mathbb{R}
%EndExpansion
}\left[ u\left( x,t\right) \left( \phi _{t}+v\phi _{x}\right) \left(
x,t\right) +f\left( x,t\right) \phi \left( x,t\right) \right] dxdt=0$. Per
la regolarit\`{a} di $u$ e $\phi $ si pu\`{o} integrare per parti
all'indietro: $\int_{0}^{+\infty }\int_{%
%TCIMACRO{\U{211d} }%
%BeginExpansion
\mathbb{R}
%EndExpansion
}u\left( x,t\right) \left( \phi _{t}+v\phi _{x}\right) \left( x,t\right)
dxdt=-\int_{0}^{+\infty }\int_{%
%TCIMACRO{\U{211d} }%
%BeginExpansion
\mathbb{R}
%EndExpansion
}\phi \left( x,t\right) \left( u_{t}+vu_{x}\right) \left( x,t\right) dxdt$,
quindi l'uguaglianza originaria diventa $\int_{0}^{+\infty }\int_{%
%TCIMACRO{\U{211d} }%
%BeginExpansion
\mathbb{R}
%EndExpansion
}\phi \left( x,t\right) \left( f\left( x,t\right) -\left(
u_{t}+vu_{x}\right) \left( x,t\right) \right) dxdt=0$ $\forall $ $\phi \in
C_{0}^{1}\left( 
%TCIMACRO{\U{211d} }%
%BeginExpansion
\mathbb{R}
%EndExpansion
\times \lbrack 0,+\infty )\right) $. Per il teorema di annullamento, essendo 
$f-\left( u_{t}+vu_{x}\right) \in C^{0}$, si ha $f\left( x,t\right) -\left(
u_{t}+vu_{x}\right) \left( x,t\right) =0$ in $%
%TCIMACRO{\U{211d} }%
%BeginExpansion
\mathbb{R}
%EndExpansion
\times (0,+\infty )$.

Mostro che $u\left( x,0\right) =g\left( x\right) $. Considerando in
particolare la funzioni test a supporto in $%
%TCIMACRO{\U{211d} }%
%BeginExpansion
\mathbb{R}
%EndExpansion
\times \lbrack 0,+\infty )$, stavolta si ha anche il termine di bordo: $%
\int_{0}^{+\infty }\int_{%
%TCIMACRO{\U{211d} }%
%BeginExpansion
\mathbb{R}
%EndExpansion
}\left[ u\left( \phi _{t}+v\phi _{x}\right) +f\phi \right] dxdt+\int_{%
%TCIMACRO{\U{211d} }%
%BeginExpansion
\mathbb{R}
%EndExpansion
}\phi \left( x,0\right) g\left( x\right) dx=0$. Integrando per parti si ha $%
\int_{0}^{+\infty }\int_{%
%TCIMACRO{\U{211d} }%
%BeginExpansion
\mathbb{R}
%EndExpansion
}u\left( x,t\right) \left( \phi _{t}+v\phi _{x}\right) \left( x,t\right)
dxdt=\int_{%
%TCIMACRO{\U{211d} }%
%BeginExpansion
\mathbb{R}
%EndExpansion
}\left[ u\left( x,t\right) \phi \left( x,t\right) \right] _{t=0}^{t=+\infty
}dx-\int_{0}^{+\infty }\int_{%
%TCIMACRO{\U{211d} }%
%BeginExpansion
\mathbb{R}
%EndExpansion
}\phi \left( x,t\right) \left( u_{t}+vu_{x}\right) \left( x,t\right) dxdt$.

Sostituendo si ottiene $\int_{0}^{+\infty }\int_{%
%TCIMACRO{\U{211d} }%
%BeginExpansion
\mathbb{R}
%EndExpansion
}\phi \left( f-\left( u_{t}+vu_{x}\right) \right) dxdt+\int_{%
%TCIMACRO{\U{211d} }%
%BeginExpansion
\mathbb{R}
%EndExpansion
}\phi \left( x,0\right) \left( g\left( x\right) -u\left( x,0\right) \right)
dx=0$ $\forall $ $\phi \in C_{0}^{1}\left( 
%TCIMACRO{\U{211d} }%
%BeginExpansion
\mathbb{R}
%EndExpansion
\times \lbrack 0,+\infty )\right) $. Il primo addendo \`{e} nullo per quanto
detto sopra: si ha quindi $\int_{%
%TCIMACRO{\U{211d} }%
%BeginExpansion
\mathbb{R}
%EndExpansion
}\phi \left( x,0\right) \left( g\left( x\right) -u\left( x,0\right) \right)
dx=0$ $\forall $ $\phi \in C_{0}^{1}\left( 
%TCIMACRO{\U{211d} }%
%BeginExpansion
\mathbb{R}
%EndExpansion
\times \lbrack 0,+\infty )\right) $, cio\`{e} come sopra $g\left( x\right)
=u\left( x,0\right) $. $\blacksquare $

Ora \`{e} naturale domandarsi se il concetto di soluzione debole abbia
effettivamente introdotto nuove soluzioni: cio\`{e} le soluzioni deboli
quando $u,f,g$ non sono molto regolari.

\textbf{Teo}%
\begin{eqnarray*}
\text{Hp}\text{: } &&g\in L^{\infty }\left( 
%TCIMACRO{\U{211d} }%
%BeginExpansion
\mathbb{R}
%EndExpansion
\right) ,f\in L^{\infty }\left( 
%TCIMACRO{\U{211d} }%
%BeginExpansion
\mathbb{R}
%EndExpansion
\times \lbrack 0,+\infty )\right) \text{ } \\
\text{Ts}\text{: } &&u\left( x,t\right) =g\left( x-vt\right)
+\int_{0}^{t}f\left( x-v\left( t-s\right) ,s\right) ds\text{ \`{e} soluzione
debole di }\left\{ 
\begin{array}{c}
u_{t}+vu_{x}=f\text{ per }x\in 
%TCIMACRO{\U{211d} }%
%BeginExpansion
\mathbb{R}
%EndExpansion
,t>0 \\ 
u\left( x,0\right) =g\text{ per }x\in 
%TCIMACRO{\U{211d}}%
%BeginExpansion
\mathbb{R}%
%EndExpansion
\end{array}%
\right.
\end{eqnarray*}

\section{Equazione delle onde}

Vedremo che l'equazione delle onde assume significati fisici molto diversi a
seconda della dimensione in cui si lavora: a differenza delle equazione gi%
\`{a} viste, non \`{e} affatto facile elaborare una teoria generale che
funzioni in dimensione qualsiasi.

\subsection{Equazione della corda vibrante}

\textbf{Modello fisico} Considero una corda sottile, elastica, tesa e
perfettamente flessibile che vibra in un piano verticale. Si indica con $%
u\left( x,t\right) $ la funzione il cui grafico, a $t$ fissato, rappresenta
la forma della corda. Si suppone che i punti della corda oscillino solo
verticalmente, e che le forze agenti siano solo la tensione della corda (la
tensione all'istante $t$ nel punto $\left( x,u\left( x,t\right) \right) $ si
indica con $\tau \left( x,t\right) $) e un carico dall'esterno, e. g. il
peso. Si indica con $\rho \left( x,t\right) $ la densit\`{a} lineare di
massa della corda, e con $\rho _{0}\left( x\right) $ la densit\`{a} a riposo 
$\rho \left( x,0\right) $.

Il vettore tensione $\tau \left( x,t\right) $ \`{e} tangente alla corda
grazie all'ipotesi di perfetta flessibilit\`{a}, e rappresenta la forza che
"il pezzo che sta a destra della corda (del punto$\left( x,u\left( x\right) \right) $%
??) esercita sul pezzo che sta a sinistra". Quindi la forza \textit{sub\`{\i}%
ta} dal punto $\left( x,u\left( x,t\right) \right) $ \`{e} $-\tau \left(
x,t\right) $.

Allora considero un tratto elementare $\left[ x,x+\Delta x\right] $ di
corda: per conservazione della massa nel tempo, la massa di tale tratto \`{e}
$dm=\rho _{0}\left( x\right) \Delta x=\rho \left( x,t\right) \Delta s$, dove 
$\Delta s$ indica la lunghezza del grafico di $u$ da $\left( x,u\left(
x\right) \right) $ a $\left( x+\Delta x,u\left( x+\Delta x\right) \right) $.
Ovviamente $\Delta s>\Delta x,\rho \left( x,t\right) <\rho _{0}\left(
x,t\right) $ (che \`{e} intuitivo: la corda viene "tirata"). Quindi $\rho
\left( x,t\right) =\rho _{0}\frac{\Delta x}{\Delta s}=\rho _{0}\cos \alpha
\left( x,t\right) $, dove $\alpha \left( x,t\right) $ \`{e} l'angolo tra
l'orizzontale e la retta tangente al grafico di $u$ nel punto $\left(
x,u\left( x,t\right) \right) $.

Per ricavare l'equazione delle onde vogliamo scrivere la seconda equazione
di Newton, scrivendo la risultante delle forze sul tratto $\left[ x,x+\Delta
x\right] $ sia in direzione verticale che in direzione orizzontale.

Per quanto detto sull'accelerazione dei punti della corda, $R_{hor}\mathbf{=}%
0$, ed essendo $\tau _{hor}\left( x,t\right) =\tau \left( x,t\right) \cos
\alpha \left( x,t\right) $, si ha $R_{hor}=\tau _{hor}\left( x+\Delta
x,t\right) -\tau _{hor}\left( x,t\right) =\mathbf{0}$, per cui $\tau
_{hor}\left( x+\Delta x,t\right) =\tau _{hor}\left( x,t\right) $: la
componente orizzontale della tensione \`{e} costante in $x$ e d'ora in poi
si indicher\`{a} con $\tau \left( t\right) =\tau \left( x,t\right) \cos
\alpha \left( x,t\right) $.

Per calcolare la risultante verticale si suppone che ci sia una forza per
unit\`{a} di massa $f\left( x,t\right) $ che agisce solo verticalmente; con $%
a\left( x,t\right) $ si indica l'accelerazione verticale. Allora la
risultante verticale, essendo $\tau _{ver}\left( x,t\right) =\tau \left(
x,t\right) \sin \alpha \left( x,t\right) $, \`{e} $a\left( x,t\right)
dm=\tau _{ver}\left( x+\Delta x,t\right) -\tau _{ver}\left( x,t\right)
+f\left( x,t\right) dm$. Sostituendo $dm=\rho _{0}\Delta x$ e $\tau \left(
x,t\right) =\frac{\tau _{hor}\left( x,t\right) }{\cos \alpha \left(
x,t\right) }$ si ottiene $\tau _{ver}\left( x,t\right) =\tau _{hor}\left(
x,t\right) \tan \alpha \left( x,t\right) =\tau _{hor}\left( x,t\right) \frac{%
\partial u}{\partial x}\left( x,t\right) $, da cui l'equazione $\frac{%
\partial ^{2}u}{\partial t^{2}}\left( x,t\right) \rho _{0}\Delta x=\tau
\left( t\right) \left( \frac{\partial u}{\partial x}\left( x+\Delta
x,t\right) -\frac{\partial u}{\partial x}\left( x,t\right) \right) +f\left(
x,t\right) \rho _{0}\Delta x$. Dividendo per $\rho _{0}\Delta x$ e
calcolando il limite per $\Delta x\rightarrow 0$ si ha infine $\frac{%
\partial ^{2}u}{\partial t^{2}}\left( x,t\right) =\frac{\tau \left( t\right) 
}{\rho _{0}}\frac{\partial ^{2}u}{\partial x^{2}}\left( x,t\right) +f\left(
x,t\right) $.

Se la corda a riposo \`{e} omogenea, $\rho _{0}$ \`{e} costante, ed \`{e}
anche ragionevole approssimare $\tau \left( t\right) $ a costante, qualora
la corda sia molto tesa ed essa oscilli di poco rispetto alla posizione di
equilibrio orizzontale. In questa situazione si indica con $c^{2}$ la
costante $\frac{\tau \left( t\right) }{\rho _{0}}$ (che dimensionalmente 
\`{e} una velocit\`{a}). Si ha quindi l'equazione della corda vibrante%
\begin{equation*}
\frac{\partial ^{2}u}{\partial t^{2}}-c^{2}\frac{\partial ^{2}u}{\partial
x^{2}}=f\left( x,t\right)
\end{equation*}

L'operatore delle onde $\frac{\partial ^{2}}{\partial t^{2}}-c^{2}\frac{%
\partial ^{2}}{\partial x^{2}}$ \`{e}, come l'operatore del calore,
invariante per traslazioni in $x$ e $t$ e per riflessioni in $x$; a
differenza dell'operatore del calore, \`{e} invariante anche per riflessioni
in $t$.

\textbf{Energia meccanica} Si calcola l'energia meccanica del tratto $\left[
0,L\right] $ della corda nel caso in cui non ci sia alcuna forza esterna.
Questo conto si riveler\`{a} utile in seguito.

L'energia cinetica elementare di un tratto $\left[ x,x+\Delta x\right] $ 
\`{e} $\frac{1}{2}dm\left( \frac{\partial u}{\partial t}\right) ^{2}=\frac{1%
}{2}\left( \frac{\partial u}{\partial t}\right) ^{2}\rho _{0}\Delta x$,
quindi l'energia cinetica totale \`{e} $E_{cin}\left( t\right) =\frac{1}{2}%
\rho _{0}\int_{0}^{L}u_{t}^{2}\left( x,t\right) dx$.

L'energia potenziale elementare \`{e} il lavoro della forza di tensione:
essendo l'allungamento e la tensione diretti parallelamente, si ha $%
\left\vert \tau \right\vert \left( \Delta s-\Delta x\right) =\sqrt{\tau
_{hor}^{2}+\tau _{ver}^{2}}\left( \sqrt{\Delta x^{2}+\Delta y^{2}}-\Delta
x\right) =\tau \sqrt{1+u_{x}^{2}}\Delta x\left( \sqrt{1+\left( \frac{\Delta y%
}{\Delta x}\right) ^{2}}-1\right) $. Se si continua a supporre che la corda
vibri con piccole oscillazioni rispetto all'orizzontale, si ha $\left\vert 
\frac{\partial u}{\partial x}\right\vert <<1$ e $\sqrt{1+u_{x}^{2}}%
\simeq 1$, per cui $\sqrt{1+u_{x}^{2}}-1\simeq \frac{1}{2}u_{x}^{2}$. Quindi
l'energia potenziale elementare \`{e} $\tau \Delta x\frac{1}{2}u_{x}^{2}$, e
l'energia potenziale totale \`{e} $E_{pot}\left( t\right) =\frac{1}{2}\tau
\int_{0}^{L}u_{x}^{2}\left( x,t\right) dx$.

L'energia meccanica totale \`{e} $E_{mecc}\left( t\right) =\frac{1}{2}\rho
_{0}\int_{0}^{L}u_{t}^{2}\left( x,t\right) dx+\frac{1}{2}\tau
\int_{0}^{L}u_{x}^{2}\left( x,t\right) dx=\frac{1}{2}\rho
_{0}\int_{0}^{L}\left( u_{t}^{2}+c^{2}u_{x}^{2}\right) \left( x,t\right) dx$.

\subsubsection{Problemi al contorno e ai valori iniziali per l'equazione
della corda vibrante}

Ci sono alcuni tipi di problemi che \`{e} naturale studiare per l'equazione
della corda vibrante.

Il problema di Cauchy globale \`{e} $\left\{ 
\begin{array}{c}
u_{tt}-c^{2}u_{xx}=f\text{ per }x\in 
%TCIMACRO{\U{211d} }%
%BeginExpansion
\mathbb{R}
%EndExpansion
,t>0 \\ 
u\left( x,0\right) =g\left( x\right) \text{ per }x\in 
%TCIMACRO{\U{211d} }%
%BeginExpansion
\mathbb{R}
%EndExpansion
\\ 
u_{t}\left( x,0\right) =h\left( x\right) \text{ per }x\in 
%TCIMACRO{\U{211d}}%
%BeginExpansion
\mathbb{R}%
%EndExpansion
\end{array}%
\right. $: essendo l'equazione del second'ordine in $t$, si deve imporre la
condizione iniziale sia su $u$ che su $u_{t}$.

Il problema di Cauchy su un intervallo con condizioni agli estremi \`{e} $%
\left\{ 
\begin{array}{c}
u_{tt}-c^{2}u_{xx}=f\text{ per }x\in \left( 0,L\right) ,t>0 \\ 
u\left( x,0\right) =g\left( x\right) \text{ per }x\in \left( 0,L\right) \\ 
u_{t}\left( x,0\right) =h\left( x\right) \text{ per }x\in \left( 0,L\right)
\\ 
\text{condizioni agli estremi}%
\end{array}%
\right. $, dove le condizioni agli estremi naturali sono del tipo Dirichlet
omogenee ($u\left( 0,t\right) =u\left( L,t\right) =0$, che ha il significato
fisico di corda fissata agli estremi) oppure Neumann omogenee ($u_{x}\left(
0,t\right) =u_{x}\left( L,t\right) =0$, che ha il significato fisico
alquanto irrealistico di corda con estremi liberi di scorrere su due guide
verticali).

\textbf{Teo (unicit\`{a} per il problema di Cauchy-Dirichlet o
Cauchy-Neumann sul segmento)}%
\begin{gather*}
\text{Hp: }\left\{ 
\begin{array}{c}
u_{tt}-c^{2}u_{xx}=f\text{ per }x\in \left( 0,L\right) ,t\in \left(
0,T\right) \\ 
u\left( x,0\right) =g\left( x\right) \text{ per }x\in \left( 0,L\right) \\ 
u_{t}\left( x,0\right) =h\left( x\right) \text{ per }x\in \left( 0,L\right)
\\ 
u\left( 0,t\right) =u\left( L,t\right) =0/u_{x}\left( 0,t\right)
=u_{x}\left( L,t\right) =0\text{ per }t\in \left( 0,T\right)%
\end{array}%
\right. \\
\text{Ts: se esiste, la soluzione del problema \`{e} unica nella classe }%
C^{2}\left( \left( 0,L\right) \times \left( 0,T\right) \right) \cap
C^{1}\left( \left[ 0,L\right] \times \left[ 0,T\right] \right) \text{ }
\end{gather*}

L'unicit\`{a} si estende al caso di $T=+\infty $.

\textbf{Dim} Siano $u_{1},u_{2}$ soluzioni del problema nella classe $%
C^{2}\left( \left( 0,L\right) \times \left( 0,T\right) \right) \cap
C^{1}\left( \left[ 0,L\right] \times \left[ 0,T\right] \right) $. Allora $%
u=u_{1}-u_{2}$ \`{e} nella stessa classe e risolve il problema con $f=g=h=0$.

Considero l'energia meccanica $E\left( t\right) =\frac{1}{2}\rho
_{0}\int_{0}^{L}\left( u_{t}^{2}+c^{2}u_{x}^{2}\right) \left( x,t\right) dx$%
: per il teorema di continuit\`{a} di una funzione definita mediante
integrale dipendente da un parametro, $E\in C^{1}\left( 0,T\right) \cap
C^{0}\left( \left[ 0,T\right] \right) $. La strategia dimostrativa \`{e}
mostrare che (1) $E$ che \`{e} costante, (2) tale costante \`{e} $0$, (3) $u$
\`{e} nulla.

(1) $E^{\prime }\left( t\right) =\frac{1}{2}\rho _{0}\int_{0}^{L}\left(
2u_{t}u_{tt}+2c^{2}u_{x}u_{xt}\right) \left( x,t\right) dx$: la derivazione
sotto il segno di integrale \`{e} giustificata dal fatto che le derivate
prime e seconde sono continue in $\left[ 0,L\right] $ (in realt\`{a} quelle
seconde in $\left( 0,L\right) $...) e dunque limitate da una costante
integrabile. Integro per parti $\int_{0}^{L}u_{x}u_{xt}dx=\left[ u_{x}u_{t}%
\right] _{x=0}^{x=L}-\int_{0}^{L}u_{xx}u_{t}dx$. Se la condizione agli
estremi \`{e} di Neumann, $\left[ u_{x}u_{t}\right] _{x=0}^{x=L}=\left(
u_{x}u_{t}\right) \left( L,t\right) -\left( u_{x}u_{t}\right) \left(
0,t\right) =0$; se la condizione \`{e} di Dirichlet, si ha $u\left(
0,t\right) =u\left( L,t\right) =0$ $\forall $ $t\in \left( 0,T\right) $,
dunque $u_{t}\left( 0,t\right) =u_{t}\left( L,t\right) =0$ $\forall $ $t\in
\left( 0,T\right) $ e $\left( u_{x}u_{t}\right) \left( L,t\right) -\left(
u_{x}u_{t}\right) \left( 0,t\right) =0$. Perci\`{o} in entrambi i casi $%
\int_{0}^{L}u_{x}u_{xt}dx=-\int_{0}^{L}u_{xx}u_{t}dx$ e $E^{\prime }\left(
t\right) =\rho _{0}\int_{0}^{L}u_{t}\left( u_{tt}-c^{2}u_{xx}\right) \left(
x,t\right) dx=0$ perch\'{e} $u$ risolve l'equazione omogenea. Quindi $%
E\left( t\right) =E\left( 0\right) $ $\forall $ $t$.

(2) $E\left( 0\right) =\frac{1}{2}\rho _{0}\int_{0}^{L}\left(
u_{t}^{2}\left( x,0\right) +c^{2}u_{x}^{2}\left( x,0\right) \right) dx$. Per
le condizioni iniziali $u_{t}\left( x,0\right) =0$; inoltre $u\left(
x,0\right) =0$ $\forall $ $x$, quindi $u_{x}\left( x,0\right) =0$ $\forall $ 
$x$ e $E\left( 0\right) =0$.

(3) Quindi $E\left( t\right) =\frac{1}{2}\rho _{0}\int_{0}^{L}\left(
u_{t}^{2}+c^{2}u_{x}^{2}\right) \left( x,t\right) dx=0$ $\forall $ $t\in
\left( 0,T\right) $: fissato $t$, si ha che l'integrale di una funzione
continua nonnegativa \`{e} nullo, il che implica che la funzione integranda 
\`{e} nulla $\forall $ $x$. Allora $\left( u_{t}^{2}+c^{2}u_{x}^{2}\right)
\left( x,t\right) =0$ $\forall $ $t,x$, e dunque $u$ \`{e} costante sia in $%
y $ che in $x$; poich\'{e} $u\left( x,0\right) =0$, tale costante \`{e} $0$
e $u\left( x,t\right) =0$ $\forall $ $x\in \left( 0,L\right) ,\forall $ $%
t\in \left( 0,T\right) $. Dunque $u_{1}=u_{2}$. $\blacksquare $

La tecnica dimostrativa che utilizza l'integrale dell'energia si ritrover%
\`{a}, indipendentemente dal suo significato fisico di energia meccanica.

\subsubsection{Corda vibrante fissata agli estremi}

Considero il problema di Cauchy per l'equazione omogenea sul segmento con
condizioni di Dirichlet omogenee e condizioni di Cauchy:%
\begin{equation*}
\left\{ 
\begin{array}{c}
u_{tt}-c^{2}u_{xx}=0\text{ per }x\in \left( 0,L\right) ,t>0 \\ 
u\left( 0,t\right) =u\left( L,t\right) =0\text{ per }t>0 \\ 
u\left( x,0\right) =g\left( x\right) \text{ per }x\in \left( 0,L\right) \\ 
u_{t}\left( x,0\right) =h\left( x\right) \text{ per }x\in \left( 0,L\right)%
\end{array}%
\right.
\end{equation*}

Si risolve l'equazione per separazione di variabili: si suppone $u\left(
x,t\right) =X\left( x\right) T\left( t\right) $ e si risolve $XT^{\prime
\prime }-c^{2}X^{\prime \prime }T=0$: dividendo per $XT$ si ottiene $\frac{%
T^{\prime \prime }}{c^{2}T}=\frac{X^{\prime \prime }}{X}$. Due funzioni di
due variabili diverse possono essere uguali $\forall $ $x\in \left(
0,L\right) ,\forall $ $t>0$ se e solo se hanno entrambi un valore costante $%
\lambda \in 
%TCIMACRO{\U{211d} }%
%BeginExpansion
\mathbb{R}
%EndExpansion
$. Allora si risolvono due problemi disaccoppiati. Il primo, con le
condizioni agli estremi, \`{e} $\left\{ 
\begin{array}{c}
X^{\prime \prime }\left( x\right) -\lambda X\left( x\right) =0\text{ per }%
x\in \left( 0,L\right) \\ 
X\left( 0\right) =X\left( L\right) =0%
\end{array}%
\right. $, con $\lambda $ da determinare: se ne cerca una soluzione $X$ non
identicamente nulla. E' esattamente lo stesso problema visto per l'equazione
di Cauchy sul segmento: si trova la successione di soluzioni $X_{n}\left(
t\right) =\sin \frac{n\pi x}{L}$ con $\lambda _{n}=-\left( \frac{n\pi }{L}%
\right) ^{2}$.

Sostituendo $\lambda _{n}$ in $T^{\prime \prime }\left( t\right) =\lambda
c^{2}T\left( t\right) $ si ha $T^{\prime \prime }\left( t\right) =-\left( 
\frac{n\pi }{L}\right) ^{2}c^{2}T\left( t\right) $, per cui $T_{n}\left(
t\right) =a_{n}\cos \frac{n\pi c}{L}t+b_{n}\sin \frac{n\pi c}{L}t$.

Si \`{e} trovata una successione di soluzione $u_{n}\left( x,t\right) =\sin 
\frac{n\pi x}{L}\left( a_{n}\cos \frac{n\pi c}{L}t+b_{n}\sin \frac{n\pi c}{L}%
t\right) $. Ciascuna delle $u_{n}$ ha il significato fisico di vibrazione
stazionaria o armonica elementare, come si spiegher\`{a} meglio pi\`{u}
avanti.

Come al solito, per avere qualche speranza di soddisfare le condizioni
iniziali, con una scelta opportuna di $a_{n}$ e $b_{n}$, si considera la
serie $u\left( x,t\right) =\sum_{n=1}^{+\infty }u_{n}\left( x,t\right)
=\sum_{n=1}^{+\infty }\sin \frac{n\pi x}{L}\left( a_{n}\cos \frac{n\pi c}{L}%
t+b_{n}\sin \frac{n\pi c}{L}t\right) $.

$u\left( x,0\right) =\sum_{n=1}^{+\infty }a_{n}\sin \frac{n\pi x}{L}$:
affinch\'{e} questa sia la serie di $g$ in $\left( 0,L\right) $ si deve
sviluppare $g$ in serie di soli seni, per cui i coefficienti $a_{n}$ sono i
coefficienti di Fourier $\alpha _{n}$ della simmetrizzata dispari di $g$ in $%
\left( -L,L\right) $. Quindi $a_{n}=\frac{2}{L}\int_{0}^{L}g\left( x\right)
\sin \frac{n\pi x}{L}dx$.

$u_{t}\left( x,0\right) =\sum_{n=1}^{+\infty }b_{n}\frac{n\pi c}{L}\sin 
\frac{n\pi x}{L}$, che \`{e} simile alla forma di $h$ sviluppata in serie di
soli seni: $h\left( x\right) =\sum_{n=1}^{+\infty }\beta _{n}\sin \frac{n\pi
x}{L}$. E' quindi sufficiente imporre $\frac{n\pi c}{L}b_{n}=\beta _{n}$, da
cui $b_{n}=\frac{L}{n\pi c}\beta _{n}$.

Dunque la candidata soluzione \`{e} 
\begin{equation*}
u\left( x,t\right) =\sum_{n=1}^{+\infty }\sin \frac{n\pi x}{L}\left(
a_{n}\cos \frac{n\pi c}{L}t+b_{n}\sin \frac{n\pi c}{L}t\right)
\end{equation*}

dove $a_{n}$ sono i coefficienti di Fourier della simmetrizzata dispari di $%
g $ e $b_{n}=\frac{L}{n\pi c}\beta _{n}$ con $\beta _{n}$ coefficienti di
Fourier della simmetrizzata dispari di $h$.

\textbf{Analisi critica} Le altre volte in cui si \`{e} utilizzata questa
tecnica si aveva una funzione potenza oppure esponenziale che aiutava la
convergenza della serie candidata; in questo caso si ha per la prima volta
un'altra funzione trigonometrica. Questo significa che, ammesso che i
coefficienti di Fourier esistano, non c'\`{e} speranza di far convergere la
serie senza fare ipotesi su $a_{n}$ e $b_{n}$, e non aiuta neanche fare
distinzioni tra intervallo aperto e chiuso. Quindi, se $\sum_{n=1}^{+\infty
}\left( \left\vert a_{n}\right\vert +\left\vert b_{n}\right\vert \right)
<+\infty $, la serie che assegna $u$ converge totalmente: $u$ \`{e} ben
definita e continua in $\left[ 0,L\right] \times \left[ 0,T\right] $.
Ovviamente questo non basta: per verificare che $u$ risolva l'equazione
chiediamo anche $u\in C^{2}\left( \left[ 0,L\right] \times \left[ 0,T\right]
\right) $. Ci\`{o} \`{e} vero se $\sum_{n=1}^{+\infty }n^{2}\left(
\left\vert a_{n}\right\vert +\left\vert b_{n}\right\vert \right) <+\infty $,
e in tal caso $u\in C^{2}\left( \left[ 0,L\right] \times \left[ 0,T\right]
\right) $ \`{e} soluzione classica; non c'\`{e} speranza di avere una
soluzione dell'equazione che assuma il dato al bordo in senso $L^{2}$. In
termini di coefficienti di Fourier di $g$ e $h$, si sta chiedendo che $%
\sum_{n=1}^{+\infty }n^{2}\alpha _{n}<+\infty ,\sum_{n=1}^{+\infty }n\beta
_{n}<+\infty $.

Per ottenere ci\`{o}, quali ipotesi occorrono sulle riflesse dispari di $g$
e $h$? Serve un teorema che, con qualche ipotesi sulla regolarit\`{a} della
funzione, dica con quale rapidit\`{a} tendono a $0$ i suoi coefficienti di
Fourier. Non basta trovare ipotsi che permettano di dire che $\alpha
_{n}=o\left( \frac{1}{n^{2}}\right) $: infatti, se $\sum_{n=1}^{+\infty
}n^{2}\alpha _{n}<+\infty $, $\alpha _{n}=o\left( \frac{1}{n^{2}}\right) $,
ma il viceversa non \`{e} vero (ma con $n^{3}$); ci serve un teorema pi\`{u}
preciso.

\textbf{Teo (rapidit\`{a} di convergenza a zero dei coefficienti di Fourier)}%
\begin{gather*}
\text{Hp: }\exists \text{ }s\in 
%TCIMACRO{\U{2115} }%
%BeginExpansion
\mathbb{N}
%EndExpansion
:f\in C^{s}\left( \left[ 0,L\right] \right) \text{; }f^{\left( s\right) }%
\text{ \`{e} regolare a tratti in }\left[ 0,L\right] \text{;} \\
f\left( 0\right) =f\left( L\right) ,f^{\prime }\left( 0\right) =f^{\prime
}\left( L\right) ,...,f^{\left( s\right) }\left( 0\right) =f^{\left(
s\right) }\left( L\right) \text{; }a_{n},b_{n}\text{ sono i coefficienti di
Fourier di }f \\
\text{Ts: }\sum_{n=1}^{+\infty }n^{s}\left( \left\vert a_{n}\right\vert
+\left\vert b_{n}\right\vert \right) <+\infty
\end{gather*}


Applico il teorema alla riflessa dispari di $g$ per $s=2$ e alla riflessa
dispari di $h$ per $s=1$. Se $g^{\ast }\in C^{2}\left( \left[ -L,L\right]
\right) $ (e $g^{\ast \prime \prime }$ \`{e} regolare a tratti??), $g^{\ast
}\left( -L\right) =g^{\ast }\left( L\right) ,g^{\ast \prime }\left(
-L\right) =g^{\ast \prime }\left( L\right) ,g^{\ast \prime \prime }\left(
-L\right) =g^{\ast \prime \prime }\left( L\right) $. Poich\'{e} $g^{\ast }$ 
\`{e} dispari, $g^{\ast \prime }$ \`{e} pari e la seconda condizione di
raccordo \`{e} sicuramente vera.

Queste richieste si riflettono nelle seguenti richieste su $g$: $g\in
C^{2}\left( \left[ 0,L\right] \right) $, $g\left( 0\right) =g^{\prime \prime
}\left( 0\right) =0$, $g^{\prime \prime }$ regolare a tratti, $g\left(
L\right) =0=g^{\prime \prime }\left( L\right) $.

Analogamente, si chiede $h\in C^{1}\left( \left[ 0,L\right] \right) $, $%
h\left( 0\right) =0$, $h^{\prime }$ regolare a tratti, $h\left( L\right) =0$.

Sotto queste ipotesi $u$ assegnata dalla serie \`{e} $C^{2}\left( \left[ 0,L%
\right] \times \left[ 0,T\right] \right) $ e risolve l'equazione in senso
classico: questo \`{e} un teorema di esistenza della soluzione (che ha
ipotesi molto pi\`{u} forti di quelle del teorema di unicit\`{a}). Tale
soluzione \`{e} unica, perch\'{e} rientra nella classe oggetto del teorema
di unicit\`{a}.

Si noti che non c'\`{e} nessun segnale che porti a pensare di poter
indebolire le ipotesi su $g$ e $h$: come l'equazione del calore, l'equazione
della corda vibrante non regolarizza. Infatti in questo caso, a differenza
di quanto visto per l'equazione di Laplace sul cerchio e l'equazione del
calore sul segmento, l'uso delle serie di Fourier non \`{e} conseguenza
della scelta della tecnica risolutiva: sono lo strumento privilegiato per
comprendere il fenomeno.

Nel prossimo teorema e nel successivo ragionamento si suppone di essere
nelle suddette ipotesi su $g$ e $h$.

\textbf{Teo (stima di stabilit\`{a})}%
\begin{eqnarray*}
\text{Hp}\text{: } &&u\text{ \`{e} soluzione classica del problema omogeneo
di Cauchy-Dirichlet per la corda vibrante} \\
\text{Ts}\text{: } &&\left\vert \left\vert u\left( \cdot ,t\right)
\right\vert \right\vert _{L^{2}\left( 0,L\right) }^{2}\leq \left\vert
\left\vert g\right\vert \right\vert _{L^{2}\left( 0,L\right) }^{2}+\left( 
\frac{L}{\pi c}\right) ^{2}\left\vert \left\vert h\right\vert \right\vert
_{L^{2}\left( 0,L\right) }^{2}\text{ }\forall \text{ }t>0
\end{eqnarray*}

Si noti che se $u$ \`{e} soluzione classica allora $u\left( \cdot ,t\right)
\in C^{2}\left( \left[ 0,L\right] \right) $, per cui $u\left( \cdot
,t\right) $ \`{e} sicuramente $L^{2}\left( 0,L\right) $. Con le ipotesi
viste su $h,g$ sicuramente anche loro sono $L^{2}$.

Poich\'{e} la tesi vale $\forall $ $t>0$, si pu\`{o} passare al sup: $%
\sup_{t>0}\left\vert \left\vert u\left( \cdot ,t\right) \right\vert
\right\vert _{L^{2}\left( 0,L\right) }^{2}\leq \left\vert \left\vert
g\right\vert \right\vert _{L^{2}\left( 0,L\right) }^{2}+\left( \frac{L}{\pi c%
}\right) ^{2}\left\vert \left\vert h\right\vert \right\vert _{L^{2}\left(
0,L\right) }^{2}$.

\textbf{Dim} Sia $t$ fissato. $\left\vert \left\vert u\left( \cdot ,t\right)
\right\vert \right\vert _{L^{2}\left( 0,L\right)
}^{2}=\int_{0}^{L}\left\vert u\left( x,t\right) \right\vert ^{2}dx$: per il
teorema di Pitagora in $L^{2}\left( 0,L\right) $, quanto scritto \`{e}
uguale a $\sum_{n=1}^{+\infty }\left( a_{n}\cos \frac{n\pi c}{L}t+b_{n}\sin 
\frac{n\pi c}{L}t\right) ^{2}\left\vert \left\vert \sin \frac{n\pi x}{L}%
\right\vert \right\vert _{L^{2}}^{2}$: la norma quadrata del seno \`{e} $%
\frac{L}{2}$ e $\left\vert a_{n}\cos \frac{n\pi c}{L}t+b_{n}\sin \frac{n\pi c%
}{L}t\right\vert \leq \sqrt{a_{n}^{2}+b_{n}^{2}}$ per Cauchy-Schwarz, quindi 
$\left\vert \left\vert u\left( \cdot ,t\right) \right\vert \right\vert
_{L^{2}\left( 0,L\right) }^{2}\leq \frac{L}{2}\sum_{n=1}^{+\infty }\left(
a_{n}^{2}+b_{n}^{2}\right) $. Ma $g\left( x\right)
=^{L^{2}}\sum_{n=1}^{+\infty }a_{n}\sin \frac{n\pi c}{L}x$, perci\`{o} $%
\left\vert \left\vert g\right\vert \right\vert _{L^{2}}^{2}=\frac{L}{2}%
\sum_{n=1}^{+\infty }a_{n}^{2}$; $h\left( x\right)
=^{L^{2}}\sum_{n=1}^{+\infty }\frac{cn\pi }{L}b_{n}\sin \frac{n\pi c}{L}x$,
perci\`{o} $\left\vert \left\vert h\right\vert \right\vert _{L^{2}}^{2}=%
\frac{L}{2}\sum_{n=1}^{+\infty }\left( \frac{cn\pi }{L}\right)
^{2}b_{n}^{2}\geq \frac{L}{2}\sum_{n=1}^{+\infty }\left( \frac{c\pi }{L}%
\right) ^{2}b_{n}^{2}$. Si ha infine $\left\vert \left\vert u\left( \cdot
,t\right) \right\vert \right\vert _{L^{2}\left( 0,L\right) }^{2}\leq
\left\vert \left\vert g\right\vert \right\vert _{L^{2}}^{2}+\left( \frac{L}{%
c\pi }\right) ^{2}\left\vert \left\vert h\right\vert \right\vert
_{L^{2}}^{2} $. $\blacksquare $

Coerentemente con quanto detto sul fatto che le serie di Fourier sono lo
strumento privilegiato per comprendere il fenomeno, anche i singoli addendi
della serie soluzione hanno un importante significato fisico. $u_{n}\left(
x,t\right) =\sin \frac{n\pi x}{L}\left( a_{n}\cos \frac{n\pi c}{L}%
t+b_{n}\sin \frac{n\pi c}{L}t\right) $ (che, si ricorda, si pu\`{o} anche
scrivere come $\sin \frac{n\pi x}{L}\sqrt{a_{n}^{2}+b_{n}^{2}}\cos \left( 
\frac{n\pi c}{L}t+\phi \right) $): per ogni $x$ fissato $u_{n}\left(
x,t\right) $ \`{e} una funzione trigonometrica, quindi ogni punto della
corda oscilla in modo periodico con $\omega =\frac{\pi c}{L},T=\frac{2L}{nc}=%
\frac{T_{0}}{n},f=\frac{nc}{2L}=n\nu _{0}$ ($\nu _{0}$ si dice frequenza
fondamentale), ampiezza $\sqrt{a_{n}^{2}+b_{n}^{2}}$. $u_{1},...,u_{n}$ si
dicono armoniche successive, e ogni $u_{n}$ \`{e} una vibrazione stazionaria.

Ogni $u_{n}$ ha inoltre $n+1$ punti fissati che non vibrano nel tempo, detti
nodi, che sono gli $x$ che annullano $u_{n}$ (tra questi ci sono anche gli
estremi dell'intervallo, fissati a $0$ dalla condizione al bordo): $\frac{%
n\pi x}{L}=m\pi \Longleftrightarrow x=\frac{mL}{n}$, con $m=0,1,..,n$.

Poich\'{e} $u_{1}$ ha periodo $T_{0}$ e frequenza $\nu _{0}$, $u_{2}$ $\frac{%
T_{0}}{2}$ e $2\nu _{0}$, ecc., tutta la $u$, somma delle $u_{n}$, ha
periodo $T_{0}$ (e frequenza $\nu _{0}$): la vibrazione complessiva della
corda fissata agli estremi $u$ \`{e} periodica. REC

\subsubsection{Corda vibrante illimitata}

La prima trattazione di questo problema \`{e} dovuta a D'Alembert (1750).
Considero il problema di Cauchy per l'equazione omogenea sulla retta con
condizioni di Cauchy:%
\begin{equation*}
\left\{ 
\begin{array}{c}
u_{tt}-c^{2}u_{xx}=0\text{ per }x\in 
%TCIMACRO{\U{211d} }%
%BeginExpansion
\mathbb{R}
%EndExpansion
,t>0 \\ 
u\left( x,0\right) =g\left( x\right) \text{ per }x\in 
%TCIMACRO{\U{211d} }%
%BeginExpansion
\mathbb{R}
%EndExpansion
\\ 
u_{t}\left( x,0\right) =h\left( x\right) \text{ per }x\in 
%TCIMACRO{\U{211d}}%
%BeginExpansion
\mathbb{R}%
%EndExpansion
\end{array}%
\right.
\end{equation*}

Per risolvere l'equazione si usa il metodo del cambio di variabili di
D'Alembert: si definisco $\xi ,\eta $ come $\left\{ 
\begin{array}{c}
\xi \left( x,t\right) =x+ct \\ 
\eta \left( x,t\right) =x-ct%
\end{array}%
\right. $. Allora $u\left( x,t\right) =u\left( \frac{\xi -\eta }{2},\frac{%
\eta +\xi }{2c}\right) =v\left( \xi \left( x,t\right) ,\eta \left(
x,t\right) \right) $. Con abuso di notazione, si indica con $u$ la funzione $%
v$. Allora per la regola di derivazione della funzione composta $u_{x}=\frac{%
\partial u}{\partial \xi }\frac{\partial \xi }{\partial x}+\frac{\partial u}{%
\partial \eta }\frac{\partial \eta }{\partial x}=\frac{\partial u}{\partial
\xi }+\frac{\partial u}{\partial \eta }$; il quadrato operatoriale \`{e} $%
\frac{\partial ^{2}}{\partial x^{2}}=\left( \frac{\partial u}{\partial \xi }+%
\frac{\partial u}{\partial \eta }\right) \left( \frac{\partial u}{\partial
\xi }+\frac{\partial u}{\partial \eta }\right) =\frac{\partial ^{2}u}{%
\partial \xi ^{2}}+\frac{\partial ^{2}u}{\partial \eta ^{2}}+2\frac{\partial
^{2}u}{\partial \xi \partial \eta }$. Analogamente $u_{t}=c\frac{\partial u}{%
\partial \xi }-c\frac{\partial u}{\partial \eta }$ e $\frac{\partial ^{2}}{%
\partial t^{2}}=c^{2}\frac{\partial ^{2}u}{\partial \xi ^{2}}+c^{2}\frac{%
\partial ^{2}u}{\partial \eta ^{2}}-2c^{2}\frac{\partial ^{2}u}{\partial \xi
\partial \eta }$. Quindi $u_{tt}-c^{2}u_{xx}=-4c^{2}\frac{\partial ^{2}u}{%
\partial \xi \partial \eta }=0$: l'equazione \`{e} diventata molto pi\`{u}
semplice. Si nota che $\frac{\partial ^{2}u}{\partial \xi \partial \eta }%
=0\Longleftrightarrow \frac{\partial }{\partial \xi }\left( \frac{\partial u%
}{\partial \eta }\right) =0$, cio\`{e} $\frac{\partial u}{\partial \eta }%
=F_{1}\left( \eta \right) $ con $F_{1}$ qualsiasi: perci\`{o} $u\left( \xi
,\eta \right) =\int F_{1}\left( \eta \right) d\eta +G\left( \xi \right)
=F\left( \eta \right) +G\left( \xi \right) $ con $F,G$ qualsiasi. Cambiando
di nuovo variabili, $u\left( x,t\right) =F\left( x+ct\right) +G\left(
x-ct\right) $: si \`{e} mostrato che $u$ risolve l'equazione se e solo se $u$
ha tale forma.

L'integrale generale dell'equazione della corda vibrante sulla retta \`{e}
dunque $u\left( x,t\right) =F\left( x+ct\right) +G\left( x-ct\right) $ al
variare di $F,G\in C^{2}\left( 
%TCIMACRO{\U{211d} }%
%BeginExpansion
\mathbb{R}
%EndExpansion
\right) $ (quest'ultima richiesta serve affinch\'{e} $u$ sia soluzione
classica).

Si noti che, fissato $t$, $F\left( x+ct\right) $ \`{e} una traslata
all'indietro di $F$, mentre $G\left( x-ct\right) $ \`{e} una traslata in
avanti di $G$: si dice quindi che $u$ \`{e} somma di un'onda regressiva e
una progressiva.

Si impongono le condizioni iniziali: $u\left( x,0\right) =F\left( x\right)
+G\left( x\right) =g\left( x\right) $, $u_{t}\left( x,0\right) =c\left(
F^{\prime }\left( x\right) -G^{\prime }\left( x\right) \right) =h\left(
x\right) $, da cui il sistema differenziale $\left\{ 
\begin{array}{c}
F+G=g \\ 
F^{\prime }-G^{\prime }=\frac{h}{c}%
\end{array}%
\right. $ nelle incognite $F,G$. Sommando la prima equazione derivata alla
seconda si ha $F^{\prime }=\frac{g^{\prime }+h/c}{2}$, e quindi $G^{\prime }=%
\frac{g^{\prime }-h/c}{2}$. Allora $F\left( x\right) =\frac{1}{2}g\left(
x\right) +\frac{1}{2c}\int_{0}^{x}h\left( y\right) dy+c_{1}$, $G\left(
x\right) =\frac{1}{2}g\left( x\right) -\frac{1}{2c}\int_{0}^{x}h\left(
y\right) dy+c_{1}$: $F+G=g+c_{1}+c_{2}=g\Longleftrightarrow c_{1}+c_{2}=0$ (%
\`{e} soddisfatta anche la prima equazione, derivando la quale si era persa
parte dell'informazione).

Dunque la soluzione del problema \`{e}%
\begin{equation*}
u\left( x,t\right) =F\left( x+ct\right) +G\left( x-ct\right) =\frac{g\left(
x+ct\right) +g\left( x-ct\right) }{2}+\frac{1}{2c}\int_{x-ct}^{x+ct}h\left(
y\right) dy
\end{equation*}

che si dice formula di D'Alembert, sotto le ipotesi di $g\in C^{2}\left( 
%TCIMACRO{\U{211d} }%
%BeginExpansion
\mathbb{R}
%EndExpansion
\right) ,h\in C^{1}\left( 
%TCIMACRO{\U{211d} }%
%BeginExpansion
\mathbb{R}
%EndExpansion
\right) $ (la funzione integrale concede ad $h$ un grado di regolarit\`{a}
in meno). Abbiamo dimostrato che esiste ed \`{e} unica la soluzione al
problema della corda vibrante.

\textbf{Teo (esistenza e unicit\`{a} della soluzione del problema della
corda vibrante sulla retta)}%
\begin{gather*}
\text{Hp: }g\in C^{2}\left( 
%TCIMACRO{\U{211d} }%
%BeginExpansion
\mathbb{R}
%EndExpansion
\right) ,h\in C^{1}\left( 
%TCIMACRO{\U{211d} }%
%BeginExpansion
\mathbb{R}
%EndExpansion
\right) \text{, }\left\{ 
\begin{array}{c}
u_{tt}-c^{2}u_{xx}=0\text{ per }x\in 
%TCIMACRO{\U{211d} }%
%BeginExpansion
\mathbb{R}
%EndExpansion
,t>0 \\ 
u\left( x,0\right) =g\left( x\right) \text{ per }x\in 
%TCIMACRO{\U{211d} }%
%BeginExpansion
\mathbb{R}
%EndExpansion
\\ 
u_{t}\left( x,0\right) =h\left( x\right) \text{ per }x\in 
%TCIMACRO{\U{211d}}%
%BeginExpansion
\mathbb{R}%
%EndExpansion
\end{array}%
\right. \\
\text{Ts: il problema ha in }C^{2}\left( 
%TCIMACRO{\U{211d} }%
%BeginExpansion
\mathbb{R}
%EndExpansion
\times \lbrack 0,+\infty )\right) \text{ l'unica soluzione }u\left(
x,t\right) =\frac{g\left( x+ct\right) +g\left( x-ct\right) }{2}+\frac{1}{2c}%
\int_{x-ct}^{x+ct}h\left( y\right) dy
\end{gather*}

Questo risultato \`{e} ottimale: le ipotesi non possono essere migliorate
(sono nella natura del problema).

\begin{enumerate}
\item Verifico che la formula risolve il problema nel caso $g=0$. E'
evidente che $u\left( x,0\right) =g\left( x\right) $. Prendo $u\left(
x,t\right) =\frac{1}{2c}\int_{x-ct}^{x+ct}h\left( y\right) dy$. $u_{x}\left(
x,t\right) =\frac{1}{2c}\left( h\left( x+ct\right) -h\left( x-ct\right)
\right) $, $u_{xx}\left( x,t\right) =\frac{1}{2c}\left( h^{\prime }\left(
x+ct\right) -h^{\prime }\left( x-ct\right) \right) $, mentre $u_{t}\left(
x,t\right) =\frac{1}{2c}c\left( h\left( x+ct\right) -h\left( x-ct\right)
\right) $, $u_{tt}\left( x,t\right) =\frac{1}{2c}c^{2}\left( h^{\prime
}\left( x+ct\right) -h^{\prime }\left( x-ct\right) \right) $: evidentemente $%
u_{tt}-c^{2}u_{xx}=0$.
\end{enumerate}

\textbf{Teo (stima di stabilit\`{a})}%
\begin{eqnarray*}
\text{Hp} &\text{: }&u\left( x,t\right) =\frac{g\left( x+ct\right) +g\left(
x-ct\right) }{2}+\frac{1}{2c}\int_{x-ct}^{x+ct}h\left( y\right) dy\text{, }h%
\text{ e }g\text{ sono limitate} \\
\text{Ts} &\text{: }&\left\vert u\left( x,t\right) \right\vert \leq
\sup_{x\in 
%TCIMACRO{\U{211d} }%
%BeginExpansion
\mathbb{R}
%EndExpansion
}\left\vert g\left( x\right) \right\vert +t\sup_{x\in 
%TCIMACRO{\U{211d} }%
%BeginExpansion
\mathbb{R}
%EndExpansion
}\left\vert h\left( x\right) \right\vert
\end{eqnarray*}

In questo caso (a differenza dell'equazione sul segmento) si riesce a fare
una stima puntuale.

\textbf{Dim} $\left\vert \frac{g\left( x+ct\right) +g\left( x-ct\right) }{2}+%
\frac{1}{2c}\int_{x-ct}^{x+ct}h\left( y\right) dy\right\vert \leq \frac{1}{2}%
2\sup_{x\in 
%TCIMACRO{\U{211d} }%
%BeginExpansion
\mathbb{R}
%EndExpansion
}\left\vert g\left( x\right) \right\vert +\frac{1}{2c}\sup_{x\in 
%TCIMACRO{\U{211d} }%
%BeginExpansion
\mathbb{R}
%EndExpansion
}\left\vert h\left( x\right) \right\vert \left( x+ct-x+ct\right) =\sup_{x\in 
%TCIMACRO{\U{211d} }%
%BeginExpansion
\mathbb{R}
%EndExpansion
}\left\vert g\left( x\right) \right\vert +t\sup_{x\in 
%TCIMACRO{\U{211d} }%
%BeginExpansion
\mathbb{R}
%EndExpansion
}\left\vert h\left( x\right) \right\vert $. $\blacksquare $

\textbf{Dominio di dipendenza e influenza} Considero $x^{\ast },t^{\ast }$
fissati e mi chiedo: la soluzione $u\left( x^{\ast },t^{\ast }\right) $
dipende dai valori delle condizioni iniziali $g,h$ in quali punti? $u\left(
x^{\ast },t^{\ast }\right) $ dipende da $g$ in $x^{\ast }+ct^{\ast },x^{\ast
}-ct^{\ast }$ e da $h$ in tutto l'intervallo $\left[ x^{\ast }-ct^{\ast
},x^{\ast }+ct^{\ast }\right] $ (e non da altri valori): l'intervallo $\left[
x^{\ast }-ct^{\ast },x^{\ast }+ct^{\ast }\right] $ si dice dominio di
dipendenza di $u$. Quindi abbiamo a che fare con un segnale che viaggia a
velocit\`{a} finita: ci\`{o} che si riceve in $\left( x^{\ast },t^{\ast
}\right) $ \`{e} ci\`{o} che \`{e} partito all'istante iniziale in $\left[
x^{\ast }-ct^{\ast },x^{\ast }+ct^{\ast }\right] $. C'\`{e} quindi un "tempo
di attesa".\FRAME{dtbpFX}{3.1531in}{2.1015in}{0pt}{}{}{Plot}{\special%
{language "Scientific Word";type "MAPLEPLOT";width 3.1531in;height
2.1015in;depth 0pt;display "USEDEF";plot_snapshots TRUE;mustRecompute
FALSE;lastEngine "MuPAD";xmin "2";xmax "8";xviewmin "2";xviewmax
"8";yviewmin "-1";yviewmax "2.001000";viewset"XY";rangeset"X";plottype
4;labeloverrides 2;y-label "t";axesFont "Times New
Roman,12,0000000000,useDefault,normal";numpoints 100;plotstyle
"patch";axesstyle "normal";axestips FALSE;xis \TEXUX{x};var1name
\TEXUX{$x$};function \TEXUX{$x-3$};linecolor "black";linestyle 1;pointstyle
"point";linethickness 1;lineAttributes "Solid";var1range
"2,8";num-x-gridlines 100;curveColor "[flat::RGB:0000000000]";curveStyle
"Line";function \TEXUX{$-\left( x-7\right) $};linecolor "black";linestyle
1;pointstyle "point";linethickness 1;lineAttributes "Solid";var1range
"2,8";num-x-gridlines 100;curveColor "[flat::RGB:0000000000]";curveStyle
"Line";function \TEXUX{$\left( \left( 3,0\right) ,\left( 7,0\right) \right)
$};linecolor "blue";linestyle 1;pointstyle "point";linethickness
3;lineAttributes "Solid";curveColor "[flat::RGB:0x000000ff]";curveStyle
"Line";function \TEXUX{$\left( 5,2\right) $};linecolor "blue";linestyle
1;pointplot TRUE;pointstyle "point";linethickness 1;lineAttributes
"Solid";curveColor "[flat::RGB:0x000000ff]";curveStyle "Point";VCamFile
'SAEREN0E.xvz';valid_file "T";tempfilename
'SAEREN03.wmf';tempfile-properties "XPR";}}

Nell'equazione del calore invece il segnale viaggia a velocit\`{a} infinita
(anche se la gaussiana decresce cos\`{\i} velocemente che s\`{\i}, $u\left(
x,t\right) $ dipende dalla temperatura iniziale in un istante lontanissimo
da quello presente, ma in modo tanto pi\`{u} trascurabile quanto pi\`{u}
quell'istante \`{e} lontano).

Viceversa, i valori che $g$ e $h$ hanno in $x^{\ast }$ influenzano $u$ in
quali $\left( x,t\right) $? Osservando il grafico al contrario, si vede che
sono tutti gli $\left( x,t\right) $ tali che il triangolo generato da $%
\left( x,t\right) $ contiene $x^{\ast }$: $\left\{ \left( x,t\right)
:x^{\ast }-ct\leq x\leq x^{\ast }+ct\right\} $, detto dominio di influenza.

\FRAME{dtbpFX}{3.1531in}{2.1015in}{0pt}{}{}{Plot}{\special{language
"Scientific Word";type "MAPLEPLOT";width 3.1531in;height 2.1015in;depth
0pt;display "USEDEF";plot_snapshots TRUE;mustRecompute FALSE;lastEngine
"MuPAD";xmin "3";xmax "7";xviewmin "3";xviewmax "7";yviewmin "0";yviewmax
"2.001000";viewset"XY";rangeset"X";plottype 4;labeloverrides 2;y-label
"t";axesFont "Times New Roman,12,0000000000,useDefault,normal";numpoints
100;plotstyle "patchnogrid";axesstyle "normal";axestips FALSE;xis
\TEXUX{x};var1name \TEXUX{$x$};function \TEXUX{$\left( 5,0\right)
$};linecolor "blue";linestyle 1;pointplot TRUE;pointstyle
"point";linethickness 1;lineAttributes "Solid";curveColor
"[flat::RGB:0x000000ff]";curveStyle "Point";function \TEXUX{$\left\vert
x-5\right\vert $};linecolor "black";linestyle 1;discont FALSE;pointstyle
"point";linethickness 1;lineAttributes "Solid";var1range
"3,7";num-x-gridlines 100;curveColor "[flat::RGB:0000000000]";curveStyle
"Line";discont FALSE;VCamFile 'SAES0Y0L.xvz';valid_file "T";tempfilename
'SAERZD04.wmf';tempfile-properties "XPR";}}

Le rette raffigurate sono $t=\frac{x-x^{\ast }}{c}$ e $t=\frac{x^{\ast }-x}{c%
}$.

Quindi e. g. il valore di $u\left( x,t\right) $ \`{e} influenzato dalle
condizioni iniziali in $x^{\ast }$ solo limitatamente agli $\left(
x,t\right) $ appartenente alla regione triangolare individuata dalle rette $%
t=\frac{x-x^{\ast }}{c}$ e $t=\frac{x^{\ast }-x}{c}$.

Ora si tratta l'equazione non omogenea $u_{tt}\left( x,t\right)
-c^{2}u_{xx}\left( x,t\right) =f\left( x,t\right) $ e si risolve col metodo
di Duhamel. Suppongo che la funzione $%
w\left( x,t,s\right) $ sia soluzione del problema di Cauchy $\left\{ 
\begin{array}{c}
w_{tt}-c^{2}w_{xx}=0\text{ per }x\in 
%TCIMACRO{\U{211d} }%
%BeginExpansion
\mathbb{R}
%EndExpansion
,t>s \\ 
w\left( x,s,s\right) =0 \\ 
w_{t}\left( x,s,s\right) =f\left( x,s\right)%
\end{array}%
\right. $. Allora $u\left( x,t\right) =\int_{0}^{t}w\left( x,t,s\right) ds$
risolve il problema iniziale $\left\{ 
\begin{array}{c}
u_{tt}-c^{2}u_{xx}=f\text{ per }x\in 
%TCIMACRO{\U{211d} }%
%BeginExpansion
\mathbb{R}
%EndExpansion
,t>0 \\ 
u\left( x,0\right) =0 \\ 
u_{t}\left( x,0\right) =0%
\end{array}%
\right. $. Una volta nota la soluzione di tale problema, grazie al principio
di sovrapposizione si pu\`{o} risolvere qualsiasi problema, anche con
condizioni iniziali non omogenee. Verifichiamo, solo formalmente, che $u$ cos%
\`{\i} definita risolve il problema.

Se $u\left( x,t\right) =\int_{0}^{t}w\left( x,t,s\right) ds$, $u_{t}\left(
x,t\right) =w\left( x,t,t\right) +\int_{0}^{t}\frac{\partial w}{\partial t}%
\left( x,t,s\right) ds=\int_{0}^{t}w_{t}\left( x,t,s\right) ds$ (il primo
addendo \`{e} nullo per la condizione iniziale su $w$). Si nota subito che $%
u\left( x,0\right) =u_{t}\left( x,0\right) =0$. $u_{tt}\left( x,t\right)
=w_{t}\left( x,t,t\right) +\int_{0}^{t}w_{tt}\left( x,t,s\right) ds=f\left(
x,t\right) +\int_{0}^{t}w_{tt}\left( x,t,s\right) ds$, mentre $\Delta
u\left( x,t\right) =\int_{0}^{t}\Delta w\left( x,t,s\right) ds$. Allora $%
u_{tt}-c^{2}u_{xx}=f+\int_{0}^{t}\left( w_{tt}\left( x,t,s\right)
-c^{2}\Delta w\right) ds=f$ perch\'{e} il secondo addendo \`{e} nullo per
l'ipotesi su $w$, supponendo di avere $n=1$.

Per la formula di D'Alembert con $g=0$, $w\left( x,t,s\right) =\frac{1}{2c}%
\int_{x-c\left( t-s\right) }^{x+c\left( t-s\right) }f\left( y,s\right) dy$:
quindi la candidata soluzione del problema di Cauchy per la corda vibrante
non omogeneo con condizioni iniziali omogenee \`{e}%
\begin{equation*}
u\left( x,t\right) =\frac{1}{2c}\int_{0}^{t}\int_{x-c\left( t-s\right)
}^{x+c\left( t-s\right) }f\left( y,s\right) dyds
\end{equation*}

Il seguente teorema chiarisce sotto quali ipotesi tale funzione \`{e}
effettivamente soluzione.

\textbf{Teo (soluzione del problema di Cauchy per la corda vibrante non
omogeneo)}%
\begin{eqnarray*}
\text{Hp} &\text{: }&\left\{ 
\begin{array}{c}
u_{tt}-c^{2}u_{xx}=f\text{ per }x\in 
%TCIMACRO{\U{211d} }%
%BeginExpansion
\mathbb{R}
%EndExpansion
,t>0 \\ 
u\left( x,0\right) =0 \\ 
u_{t}\left( x,0\right) =0%
\end{array}%
\right. \text{, }f\in C^{0}\left( 
%TCIMACRO{\U{211d} }%
%BeginExpansion
\mathbb{R}
%EndExpansion
\times \lbrack 0,+\infty )\right) ,\exists \text{ }\frac{\partial f}{%
\partial x}\in C^{0}\left( 
%TCIMACRO{\U{211d} }%
%BeginExpansion
\mathbb{R}
%EndExpansion
\times \lbrack 0,+\infty )\right) \\
\text{Ts} &\text{: }&u\left( x,t\right) =\frac{1}{2c}\int_{0}^{t}\int_{x-c%
\left( t-s\right) }^{x+c\left( t-s\right) }f\left( y,s\right) dyds\text{
risolve il problema ed \`{e} }C^{2}\left( 
%TCIMACRO{\U{211d} }%
%BeginExpansion
\mathbb{R}
%EndExpansion
\times (0,+\infty )\right) \cap C^{1}\left( 
%TCIMACRO{\U{211d} }%
%BeginExpansion
\mathbb{R}
%EndExpansion
\times \lbrack 0,+\infty )\right)
\end{eqnarray*}

Si noti che $f\in C^{0}\left( 
%TCIMACRO{\U{211d} }%
%BeginExpansion
\mathbb{R}
%EndExpansion
\times \lbrack 0,+\infty )\right) ,u\in C^{2}\left( 
%TCIMACRO{\U{211d} }%
%BeginExpansion
\mathbb{R}
%EndExpansion
\times (0,+\infty )\right) \left( \cap C^{1}\left( 
%TCIMACRO{\U{211d} }%
%BeginExpansion
\mathbb{R}
%EndExpansion
\times \lbrack 0,+\infty )\right) \right) $ sono le propriet\`{a} minime
perch\'{e} si possa parlare di $u$ come soluzione classica (l'equazione
differenziale \`{e} soddisfatta punto per punto ed \`{e} un'identit\`{a} tra
funzione continue); $\frac{\partial f}{\partial x}\in C^{0}\left( 
%TCIMACRO{\U{211d} }%
%BeginExpansion
\mathbb{R}
%EndExpansion
\times \lbrack 0,+\infty )\right) $ \`{e} un'ipotesi che serve per fare la
dimostrazione.

Nella classe $C^{2}\left( 
%TCIMACRO{\U{211d} }%
%BeginExpansion
\mathbb{R}
%EndExpansion
\times (0,+\infty )\right) \cap C^{1}\left( 
%TCIMACRO{\U{211d} }%
%BeginExpansion
\mathbb{R}
%EndExpansion
\times \lbrack 0,+\infty )\right) $ c'\`{e} unicit\`{a} grazie all'unicit%
\`{a} della soluzione del problema omogeneo.

La dimostrazione \`{e} una verifica e procede come la dimostrazione del
teorema analogo visto per l'equazione del trasporto.

\textbf{Dim} Si applica due volte il teorema fondamentale del calcolo
integrale. $u_{t}\left( x,t\right) =\frac{1}{2c}\frac{d}{dt}%
\int_{0}^{t}\left( \int_{x-c\left( t-s\right) }^{x+c\left( t-s\right)
}f\left( y,s\right) dy\right) ds=\frac{1}{2c}\int_{0}^{t}\left( \frac{%
\partial }{\partial t}\int_{x-c\left( t-s\right) }^{x+c\left( t-s\right)
}f\left( y,s\right) dy\right) ds$. Vale $\frac{\partial }{\partial t}%
\int_{x-c\left( t-s\right) }^{x+c\left( t-s\right) }f\left( y,s\right)
dy=c\left( f\left( x+c\left( t-s\right) ,s\right) +f\left( x-c\left(
t-s\right) ,s\right) \right) $, quindi $u_{t}\left( x,t\right) =\frac{1}{2}%
\int_{0}^{t}\left[ f\left( x+c\left( t-s\right) ,s\right) +f\left( x-c\left(
t-s\right) ,s\right) \right] ds$. Allora $u_{tt}\left( x,t\right) =f\left(
x,t\right) +\frac{1}{2}\int_{0}^{t}\frac{\partial }{\partial t}\left[
f\left( x+c\left( t-s\right) ,s\right) +f\left( x-c\left( t-s\right)
,s\right) \right] ds$, cio\`{e} $f\left( x,t\right) +\frac{1}{2}c\int_{0}^{t}%
\left[ \frac{\partial }{\partial x}f\left( x+c\left( t-s\right) ,s\right) -%
\frac{\partial }{\partial x}f\left( x-c\left( t-s\right) ,s\right) \right]
ds $. D'altra parte $u_{x}\left( x,t\right) =\frac{1}{2c}\frac{d}{dx}%
\int_{x-c\left( t-s\right) }^{x+c\left( t-s\right) }\left(
\int_{0}^{t}f\left( y,s\right) ds\right) dy=\frac{1}{2c}\int_{0}^{t}\left[
f\left( x+c\left( t-s\right) ,s\right) -f\left( x-c\left( t-s\right)
,s\right) \right] ds$ e quindi $u_{xx}\left( x,t\right) =\frac{1}{2c}%
\int_{0}^{t}\left[ \frac{\partial }{\partial x}f\left( x+c\left( t-s\right)
,s\right) -\frac{\partial }{\partial x}f\left( x-c\left( t-s\right)
,s\right) \right] ds$.

\subsubsection{Soluzioni deboli}

La definizione di soluzione debole ha la stessa motivazione vista per
l'equazione del trasporto, e si ispira a passaggi simili.

Suppongo che $u\in C^{2}\left( 
%TCIMACRO{\U{211d} }%
%BeginExpansion
\mathbb{R}
%EndExpansion
\times \lbrack 0,+\infty )\right) $ risolva in senso classico $\left\{ 
\begin{array}{c}
u_{tt}-c^{2}u_{xx}=0 \\ 
u\left( x,0\right) =g\left( x\right) \\ 
u_{t}\left( x,0\right) =h\left( x\right)%
\end{array}%
\right. $ (1), con $g\in C^{2}\left( 
%TCIMACRO{\U{211d} }%
%BeginExpansion
\mathbb{R}
%EndExpansion
\right) ,h\in C^{1}\left( 
%TCIMACRO{\U{211d} }%
%BeginExpansion
\mathbb{R}
%EndExpansion
\right) $. Sia $\phi \in C_{0}^{2}\left( 
%TCIMACRO{\U{211d} }%
%BeginExpansion
\mathbb{R}
%EndExpansion
\times \lbrack 0,+\infty \right) )$: moltiplico entrambi i lati
dell'equazione per $\phi $ e integro in spazio e tempo. Ottengo $%
\int_{0}^{+\infty }\int_{%
%TCIMACRO{\U{211d} }%
%BeginExpansion
\mathbb{R}
%EndExpansion
}\phi \left( u_{tt}-c^{2}u_{xx}\right) \left( x,t\right) dxdt=0$. Integro
per parti entrambi gli addendi per scaricare le derivate su $\phi $: $%
\int_{0}^{+\infty }\int_{%
%TCIMACRO{\U{211d} }%
%BeginExpansion
\mathbb{R}
%EndExpansion
}\phi u_{xx}dx=\int_{0}^{+\infty }\int_{%
%TCIMACRO{\U{211d} }%
%BeginExpansion
\mathbb{R}
%EndExpansion
}\phi _{xx}u$, dato che $\phi $ \`{e} nulla $+\infty $ e $-\infty $. Invece $%
\int_{0}^{+\infty }\phi u_{tt}dt=\left[ \phi u_{t}\right] _{0}^{+\infty
}-\int_{0}^{+\infty }\phi _{t}u_{t}dt=-\phi \left( x,0\right) u_{t}\left(
x,0\right) -\int_{0}^{+\infty }\phi _{t}u_{t}dt$, cio\`{e} $-\phi \left(
x,0\right) h\left( x\right) -\left\{ \left[ \phi _{t}u\right] _{0}^{+\infty
}-\int_{0}^{+\infty }\phi _{tt}udt\right\} =-\phi \left( x,0\right) h\left(
x\right) +\phi _{t}\left( x,0\right) g\left( x\right) +\int_{0}^{+\infty
}\phi _{tt}udt$.

Quindi complessivamente $\int_{%
%TCIMACRO{\U{211d} }%
%BeginExpansion
\mathbb{R}
%EndExpansion
}\left( -\phi \left( x,0\right) h\left( x\right) +\phi _{t}\left( x,0\right)
g\left( x\right) \right) dx+\int_{0}^{+\infty }\int_{%
%TCIMACRO{\U{211d} }%
%BeginExpansion
\mathbb{R}
%EndExpansion
}u\left( \phi _{tt}-c^{2}\phi _{xx}\right) dxdt=0$ $\forall $ $\phi \in
C_{0}^{2}\left( 
%TCIMACRO{\U{211d} }%
%BeginExpansion
\mathbb{R}
%EndExpansion
\times \lbrack 0,+\infty \right) )$ (2).

Abbiamo dunque mostrato che se $u$ \`{e} una soluzione classica di (1), con $%
u\in C^{2}\left( 
%TCIMACRO{\U{211d} }%
%BeginExpansion
\mathbb{R}
%EndExpansion
\times \lbrack 0,+\infty )\right) ,g\in C^{2}\left( 
%TCIMACRO{\U{211d} }%
%BeginExpansion
\mathbb{R}
%EndExpansion
\right) ,h\in C^{1}\left( 
%TCIMACRO{\U{211d} }%
%BeginExpansion
\mathbb{R}
%EndExpansion
\right) $, allora vale (2).

Quali sono le ipotesi minime sotto cui (2) ha senso? E' sufficiente che $%
u,g,h$ siano localmente integrabili. In realt\`{a} per\`{o}, dato che $u$ ha
pur sempre il significato di forma di una corda e $g$ di condizione iniziale
di una corda, non ha senso suppore $u,g$ meno che continue (supponendo che
la corda non si spezzi): si prenderanno quindi $u\in C^{0}\left( 
%TCIMACRO{\U{211d} }%
%BeginExpansion
\mathbb{R}
%EndExpansion
\times \lbrack 0,+\infty )\right) ,g\in C^{0}\left( 
%TCIMACRO{\U{211d} }%
%BeginExpansion
\mathbb{R}
%EndExpansion
\right) $, e chiediamo $h\in L^{\infty }\left( 
%TCIMACRO{\U{211d} }%
%BeginExpansion
\mathbb{R}
%EndExpansion
\right) $ (che \`{e} pi\`{u} forte di localmente integrabile: ma vogliamo
che la velocit\`{a} iniziale della corda sia essenzialmente limitata).
Questo giustifica la seguente

\textbf{Def} Date $u\in C^{0}\left( 
%TCIMACRO{\U{211d} }%
%BeginExpansion
\mathbb{R}
%EndExpansion
\times \lbrack 0,+\infty )\right) ,g\in C^{0}\left( 
%TCIMACRO{\U{211d} }%
%BeginExpansion
\mathbb{R}
%EndExpansion
\right) ,h\in L^{\infty }\left( 
%TCIMACRO{\U{211d} }%
%BeginExpansion
\mathbb{R}
%EndExpansion
\right) $, si dice che $u$ \`{e} soluzione debole di $\left\{ 
\begin{array}{c}
u_{tt}-c^{2}u_{xx}=0 \\ 
u\left( x,0\right) =g\left( x\right) \\ 
u_{t}\left( x,0\right) =h\left( x\right)%
\end{array}%
\right. $ se $\int_{%
%TCIMACRO{\U{211d} }%
%BeginExpansion
\mathbb{R}
%EndExpansion
}\left( -\phi \left( x,0\right) h\left( x\right) +\phi _{t}\left( x,0\right)
g\left( x\right) \right) dx+\int_{0}^{+\infty }\int_{%
%TCIMACRO{\U{211d} }%
%BeginExpansion
\mathbb{R}
%EndExpansion
}u\left( \phi _{tt}-c^{2}\phi _{xx}\right) dxdt=0$ $\forall $ $\phi \in
C_{0}^{2}\left( 
%TCIMACRO{\U{211d} }%
%BeginExpansion
\mathbb{R}
%EndExpansion
\times \lbrack 0,+\infty \right) )$.

Sopra abbiamo quindi mostrato che se $u$ \`{e} soluzione classica allora 
\`{e} anche soluzione debole.

\textbf{Teo (una soluzione classica \`{e} anche soluzione debole)}%
\begin{eqnarray*}
\text{Hp} &\text{: }&u\in C^{2}\left( 
%TCIMACRO{\U{211d} }%
%BeginExpansion
\mathbb{R}
%EndExpansion
\times \lbrack 0,+\infty )\right) ,g\in C^{2}\left( 
%TCIMACRO{\U{211d} }%
%BeginExpansion
\mathbb{R}
%EndExpansion
\right) ,h\in C^{1}\left( 
%TCIMACRO{\U{211d} }%
%BeginExpansion
\mathbb{R}
%EndExpansion
\right) \text{, }u\text{ \`{e} soluzione classica di (1)} \\
\text{Ts} &\text{: }&u\text{ \`{e} anche soluzione debole di (1)}
\end{eqnarray*}

ma $h\in C^{1}$ non implica limitata...

\textbf{Teo (una soluzione debole con ingredienti abbastanza regolari \`{e}
anche soluzione classica)}%
\begin{eqnarray*}
\text{Hp}\text{: } &&u\in C^{2}\left( 
%TCIMACRO{\U{211d} }%
%BeginExpansion
\mathbb{R}
%EndExpansion
\times \lbrack 0,+\infty )\right) ,g\in C^{2}\left( 
%TCIMACRO{\U{211d} }%
%BeginExpansion
\mathbb{R}
%EndExpansion
\right) ,h\in C^{1}\left( 
%TCIMACRO{\U{211d} }%
%BeginExpansion
\mathbb{R}
%EndExpansion
\right) \text{, }u\text{ \`{e} soluzione debole di (1)} \\
\text{Ts}\text{: } &&u\text{ \`{e} anche soluzione classica di (1)}
\end{eqnarray*}

Si pu\`{o} dimostrare facendo parlare la definizione di soluzione debole,
analogamente a quanto visto per l'equazione del trasporto.

\textbf{Teo (la soluzione di d'Alembert \`{e} soluzione debole)}%
\begin{eqnarray*}
\text{Hp} &\text{: }&g\in C^{0}\left( 
%TCIMACRO{\U{211d} }%
%BeginExpansion
\mathbb{R}
%EndExpansion
\right) ,h\in L^{\infty }\left( 
%TCIMACRO{\U{211d} }%
%BeginExpansion
\mathbb{R}
%EndExpansion
\right) ,u\left( x,t\right) =\frac{g\left( x-ct\right) +g\left( x+ct\right) 
}{2}+\frac{1}{2c}\int_{x-ct}^{x+ct}h\left( y\right) dy \\
\text{Ts} &\text{: }&u\text{ \`{e} soluzione debole di (1)}
\end{eqnarray*}

Si noti che in tal caso $u$ \`{e} continua e quindi le ipotesi della
definizione di soluzione debole sono soddisfatte.

La questione dell'unicit\`{a} della soluzione debole \`{e} delicata, e
dunque non la discutiamo.

\subsection{Equazione delle onde in un dominio qualsiasi}

L'equazione della corda vibrante \`{e} il caso $n=1$ dell'equazione delle
onde $u_{tt}-c^{2}\Delta u=f$ in $%
%TCIMACRO{\U{211d} }%
%BeginExpansion
\mathbb{R}
%EndExpansion
^{n}$; essa nasce con $n=2$ dal modello fisico della membrana vibrante, che
porta a studiare $u_{tt}-c^{2}\left( u_{xx}+u_{yy}\right) =0$. $c^{2}=\frac{%
\tau }{\rho _{0}}$, dove $\tau $ \`{e} una forza per unit\`{a} di lunghezza
e $\rho _{0}$ la densit\`{a} di massa superficiale per la membrana, ha
ancora le dimensioni di una velocit\`{a} al quadrato.

Ci sono alcuni tipi di problemi che \`{e} naturale studiare per l'equazione
della membrana vibrante.

Il problema di Cauchy globale \`{e} $\left\{ 
\begin{array}{c}
u_{tt}-c^{2}\left( u_{xx}+u_{yy}\right) =0\text{ per }x\in 
%TCIMACRO{\U{211d} }%
%BeginExpansion
\mathbb{R}
%EndExpansion
^{2},t>0 \\ 
u\left( x,y,0\right) =g\left( x,y\right) \text{ per }x\in 
%TCIMACRO{\U{211d} }%
%BeginExpansion
\mathbb{R}
%EndExpansion
^{2} \\ 
u_{t}\left( x,y,0\right) =h\left( x,y\right) \text{ per }x\in 
%TCIMACRO{\U{211d} }%
%BeginExpansion
\mathbb{R}
%EndExpansion
^{2}%
\end{array}%
\right. $.

Il problema di Cauchy su un dominio con condizioni al bordo \`{e} $\left\{ 
\begin{array}{c}
u_{tt}-c^{2}\left( u_{xx}+u_{yy}\right) =0\text{ per }x\in \Omega ,t>0 \\ 
u\left( x,y,0\right) =g\left( x,y\right) \text{ per }x\in \Omega \\ 
u_{t}\left( x,y,0\right) =h\left( x,y\right) \text{ per }x\in \Omega \\ 
\text{condizioni al bordo}%
\end{array}%
\right. $, dove le condizioni al bordo naturali sono del tipo Dirichlet
omogenee ($u\left( x,y,t\right) =0$ se $\left( x,y\right) \in \partial
\Omega ,t>0$) oppure Neumann omogenee ($\frac{\partial }{\partial n}u\left(
x,y,t\right) =0$ per $\left( x,y\right) \in \partial \Omega ,t>0$), che ha
il significato fisico alquanto irrealistico di corda con estremi liberi di
scorrere su due guide verticali).

Per $n=3$ l'equazione delle onde nasce da modelli fisici delle onde sonore
(di pressione) o elettromagnetiche (ogni componente del campo
elettromagnetico $\left( \mathbf{E,B}\right) $ nel vuoto in assenza di
sorgenti soddisfa $u_{tt}-c^{2}\Delta u=0$ in $%
%TCIMACRO{\U{211d} }%
%BeginExpansion
\mathbb{R}
%EndExpansion
^{3}\times \lbrack 0,+\infty )$).

Con il metodo dell'energia si dimostra un risultato di unicit\`{a} per
l'equazione delle onde in $%
%TCIMACRO{\U{211d} }%
%BeginExpansion
\mathbb{R}
%EndExpansion
^{n}$.

\textbf{Teo (unicit\`{a} per il problema di Cauchy-Dirichlet o
Cauchy-Neumann in }$Q_{T}$\textbf{)}%
\begin{gather*}
\text{Hp: }\Omega \subseteq 
%TCIMACRO{\U{211d} }%
%BeginExpansion
\mathbb{R}
%EndExpansion
^{n}\text{ dominio limitato e lipschitziano, }Q_{T}=\Omega \times \left(
0,T\right) \text{, }\left\{ 
\begin{array}{c}
u_{tt}-c^{2}\Delta u=f\text{ in }Q_{T} \\ 
u\left( \mathbf{x},0\right) =g\left( \mathbf{x}\right) \text{ per }\mathbf{x}%
\in \Omega \\ 
u_{t}\left( \mathbf{x},0\right) =h\left( \mathbf{x}\right) \text{ per }%
\mathbf{x}\in \Omega \\ 
u\left( \mathbf{x},t\right) /\frac{\partial u}{\partial n}\left( \mathbf{x}%
,t\right) =0\text{ se }\mathbf{x}\in \partial \Omega ,t\in \left( 0,T\right)%
\end{array}%
\right. \\
\text{Ts: se esiste, la soluzione del problema \`{e} unica nella classe }%
C^{2}\left( Q_{T}\right) \cap C^{1}\left( \bar{Q}_{T}\right) \text{ }
\end{gather*}

Questa \`{e} una generalizzazione del teorema di unicit\`{a} gi\`{a} visto
nel caso $n=1$. L'unicit\`{a} si estende al caso di $T=+\infty $.

\textbf{Dim} Siano $u_{1},u_{2}$ soluzioni del problema nella classe $%
C^{2}\left( Q_{T}\right) \cap C^{1}\left( \bar{Q}_{T}\right) $. Allora $%
u=u_{1}-u_{2}$ \`{e} nella stessa classe e risolve il problema con $f=g=h=0$.

Definisco l'energia $E\left( t\right) =\frac{1}{2}\int_{\Omega }\left(
u_{t}^{2}+c^{2}\left\vert \left\vert \nabla u\right\vert \right\vert
^{2}\right) \left( \mathbf{x},t\right) d\mathbf{x}$: per il teorema di
continuit\`{a} di una funzione definita mediante integrale dipendente da un
parametro, $E\in C^{1}\left( 0,T\right) \cap C^{0}\left( \left[ 0,T\right]
\right) $. La strategia dimostrativa \`{e} mostrare che (1) $E$ che \`{e}
costante, (2) tale costante \`{e} $0$, (3) $u$ \`{e} nulla.

(1) Poich\'{e} $\frac{\partial }{\partial t}\left\vert \left\vert \nabla
u\right\vert \right\vert ^{2}=\frac{\partial }{\partial t}%
\sum_{i=1}^{n}u_{x_{i}}^{2}=\sum_{i=1}^{n}2u_{x_{i}}u_{x_{i}t}=2\left\langle
\nabla u,\nabla u_{t}\right\rangle $, $E^{\prime }\left( t\right) =\frac{1}{2%
}\int_{\Omega }\left( 2u_{t}u_{tt}+2c^{2}\left\langle \nabla u,\nabla
u_{t}\right\rangle \right) \left( \mathbf{x},t\right) d\mathbf{x}$: la
derivazione sotto il segno di integrale \`{e} giustificata dal fatto che le
derivate prime e seconde sono continue in $\bar{\Omega}$ (in realt\`{a}
quelle seconde in $\Omega $...) e dunque limitate da una costante
integrabile. Applico la prima identit\`{a} di Green a $\int_{\Omega
}\left\langle \nabla u,\nabla u_{t}\right\rangle \left( \mathbf{x},t\right) d%
\mathbf{x}$ ($u$ dovrebbe essere $C^{2}$ fino al bordo: ma si pu\`{o} usare
la stessa argomentazione di estensione vista all'inizio del corso): $%
\int_{\Omega }\left\langle \nabla u,\nabla u_{t}\right\rangle \left( \mathbf{%
x},t\right) d\mathbf{x}=-\int_{\Omega }u_{t}\Delta ud\mathbf{x}%
+\int_{\partial \Omega }u_{t}\frac{\partial u}{\partial n}d\sigma $. Dico
che il secondo addendo \`{e} nullo. Se il problema che risolve $u$ \`{e} di
Neumann, \`{e} ovvio; se \`{e} di Dirichlet, $u\left( \mathbf{x},t\right) =0$
se $\mathbf{x}\in \partial \Omega $ $\forall $ $t$, quindi $u_{t}\left( 
\mathbf{x},t\right) =0$ se $\mathbf{x}\in \partial \Omega $. Si ha quindi $E^{\prime }\left( t\right) =\frac{%
1}{2}\int_{\Omega }\left( 2u_{t}u_{tt}-c^{2}u_{t}\Delta u\right) \left( 
\mathbf{x},t\right) d\mathbf{x}=0$ perch\'{e} $u$ risolve l'equazione.
Quindi $E\left( t\right) =E\left( 0\right) $ $\forall $ $t$.

(2) $E\left( 0\right) =\frac{1}{2}\int_{\Omega }\left( u_{t}^{2}\left( 
\mathbf{x},0\right) +c^{2}\left\vert \left\vert \nabla u\left( \mathbf{x}%
,0\right) \right\vert \right\vert ^{2}\right) d\mathbf{x}$. Per le
condizioni iniziali $u_{t}\left( \mathbf{x},0\right) =0$; inoltre $u\left( 
\mathbf{x},0\right) =0$ $\forall $ $\mathbf{x}$, quindi $\nabla u\left( 
\mathbf{x},0\right) =0$ $\forall $ $\mathbf{x}$ e $E\left( 0\right) =0$.

(3) Quindi $E\left( t\right) =\frac{1}{2}\int_{\Omega }\left(
u_{t}^{2}\left( \mathbf{x},t\right) +c^{2}\left\vert \left\vert \nabla
u\left( \mathbf{x},t\right) \right\vert \right\vert ^{2}\right) d\mathbf{x}%
=0 $ $\forall $ $t\in \left( 0,T\right) $: fissato $t$, si ha che
l'integrale di una funzione continua nonnegativa \`{e} nullo, il che implica
che la funzione integranda \`{e} nulla $\forall $ $\mathbf{x}$. Allora $%
u_{t}^{2}\left( \mathbf{x},t\right) +c^{2}\left\vert \left\vert \nabla
u\left( \mathbf{x},t\right) \right\vert \right\vert ^{2}=0$ $\forall $ $t,%
\mathbf{x}$, e dunque $u$ \`{e} costante sia in $t$ che in $\mathbf{x}$;
poich\'{e} $u\left( \mathbf{x},0\right) =0$, tale costante \`{e} $0$ e $%
u\left( \mathbf{x},t\right) =0$ $\forall $ $\mathbf{x}\in Q_{T}$. Dunque $%
u_{1}=u_{2}$. $\blacksquare $

\subsubsection{Problema di Cauchy-Dirichlet}

Risolviamo il problema di Cauchy-Dirichlet omogeneo con condizioni di
Dirichlet omogenee 
\begin{equation*}
\left\{ 
\begin{array}{c}
u_{tt}-c^{2}\Delta u=0\text{ in }Q_{T}=\Omega \times \left( 0,T\right) \\ 
u\left( \mathbf{x},0\right) =g\left( \mathbf{x}\right) \text{ per }\mathbf{x}%
\in \Omega \\ 
u_{t}\left( \mathbf{x},0\right) =h\left( \mathbf{x}\right) \text{ per }%
\mathbf{x}\in \Omega \\ 
u\left( \mathbf{x},t\right) =0\text{ per }\mathbf{x\in \partial }\Omega
,t\in \left( 0,T\right)%
\end{array}%
\right. 
\end{equation*}
con $\Omega \subseteq 
%TCIMACRO{\U{211d} }%
%BeginExpansion
\mathbb{R}
%EndExpansion
^{n}$ dominio, usando la separazione di variabili. Se $u\left( \mathbf{x}%
,t\right) =X\left( \mathbf{x}\right) T\left( t\right) $, l'equazione diventa 
$XT^{\prime \prime }-c^{2}T\Delta X=0$, cio\`{e} $\frac{T^{\prime \prime }}{%
c^{2}T}=\frac{\Delta X}{X}$ in $Q_{T}=\Omega \times \left( 0,T\right) $.
Allora $\exists $ $\lambda \in 
%TCIMACRO{\U{211d} }%
%BeginExpansion
\mathbb{R}
%EndExpansion
:\frac{T^{\prime \prime }}{c^{2}T}=\frac{\Delta X}{X}=\lambda $ in $Q_{T}$,
e il problema si \`{e} disaccoppiato in $\left\{ 
\begin{array}{c}
\Delta X\left( \mathbf{x}\right) =\lambda X\left( \mathbf{x}\right) \text{
in }\Omega \\ 
X\left( \mathbf{x}\right) =0\text{ in }\partial \Omega%
\end{array}%
\right. ,T^{\prime \prime }\left( t\right) =c^{2}\lambda T\left( t\right) $
in $\left( 0,T\right) $. Il primo problema si dice problema agli autovalori
per il laplaciano con condizione al contorno di Dirichlet omogenea, e
l'equazione, in dimensione $n=1$, diventa un'EDO lineare del second'ordine a
coefficienti costanti, gi\`{a} risolta molte volte in base al segno di $%
\lambda $.

Il problema agli autovalori \`{e} un problema che emerge non solo per
l'equazione delle onde. Si considera un analogo problema per l'equazione del
calore: $\left\{ 
\begin{array}{c}
u_{t}-D\Delta u=0\text{ in }Q_{T} \\ 
u\left( \mathbf{x},0\right) =g\left( \mathbf{x}\right) \text{ in }\Omega \\ 
u\left( \mathbf{x},t\right) =0\text{ in }\partial \Omega%
\end{array}%
\right. $; sarebbe analogo al problema sul segmento in dimensione $n$ (e. g.
se $n=2$ ho una lastra metallica isolata sopra e sotto che pu\`{o} scambiare
calore solo sul bordo, su cui \`{e} termostatata in modo da avere
temperatura nulla), ma non l'abbiamo risolto. Tuttavia, se ripetiamo la
separazione di variabili $u\left( \mathbf{x},t\right) =X\left( \mathbf{x}%
\right) T\left( t\right) $, l'equazione diventa $XT^{\prime }-DT\Delta X=0$,
cio\`{e} $\frac{T^{\prime }}{DT}=\frac{\Delta X}{X}$ in $Q_{T}=\Omega \times
\left( 0,T\right) $, da cui $\left\{ 
\begin{array}{c}
\Delta X\left( \mathbf{x}\right) =\lambda X\left( \mathbf{x}\right) \text{
in }\Omega \\ 
X\left( \mathbf{x}\right) =0\text{ in }\partial \Omega%
\end{array}%
\right. ,T^{\prime }\left( t\right) =D\lambda T\left( t\right) $ in $\left(
0,T\right) $. Il problema in $X$ \`{e} proprio lo stesso problema ricavato
per l'equazione delle onde! E' quindi ovvio che \`{e} utile saperlo
risolvere.

\textbf{Problema agli autovalori per il laplaciano} Il problema $\left\{ 
\begin{array}{c}
\Delta X\left( \mathbf{x}\right) =\lambda X\left( \mathbf{x}\right) \text{
in }\Omega \\ 
X\left( \mathbf{x}\right) =0\text{ in }\partial \Omega%
\end{array}%
\right. $ si dice problema agli autovalori perch\'{e} l'equazione \`{e}
quella che definisce autovalori e autovettori di un operatore lineare $L$:
si cercano $X,\lambda :LX=\lambda X$. In questo caso, se $\Delta X=\lambda X$
in $\Omega $, $X=0$ in $\partial \Omega $ e $X\neq 0$, $\lambda $ si dice
autovalore e $X$ autofunzione del laplaciano; $X$ si dice anche autofunzione
di $\Delta $ relativa all'autovalore $\lambda $.

Il seguente teorema presenta alcune propriet\`{a} di autovalori e
autofunzioni di $\Delta $.

\textbf{Teo (propriet\`{a} di autovalori e autofunzioni di }$\Delta $\textbf{%
)} 
\begin{gather*}
\text{(1) Hp: }\Omega \text{ \`{e} un dominio limitato lipschitziano, }%
\exists \text{ }\lambda \in 
%TCIMACRO{\U{211d} }%
%BeginExpansion
\mathbb{R}
%EndExpansion
:u\in C^{2}\left( \bar{\Omega}\right) ,u\neq 0\text{ risolve }\left\{ 
\begin{array}{c}
\Delta u\left( \mathbf{x}\right) =\lambda u\left( \mathbf{x}\right) \text{
in }\Omega \\ 
u\left( \mathbf{x}\right) =0\text{ in }\partial \Omega%
\end{array}%
\right. \\
\text{Ts: }\lambda <0 \\
\text{(2) Hp: }\Omega \text{ \`{e} un dominio limitato lipschitziano, }%
u,v\in C^{2}\left( \bar{\Omega}\right) ,u,v\neq 0\text{ risolvono } \\
\left\{ 
\begin{array}{c}
\Delta u\left( \mathbf{x}\right) =\lambda u\left( \mathbf{x}\right) \text{
in }\Omega \\ 
u\left( \mathbf{x}\right) =0\text{ in }\partial \Omega%
\end{array}%
\right. ,\left\{ 
\begin{array}{c}
\Delta v\left( \mathbf{x}\right) =\mu v\left( \mathbf{x}\right) \text{ in }%
\Omega \\ 
v\left( \mathbf{x}\right) =0\text{ in }\partial \Omega%
\end{array}%
\right. \text{ con }\lambda \neq \mu \\
\text{Ts: }\int_{\Omega }uvd\mathbf{x}=0
\end{gather*}

La ipotesi di (2) si esprime sinteticamente dicendo che $u,v\in C^{2}\left( 
\bar{\Omega}\right) $ sono autofunzioni di $\Delta $ relative agli
autovalori $\lambda ,\mu :\lambda \neq \mu $; la tesi \`{e} che $u$ e $v$
sono ortogonali in $L^{2}\left( \Omega \right) $ (essendo $u,v\in
C^{2}\left( \bar{\Omega}\right) $ con $\Omega $ limitato, sicuramente sono
in $L^{2}\left( \Omega \right) $)

\textbf{Dim} (1) L'idea \`{e} applicare la prima identit\`{a} di Green a $u$
e $u$. Moltiplicando entrambi i lati dell'equazione per $u$ si ottiene $%
u\Delta u=\lambda u^{2}$: integrando su $\Omega $, $\int_{\Omega }u\Delta ud%
\mathbf{x}=\int_{\Omega }\lambda u^{2}d\mathbf{x}$. Ma per Green $%
\int_{\Omega }u\Delta ud\mathbf{x}=\int_{\partial \Omega }u\frac{\partial u}{%
\partial n}d\sigma -\int_{\Omega }\left\vert \left\vert \nabla u\right\vert
\right\vert ^{2}d\mathbf{x}=-\int_{\Omega }\left\vert \left\vert \nabla
u\right\vert \right\vert ^{2}d\mathbf{x}$, essendo $u=0$ su $\partial \Omega 
$. Allora si ha $\int_{\Omega }\lambda u^{2}d\mathbf{x}=-\int_{\Omega
}\left\vert \left\vert \nabla u\right\vert \right\vert ^{2}d\mathbf{x}$, cio%
\`{e} $\lambda =\frac{-\int_{\Omega }\left\vert \left\vert \nabla
u\right\vert \right\vert ^{2}d\mathbf{x}}{\int_{\Omega }u^{2}d\mathbf{x}}%
\leq 0$ (il rapporto \`{e} ben definito perch\'{e} $u$ \`{e} non nulla). Se
per assurdo fosse $\int_{\Omega }\left\vert \left\vert \nabla u\right\vert
\right\vert ^{2}d\mathbf{x}=0$, si avrebbe $\left\vert \left\vert \nabla
u\right\vert \right\vert ^{2}=0$ e quindi $u$ costante, ma essendo $u=0$ in $%
\partial \Omega $ tale costante dovrebbe essere $0$, il che \`{e} assurdo
perch\'{e} per ipotesi $u\neq 0$. Allora $\lambda <0$.

(2) L'idea \`{e} usare la seconda identit\`{a} di Green. Moltiplicando la
prima equazione per $v$ e la seconda per $u$ e sottraendo membro a membro si
ottiene $v\Delta u-u\Delta v=\left( \lambda -\mu \right) uv$. Integrando, $%
\int_{\Omega }\left( v\Delta u-u\Delta v\right) d\mathbf{x}=\left( \lambda
-\mu \right) \int_{\Omega }uvd\mathbf{x}$; ma per per Green $\int_{\Omega
}\left( v\Delta u-u\Delta v\right) d\mathbf{x=}\int_{\partial \Omega }\left(
v\frac{\partial u}{\partial n}-u\frac{\partial v}{\partial n}\right) d\sigma
=0$ perch\'{e} $u,v=0$ su $\partial \Omega $. Quindi, essendo $\lambda \neq
\mu $, $\int_{\Omega }uvd\mathbf{x}=0$. $\blacksquare $

Il rapporto al lato destro di $\lambda =\frac{-\int_{\Omega }\left\vert
\left\vert \nabla u\right\vert \right\vert ^{2}d\mathbf{x}}{\int_{\Omega
}u^{2}d\mathbf{x}}$ si dice quoziente di Rayleigh (approfondisci).

Torniamo ora al problema di Cauchy-Dirichlet per le onde: $\left\{ 
\begin{array}{c}
\Delta X\left( \mathbf{x}\right) =\lambda X\left( \mathbf{x}\right) \text{
in }\Omega \\ 
X\left( \mathbf{x}\right) =0\text{ in }\partial \Omega%
\end{array}%
\right. ,T^{\prime \prime }\left( t\right) =c^{2}\lambda T\left( t\right) $
in $\left( 0,T\right) $. Poich\'{e} ora si sa che $\lambda <0$, si pu\`{o}
risolvere la seconda equazione: posto $\omega ^{2}=-\lambda $, $T\left(
t\right) =a\cos \omega ct+b\sin \omega ct$, per cui $u\left( \mathbf{x}%
,t\right) =X\left( \mathbf{x}\right) \left( a\cos \omega ct+b\sin \omega
ct\right) $. Ad ora in realt\`{a} non sappiamo niente su \textit{quanti}
sono gli autovalori $\lambda $: se ne esistesse una successione, e anche una
successione di autofunzioni, si potrebbe scrivere $u$ come serie di
autofunzioni moltiplicate per funzioni trigonometriche; cos\`{\i} si
potrebbe tentare di imporre le condizioni iniziali.

Questa speranza \`{e} avallata dal risultato (2) del teorema sopra, che fa
sperare che esista una base ortonormale di $L^{2}$ formata da autofunzioni
del laplaciano (quando si \`{e} risolta l'equazione della corda vibrante in
effetti si \`{e} scritta la soluzione come serie che in spazio \`{e} di
autofunzioni del laplaciano unidimensionale sul segmento, $\sin \frac{n\pi x%
}{L}$). Questo \`{e} garantito dal seguente teorema.

\textbf{Teo}%
\begin{gather*}
\text{Hp: }\Omega \subseteq 
%TCIMACRO{\U{211d} }%
%BeginExpansion
\mathbb{R}
%EndExpansion
^{n}\text{ dominio limitato} \\
\text{Ts: esiste una successione di autovalori di }\Delta \text{ }\left\{
\lambda _{k}\right\} _{k}:\lambda _{k}\rightarrow ^{k\rightarrow +\infty
}-\infty \text{ e un sistema} \\
\text{ortonormale completo di }L^{2}\left( \Omega \right) \text{ }\left\{
X_{k}\right\} _{k=1}^{+\infty }\text{ formato da autofunzioni a essi relative%
}
\end{gather*}

Questo significa che la soluzione $u$ potr\`{a} essere scritta come $u\left( 
\mathbf{x},t\right) =\sum_{k=1}^{+\infty }X_{k}\left( \mathbf{x}\right)
\left( a_{k}\cos \omega _{k}ct+b_{k}\sin \omega _{k}ct\right) $ con $\omega
_{k}^{2}=-\lambda _{k}$. Si cerca una soluzione con tale forma, dove $%
a_{k},b_{k}$ sono da determinare imponendo le condizioni iniziali (che
dovranno anch'esse essere sviluppate in serie di autofunzioni $X_{k}$).
Ovviamente la forma dello sviluppo di funzioni in serie di autofunzioni
dipende dal dominio $\Omega $: in geometrie particolarmente semplici si
trovaranno ad esempio le serie di Fourier.

Si noti che questa strategia si applica per $\Omega $ dominio limitato
qualsiasi.

Il problema di trovare autovalori e autofunzioni si complica al complicarsi
del dominio $\Omega $; il problema si sa risolvere in alcuni dom\`{\i}ni
particolarmente semplici: in $n=2$, sul rettangolo, il cerchio (da cui, per
separazione di variabili, si trovano le funzioni di Bessel) e il triangolo
equilatero; in $n=3$ sul parallelepipedo, la sfera e il cilindro. In
generale autofunzioni del laplaciano in vari dominii spesso diventano
categorie di funzioni note, che prendono il nome di funzioni speciali%
\footnote{%
di cui si occupa l'analisi spettrale}: come le funzioni di Bessel, i
polinomi di Chebychev...

Si risolve il problema per il caso del rettangolo in $n=2$. \footnote{%
La risoluzione del problema agli autovalori in un dominio $\Omega $ \`{e}
collegata alla risoluzione di problemi variazionali. Ad esempio in $n=2$,
fissato il perimetro (o l'area) di $\Omega =\left( 0,a\right) \times \left(
0,a\right) $, quali sono $a,b$ che minimizzano $\nu _{1,1}$? \ Fissata
l'area di $\Omega $, qual \`{e} la forma di $\Omega $ che minimizza $\nu
_{1,1}$? Pensando al significato musicale, ci\`{o} significa cercare la
forma delle membrana che minimizza la (?) della nota pi\`{u} bassa. La
risposta, nota come teorema del tamburo, \`{e} il cerchio.
\par
Inoltre ci si \`{e} chiesto se la successione di autovalori del laplaciano
per $\Omega $, fissata l'area di $\Omega $, ne determini la forma (\textit{%
can one hear the shape of a drum?}). La risposta \`{e} no: esistono domini
di forma diversa che producono la stessa successione di autovalori (\textit{%
one cannot hear the shape of a drum}).}

\textbf{Membrana vibrante rettangolare} Sia $\Omega =\left( 0,a\right)
\times \left( 0,b\right) \subseteq 
%TCIMACRO{\U{211d} }%
%BeginExpansion
\mathbb{R}
%EndExpansion
^{2}$. Il problema \`{e} $\left\{ 
\begin{array}{c}
u_{tt}-c^{2}\Delta u=0\text{ in }\Omega \times \left( 0,T\right) \\ 
u\left( x,y,0\right) =g\left( x,y\right) \text{ in }\Omega \\ 
u_{t}\left( x,y,0\right) =h\left( x,y\right) \text{ in }\Omega \\ 
u\left( x,y,t\right) =0\text{ in }\mathbf{\partial }\Omega \times \left(
0,T\right)%
\end{array}%
\right. $; con la separazione di variabili si trova $\left\{ 
\begin{array}{c}
\Delta U\left( \mathbf{x}\right) =\lambda U\left( \mathbf{x}\right) \text{
in }\Omega \\ 
U\left( \mathbf{x}\right) =0\text{ in }\partial \Omega%
\end{array}%
\right. ,T^{\prime \prime }\left( t\right) =c^{2}\lambda T\left( t\right) $
in $\left( 0,T\right) $. Per trovare le autofunzioni del laplaciano sul
rettangolo si separano di nuovo le variabili: $U\left( x,y\right) =X\left(
x\right) Y\left( y\right) $, da cui $X^{\prime \prime }Y+XY^{\prime \prime
}=\lambda XY$, cio\`{e} $\frac{X^{\prime \prime }}{X}=\lambda -\frac{%
Y^{\prime \prime }}{Y}$, che devono quindi essere uguali a $\mu $. Si
trovano i due sottoproblemi 
\begin{equation*}
\left\{ 
\begin{array}{c}
X^{\prime \prime }\left( x\right) =\mu X\left( x\right) \\ 
X\left( 0\right) =X\left( a\right) =0%
\end{array}%
\right. ,\left\{ 
\begin{array}{c}
Y^{\prime \prime }\left( y\right) =\left( \lambda -\mu \right) Y\left(
y\right) \\ 
Y\left( 0\right) =Y\left( b\right) =0%
\end{array}%
\right.
\end{equation*}
che sono perfettamente analoghi al gi\`{a} risolto problema della
corda vibrante fissata agli estremi: la soluzione \`{e} $X\left( x\right)
=\sin \frac{n\pi x}{a}$ con $\mu =-\left( \frac{n\pi }{a}\right)
^{2},Y\left( y\right) =\sin \frac{m\pi y}{b}$ con $\gamma =\lambda -\mu
=-\left( \frac{m\pi }{b}\right) ^{2}$ (il segno di questi autovalori segue
dalla soluzione diretta delle equazioni, non dalla teoria generale). Allora
la successione di autovalori \`{e} $\lambda \,_{n,m}=\mu +\gamma =-\left( 
\frac{n\pi }{a}\right) ^{2}-\left( \frac{m\pi }{b}\right) ^{2}$ (che \`{e}
una successione a due indici, coerentemente col fatto che si lavora in $%
%TCIMACRO{\U{211d} }%
%BeginExpansion
\mathbb{R}
%EndExpansion
^{2}$) e le autofunzioni sono $U_{n,m}\left( x,y\right) =\sin \frac{n\pi x}{a%
}\sin \frac{m\pi y}{b}$. Si \`{e} risolto il problema agli autovalori per il
laplaciano sul rettangolo.

Ora si pu\`{o} risolvere il problema iniziale: avendo $\lambda _{n,m}=-\pi
^{2}\left( \frac{n^{2}}{a^{2}}+\frac{m^{2}}{b^{2}}\right) $, si pone $\omega
^{2}=-\lambda \,_{n,m}$ e allora $T\left( t\right) =a\cos \omega ct+b\sin
\omega ct$.

Quindi $u_{n,m}\left( x,y,t\right) =U_{n,m}\left( \mathbf{x}\right) T\left(
t\right) =\sin \frac{n\pi x}{a}\sin \frac{m\pi y}{b}\left( a_{n,m}\cos \pi 
\sqrt{\frac{n^{2}}{a^{2}}+\frac{m^{2}}{b^{2}}}ct+b_{n,m}\sin \pi \sqrt{\frac{%
n^{2}}{a^{2}}+\frac{m^{2}}{b^{2}}}ct\right) $ e come al solito $u\left(
x,y,t\right) =\sum_{n,m=1}^{+\infty }u_{n,m}\left( x,y,t\right) $: la
candidata soluzione \`{e}%
\begin{equation*}
u\left( x,y,t\right) =\sum_{n,m=1}^{+\infty }\sin \frac{n\pi x}{a}\sin \frac{%
m\pi y}{b}\left( a_{n,m}\cos \pi \sqrt{\frac{n^{2}}{a^{2}}+\frac{m^{2}}{b^{2}%
}}ct+b_{n,m}\sin \pi \sqrt{\frac{n^{2}}{a^{2}}+\frac{m^{2}}{b^{2}}}ct\right)
\end{equation*}

Ora si possono imporre la condizioni iniziali. $g\left( x,y\right) =u\left(
x,y,0\right) =\sum_{n,m=1}^{+\infty }a_{n,m}\sin \frac{n\pi x}{a}\sin \frac{%
m\pi y}{b}$: per determinare $a_{n,m}$ occorre sviluppare $g\left(
x,y\right) $ in serie di Fourier di soli seni in $\left( 0,a\right) \times
\left( 0,b\right) $, la quale - essendo in $%
%TCIMACRO{\U{211d} }%
%BeginExpansion
\mathbb{R}
%EndExpansion
^{2}$ - sar\`{a} una serie in due variabili, proprio come quella sopra.
Analogamente a quanto visto in passato, $a_{n,m}=\frac{2}{a}\frac{2}{b}%
\int_{0}^{a}\int_{0}^{b}g\left( x,y\right) \sin \frac{n\pi x}{a}\sin \frac{%
m\pi y}{b}dydx$.

Invece $h\left( x,y\right) =u_{t}\left( x,y,0\right) =\sum_{n,m=1}^{+\infty
}b_{n,m}\pi c\sqrt{\frac{n^{2}}{a^{2}}+\frac{m^{2}}{b^{2}}}\sin \frac{n\pi x%
}{a}\sin \frac{m\pi y}{b}$, e quindi $b_{n,m}=\frac{1}{\pi c\sqrt{\frac{n^{2}%
}{a^{2}}+\frac{m^{2}}{b^{2}}}}\beta _{n,m}=\frac{1}{\pi c\sqrt{\frac{n^{2}}{%
a^{2}}+\frac{m^{2}}{b^{2}}}}\frac{4}{ab}\int_{0}^{a}\int_{0}^{b}h\left(
x,y\right) \sin \frac{n\pi x}{a}\sin \frac{m\pi y}{b}dydx$.

Sotto opportune ipotesi di regolarit\`{a}, che sono simili a quelle viste
per la corda vibrante ($g$ un po' meglio di $C^{2}$, $h$ un po' meglio di $%
C^{1}$, soddisfacenti alcune condizioni di raccordo), questa \`{e} la%
\footnote{%
ne abbiamo dimostrato l'unicit\`{a}!} soluzione classica del problema di
Cauchy-Dirichlet per l'equazione delle onde sulla membrana rettangolare.

Proprio come le armoniche fondamentali per la corda vibrante, i singoli
addendi $u_{n,m}\left( x,y,t\right) $, che si dicono parziali, hanno il
significato di vibrazioni stazionarie. Per la corda vibrante, il grafico di $%
u_{n}$ a $t$ fissato presentava $n-2$ punti nodali: per la membrana $u_{n,m}$
ha $n-1,m-1$ punti fissati, quindi presenta $n-1$ linee nodali interne
del tipo $x=...$ e $m-1$ linee nodali interne del tipo $y=...$, che \textit{%
non vibrano}.

Invece, fissato $\left( x,y\right) $, il seno e il coseno di $u_{n,m}$ hanno
pulsazione $\omega c=\pi c\sqrt{\frac{n^{2}}{a^{2}}+\frac{m^{2}}{b^{2}}}$,
periodo $T_{n,m}=\frac{2}{c\sqrt{\frac{n^{2}}{a^{2}}+\frac{m^{2}}{b^{2}}}}$
e frequenza $\nu _{n,m}=\frac{c}{2}\sqrt{\frac{n^{2}}{a^{2}}+\frac{m^{2}}{%
b^{2}}}$; la frequenza fondamentale di vibrazione \`{e} $\nu _{1,1}=\frac{c}{%
2}\sqrt{\frac{1}{a^{2}}+\frac{1}{b^{2}}}$. Per la membrana, le frequenze pi%
\`{u} alte - a differenza della corda vibrante - non sono multipli interi
della frequenza fondamentale, e d'altro canto i periodi delle $u_{n,m}$ non
sono sottomultipli di $T_{1,1}$. Quindi la vibrazione complessiva non \`{e}
periodica.

A questo punto si sa risolvere anche il problema del calore sul rettangolo $%
\left\{ 
\begin{array}{c}
u_{t}-D\Delta u=0\text{ in }\Omega \times \left( 0,T\right) =\left(
0,a\right) \times \left( 0,b\right) \times \left( 0,T\right) \\ 
u\left( \mathbf{x},0\right) =g\left( \mathbf{x}\right) \text{ in }\Omega \\ 
u=0\text{ in }\partial \Omega%
\end{array}%
\right. $, che d\`{a} luogo ai problemi $\left\{ 
\begin{array}{c}
\Delta X\left( \mathbf{x}\right) =\lambda X\left( \mathbf{x}\right) \text{
in }\Omega \\ 
X\left( \mathbf{x}\right) =0\text{ in }\partial \Omega%
\end{array}%
\right. ,T^{\prime }\left( t\right) =D\lambda T\left( t\right) $ in $\left(
0,T\right) $. Allora le autofunzioni sono $U_{n,m}\left( x,y\right) =\sin 
\frac{n\pi x}{a}\sin \frac{m\pi y}{b}$ e $\lambda =-\pi ^{2}\left( \frac{%
n^{2}}{a^{2}}+\frac{m^{2}}{b^{2}}\right) $, quindi $T\left( t\right)
=ce^{-\pi ^{2}\left( \frac{n^{2}}{a^{2}}+\frac{m^{2}}{b^{2}}\right) Dt}$ e 
\begin{equation*}
u\left( x,y,t\right) =\sum_{n,m=1}^{+\infty }c_{n,m}e^{-\pi ^{2}\left( \frac{%
n^{2}}{a^{2}}+\frac{m^{2}}{b^{2}}\right) Dt}\sin \frac{n\pi x}{a}\sin \frac{%
m\pi y}{b} 
\end{equation*}
Imponendo la condizione iniziale si ha $u\left( x,y,0\right)
=\sum_{n,m=1}^{+\infty }c_{n,m}\sin \frac{n\pi x}{a}\sin \frac{m\pi y}{b}%
=g\left( x,y\right) $, e come prima i coefficienti in serie di Fourier di
soli seni di $g$ sul rettangolo sono $c_{n,m}=\frac{4}{ab}%
\int_{0}^{a}\int_{0}^{b}g\left( x,y\right) \sin \frac{n\pi x}{a}\sin \frac{%
m\pi y}{b}dydx$.

\subsection{Problema di Cauchy globale}

Ora si risolve il problema di Cauchy in tutto $%
%TCIMACRO{\U{211d} }%
%BeginExpansion
\mathbb{R}
%EndExpansion
^{n}$: $\left\{ 
\begin{array}{c}
u_{tt}-c^{2}\Delta u=f\text{ per }\mathbf{x}\in 
%TCIMACRO{\U{211d} }%
%BeginExpansion
\mathbb{R}
%EndExpansion
^{n},t>0 \\ 
u\left( \mathbf{x},0\right) =g\left( \mathbf{x}\right) \text{ in }%
%TCIMACRO{\U{211d} }%
%BeginExpansion
\mathbb{R}
%EndExpansion
^{n} \\ 
u_{t}\left( \mathbf{x},0\right) =h\left( \mathbf{x}\right) \text{ in }%
%TCIMACRO{\U{211d} }%
%BeginExpansion
\mathbb{R}
%EndExpansion
^{n}%
\end{array}%
\right. $. A differenza di quanto visto per altre equazioni, la soluzione
del problema di Cauchy globale per l'equazione delle onde ha forma e propriet%
\`{a} molto diverse a seconda di $n$. Noi faremo una trattazione iniziale
generale, che proseguiremo nel caso $n=3$; la soluzione ottenuta per $n=3$
permetter\`{a} di trovare anche quella per $n=2$.

Abbiamo dimostrato l'unicit\`{a} della soluzione su cilindri limitati;
ovviamente serve un risultato del genere anche in questo caso.

\textbf{Teo (unicit\`{a} della soluzione del problema di Cauchy globale) }

\begin{gather*}
\text{Hp: }\left\{ 
\begin{array}{c}
u_{tt}-c^{2}\Delta u=f\text{ per }\mathbf{x}\in 
%TCIMACRO{\U{211d} }%
%BeginExpansion
\mathbb{R}
%EndExpansion
^{n},t>0 \\ 
u\left( \mathbf{x},0\right) =g\left( \mathbf{x}\right) \text{ in }%
%TCIMACRO{\U{211d} }%
%BeginExpansion
\mathbb{R}
%EndExpansion
^{n} \\ 
u_{t}\left( \mathbf{x},0\right) =h\left( \mathbf{x}\right) \text{ in }%
%TCIMACRO{\U{211d} }%
%BeginExpansion
\mathbb{R}
%EndExpansion
^{n}%
\end{array}%
\right. \\
\text{Ts: nella classe }C^{2}\left( 
%TCIMACRO{\U{211d} }%
%BeginExpansion
\mathbb{R}
%EndExpansion
^{n}\times \lbrack 0,+\infty )\right) \text{ la soluzione del problema, se
esiste, \`{e} unica}
\end{gather*}

Si noti che la classe in cui vale l'unicit\`{a} \`{e} piuttosto piccola: 
\`{e} richiesta continuit\`{a} fino alla derivata seconda fino al bordo.

Si comincia a presentare un insieme di risultati che permettono di risolvere
il problema

\textbf{Proposizione}%
\begin{eqnarray*}
\text{Hp} &\text{: }&u\in C^{3}\left( 
%TCIMACRO{\U{211d} }%
%BeginExpansion
\mathbb{R}
%EndExpansion
^{n+1}\right) \text{ risolve }\left\{ 
\begin{array}{c}
u_{tt}-c^{2}\Delta u=0\text{ per }\mathbf{x}\in 
%TCIMACRO{\U{211d} }%
%BeginExpansion
\mathbb{R}
%EndExpansion
^{n},t>0 \\ 
u\left( \mathbf{x},0\right) =0\text{ in }%
%TCIMACRO{\U{211d} }%
%BeginExpansion
\mathbb{R}
%EndExpansion
^{n} \\ 
u_{t}\left( \mathbf{x},0\right) =h\left( \mathbf{x}\right) \text{ in }%
%TCIMACRO{\U{211d} }%
%BeginExpansion
\mathbb{R}
%EndExpansion
^{n}%
\end{array}%
\right. \text{ (1)} \\
\text{Ts} &\text{: }&u_{t}\text{ risolve }\left\{ 
\begin{array}{c}
w_{tt}-c^{2}\Delta w=0\text{ per }\mathbf{x}\in 
%TCIMACRO{\U{211d} }%
%BeginExpansion
\mathbb{R}
%EndExpansion
^{n},t>0 \\ 
w\left( \mathbf{x},0\right) =h\left( \mathbf{x}\right) \text{ in }%
%TCIMACRO{\U{211d} }%
%BeginExpansion
\mathbb{R}
%EndExpansion
^{n} \\ 
w_{t}\left( \mathbf{x},0\right) =0\text{ in }%
%TCIMACRO{\U{211d} }%
%BeginExpansion
\mathbb{R}
%EndExpansion
^{n}%
\end{array}%
\right. \text{ (2)}
\end{eqnarray*}

\textbf{Dim} Sia $w\left( \mathbf{x},t\right) =u_{t}\left( \mathbf{x}%
,t\right) $. E' ovvio che $w\left( \mathbf{x},0\right) =h\left( \mathbf{x}%
\right) $; inoltre $\frac{\partial }{\partial t}\left( u_{tt}-c^{2}\Delta
u\right) =0$ per ipotesi, quindi $w_{tt}-c^{2}\frac{\partial }{\partial t}%
\Delta u=w_{tt}-c^{2}\Delta w=0$, dove si \`{e} usato il teorema di Schwarz
per scambiare $\frac{\partial }{\partial t}$ e $\Delta $. Infine, vale $%
w_{t}\left( \mathbf{x},0\right) =u_{tt}\left( \mathbf{x},0\right) =\footnote{%
l'equazione differenziale, per continuit\`{a} di $u$, \`{e} soddisfatta
anche per $t=0$}c^{2}\Delta u\left( \mathbf{x},0\right) $: ma $u\left( 
\mathbf{x},0\right) =0$ in $%
%TCIMACRO{\U{211d} }%
%BeginExpansion
\mathbb{R}
%EndExpansion
^{n}$, quindi $\Delta u\left( \mathbf{x},0\right) =0$ e $w_{t}\left( \mathbf{%
x},0\right) =0$ in $%
%TCIMACRO{\U{211d} }%
%BeginExpansion
\mathbb{R}
%EndExpansion
^{n}$. $\blacksquare $

Quindi, se si sa risolvere il problema in ipotesi, grazie a questo risultato
si pu\`{o} risolvere anche 
\begin{equation*}
\left\{ 
\begin{array}{c}
u_{tt}-c^{2}\Delta u=0\text{ per }\mathbf{x}\in 
%TCIMACRO{\U{211d} }%
%BeginExpansion
\mathbb{R}
%EndExpansion
^{n},t>0 \\ 
u\left( \mathbf{x},0\right) =g\left( \mathbf{x}\right) \text{ in }%
%TCIMACRO{\U{211d} }%
%BeginExpansion
\mathbb{R}
%EndExpansion
^{n} \\ 
u_{t}\left( \mathbf{x},0\right) =h\left( \mathbf{x}\right) \text{ in }%
%TCIMACRO{\U{211d} }%
%BeginExpansion
\mathbb{R}
%EndExpansion
^{n}%
\end{array}%
\right.
\end{equation*}
per il principio di sovrapposizione. Infine, per risolvere il
problema con l'equazione non omogenea si pu\`{o} usare il metodo di Duhamel.

\textbf{Proposizione (metodo di Duhamel)}%
\begin{eqnarray*}
\text{Hp}\text{: } &&w=w\left( \mathbf{x},t,s\right) \text{ risolve }\left\{ 
\begin{array}{c}
w_{tt}-c^{2}\Delta w=0\text{ per }\mathbf{x}\in 
%TCIMACRO{\U{211d} }%
%BeginExpansion
\mathbb{R}
%EndExpansion
^{n},t>s \\ 
w\left( \mathbf{x},s,s\right) =0\text{ in }%
%TCIMACRO{\U{211d} }%
%BeginExpansion
\mathbb{R}
%EndExpansion
^{n} \\ 
w_{t}\left( \mathbf{x},s,s\right) =f\left( \mathbf{x},s\right) \text{ in }%
%TCIMACRO{\U{211d} }%
%BeginExpansion
\mathbb{R}
%EndExpansion
^{n}%
\end{array}%
\right. \\
\text{Ts}\text{: } &&u\left( \mathbf{x},t\right) =\int_{0}^{t}w\left( 
\mathbf{x},t,s\right) ds\text{ risolve }\left\{ 
\begin{array}{c}
u_{tt}-c^{2}\Delta u=f\text{ per }\mathbf{x}\in 
%TCIMACRO{\U{211d} }%
%BeginExpansion
\mathbb{R}
%EndExpansion
^{n},t>0 \\ 
u\left( \mathbf{x},0\right) =0\text{ in }%
%TCIMACRO{\U{211d} }%
%BeginExpansion
\mathbb{R}
%EndExpansion
^{n} \\ 
u_{t}\left( \mathbf{x},0\right) =0\text{ in }%
%TCIMACRO{\U{211d} }%
%BeginExpansion
\mathbb{R}
%EndExpansion
^{n}%
\end{array}%
\right.
\end{eqnarray*}

La dimostrazione \`{e} la solita verifica formale.

E' quindi sufficiente concentrarsi sulla risoluzione di $\left\{ 
\begin{array}{c}
u_{tt}-c^{2}\Delta u=0\text{ per }\mathbf{x}\in 
%TCIMACRO{\U{211d} }%
%BeginExpansion
\mathbb{R}
%EndExpansion
^{n},t>0 \\ 
u\left( \mathbf{x},0\right) =0\text{ in }%
%TCIMACRO{\U{211d} }%
%BeginExpansion
\mathbb{R}
%EndExpansion
^{n} \\ 
u_{t}\left( \mathbf{x},0\right) =h\left( \mathbf{x}\right) \text{ in }%
%TCIMACRO{\U{211d} }%
%BeginExpansion
\mathbb{R}
%EndExpansion
^{n}%
\end{array}%
\right. $, problema che affrontiamo col metodo delle medie sferiche (che
nasce dall'intuizione fisica: le onde nello spazio sono tipicamente
sferiche).

\textbf{Def} Data $u\in C^{0}\left( 
%TCIMACRO{\U{211d} }%
%BeginExpansion
\mathbb{R}
%EndExpansion
^{n+1}\right) $ e $r>0$, si dice media sferica di $u=u\left( \mathbf{x}%
,t\right) $, e si indica con $M_{u}\left( \mathbf{x},r,t\right) $, la
funzione $\frac{1}{\left\vert \partial B_{r}\left( \mathbf{x}\right)
\right\vert }\int_{\partial B_{r}\left( \mathbf{x}\right) }u\left( \mathbf{y}%
,t\right) dS\left( \mathbf{y}\right) $.

Si riscrive $M_{u}$ in modo che $\mathbf{x}$ non compaia pi\`{u} nel dominio
d'integrazione: con il cambio di variabili $\mathbf{y}=\mathbf{x}+r\mathbf{z}
$ si ha $\frac{1}{\left\vert \partial B_{r}\left( \mathbf{x}\right)
\right\vert }\int_{\partial B_{r}\left( \mathbf{x}\right) }u\left( \mathbf{y}%
,t\right) dS\left( \mathbf{y}\right) =\frac{1}{\omega _{n}r^{n-1}}%
\int_{\partial B_{1}\left( \mathbf{0}\right) }u\left( \mathbf{x}+r\mathbf{z}%
,t\right) r^{n-1}dS\left( \mathbf{z}\right) =\frac{1}{\omega _{n}}%
\int_{\partial B_{1}\left( \mathbf{0}\right) }u\left( \mathbf{x}+r\mathbf{z}%
,t\right) dS\left( \mathbf{z}\right) $. L'ultimo integrale, a differenza del
primo, \`{e} evidentemente ben definito $\forall $ $r$, e dunque si prender%
\`{a} questo come definizione operativa di media sferica.

Si osservano alcune propriet\`{a} di $M_{u}$.

\begin{description}
\item[i] $M_{u}\left( \mathbf{x},0,t\right) =\frac{1}{\omega _{n}}%
\int_{\partial B_{1}\left( \mathbf{0}\right) }u\left( \mathbf{x},t\right)
dS\left( \mathbf{z}\right) =u\left( \mathbf{x},t\right) $.

\item[ii] $M_{u}\left( \mathbf{x},-r,t\right) =\frac{1}{\omega _{n}}%
\int_{\partial B_{1}\left( \mathbf{0}\right) }u\left( \mathbf{x}-r\mathbf{z}%
,t\right) dS\left( \mathbf{z}\right) =^{}\frac{1}{\omega _{n}}%
\int_{\partial B_{1}\left( \mathbf{0}\right) }u\left( \mathbf{x}+r\mathbf{w}%
,t\right) dS\left( \mathbf{w}\right) =M_{u}\left( \mathbf{x},r,t\right) $,
grazie al cambio di variabile $\mathbf{w=-z}$. Dunque $M_{u}\left( \mathbf{x}%
,\cdot ,t\right) $ \`{e} una funzione pari.
\end{description}

L'idea \`{e} ora scoprire quali equazioni risolve $M_{u}$, e come queste
sono legate alle equazioni che risolve $u$.

\textbf{Teo (equazione di Darboux)}%
\begin{eqnarray*}
\text{Hp}\text{: } &&u\in C^{2}\left( 
%TCIMACRO{\U{211d} }%
%BeginExpansion
\mathbb{R}
%EndExpansion
^{n+1}\right) ,M_{u}\text{ \`{e} la sua media sferica} \\
\text{Ts}\text{: } &&\Delta _{\mathbf{x}}M_{u}\left( \mathbf{x},r,t\right)
=\left( \frac{\partial ^{2}}{\partial r^{2}}+\frac{n-1}{r}\frac{\partial }{%
\partial r}\right) M_{u}\left( \mathbf{x},r,t\right)
\end{eqnarray*}

Quindi il laplaciano rispetto alle coordinate cartesiane di $M_{u}$ coincide
con il laplaciano di $M_{u}$ vista come funzione radiale di $r$. Si noti che 
$C^{2}\left( 
%TCIMACRO{\U{211d} }%
%BeginExpansion
\mathbb{R}
%EndExpansion
^{n+1}\right) $ \`{e} la regolarit\`{a} necessaria affinch\'{e} l'equazione
della tesi sia un'uguaglianza tra funzioni continue.

\textbf{Teo (se }$u$ \textbf{risolve l'equazione delle onde omogenea, anche }%
$M_{u}$\textbf{\ la risolve)}%
\begin{eqnarray*}
\text{Hp}\text{: } &&u\in C^{2}\left( 
%TCIMACRO{\U{211d} }%
%BeginExpansion
\mathbb{R}
%EndExpansion
^{n+1}\right) \text{ risolve }u_{tt}-c^{2}\Delta _{\mathbf{x}}u=0\text{ per }%
\mathbf{x}\in 
%TCIMACRO{\U{211d} }%
%BeginExpansion
\mathbb{R}
%EndExpansion
^{n},t>0\text{, } \\
&&M_{u}\text{ \`{e} la sua media sferica} \\
\text{Ts}\text{: } &&M_{u}\text{ risolve }\frac{\partial ^{2}M_{u}\left( 
\mathbf{x},r,t\right) }{\partial t^{2}}-c^{2}\Delta _{\mathbf{x}}M_{u}\left( 
\mathbf{x},r,t\right) =0\text{ }\forall \text{ }r\in 
%TCIMACRO{\U{211d}}%
%BeginExpansion
\mathbb{R}%
%EndExpansion
\end{eqnarray*}

Quindi, se $u$ risolve l'equazione delle onde omogenea, anche $M_{u}$ la
risolve.

\textbf{Dim} $M_{u}\left( \mathbf{x},r,t\right) =\frac{1}{\omega _{n}}%
\int_{\partial B_{1}\left( \mathbf{0}\right) }u\left( \mathbf{x}+r\mathbf{z}%
,t\right) dS\left( \mathbf{z}\right) $, quindi $\Delta M_{u}\left( \mathbf{x}%
,r,t\right) =\frac{1}{\omega _{n}}\int_{\partial B_{1}\left( \mathbf{0}%
\right) }\Delta u\left( \mathbf{x}+r\mathbf{z},t\right) dS\left( \mathbf{z}%
\right) $, dato che $\Delta u$ \`{e} continua su un compatto. Invece $\frac{%
\partial ^{2}}{\partial t^{2}}M_{u}\left( \mathbf{x},r,t\right) =\frac{1}{%
\omega _{n}}\int_{\partial B_{1}\left( \mathbf{0}\right) }\frac{\partial ^{2}%
}{\partial t^{2}}u\left( \mathbf{x}+r\mathbf{z},t\right) dS\left( \mathbf{z}%
\right) $ per lo stesso motivo, quindi $\frac{\partial ^{2}M_{u}\left( 
\mathbf{x},r,t\right) }{\partial t^{2}}-c^{2}\Delta _{\mathbf{x}}M_{u}\left( 
\mathbf{x},r,t\right) =\frac{1}{\omega _{n}}\int_{\partial B_{1}\left( 
\mathbf{0}\right) }\left[ \frac{\partial ^{2}}{\partial t^{2}}u\left( 
\mathbf{x}+r\mathbf{z},t\right) -c^{2}\Delta u\left( \mathbf{x}+r\mathbf{z}%
,t\right) \right] dS\left( \mathbf{z}\right) =0$ perch\'{e} $u$ risolve
l'equazione. $\blacksquare $

Ora, supponendo che $u$ risolva il problema di interesse $\left\{ 
\begin{array}{c}
u_{tt}-c^{2}\Delta u=0\text{ per }\mathbf{x}\in 
%TCIMACRO{\U{211d} }%
%BeginExpansion
\mathbb{R}
%EndExpansion
^{n},t>0 \\ 
u\left( \mathbf{x},0\right) =0\text{ in }%
%TCIMACRO{\U{211d} }%
%BeginExpansion
\mathbb{R}
%EndExpansion
^{n} \\ 
u_{t}\left( \mathbf{x},0\right) =h\left( \mathbf{x}\right) \text{ in }%
%TCIMACRO{\U{211d} }%
%BeginExpansion
\mathbb{R}
%EndExpansion
^{n}%
\end{array}%
\right. $, vogliamo trovare qualche equazione semplice risolta da $M_{u}$.
Per i due teoremi sopra vale $\frac{\partial ^{2}M_{u}\left( \mathbf{x}%
,r,t\right) }{\partial t^{2}}=c^{2}\Delta _{\mathbf{x}}M_{u}\left( \mathbf{x}%
,r,t\right) =c^{2}\left( \frac{\partial ^{2}}{\partial r^{2}}+\frac{n-1}{r}%
\frac{\partial }{\partial r}\right) M_{u}$: l'equazione $\frac{\partial
^{2}M_{u}\left( \mathbf{x},r,t\right) }{\partial t^{2}}=c^{2}\left( \frac{%
\partial ^{2}}{\partial r^{2}}+\frac{n-1}{r}\frac{\partial }{\partial r}%
\right) M_{u}$ \`{e} nelle sole variabili $r,t$; $\mathbf{x}$ \`{e} un
parametro. Tale equazione assomiglia all'equazione della corda vibrante,
alla quale ci vorremmo ricondurre. Moltiplicando per $r$ si ottiene $r\frac{%
\partial ^{2}M_{u}\left( \mathbf{x},r,t\right) }{\partial t^{2}}=c^{2}\left(
r\frac{\partial ^{2}}{\partial r^{2}}+\left( n-1\right) \frac{\partial }{%
\partial r}\right) M_{u}$: voglio vedere $rM_{u}$ come soluzione, e quindi
riscrivere il lato destro come operatore differenziale semplice.

$\frac{\partial ^{2}}{\partial r^{2}}rM_{u}=\frac{\partial }{\partial r}%
\left( M_{u}+r\frac{\partial M_{u}}{\partial r}\right) =2\frac{\partial M_{u}%
}{\partial r}+r\frac{\partial ^{2}M_{u}}{\partial r^{2}}$: ma questo \`{e}
proprio il lato destro dell'equazione sopra, se $n=3$. Si \`{e} quindi
mostrato che, se e solo se $n=3$, se $u$ risolve il problema allora $v\left(
r,t\right) =rM_{u}\left( \mathbf{x},r,t\right) $ risolve l'equazione della
corda vibrante illimitata $\frac{\partial ^{2}v}{\partial t^{2}}=c^{2}\frac{%
\partial ^{2}v}{\partial r^{2}}$.

Per scoprire quali condizioni iniziali soddisfa $v$, osservo quelle di $u$: $%
u\left( \mathbf{x},0\right) =0$, quindi $v\left( r,0\right) =r\frac{1}{%
\omega _{n}}\int_{\partial B_{1}\left( \mathbf{0}\right) }u\left( \mathbf{x}%
+r\mathbf{z},0\right) dS\left( \mathbf{z}\right) =0$, mentre $v_{t}\left(
r,0\right) =r\frac{1}{\omega _{n}}\int_{\partial B_{1}\left( \mathbf{0}%
\right) }u_{t}\left( \mathbf{x}+r\mathbf{z},0\right) dS\left( \mathbf{z}%
\right) =r\frac{1}{\omega _{n}}\int_{\partial B_{1}\left( \mathbf{0}\right)
}h\left( \mathbf{x}+r\mathbf{z}\right) dS\left( \mathbf{z}\right)
=rM_{h}\left( \mathbf{x},r\right) $ (si noti che $h$ non dipende da $t$, che
dunque non \`{e} pi\`{u} un argomento). Dunque $v$ risolve $\left\{ 
\begin{array}{c}
v_{tt}-c^{2}v_{rr}=0\text{ se }r\in 
%TCIMACRO{\U{211d} }%
%BeginExpansion
\mathbb{R}
%EndExpansion
,t>0 \\ 
v\left( r,0\right) =0\text{ se }r\in 
%TCIMACRO{\U{211d} }%
%BeginExpansion
\mathbb{R}
%EndExpansion
\\ 
v_{t}\left( r,0\right) =rM_{h}\text{ se }r\in 
%TCIMACRO{\U{211d}}%
%BeginExpansion
\mathbb{R}%
%EndExpansion
\end{array}%
\right. $, che \`{e} il problema della corda vibrante illimitata, la cui
soluzione \`{e}, per la formula di D'Alembert, $v\left( r,t\right) =\frac{1}{%
2c}\int_{r-ct}^{r+ct}sM_{h}\left( \mathbf{x},s\right) ds=rM_{u}\left( 
\mathbf{x},r,t\right) $. Allora $M_{u}\left( \mathbf{x},r,t\right) =\frac{1}{%
2cr}\int_{r-ct}^{r+ct}sM_{h}\left( \mathbf{x},s\right) ds$, e $u\left( 
\mathbf{x},t\right) =M_{u}\left( \mathbf{x},0,t\right) $: in realt\`{a} si
deve calcolare il limite, non essendo $M_{u}$ definita in $r=0$. Ma $%
\int_{-ct}^{ct}sM_{h}\left( \mathbf{x},s\right) ds=0$ perch\'{e} integrale
di una funzione dispari su un intervallo simmetrico: il limite d\`{a} una
forma d'indeterminazione del tipo $\frac{0}{0}$, che si scioglie con il
teorema di De L'Hopital. $\lim_{r\rightarrow 0^{+}}M_{u}\left( \mathbf{x}%
,r,t\right) =\lim_{r\rightarrow 0^{+}}\frac{1}{2cr}\int_{r-ct}^{r+ct}sM_{h}%
\left( \mathbf{x},s\right) ds=\lim_{r\rightarrow 0^{+}}\frac{1}{2c}\left(
\left( r+ct\right) M_{h}\left( \mathbf{x},r+ct\right) -\left( r-ct\right)
M_{h}\left( \mathbf{x},r-ct\right) \right) $: la funzione ad argomento \`{e} 
\begin{equation*}
\frac{1}{2c}\left[ r\left( M_{h}\left( \mathbf{x},r+ct\right) -M_{h}\left( 
\mathbf{x},r-ct\right) \right) +ct\left( M_{h}\left( \mathbf{x},r+ct\right)
+M_{h}\left( \mathbf{x},r-ct\right) \right) \right]
\end{equation*}
che per $r\rightarrow
0^{+}$ tende a $tM_{h}\left( \mathbf{x},ct\right) $.

Quindi la candidata soluzione del problema (1) \`{e}%
\begin{equation*}
u\left( \mathbf{x},t\right) =tM_{h}\left( \mathbf{x},ct\right) =\frac{t}{%
\left\vert \partial B_{ct}\left( \mathbf{x}\right) \right\vert }%
\int_{\partial B_{ct}\left( \mathbf{x}\right) }h\left( \mathbf{y}\right)
dS\left( \mathbf{y}\right)
\end{equation*}

e quella del problema (2)%
\begin{equation*}
u\left( \mathbf{x},t\right) =\frac{\partial }{\partial t}\left( \frac{t}{%
\left\vert \partial B_{ct}\left( \mathbf{x}\right) \right\vert }%
\int_{\partial B_{ct}\left( \mathbf{x}\right) }g\left( \mathbf{y}\right)
dS\left( \mathbf{y}\right) \right)
\end{equation*}

Si \`{e} dimostrato il seguente

\textbf{Teo (formula di Kirchoff)}%
\begin{eqnarray*}
\text{Hp} &\text{: }&\left\{ 
\begin{array}{c}
u_{tt}-c^{2}\Delta u=0\text{ per }\mathbf{x}\in 
%TCIMACRO{\U{211d} }%
%BeginExpansion
\mathbb{R}
%EndExpansion
^{3},t>0 \\ 
u\left( \mathbf{x},0\right) =g\left( \mathbf{x}\right) \text{ in }%
%TCIMACRO{\U{211d} }%
%BeginExpansion
\mathbb{R}
%EndExpansion
^{3} \\ 
u_{t}\left( \mathbf{x},0\right) =h\left( \mathbf{x}\right) \text{ in }%
%TCIMACRO{\U{211d} }%
%BeginExpansion
\mathbb{R}
%EndExpansion
^{3}%
\end{array}%
\right. ,h\in C^{2}\left( 
%TCIMACRO{\U{211d} }%
%BeginExpansion
\mathbb{R}
%EndExpansion
^{3}\right) ,g\in C^{3}\left( 
%TCIMACRO{\U{211d} }%
%BeginExpansion
\mathbb{R}
%EndExpansion
^{3}\right) \\
\text{Ts} &\text{: }&u\left( \mathbf{x},t\right) =\frac{1}{4\pi c^{2}t}%
\int_{\partial B_{ct}\left( \mathbf{x}\right) }h\left( \mathbf{y}\right)
dS\left( \mathbf{y}\right) +\frac{\partial }{\partial t}\left( \frac{t}{4\pi
c^{2}t^{2}}\int_{\partial B_{ct}\left( \mathbf{x}\right) }g\left( \mathbf{y}%
\right) dS\left( \mathbf{y}\right) \right) \text{ risolve l'equazione}
\end{eqnarray*}

Le ipotesi sono quelle delle due proposizioni. Si noti che su $g$ serve un
grado di regolarit\`{a} in pi\`{u} perch\'{e} per calcolare $u_{tt}$ serve
la derivata terza di $g$. La dimostrazione si fa con il solito cambio di
variabili: se $\mathbf{y=x}+ct\mathbf{z}$, $dS\left( \mathbf{y}\right)
=\left( ct\right) ^{2}dS\left( \mathbf{z}\right) $ (perch\'{e} $n=3$) e $%
u\left( \mathbf{x},t\right) =\frac{t}{4\pi }\int_{\partial B_{1}\left( 
\mathbf{0}\right) }h\left( \mathbf{x+}ct\mathbf{z}\right) dS\left( \mathbf{z}%
\right) +\frac{\partial }{\partial t}\left( \frac{t}{4\pi }\int_{\partial
B_{1}\left( \mathbf{0}\right) }g\left( \mathbf{x+}ct\mathbf{z}\right)
dS\left( \mathbf{z}\right) \right) $; il secondo addendo \`{e} $\frac{1}{%
4\pi }\int_{\partial B_{1}\left( \mathbf{0}\right) }g\left( \mathbf{x+}ct%
\mathbf{z}\right) dS\left( \mathbf{z}\right) +\frac{t}{4\pi }\int_{\partial
B_{1}\left( \mathbf{0}\right) }\left\langle \nabla g\left( \mathbf{x+}ct%
\mathbf{z,}c\mathbf{z}\right) \right\rangle dS\left( \mathbf{z}\right) $.

Quindi si chiede $g\in C^{3},h\in C^{2}$ (ipotesi che non possono essere
indebolite), e nonostante ci\`{o} $u$ \`{e} solo $C^{2}$. Perlomeno
limitatamente a $g$, l'equazione delle onde in $%
%TCIMACRO{\U{211d} }%
%BeginExpansion
\mathbb{R}
%EndExpansion
^{3}$ mostra un fenomeno di perdita di regolarit\`{a}: $u$ \`{e} meno
regolare di $g$.

Si esamina il dominio di dipendenza. Il valore di $u\left( \mathbf{x}%
,t\right) $ dipende dai valori di $h$ e $g$ nei punti di $\partial
B_{ct}\left( \mathbf{x}\right) $, cio\`{e} il dominio \`{e} $\left\{ \mathbf{%
y}\in 
%TCIMACRO{\U{211d} }%
%BeginExpansion
\mathbb{R}
%EndExpansion
^{3}:\left\vert \left\vert \mathbf{x-y}\right\vert \right\vert =ct\right\} $%
. Questo significa che un segnale che a $t=0$ sia concentrato in $\mathbf{y}$
\`{e} avvertito al tempo $t=\frac{\left\vert \left\vert \mathbf{x-y}%
\right\vert \right\vert }{c}$ (principio di Huygens forte).

Per risolvere il problema non omogeneo basta applicare il principio di
Duhamel.

\textbf{Metodo della discesa} Ora si risolve il problema per $n=2$ con il
metodo della discesa\footnote{%
si intende dimensionale} di Hadamard, che permette di dedurre dalla formula
con $n=3$ il risultato per $n=2$.

Sia $u\left( \mathbf{x,}t\right) $, $\mathbf{x}\in 
%TCIMACRO{\U{211d} }%
%BeginExpansion
\mathbb{R}
%EndExpansion
^{2}$, soluzione di $\left\{ 
\begin{array}{c}
u_{tt}-c^{2}\Delta u=0\text{ per }\mathbf{x}\in 
%TCIMACRO{\U{211d} }%
%BeginExpansion
\mathbb{R}
%EndExpansion
^{2},t>0 \\ 
u\left( \mathbf{x},0\right) =0\text{ in }%
%TCIMACRO{\U{211d} }%
%BeginExpansion
\mathbb{R}
%EndExpansion
^{2} \\ 
u_{t}\left( \mathbf{x},0\right) =h\left( \mathbf{x}\right) \text{ in }%
%TCIMACRO{\U{211d} }%
%BeginExpansion
\mathbb{R}
%EndExpansion
^{2}%
\end{array}%
\right. $. $u$ pu\`{o} essere vista anche come soluzione del problema $%
\left\{ 
\begin{array}{c}
u_{tt}-c^{2}\left( u_{x_{1}x_{1}}+u_{x_{2}x_{2}}+u_{x_{3}x_{3}}\right) =0%
\text{ per }\mathbf{x}\in 
%TCIMACRO{\U{211d} }%
%BeginExpansion
\mathbb{R}
%EndExpansion
^{3},t>0 \\ 
u\left( \mathbf{x},0\right) =0\text{ in }%
%TCIMACRO{\U{211d} }%
%BeginExpansion
\mathbb{R}
%EndExpansion
^{3} \\ 
u_{t}\left( \mathbf{x},0\right) =h\left( \mathbf{x}\right) \text{ in }%
%TCIMACRO{\U{211d} }%
%BeginExpansion
\mathbb{R}
%EndExpansion
^{3}%
\end{array}%
\right. $, dato che $u_{x_{3}x_{3}}=0$. Ma allora la soluzione \`{e} $%
u\left( \mathbf{x},t\right) =\frac{1}{4\pi c^{2}t}\int_{\partial
B_{ct}\left( \mathbf{x}\right) }h\left( y_{1},y_{2}\right) dS\left(
y_{1},y_{2},y_{3}\right) $: \`{e} un integrale di superficie che si pu\`{o}
scrivere come integrale doppio sfruttando la parametrizzazione della
superficie della sfera. $\left( x_{1}-y_{1}\right) ^{2}+\left(
x_{2}-y_{2}\right) ^{2}+\left( x_{3}-y_{3}\right) ^{2}=r^{2}+\left(
x_{3}-y_{3}\right) ^{2}=c^{2}t^{2}$, ma $x_{3}$ pu\`{o} essere considerata
nulla perch\'{e} $u$ non dipende da $x_{3}$. Quindi $%
y_{3}^{2}=c^{2}t^{2}-r^{2}$ e il dominio d'integrazione pu\`{o} essere
scritto come l'unione delle due superfici $\Sigma _{+}=\left\{ \mathbf{y}\in 
%TCIMACRO{\U{211d} }%
%BeginExpansion
\mathbb{R}
%EndExpansion
^{3}:\left( y_{1},y_{2}\right) \in 
%TCIMACRO{\U{211d} }%
%BeginExpansion
\mathbb{R}
%EndExpansion
^{2},y_{3}=\sqrt{c^{2}t^{2}-r^{2}}\right\} $, $\Sigma _{-}=\left\{ \mathbf{y}%
\in 
%TCIMACRO{\U{211d} }%
%BeginExpansion
\mathbb{R}
%EndExpansion
^{3}:\left( y_{1},y_{2}\right) \in 
%TCIMACRO{\U{211d} }%
%BeginExpansion
\mathbb{R}
%EndExpansion
^{2},y_{3}=-\sqrt{c^{2}t^{2}-r^{2}}\right\} $. Ponendo $f\left(
y_{1},y_{2}\right) =\sqrt{c^{2}t^{2}-r^{2}}=\sqrt{c^{2}t^{2}-\left(
x_{1}-y_{1}\right) ^{2}-\left( x_{2}-y_{2}\right) ^{2}}$, vale $\frac{%
\partial f}{\partial y_{1}}=\frac{x_{1}-y_{1}}{\sqrt{c^{2}t^{2}-r^{2}}}$, $%
\frac{\partial f}{\partial y_{2}}=\frac{x_{2}-y_{2}}{\sqrt{c^{2}t^{2}-r^{2}}}
$, per cui $dS=\sqrt{1+\frac{r^{2}}{c^{2}t^{2}-r^{2}}}dy_{1}dy_{2}=\frac{ct}{%
\sqrt{c^{2}t^{2}-r^{2}}}dy_{1}dy_{2}$ e l'integrale diventa $u\left( \mathbf{%
x},t\right) =\frac{1}{4\pi c^{2}t}\left( \int_{\Sigma _{+}}+\int_{\Sigma
_{-}}\right) =\frac{1}{4\pi c^{2}t}2\int_{r^{2}<c^{2}t^{2}}h\left(
y_{1},y_{2}\right) \frac{ct}{\sqrt{c^{2}t^{2}-r^{2}}}dy_{1}dy_{2}=\frac{1}{%
2\pi c}\int_{\left\vert \left\vert \mathbf{x-y}\right\vert \right\vert <ct}%
\frac{h\left( \mathbf{y}\right) }{\sqrt{c^{2}t^{2}-\left\vert \left\vert 
\mathbf{x-y}\right\vert \right\vert ^{2}}}dy_{1}dy_{2}$. I due integrali di
superficie coincidono perch\'{e} la funzione integranda non dipende dalla
terza variabile (che nelle due superfici ha segno diverso). Quindi la
soluzione al problema sopra \`{e}%
\begin{equation*}
u\left( \mathbf{x},t\right) =\frac{1}{2\pi c}\int_{\left\vert \left\vert 
\mathbf{x-y}\right\vert \right\vert <ct}\frac{h\left( \mathbf{y}\right) }{%
\sqrt{c^{2}t^{2}-\left\vert \left\vert \mathbf{x-y}\right\vert \right\vert
^{2}}}d\mathbf{y}
\end{equation*}

che si dice formula di Poisson. Quindi la soluzione di $\left\{ 
\begin{array}{c}
u_{tt}-c^{2}\Delta u=0\text{ per }\mathbf{x}\in 
%TCIMACRO{\U{211d} }%
%BeginExpansion
\mathbb{R}
%EndExpansion
^{2},t>0 \\ 
u\left( \mathbf{x},0\right) =g\left( \mathbf{x}\right) \text{ in }%
%TCIMACRO{\U{211d} }%
%BeginExpansion
\mathbb{R}
%EndExpansion
^{2} \\ 
u_{t}\left( \mathbf{x},0\right) =h\left( \mathbf{x}\right) \text{ in }%
%TCIMACRO{\U{211d} }%
%BeginExpansion
\mathbb{R}
%EndExpansion
^{2}%
\end{array}%
\right. $ \`{e} 
\begin{equation*}
u\left( \mathbf{x},t\right) =\frac{1}{2\pi c}\int_{\left\vert \left\vert 
\mathbf{x-y}\right\vert \right\vert <ct}\frac{h\left( \mathbf{y}\right) }{%
\sqrt{c^{2}t^{2}-\left\vert \left\vert \mathbf{x-y}\right\vert \right\vert
^{2}}}d\mathbf{y}+\frac{1}{2\pi c}\frac{\partial }{\partial t}%
\int_{\left\vert \left\vert \mathbf{x-y}\right\vert \right\vert <ct}\frac{%
g\left( \mathbf{y}\right) }{\sqrt{c^{2}t^{2}-\left\vert \left\vert \mathbf{%
x-y}\right\vert \right\vert ^{2}}}d\mathbf{y}
\end{equation*}

Si noti in particolare che il dominio d'integrazione \`{e} molto diverso da
quello della formula di Kirchoff: ora $u\left( \mathbf{x},t\right) $ dipende
dai valori di $g$ e $h$ negli $\mathbf{y}:\left\vert \left\vert \mathbf{x-y}%
\right\vert \right\vert \leq ct$, per cui il valore delle condizioni
iniziali in $\mathbf{y}$ influenza il valore di $u$ in $\mathbf{x}$ per
tutti i $t>\frac{\left\vert \left\vert \mathbf{x-y}\right\vert \right\vert }{%
c}$.

\section{Intermezzi}

\subsection{Significati probabilistici delle EDP}

Nel 1826 il botanico scozzese Robert Brown osserv\`{o} al microscopio che
piccoli granelli di polline sospesi nell'acqua sono soggetti a un perpetuo
movimento, con piccoli spostamenti lungo traiettorie apparentemente caotiche
e irregolari. Varie interpretazioni del fenomeno furono proposte nei decenni
seguenti; nel 1877, Delsaux espresse per la prima volta l'idea che questo
movimento fosse causato dagli urti delle molecole del liquido con le
particelle in sospensione. Ma le molecole dell'acqua, a differenza dei
granelli di polline, sono troppo piccole per poter essere direttamente
osservate al microscopio: per confermare quest'ipotesi occorreva
un'argomentazione teorica che permettesse di prevedere quantitativamente il
fenomeno osservato a partire da una teoria sui fenomeni che accadono a
livello molecolare.

Immaginiamo che in un liquido avente coefficiente di viscosit\`{a} $k$ siano
sospese $n$ particelle sferiche identiche, di raggio $a$ (i
\textquotedblleft granelli di polline\textquotedblright\ di Brown); indicato
con $f(x,t)$ il numero di particelle per unit\`{a} di volume nel punto $x$
all'istante $t$, si ha $\int f\left( x,t\right) dx=n$ $\forall $ $t$.
Einstein dimostra a partire dalla teoria cinetica del calore che $f$
soddisfa l'equazione di diffusione $\frac{\partial f}{\partial t}\left(
x,t\right) -D\frac{\partial ^{2}f}{\partial x^{2}}=0$, con $D$ noto. Se si
fa l'ipotesi (ideale) che $f\left( x,0\right) =0$ se $x\neq 0$, cio\`{e}
tutte le particelle all'istante iniziale si trovano nell'origine ($f\left(
x,0\right) =\delta _{0}$), allora $f\left( x,t\right) =\frac{n}{\sqrt{4\pi Dt%
}}e^{-\frac{x^{2}}{4Dt}}$: $n$ volte il nucleo del calore unidimensionale,
perch\'{e} la convoluzione del nucleo con la delta restituisce il nucleo. Se
si normalizza $f$ dividendo per $n$ si trova la densit\`{a} di probabilit%
\`{a} $p\left( x,t\right) =\frac{1}{\sqrt{4\pi Dt}}e^{-\frac{x^{2}}{4Dt}}$,
che \`{e} la densit\`{a} di una v. a. gaussiana con valore atteso nullo e
varianza $2Dt$ e ha il significato di densit\`{a} di probabilit\`{a} che una 
\textit{singola} particella, che per $t=0$ si trova in $x=0$, si trovi nel
punto $x$ all'istante $t$.

Chiediamoci ora: dopo un tempo $t$, dove si trover\`{a} la particella?
Naturalmente se le oscillazioni sono caotiche con ugual probabilit\`{a} di
muoversi in ogni direzione, il valore atteso della posizione sar\`{a} 0.
Difatti (sempre nel caso unidimensionale, per semplicit\`{a}) $\mathbf{E}%
\left( X_{t}\right) =\int_{%
%TCIMACRO{\U{211d} }%
%BeginExpansion
\mathbb{R}
%EndExpansion
}x\frac{\text{ }1}{\sqrt{4\pi Dt}}e^{-\frac{x^{2}}{4Dt}}dx=0$ $\forall $ $t$.

La domanda pi\`{u} interessante \`{e}, invece: dopo un tempo t, a quale
distanza dal punto di partenza si trover\`{a} la particella? Pi\`{u}
precisamente, calcoliamo la deviazione standard $\sqrt{\mathbf{E}\left(
X_{t}^{2}\right) }=\left( \int_{%
%TCIMACRO{\U{211d} }%
%BeginExpansion
\mathbb{R}
%EndExpansion
}x^{2}\frac{\text{ }1}{\sqrt{4\pi Dt}}e^{-\frac{x^{2}}{4Dt}}dx\right) ^{%
\frac{1}{2}}=\sqrt{2Dt}$.

Quindi Einstein ha dimostrato che una particella ("di polline") che
all'istante $0$ si trova nell'origine allo scorrere del tempo si muover\`{a}
(sotto l'azione del bombardamento delle molecole d'acqua soggette ad
agitazione termica) con un moto caotico e irregolare, con una densit\`{a} di
probabilit\`{a} di transizione (cio\`{e} la densit\`{a} di probabilit\`{a}
che la particella al tempo $t$ si trovi in $x$) che \`{e} gaussiana e - a
meno di una costante - risolve l'equazione di diffusione. E' un esempio di
processo stocastico continuo, che prende il nome di moto browniano: la
posizione nello spazio della particella \`{e} una famiglia $X_{t}$ di
variabili aleatorie (in generale vettori aleatori) dipendenti dal parametro $%
t$, con la condizione $X_{0}=0$ e densit\`{a} di probabilit\`{a} di
transizione $p(x,t)=\frac{1}{\left( 4\pi Dt\right) ^{n/2}}e^{-\frac{x^{2}}{%
4Dt}}$. Se si rimuove la condizione $X_{0}=0$ e suppone invece $X_{0}=x$, la
densit\`{a} \`{e} la traslata $p_{x}\left( y,t\right) =\frac{1}{\left( 4\pi
Dt\right) ^{n/2}}e^{-\frac{\left\vert x-y\right\vert ^{2}}{4Dt}}$.

D'altra parte sappiamo dalla teoria dell'equazione del calore che $u\left(
x,t\right) =k\left( x,t\right) \ast g\left( x\right) =\int_{%
%TCIMACRO{\U{211d} }%
%BeginExpansion
\mathbb{R}
%EndExpansion
^{n}}\left( \frac{1}{4\pi Dt}\right) ^{\frac{n}{2}}e^{-\frac{\left\vert
\left\vert x-y\right\vert \right\vert ^{2}}{4Dt}}g\left( y\right) dy$
risolve $\left\{ 
\begin{array}{c}
u_{t}-D\Delta u=0 \\ 
u\left( x,0\right) =g\left( x\right)%
\end{array}%
\right. $: la soluzione ha il significato probabilistico $u\left( x,t\right)
=\mathbf{E}^{x}\left( g\left( X_{t}\right) \right) =\mathbf{E}\left( g\left(
X_{t}\right) |X_{0}=x\right) $, valore atteso di $g\left( X_{t}\right) $
condizionato al fatto che $X_{0}=x$. Dunque $u$ risolve $\left\{ 
\begin{array}{c}
u_{t}-D\Delta u=0 \\ 
u\left( x,0\right) =g\left( x\right)%
\end{array}%
\right. $ se e solo se $u\left( x,t\right) =\mathbf{E}^{x}\left( g\left(
X_{t}\right) \right) $ con $\left\{ X_{t}\right\} _{t>0}$ processo browniano
e $X_{0}=x$.

\textbf{Equazione di Laplace} Consideriamo ora una particella che si muove
di moto browniano in un dominio limitato $%
%TCIMACRO{\U{3a9} }%
%BeginExpansion
\Omega
%EndExpansion
\subset 
%TCIMACRO{\U{211d} }%
%BeginExpansion
\mathbb{R}
%EndExpansion
^{n}$, partendo da un punto $x\in 
%TCIMACRO{\U{3a9} }%
%BeginExpansion
\Omega
%EndExpansion
$ all'istante $t=0$. Poich\'{e} lo spostamento medio atteso dopo un tempo $t$
\`{e} proporzionale a $\sqrt{t}$, ci aspettiamo che prima o poi la
particella uscir\`{a} da $%
%TCIMACRO{\U{3a9} }%
%BeginExpansion
\Omega
%EndExpansion
$. Definiamo la variabile aleatoria \textquotedblleft istante di prima
uscita" $T_{\Omega }=\inf \left\{ t>0:X_{t}\not\in \Omega \right\} $: $%
X_{T_{\Omega }}$ \`{e} il punto (aleatorio) del bordo da cui esce la
particella.

Consideriamo ora questo \textquotedblleft gioco d'azzardo\textquotedblright
: si assegna una funzione continua $f$ su $\partial 
%TCIMACRO{\U{3a9} }%
%BeginExpansion
\Omega
%EndExpansion
$; si colloca la particella all'istante $t=0$ in un punto $x\in 
%TCIMACRO{\U{3a9} }%
%BeginExpansion
\Omega
%EndExpansion
$ e la si lascia muovere di moto browniano finch\'{e} raggiunge per la prima
volta $\partial 
%TCIMACRO{\U{3a9} }%
%BeginExpansion
\Omega
%EndExpansion
$; quando questo accade, si calcola il valore di $f$ nel punto di $\partial 
%TCIMACRO{\U{3a9} }%
%BeginExpansion
\Omega
%EndExpansion
$ da cui la particella \`{e} uscita, e tale valore rappresenta la vincita (o
perdita, se \`{e} negativo) del nostro gioco. Ci chiediamo quale sia la
vincita attesa: ovviamente dipende dal punto $x$ di partenza, quindi sar\`{a}
una funzione $u(x)$. Per come si \`{e} definita $T_{\Omega }$, la vincita
(aleatoria) \`{e} $f\left( X_{T_{\Omega }}\right) $ e la vincita attesa \`{e}
$u\left( x\right) =\mathbf{E}^{x}\left( f\left( X_{T_{\Omega }}\right)
\right) $.

Affermiamo che analiticamente, questa funzione \`{e} anche soluzione del
problema di Dirichlet $\left\{ 
\begin{array}{c}
\Delta u=0\text{ in }\Omega \\ 
u=f\text{ su }\partial \Omega%
\end{array}%
\right. $. La condizione al contorno \`{e} ovvia per il signicato
probabilistico: se $x\in \partial \Omega $, cio\`{e} la particella parte da
un punto che \`{e} gi\`{a} sul bordo, l'istante di prima uscita \`{e} $t=0$
e il punto di prima uscita \`{e} certamente $x$ stesso, perci\`{o} in questo
caso $f(X_{T_{\Omega }})=f(x)$ (il valore non \`{e} aleatorio ma certo),
quindi coincide anche con il suo valore atteso: $\mathbf{E}^{x}\left(
f\left( X_{T_{\Omega }}\right) \right) =f(x)$. Per provare che $u$ \`{e}
armonica, si prova che $u$ soddisfa la propriet\`{a} di media (di
superficie): per ogni sfera $B_{r}\left( x\right) $ la cui chiusura \`{e} in 
$\Omega $ risulta $u\left( x\right) =\frac{1}{\left\vert \partial
B_{r}\left( x\right) \right\vert }\int_{\partial B_{r}\left( x\right)
}u\left( y\right) d\sigma \left( y\right) $. Quindi si sfrutta il risultato
noto (teorema inverso rispetto al teorema della media delle funzioni
armoniche) che afferma che ogni funzione continua che soddisfa la propriet%
\`{a} di media su tutte le sfere contenute in un dominio \`{e} armonica in
quel dominio.

Per dimostrare la propriet\`{a} di media si calcola il valore atteso $%
\mathbf{E}^{x}\left( f\left( X_{T_{\Omega }}\right) \right) $ con un
argomento di valore atteso condizionato che segue dalla versione nel
continuo del teorema delle probabilit\`{a} totali. Consideriamo una sferetta 
$B_{r}\left( x\right) $ tale che $\bar{B}_{r}\left( x\right) \subseteq
\Omega $ e seguiamo nel tempo la nostra particella che, partita da $x$,
prima o poi raggiunge $\partial \Omega $. Prima di raggiungere $\partial
\Omega $ dovr\`{a} necessariamente attraversare $\partial B_{r}\left(
x\right) $ in qualche punto $z$. Indicando con $p(x,z)$ la densit\`{a} di
probabilit\`{a} che la particella partita da $x$ esca da $B_{r}(x)$ per la
prima volta nel punto $z\in \partial B_{r}\left( x\right) $, si pu\`{o}
calcolare $\mathbf{E}^{x}\left( f\left( X_{T_{\Omega }}\right) \right) $
spezzando la traiettoria della particella in due parti: la prima da $x$ a un
punto (aleatorio) $z\in \partial B_{r}\left( x\right) $ e la seconda da $z$
a $\partial \Omega $ (questa seconda parte della traiettoria potrebbe anche,
eventualmente, riattraversare la sferetta $B_{r}\left( x\right) $). Allora $%
u\left( x\right) =\mathbf{E}^{x}\left( f\left( X_{T_{\Omega }}\right)
\right) =\int_{\partial B_{r}\left( x\right) }p\left( x,z\right) \mathbf{E}%
^{z}\left( f\left( X_{T_{\Omega }}\right) \right) dz$; infatti vale $\mathbf{%
E}\left( f\left( X_{T_{\Omega }}\right) |X_{0}=x\right) =\mathbf{E}\left( 
\mathbf{E}\left( f\left( X_{T_{\Omega }}\right) |X_{T_{B_{r}\left( x\right)
}}=z\right) |X_{0}=x\right) $, che - ponendo $\mathbf{E}\left( f\left(
X_{T_{\Omega }}\right) |X_{T_{B_{r}\left( x\right) }}\right) =h\left(
Z\right) $, con $Z=X_{T_{B_{r}\left( x\right) }}$ variabile aleatoria a
supporto in $\partial B_{r}\left( x\right) $ che \`{e} il punto di prima
uscita da $B_{r}\left( x\right) $ - si calcola come $\int_{\partial
B_{r}\left( x\right) }\mathbf{E}^{z}\left( f\left( X_{T_{\Omega }}\right)
\right) p\left( x,z\right) dz$, dove $p\left( x,z\right) $ \`{e} la legge di 
$Z$ condizionata al fatto che $X_{0}=x$.

Ma tale legge \`{e} uniforme, perch\'{e} la particella che parte da $x$ si
muove con probabilit\`{a} uniforme in ogni direzione: allora $p\left(
x,z\right) dz=\frac{1}{\left\vert \partial B_{r}\left( x\right) \right\vert }%
d\sigma \left( z\right) $ e si ottiene $u\left( x\right) =\frac{1}{%
\left\vert \partial B_{r}\left( x\right) \right\vert }\int_{\partial
B_{r}\left( x\right) }\mathbf{E}^{z}\left( f\left( X_{T_{\Omega }}\right)
\right) d\sigma \left( z\right) =\frac{1}{\left\vert \partial B_{r}\left(
x\right) \right\vert }\int_{\partial B_{r}\left( x\right) }u\left( z\right)
d\sigma \left( z\right) $, come si voleva dimostrare.

Quindi $u$ risolve $\left\{ 
\begin{array}{c}
\Delta u=0\text{ in }\Omega \\ 
u=f\text{ su }\partial \Omega%
\end{array}%
\right. \Longleftrightarrow u\left( x\right) =\mathbf{E}^{x}\left( f\left(
X_{T_{\Omega }}\right) \right) $.

\subsection{Classificazione delle EDP lineari del second'ordine}

Abbiamo studiato 3 equazioni lineari del second'ordine a coefficienti
costanti: l'equazione di Laplace, del calore, delle onde. Queste sono i
prototipi di tre tipologie di equazioni, che procediamo a definire dopo aver
ripercorso le caratteristiche delle 3 equazioni viste.

Le tre equazioni menzionate hanno propriet\`{a} molto diverse. Infatti
l'equazione di Laplace \`{e} stazionaria, mentre le equazioni del calore e
delle onde sono di evoluzione: anche prescindendo dal significato fisico di
una variabile che si pensa come \textquotedblleft variabile
tempo\textquotedblright , dal punto di vista matematico per le equazioni di
evoluzione sono diversi i problemi al contorno che \`{e} naturale studiare,
e per i quali si dimostra che il problema \`{e} ben posto.

\begin{enumerate}
\item Pensando a un dominio rettangolare $\Omega =\left( 0,a\right) \times
\left( 0,b\right) $ per l'equazione di Laplace e analogamente $\Omega
=\left( 0,L\right) \times \left( 0,T\right) $ per calore e onde, tipicamente
nel primo caso \`{e} assegnata una condizione relativa a $u$ su tutto il
bordo di $\Omega $; nel secondo una condizione relativa a $u$ sulla
frontiera parabolica; nel terzo \`{e} assegnata una condizione relativa a $u$
sulla frontiera parabolica, e inoltre una condizione su $u_{t}$.
\end{enumerate}

L'equazione del calore incorpora in s\'{e} "la freccia del tempo", infatti
il problema all'indietro \`{e} malposto; al contrario, l'equazione delle
onde \`{e} invariante anche per riflessioni temporali, e il problema in
avanti e all'indietro si equivalgono.

L'equazione di Laplace e l'equazione del calore regolarizzano; invece la
soluzione dell'equazione delle onde \`{e} regolare quanto i dati, o
addirittura meno: questo fenomeno \`{e} dovuto al fatto che l'informazione 
\`{e} trasportata lungo linee caratteristiche, e quindi non c'\`{e} modo di
guadagnare regolarit\`{a} (come accade per l'equazione del trasporto), anche
se d'altro canto lo stesso fenomeno permette di scrivere l'integrale
generale dell'equazione.

L'unicit\`{a} per l'equazione di Laplace e del calore si \`{e} stabilita con
princ\`{\i}pii di massimo, che nascono da un'idea geometrica, mentre per
l'equazione delle onde si \`{e} usato il metodo dell'energia, che nasce da
un'idea fisica\footnote{%
In effetti l'intuizione fisica suggerisce anche che non ha senso appellarsi
a princ\`{\i}pii di massimo: \`{e} naturale che le soluzioni abbiano punti
di massimo anche interni al dominio.}.

Ci sono quindi differenze profonde tra queste tre equazioni, anche se sono
tutte lineari del second'ordine a coefficienti costanti. A uno sguardo
superficiale si potrebbero pensare molto simili: perch\'{e} dovremmo
aspettarci grandi differenze tra $u_{xx}+u_{yy}$ o $u_{xx}-u_{tt}$? Invece
questa differenza di segno \`{e} la differenza tra l'equazione di Laplace e
quella delle onde, ed \`{e} dunque naturale chiedersi in che modo
l'informazione di questa differenza sia scritta nella forma dell'equazione.

\textbf{Def} Date le funzioni $a_{ij},b_{k},c:%
%TCIMACRO{\U{211d} }%
%BeginExpansion
\mathbb{R}
%EndExpansion
^{n}\rightarrow 
%TCIMACRO{\U{211d} }%
%BeginExpansion
\mathbb{R}
%EndExpansion
$ al variare di $i,j,k=1,...,n$, si dice operatore lineare del second'ordine
un operatore del tipo $Lu=\sum_{i,j=1}^{n}a_{ij}\left( \mathbf{x}\right)
u_{x_{i}x_{j}}+\sum_{k=1}^{n}b_{k}\left( \mathbf{x}\right) u_{x_{k}}+c\left( 
\mathbf{x}\right) u$, qualora tale espressione sia ben definita. Si dice
parte principale dell'operatore $L$ l'operatore $Au:=\sum_{i,j=1}^{n}a_{ij}%
\left( \mathbf{x}\right) u_{x_{i}x_{j}}$.

Ci stiamo occupando quindi delle EDP lineari del second'ordine, con
coefficienti eventualmente non costanti. La parte principale dell'operatore $%
L$ \`{e} la parte del second'ordine.

In realt\`{a}, per motivi legati alla deduzione fisica del modello
differenziale (per esempio nel caso dell'equazione di diffusione), a volte
l'equazione si presenta scritta nella forma \textquotedblleft di
divergenza\textquotedblright : $Lu=\sum_{i,j=1}^{n}\left( a_{ij}\left( 
\mathbf{x}\right) u_{x_{i}}+b_{j}\left( \mathbf{x}\right) u\right)
_{x_{j}}+\sum_{k=1}^{n}c_{k}\left( \mathbf{x}\right) u_{x_{k}}+d\left( 
\mathbf{x}\right) u$. In questo caso si chiama ancora parte principale
dell'operatore la parte del second'ordine, cio\`{e} $Au:=\sum_{i,j=1}^{n}%
\left( a_{ij}\left( \mathbf{x}\right) u_{x_{i}}+b_{j}\left( \mathbf{x}%
\right) u\right) _{x_{j}}$. Si potrebbe pensare: calcolando le derivate,
ogni equazione in forma di divergenza si pu\`{o} riscrivere nella forma
standard (detta \textquotedblleft di non divergenza\textquotedblright ). Si
preferisce per\`{o} non fare quest'operazione per non dover scrivere le
derivate dei coefficienti e tenersi aperta la possibilit\`{a} che queste
derivate possano non esistere, almeno in senso classico.

Ora ci concentriamo sul caso $n=2$: le variabili sono $x$ e $t$.

\textbf{Def} Dato un operatore lineare del second'ordine $L$ con parte
principale $Pu=a\left( x,t\right) u_{xx}+b\left( x,t\right) u_{xt}+c\left(
x,t\right) u_{tt}$ e fissato $\left( x,t\right) \in 
%TCIMACRO{\U{211d} }%
%BeginExpansion
\mathbb{R}
%EndExpansion
^{2}$, si dice che $L$\ in $\left( x,t\right) $ \`{e}

\begin{description}
\item[i] ellittico se $b^{2}\left( x,t\right) -4a\left( x,t\right) c\left(
x,t\right) <0$

\item[ii] parabolico se $b^{2}\left( x,t\right) -4a\left( x,t\right) c\left(
x,t\right) =0$

\item[iii] iperbolico se $b^{2}\left( x,t\right) -4a\left( x,t\right)
c\left( x,t\right) >0$.
\end{description}

Il numero $b^{2}\left( x,t\right) -4a^{2}\left( x,t\right) c^{2}\left(
x,t\right) $ prende il nome di discriminante, proprio come per le equazioni
di secondo grado. Le condizioni i, ii, iii nella definizione possono essere
riscritte equivalentemente dicendo che $A\left( x,t\right) =\left[ 
\begin{array}{cc}
a\left( x,t\right) & \frac{b\left( x,t\right) }{2} \\ 
\frac{b\left( x,t\right) }{2} & c\left( x,t\right)%
\end{array}%
\right] $ (matrice dei coefficienti della parte principale) ha determinante
positivo, nullo e negativo, rispettivamente. Se l'operatore \`{e} a
coefficienti costanti, il suo tipo sar\`{a} il medesimo $\forall $ $\left(
x,t\right) $.

\begin{enumerate}
\item L'operatore di Laplace $Lu=u_{xx}+u_{yy}$, fissato $\left( x,y\right) $
qualsiasi, ha discriminante $-4$, per cui \`{e} ellittico.

L'operatore di diffusione, trasporto e reazione $Lu=u_{t}-Du_{xx}+vu_{x}+cu$
ha parte principale $Au=-Du_{xx}$: fissato $\left( x,t\right) $ qualsiasi,
ha discriminante $0$, per cui \`{e} parabolico.

L'operatore delle onde $Lu=u_{tt}-\omega ^{2}u_{xx}$ ha discriminante $%
4\omega ^{2}$, per cui \`{e} iperbolico.
\end{enumerate}

Gli operatori a coefficienti non costanti possono cambiare tipo da punto a
punto. Questo naturalmente rende molto difficile studiare l'equazione
associata.

\begin{enumerate}
\item L'operatore di Tricomi $Lu=u_{tt}-tu_{xx}$ ha discriminante $4t$,
dunque \`{e} iperbolico per $t>0$, parabolico per $t=0$ e ellittico per $t<0$%
.
\end{enumerate}

Ora si estende la definizione al caso di $n$ variabili.

\textbf{Def} Dato un operatore lineare del second'ordine $L$ con parte
principale $\sum_{i,j=1}^{n}a_{ij}\left( \mathbf{x}\right) u_{x_{i}x_{j}}$ e
matrice $A$ e fissato $\mathbf{x}\in 
%TCIMACRO{\U{211d} }%
%BeginExpansion
\mathbb{R}
%EndExpansion
^{n}$, si dice che $L$ in $\mathbf{x}$ \`{e} ellittico se la matrice $%
A\left( \mathbf{x}\right) $ \`{e} definita positiva oppure definita
negativa; si dice ellittico in un dominio $\Omega $ se lo \`{e} in ogni
punto di $\Omega $; si dice che \`{e} uniformemente ellittico in $\Omega $
se $A$ \`{e} uniformemente definita positiva in $\Omega $, cio\`{e} $\exists 
$ $\alpha >0:\left\langle \mathbf{\xi },A\mathbf{\xi }\right\rangle \geq
\alpha \left\vert \left\vert \mathbf{\xi }\right\vert \right\vert ^{2}$ $%
\forall $ $\mathbf{\xi }\in 
%TCIMACRO{\U{211d} }%
%BeginExpansion
\mathbb{R}
%EndExpansion
^{n}$ (q. o. in $\Omega $, cio\`{e} con $\alpha $ indipendente da $\mathbf{x}
$)

L'ultima condizione \`{e} equivalente a richiede che il minimo autovalore di 
$A\left( \mathbf{x}\right) $ sia maggiore o uguale di una costante
indipendente da $\mathbf{x}$, e in tal caso non succede che per $\mathbf{x}$
tendente a un certo valore il minimo autovalore tenda a $0$ (quindi \`{e}
una condizione pi\`{u} forte dell'ellitticit\`{a}). Ovviamente, se $L$ ha
coefficienti costanti ed \`{e} ellittico in un punto, allora \`{e} ellittico
e uniformemente ellittico in $\Omega $.

\begin{enumerate}
\item L'operatore di Laplace \`{e} uniformemente ellittico.
\end{enumerate}

\textbf{Def} Dato un operatore lineare del second'ordine $L$ con parte
principale $\sum_{i,j=1}^{n}a_{ij}\left( \mathbf{x}\right) u_{x_{i}x_{j}}$ e
matrice $A$ e fissato $\mathbf{x}\in 
%TCIMACRO{\U{211d} }%
%BeginExpansion
\mathbb{R}
%EndExpansion
^{n}$, si dice che $L$ in $\mathbf{x}$ \`{e} iperbolico se la matrice $%
A\left( \mathbf{x}\right) $ \`{e} indefinita; si dice iperbolico in un
dominio $\Omega $ se lo \`{e} in ogni punto di $\Omega $.

\begin{enumerate}
\item L'operatore delle onde $Lu=c^{2}%
\sum_{i=1}^{n}u_{x_{i}x_{i}}-u_{x_{n+1}x_{n+1}}$ \`{e} iperbolico in $%
%TCIMACRO{\U{211d} }%
%BeginExpansion
\mathbb{R}
%EndExpansion
^{n+1}$: infatti $\left\langle \mathbf{\xi },A\mathbf{\xi }\right\rangle
=c^{2}\left( \xi _{1}^{2}+...+\xi _{n}^{2}\right) -\xi _{n+1}^{2}$ cambia
segno a seconda di $\xi $.
\end{enumerate}

\textbf{Def} Dato un operatore lineare del second'ordine del tipo $Lu=u_{t}-$
$\sum_{i,j=1}^{n}a_{ij}\left( \mathbf{x}\right)
u_{x_{i}x_{j}}+\sum_{k=1}^{n}b_{k}\left( \mathbf{x}\right) u_{x_{k}}+c\left( 
\mathbf{x}\right) u$ e matrice $A$ e fissato $\left( \mathbf{x},t\right) \in 
%TCIMACRO{\U{211d} }%
%BeginExpansion
\mathbb{R}
%EndExpansion
^{n+1}$, si dice che $L$ in $\left( \mathbf{x},t\right) $ \`{e} parabolico
se la matrice $A\left( \mathbf{x}\right) $ \`{e} definita positiva; si dice parabolico in un dominio $Q\subseteq 
%TCIMACRO{\U{211d} }%
%BeginExpansion
\mathbb{R}
%EndExpansion
^{n+1}$ se lo \`{e} in ogni punto di $Q$; si dice che \`{e} uniformemente
parabolico in $Q$ se $A$ \`{e} uniformemente definita positiva in $Q$, cio%
\`{e} $\exists $ $\lambda >0:\left\langle \mathbf{\xi },A\mathbf{\xi }%
\right\rangle \geq \lambda \left\vert \left\vert \mathbf{\xi }\right\vert
\right\vert ^{2}$ $\forall $ $\mathbf{\xi }\in 
%TCIMACRO{\U{211d} }%
%BeginExpansion
\mathbb{R}
%EndExpansion
^{n}$ (q. o. in $\Omega $?).

L'operatore del calore $u_{t}-D\Delta u$ \`{e} uniformemente parabolico.

Quindi ci\`{o} che decide che un operatore sia ellittico o iperbolico \`{e}
solo la sua parte principale (quella che contiene le derivate seconde); i
termini di ordine inferiore (cio\`{e} nelle derivate prime o nella $u$) sono
ininfluenti da questo punto di vista. Ad esempio: se al laplaciano sommiamo
dei termini qualsiasi contenenti le derivate prime di $u$, l'operatore
rimane uniformemente ellittico.

Per un operatore parabolico la situazione \`{e} leggermente diversa. Ci\`{o}
che conta \`{e} $u_{t}-$ $\sum_{i,j=1}^{n}a_{ij}\left( \mathbf{x}\right)
u_{x_{i}x_{j}}$: i termini di ordine inferiore sono in questo caso quelli
nelle derivate prime rispetto alle sole $x_{i}$ e nella $u$, mentre $u_{t}$ 
\`{e} nella parte principale anche se \`{e} una derivata prima. In
un'equazione parabolica la derivata rispetto al tempo \textquotedblleft
pesa\textquotedblright\ come una derivata seconda rispetto alle variabili
spaziali.

Come gi\`{a} accennato, l'equazione di Laplace, del calore e delle onde sono
i prototipi di equazioni ellittica, parabolica, iperbolica. A equazioni dei
vari tipi corrispondono, in generale, propriet\`{a} simili a quelle delle
equazioni prototipo, ad esempio dai seguenti punti di vista:

\begin{description}
\item[a] quali sono i problemi al contorno e/o ai valori iniziali che \`{e}
naturale studiare e per i quali ci si pu\`{o} aspettare che il problema sia
ben posto;

\item[b] quali sono le propriet\`{a} di regolarizzazione che l'operatore ha
o non ha;

\item[c] quali tecniche dimostrative si possono utilizzare per dimostrare
risultati di esistenza, unicit\`{a} o dipendenza continua.
\end{description}

Quello che certamente non potremo aspettarci da un operatore (ellittico,
parabolico o iperbolico) a coefficienti variabili, eventualmente non
regolari, \`{e} di essere capaci di calcolare esplicitamente la soluzione,
come accade, in situazioni geometriche semplici, per gli operatori di
Laplace, calore, onde.

Procediamo a trattare a e b pi\`{u} nel dettaglio, con una premessa su a.

\textbf{a} Cosa vuol dire \textquotedblleft problema di
Cauchy\textquotedblright\ per una generica EDP del 2%
%TCIMACRO{\U{b0} }%
%BeginExpansion
${{}^\circ}$
%EndExpansion
ordine (che magari non contiene una variabile tempo)? Significa assegnare,
lungo un'ipersuperficie $\Sigma $ di $%
%TCIMACRO{\U{211d} }%
%BeginExpansion
\mathbb{R}
%EndExpansion
^{n}$ (e. g. $t=0$) il valore di $u$ e della sua derivata normale a $\Sigma $%
. Perch\'{e} \textquotedblleft derivata normale\textquotedblright ? Se
conosco $u$ su $\U{3a3} $ posso calcolare le derivate tangenziali a $\U{3a3} 
$: la derivata normale \`{e} un'informazione indipendente da questa, e
conoscendo sia la normale sia le tangenziali si pu\`{o} calcolare qualsiasi
derivata direzionale. Il caso particolare che abbiamo visto finora
corrisponde all'ipersuperficie $t=0$, per cui derivata normale significa $%
\frac{\partial }{\partial t}$. Per sua natura, la soluzione di un problema
di Cauchy si cerca, a priori, definita (almeno) in un intorno
dell'ipersuperficie in cui si assegna il dato.

Fatta questa premessa, veniamo ai confronti tra i tipi di operatori. A
prescindere dalle ipotesi di regolarit\`{a} sui coefficienti e sul dominio,
che andranno precisate, si pu\`{o} dire che per gli operatori ellittici si
studiano problemi al contorno (di Dirichlet, Neumann...), mentre non si
studia il problema di Cauchy. \textquotedblleft Non si
studia\textquotedblright\ significa che non \`{e} naturale farlo per i
significati fisici che l'equazione ha, e che dal punto di vista matematico
non \`{e} un problema ben posto per quest'equazione.

Per un operatore iperbolico, invece, tipicamente si studia proprio un
problema di Cauchy (con eventuali condizioni al contorno se il dominio
spaziale \`{e} limitato), con condizioni iniziali assegnate su una
ipersuperficie (che generalizza il caso $t=0$ dell'equazione delle onde).
L'ipersuperficie $\U{3a3} $ su cui si assegnano i dati di Cauchy deve
soddisfare una certa condizione, per\`{o} (dipendente dallo specifico
operatore): $\U{3a3} $ dev'essere una superficie non caratteristica. Senza
dare la definizione generale di superficie caratteristica, spieghiamola su
un esempio particolare: l'equazione della corda vibrante. L'equazione della
corda vibrante $u_{tt}-c%
%TCIMACRO{\U{b2}}%
%BeginExpansion
{{}^2}%
%EndExpansion
u_{xx}=0$ ha due famiglie di rette caratteristiche, $x\pm ct=k$.

Invece di assegnare le condizioni di Cauchy sulla retta $t=0$, potremmo
assegnarle su (quasi) qualsiasi altra retta $ax+bt+c=0$, purch\'{e} non sia
proprio una retta caratteristica. Infatti lungo quelle rette si propaga
l'informazione, perci\`{o} i valori di $u$ e della derivata normale di $u$
su quelle rette non sono tra loro indipendenti, non si possono assegnare
arbitrariamente.

Per operatori parabolici si studia il problema di Cauchy globale o di Cauchy
con condizioni al contorno nel caso di domini spaziali limitati, nello
stesso senso visto per l'equazione del calore: vedendo l'operatore
parabolico definito su un dominio $Q$ dello spazio-tempo, si pu\`{o} dire
che il dato al contorno \`{e} assegnato sulla frontiera parabolica. Ci sono
alcune precisazioni da fare. Anzitutto, \textquotedblleft problema di
Cauchy\textquotedblright , per una EDP del 2%
%TCIMACRO{\U{b0} }%
%BeginExpansion
${{}^\circ}$
%EndExpansion
ordine, dovrebbe significare assegnare, per $t=0$, il valore di $u$ e di $%
u_{t}$; invece, come per l'equazione del calore, per le equazioni
paraboliche si assegna solo $u$ su $\partial _{p}Q$. Questa anomalia\
dipende da un'altra anomalia: un operatore parabolico ha una sola famiglia
di superfici caratteristiche, che sono proprio le $t=k$. Quindi si sta
assegnando la condizione di Cauchy proprio su una superficie caratteristica,
ci\`{o} che per le equazioni iperboliche \`{e} proibito fare; e infatti non 
\`{e} possibile assegnare sia $u$ che $u_{t}$, ma solo $u$. \ Rispetto alle
equazioni iperboliche, quindi, per le paraboliche si assegnano condizioni di
Cauchy \textquotedblleft ridotte\textquotedblright\ (solo sulla $u$ e non su 
$u_{t}$); rispetto alle equazioni ellittiche, per le paraboliche si assegna
la condizione di Dirichlet su una frontiera \textquotedblleft
ridotta\textquotedblright , la frontiera parabolica.

\textbf{b} Gli operatori uniformemente ellittici e uniformemente parabolici
con coefficienti costanti oppure variabili ma $C^{\infty }$ condividono con
l'operatore di Laplace e del calore una propriet\`{a} di regolarizzazione
forte: all'interno del dominio (in spazio o spazio-tempo, rispettivamente),
la soluzione dell'equazione omogenea \`{e} $C^{\infty }$. Se i coefficienti
hanno solo una regolarit\`{a} limitata, ovviamente, anche la regolarit\`{a}
della soluzione ne risentir\`{a}. Gli operatori iperbolici, invece, non sono
regolarizzanti.

Abbiamo detto che gli operatori uniformemente ellittici o parabolici
regolarizzano: il termine \textquotedblleft uniformemente\textquotedblright\
offre lo spunto per fare qualche ulteriore precisazione sulla
classificazione degli operatori descritta fin qui. Consideriamo un operatore
lineare del second'ordine la cui parte principale sia definita oppure
semidefinita (positiva, per fissare le idee): questi operatori si dicono
\textquotedblleft operatori con forma caratteristica non
negativa\textquotedblright . Gli operatori ellittici e parabolici rientrano
in questa categoria, ma non solo loro. Vediamo che fenomeni si possono
presentare per questi operatori.

a. Un operatore pu\`{o} avere forma quadratica $A\left( \mathbf{x}\right) $
che \`{e} definita positiva per quasi ogni $\mathbf{x}$ del dominio, ma in
qualche punto isolato, o lungo certe linee o superfici, la forma quadratica 
\`{e} solo semidefinita positiva. In questo caso si dice che l'operatore 
\`{e} ellittico degenere.

\begin{enumerate}
\item $Lu=u_{xx}+x^{2}u_{yy}$ \`{e} ellittico in $%
%TCIMACRO{\U{211d} }%
%BeginExpansion
\mathbb{R}
%EndExpansion
^{2}$ privato della retta $x=0$. Lungo tale retta l'operatore degenera.

\item $Lu=\func{div}\left( \left( x^{2}+y^{2}\right) \nabla u\right) $ \`{e}
ellittico in $%
%TCIMACRO{\U{211d} }%
%BeginExpansion
\mathbb{R}
%EndExpansion
^{2}\backslash \left\{ \mathbf{0}\right\} $, dove degenera.
\end{enumerate}

Per un operatore ellittico degenere, anche se a coefficienti molto regolari
come in questi due esempi, non \`{e} necessariamente vero che valgano buone
propriet\`{a} di regolarizzazione.

b. Un operatore in $n$ variabili pu\`{o} coinvolgere le derivate seconde di
un numero di variabili $k<n$, con la matrice dei coefficienti delle derivate
seconde definita positiva (su $%
%TCIMACRO{\U{211d} }%
%BeginExpansion
\mathbb{R}
%EndExpansion
^{k}$); se $k=n-1$ si tratta di un operatore parabolico; se per\`{o} $k<n-1$
l'operatore non \`{e} n\'{e} ellittico n\'{e} parabolico, e si dice
ultraparabolico.

\begin{enumerate}
\item $Lu=u_{t}+u_{x}+u_{yy}+u_{zz}$ agisce su 4 variabili ma solo 2
compaiono nelle derivate seconde; sullo spazio $%
%TCIMACRO{\U{211d} }%
%BeginExpansion
\mathbb{R}
%EndExpansion
^{2}$ delle variabili $(y,z)$ la forma quadratica delle derivate seconde 
\`{e} positiva: l'operatore \`{e} ultraparabolico.

\item $Lu=u_{t}+u_{x}+u_{yy}-u_{zz}$ non \`{e} nemmeno ultraparabolico perch%
\'{e} la parte del second'ordine non \`{e} definita positiva nella sua
dimensione: \`{e} iperbolico.
\end{enumerate}

Anche dagli operatori ultraparabolici non ci si possono aspettare a priori
le stesse propriet\`{a} di regolarit\`{a} degli operatori uniformemente
ellittici o parabolici. Lo studio delle propriet\`{a} che rendono
\textquotedblleft buoni\textquotedblright\ dal punto di vista delle propriet%
\`{a} di regolarizzazione anche certi operatori ultraparabolici o ellittici
degeneri (non tutti!) rientra in un filone di ricerca noto come teoria degli
operatori ipoellittici.

\subsection{Concetti diversi di soluzione di un'equazione alle derivate
parziali}

Il concetto di \textquotedblleft soluzione\textquotedblright\ di
un'equazione alle derivate parziali si evolve nella storia e dipende dalla
classe di equazioni considerate. Non ci sono solo nozioni pi\`{u} o meno
generali di soluzione: l'applicabilit\`{a} di un concetto di soluzione
dipende anche dalle ipotesi sui coefficienti. Vediamo alcuni concetti di
soluzione che hanno ciascuno il proprio ambito di interesse e applicabilit%
\`{a} nel contesto delle equazioni alle derivate parziali lineari.

Abbiamo visto che le equazioni ellittiche, paraboliche, iperboliche, perfino
a coefficienti costanti, possono avere propriet\`{a} molto diverse tra loro,
e questo gi\`{a} rimanendo nell'ambito delle sole equazioni lineari del
second'ordine. Ci si pu\`{o} chiedere se si possa stabilire qualche
risultato veramente generale sulle equazioni alle derivate parziali, o per
lo meno affrontarle con un approccio unitario.

\textbf{Teoria delle distribuzioni} Una possibilit\`{a} di affronto unitario
alle EDP lineari \`{e} offerta dalla teoria delle distribuzioni (Schwartz, $%
>1950$), \textit{limitatamente per\`{o} alle equazioni a coefficienti }$%
C^{\infty }$. Infatti si ricorda che se $%
%TCIMACRO{\U{3a9} }%
%BeginExpansion
\Omega
%EndExpansion
$ \`{e} un dominio di $%
%TCIMACRO{\U{211d} }%
%BeginExpansion
\mathbb{R}
%EndExpansion
^{n}$, ogni distribuzione $T\in \mathcal{D}^{\prime }(%
%TCIMACRO{\U{3a9} }%
%BeginExpansion
\Omega
%EndExpansion
)$ \`{e} derivabile infinite volte; se $T\in \mathcal{D}^{\prime }(%
%TCIMACRO{\U{3a9} }%
%BeginExpansion
\Omega
%EndExpansion
)$ e $f\in C^{\infty }\left( \Omega \right) $ allora $fT\in \mathcal{D}%
^{\prime }(%
%TCIMACRO{\U{3a9} }%
%BeginExpansion
\Omega
%EndExpansion
)$ (se $f$ \`{e} meno regolare, il prodotto $fT$ in generale non \`{e}
definito). Perci\`{o} la teoria delle distribuzioni fornisce il quadro
concettuale adatto allo studio delle EDP lineari, di qualunque ordine, a
coefficienti reali o complessi infinitamente derivabili (o, come caso
particolare, coefficienti costanti). Un operatore differenziale lineare in $%
n $ variabili e di ordine $m$ si pu\`{o} scrivere nella forma $%
Lu=\sum_{\left\vert \alpha \right\vert \leq m}c_{\alpha }\left( x\right)
D^{\alpha }u$; se i $c_{\alpha }\in C^{\infty }\left( \Omega \right) $
allora $L$ si pu\`{o} vedere come operatore lineare $L:\mathcal{D}^{\prime }(%
%TCIMACRO{\U{3a9} }%
%BeginExpansion
\Omega
%EndExpansion
)\rightarrow \mathcal{D}^{\prime }(%
%TCIMACRO{\U{3a9} }%
%BeginExpansion
\Omega
%EndExpansion
)$.

Riguardo a questi operatori differenziali ci sono due problemi molto
generali che \`{e} naturale porsi:

\begin{description}
\item[i] La risolubilit\`{a} di un operatore differenziale $L$: si pu\`{o}
affermare che per ogni termine noto $T\in \mathcal{D}^{\prime }(%
%TCIMACRO{\U{3a9} }%
%BeginExpansion
\Omega
%EndExpansion
)$ (soddisfacente eventuali ulteriori ipotesi) l'equazione $Lu=T$ in $%
\mathcal{D}^{\prime }(%
%TCIMACRO{\U{3a9} }%
%BeginExpansion
\Omega
%EndExpansion
)$ ha almeno una soluzione $u\in \mathcal{D}^{\prime }(%
%TCIMACRO{\U{3a9} }%
%BeginExpansion
\Omega
%EndExpansion
)$?

\item[ii] La regolarit\`{a} delle soluzioni: si pu\`{o} affermare che per
ogni soluzione $u\in \mathcal{D}^{\prime }(%
%TCIMACRO{\U{3a9} }%
%BeginExpansion
\Omega
%EndExpansion
)$ dell'equazione $Lu=T$ in $\mathcal{D}^{\prime }(%
%TCIMACRO{\U{3a9} }%
%BeginExpansion
\Omega
%EndExpansion
)$, se $T$ \`{e} regolare (e. g. $C^{\infty }\left( \Omega \right) $), anche 
$u\in C^{\infty }\left( \Omega \right) $?
\end{description}

Questi due problemi sono meglio precisati nelle due definizioni seguenti.

\textbf{Def} Un operatore differenziale lineare $L$ a coefficienti $%
C^{\infty }\left( \Omega \right) $ si dice localmente risolubile se $\forall 
$ $x_{0}\in \Omega $ $\exists $ $U(x_{0})$ intorno tale che $\forall $ $f\in 
\mathcal{D}\left( U\left( x_{0}\right) \right) $ $\exists $ $u\in \mathcal{D}%
^{\prime }(U\left( x_{0}\right) )$ soluzione di $Lu=u_{f}$ in $\mathcal{D}%
^{\prime }(U\left( x_{0}\right) )$.

\textbf{Def} Un operatore differenziale lineare $L$ a coefficienti $%
C^{\infty }\left( \Omega \right) $ si dice ipoellittico in $%
%TCIMACRO{\U{3a9} }%
%BeginExpansion
\Omega
%EndExpansion
$ se per ogni soluzione $u\in \mathcal{D}^{\prime }(%
%TCIMACRO{\U{3a9} }%
%BeginExpansion
\Omega
%EndExpansion
)$ dell'equazione $Lu=T$ in $\mathcal{D}^{\prime }(%
%TCIMACRO{\U{3a9} }%
%BeginExpansion
\Omega
%EndExpansion
)$, se per un certo aperto $A\subset 
%TCIMACRO{\U{3a9} }%
%BeginExpansion
\Omega
%EndExpansion
$ \`{e} $T\in C^{\infty }\left( A\right) $, allora anche $u\in C^{\infty
}\left( A\right) $ (i. e. \`{e} associata a una distribuzione $C^{\infty }$?).

In tal caso quindi ogni soluzione distribuzionale dell'equazione \`{e}
regolare negli aperti in cui il termine noto \`{e} regolare. In particolare,
se $L$ \`{e} ipoellittico, ogni distribuzione che risolve l'equazione
omogenea $Lu=0$ in un aperto $%
%TCIMACRO{\U{3a9} }%
%BeginExpansion
\Omega
%EndExpansion
$ \`{e} in realt\`{a} una funzione $C^{\infty }\left( \Omega \right) $!

Il problema di dare delle condizioni sufficienti (o meglio ancora necessarie
e sufficienti) affinch\'{e} un operatore sia localmente risolubile oppure
ipoellittico \`{e} un problema molto difficile e vasto, nella sua generalit%
\`{a}; risposte soddisfacenti si trovano delimitando pi\`{u} strettamente la
classe di operatori differenziali che si considerano.

Diamo qualche cenno riguardo a questi problemi nel caso particolare degli
operatori a coefficienti costanti.

Consideriamo quindi ora un operatore differenziale lineare in n variabili e
di ordine m, a coefficienti costanti (reali o complessi): $%
Lu=\sum_{\left\vert \alpha \right\vert \leq m}c_{\alpha }D^{\alpha }u$. In
questo contesto, un concetto importante \`{e} quello di soluzione
fondamentale.

\textbf{Def} Se $L$ \`{e} un operatore del tipo $Lu=\sum_{\left\vert \alpha
\right\vert \leq m}c_{\alpha }D^{\alpha }u$, si dice soluzione fondamentale
per $L$ una distribuzione $\Gamma \in \mathcal{D}^{\prime }\left( 
%TCIMACRO{\U{211d} }%
%BeginExpansion
\mathbb{R}
%EndExpansion
^{n}\right) $ che risolve $L\Gamma =\delta _{0}$ in $\mathcal{D}^{\prime
}\left( 
%TCIMACRO{\U{211d} }%
%BeginExpansion
\mathbb{R}
%EndExpansion
^{n}\right) $.

La soluzione fondamentale $\Gamma $, se esiste, non \`{e} in generale
unica, perch\'{e} se $u$ risolve $Lu=0$ allora $L(\Gamma +u)=\delta_0 $ e
quindi anche $\Gamma +u$ \`{e} una soluzione fondamentale.

L'interesse per le soluzioni fondamentali sta nel fatto che se $f\in 
\mathcal{D}\left( 
%TCIMACRO{\U{211d} }%
%BeginExpansion
\mathbb{R}
%EndExpansion
^{n}\right) $ allora $\Gamma \ast f$ \`{e} ben definita come distribuzione,
e soddisfa, per le propriet\`{a} della convoluzione, $L\left( \Gamma \ast
f\right) =L\Gamma \ast f=\delta \ast f=f$. Perci\`{o} la distribuzione $%
u=\Gamma \ast f$ risolve l'equazione $Lu=f$: in particolare, se l'operatore
ammette una soluzione fondamentale, allora \`{e} risolubile.

Perci\`{o} \`{e} importante il seguente risultato (tra i primi successi
della teoria delle distribuzioni):

\textbf{Teo (Malgrange-Ehrenpreis, 1954-1956)} 
\begin{eqnarray*}
\text{Hp} &\text{: }&L\text{ \`{e} un operatore differenziale lineare a
coefficienti costanti} \\
\text{Ts} &\text{: }&L\text{ ha una soluzione fondamentale (e perci\`{o} 
\`{e} risolubile)}
\end{eqnarray*}

Abbiamo determinato in precedenza le soluzioni fondamentali dell'equazione
di Laplace e del calore (in qualsiasi dimensione $n$), e delle onde (in
dimensione $n=1,2,3$).

Riguardo al problema della regolarit\`{a} (cio\`{e} l'ipoellitticit\`{a}
dell'operatore) vale il seguente:

\textbf{Teo} 
\begin{eqnarray*}
\text{Hp}\text{: } &&L\text{ \`{e} un operatore differenziale lineare a
coefficienti costanti} \\
\text{Ts}\text{: } &&L\text{ \`{e} ipoellittico }\Longleftrightarrow \text{
ha una soluzione fondamentale }\Gamma \in C^{\infty }\left( 
%TCIMACRO{\U{211d} }%
%BeginExpansion
\mathbb{R}
%EndExpansion
^{n}\backslash \left\{ \mathbf{0}\right\} \right)
\end{eqnarray*}

Si osservi che le soluzioni fondamentali che abbiamo determinato per gli
operatori di Laplace e del calore sono $C^{\infty }\left( 
%TCIMACRO{\U{211d} }%
%BeginExpansion
\mathbb{R}
%EndExpansion
^{n}\backslash \left\{ \mathbf{0}\right\} \right) $, perci\`{o} questi
operatori sono ipoellittici. Non \`{e} ipoellittico invece l'operatore delle
onde, per il quale ad esempio $\Gamma \left( \mathbf{x},t\right) =\left\{ 
\begin{array}{c}
\frac{1}{2c}u\left( ct-\left\vert x\right\vert \right) \text{ se }n=1 \\ 
\frac{1}{2\pi c\sqrt{c^{2}t^{2}-\left\vert \left\vert \mathbf{x}\right\vert
\right\vert ^{2}}}u\left( ct-\left\vert \left\vert \mathbf{x}\right\vert
\right\vert \right) \text{ se }n=2%
\end{array}%
\right. $ con $u$ gradino di Heaviside.

Dunque la teoria delle distribuzioni permette di considerare soluzioni
distribuzionali, quindi anche molto poco regolari, addirittura non
rappresentate da funzioni, purch\'{e} per\`{o} l'equazione abbia
coefficienti estremamente regolari. Perci\`{o} anche questa teoria non \`{e}
sufficiente.

\textbf{EDP ellittiche del second'ordine} D'ora in poi si considerano solo
equazioni lineari del second'ordine uniformemente ellittiche, a coefficienti
variabili ma non necessariamente $C^{\infty }$: diamo un'idea dei vari
concetti di soluzione e le corrispondenti teorie che sono state sviluppate.

Si dice che $u$ \`{e} una soluzione classica di un'equazione differenziale $%
Lu=f$ se $u$ possiede in ogni punto del dominio tutte le derivate che sono
coinvolte nell'equazione, e l'uguaglianza $Lu=f$ vale in tutti i punti del
dominio ed \`{e} un'uguaglianza tra funzioni continue.

Se l'equazione \`{e} del second'ordine, questo significa richiedere che $u$
sia almeno $C^{2}(%
%TCIMACRO{\U{3a9} }%
%BeginExpansion
\Omega
%EndExpansion
)$ e coefficienti e termine noto siano almeno continui. In realt\`{a}, si
scopre che richiedere la sola continuit\`{a} di termine noto e coefficienti 
\`{e} troppo poco per avere soluzioni classiche: esistono esempi di funzioni 
$f$ continue in $%
%TCIMACRO{\U{211d} }%
%BeginExpansion
\mathbb{R}
%EndExpansion
^{n}$ per cui l'equazione $\U{394} u=f$ non ha alcuna soluzione $C^{2}(%
%TCIMACRO{\U{211d} }%
%BeginExpansion
\mathbb{R}
%EndExpansion
^{n})$.

Per avere una buona teoria classica bisogna richiedere qualcosa di pi\`{u}
della continuit\`{a}, o meglio una forma quantitativa di continuit\`{a}. Il
concetto chiave \`{e} contenuto nella prossima:

\textbf{Def} Si dice che $f:\Omega \rightarrow 
%TCIMACRO{\U{211d} }%
%BeginExpansion
\mathbb{R}
%EndExpansion
$ \`{e} hoelderiana di esponente $\alpha \in \left( 0,1\right) $ se $\exists 
$ $c>0:\left\vert f\left( x_{1}\right) -f\left( x_{2}\right) \right\vert
\leq c\left\vert x_{1}-x_{2}\right\vert ^{\alpha }$ $\forall $ $%
x_{1},x_{2}\in \Omega $.

Si dimostra che se $f$ \`{e} hoelderiana allora $f$ \`{e} uniformemente
continua in $%
%TCIMACRO{\U{3a9} }%
%BeginExpansion
\Omega
%EndExpansion
$. L'h\"{o}lderianit\`{a} \`{e} una condizione pi\`{u} debole rispetto alla
lipschitzianit\`{a}, che \`{e} l'analogo con $\U{3b1} =1$, e implica la
derivabilit\`{a} quasi ovunque; le funzioni h\"{o}lderiane, invece, non sono
generalmente derivabili.

Se $%
%TCIMACRO{\U{3a9} }%
%BeginExpansion
\Omega
%EndExpansion
$ \`{e} un dominio limitato, allora $f$ \`{e} anche limitata e si pu\`{o}
definire la norma $\left\vert \left\vert f\right\vert \right\vert
_{C^{\alpha }\left( \bar{\Omega}\right) }=\max_{\bar{\Omega}}\left\vert
f\right\vert +\sup_{x_{1},x_{2}\in \Omega ,x_{1}\neq x_{2}}\frac{\left\vert
f\left( x_{1}\right) -f\left( x_{2}\right) \right\vert }{\left\vert
x_{1}-x_{2}\right\vert ^{\alpha }}$, che rende l'insieme delle funzioni
hoelderiane uno spazio di Banach, indicato con $C^{\alpha }\left( \Omega
\right) $. Si dice che $f\in C^{k,\alpha }\left( \Omega \right) $ se $f\in
C^{k}(%
%TCIMACRO{\U{3a9} }%
%BeginExpansion
\Omega
%EndExpansion
)$ e tutte le derivate di ordine $k$ di $f$ appartengono a $C^{\alpha }(%
%TCIMACRO{\U{3a9} }%
%BeginExpansion
\Omega
%EndExpansion
)$.

La teoria di Schauder, sviluppata negli anni 1930, \`{e} la teoria classica
standard che si applica agli operatori lineari del second'ordine
uniformemente ellittici, ed \`{e} ambientata negli spazi di tipo H\"{o}lder.
Il risultato base \`{e} il seguente:

\textbf{Teo}%
\begin{gather*}
\text{Hp: }\Omega \subseteq 
%TCIMACRO{\U{211d} }%
%BeginExpansion
\mathbb{R}
%EndExpansion
^{n}\text{ \`{e} un dominio limitato con frontiera sufficientemente\footnote{%
non specifichiamo ulteriormente quest'ipotesi} regolare,} \\
Lu=\sum_{i,j=1}^{n}a_{ij}\left( \mathbf{x}\right)
u_{x_{i}x_{j}}+\sum_{k=1}^{n}b_{k}\left( \mathbf{x}\right) u_{x_{k}}+c\left( 
\mathbf{x}\right) u\text{, }a_{ij},x_{k},c\in C^{\alpha }\left( \bar{\Omega}%
\right) \text{; }L\text{ \`{e} uniformemente } \\
\text{ellittico in }\Omega \text{ e }c\left( x\right) \leq 0\text{ in }%
\Omega \text{; }\left\{ 
\begin{array}{c}
Lu=f\text{ in }\Omega \\ 
u=g\text{ su }\partial \Omega%
\end{array}%
\right. \text{ ha }f\in C^{\alpha }\left( \bar{\Omega}\right) ,g\in
C^{2,\alpha }\left( \bar{\Omega}\right) \\
\text{Ts: esiste un'unica soluzione }u\in C^{2,\alpha }\left( \Omega \right) 
\text{ del problema di Dirichlet e vale} \\
\left\vert \left\vert u\right\vert \right\vert _{C^{2,\alpha }\left( \bar{%
\Omega}\right) }\leq c\left( \left\vert \left\vert f\right\vert \right\vert
_{C^{\alpha }\left( \bar{\Omega}\right) }+\left\vert \left\vert g\right\vert
\right\vert _{C^{2,\alpha }\left( \bar{\Omega}\right) }\right)
\end{gather*}

La teoria di Schauder \`{e} soddisfacente nel suo ambito, che richiede
coefficienti (pi\`{u} che) continui.

Se pensiamo a un'equazione ellittica in forma di divergenza come caso
stazionario di un'equazione di diffusione, per semplicit\`{a} con la sola
parte principale, $Lu=\sum_{i=1}^{n}\left( a\left( \mathbf{x}\right)
u_{x_{i}}\right) _{x_{i}}$, allora $a(\mathbf{x})$ descrive la conducibilit%
\`{a} termica del materiale, e pu\`{o} essere discontinuo. In questo caso,
per quanto regolare sia $u$, in generale $a(\mathbf{x})u_{x_{i}}$ non sar%
\`{a} derivabile, e il significato stesso dell'operatore non \`{e} chiaro.
D'altro canto, se riandiamo al procedimento con cui si \`{e} dedotta
l'equazione differenziale, vediamo che la quantit\`{a} dotata di un
significato fisico diretto \`{e} il flusso di corrente termica $a(\mathbf{x}%
)\nabla u(\mathbf{x})$ attraverso un'opportuna superficie.

L'idea quindi \`{e} indebolire la definizione di soluzione, accontentandosi
di richiedere che $a(\mathbf{x})u_{x_{i}}$ sia effettivamente una funzione
per $i=1,2,...,n$, e la derivata $\left( a\left( \mathbf{x}\right)
u_{x_{i}}\right) _{x_{i}}$ esista solo in senso distribuzionale.

Come al solito, si suppone in un primo tempo che il coefficiente $a$ e la
soluzione $u$ siano regolari quanto occorre perch\'{e} $Lu$ sia ben definito
in senso classico e, volendo interpretare l'equazione $Lu=f$ in $%
%TCIMACRO{\U{3a9} }%
%BeginExpansion
\Omega
%EndExpansion
$, moltiplichiamo ambo i membri per $\phi \in \mathcal{D}(%
%TCIMACRO{\U{3a9} }%
%BeginExpansion
\Omega
%EndExpansion
)$ e integriamo su $%
%TCIMACRO{\U{3a9} }%
%BeginExpansion
\Omega
%EndExpansion
$: si ha $\int_{\Omega }\sum_{i=1}^{n}\left( a\left( \mathbf{x}\right)
u_{x_{i}}\right) _{x_{i}}\phi d\mathbf{x}=\int_{\Omega }f\phi d\mathbf{x}$.
Integrando per parti il lato sinistro si ottiene $-\int_{\Omega
}\sum_{i=1}^{n}a\left( \mathbf{x}\right) u_{x_{i}}\phi _{x_{i}}d\mathbf{x}%
=\int_{\Omega }f\phi d\mathbf{x}$: tale uguaglianza ha senso sotto la sola
ipotesi che (ad esempio) $u\in C^{1}(%
%TCIMACRO{\U{3a9} }%
%BeginExpansion
\Omega
%EndExpansion
)$. In questo caso infatti $v_{i}:=a\left( \mathbf{x}\right) u_{x_{i}}\in
L^{\infty }(%
%TCIMACRO{\U{3a9} }%
%BeginExpansion
\Omega
%EndExpansion
)$, in particolare \`{e} una funzione localmente integrabile e quindi ha una
distribuzione associata, e l'uguaglianza precedente esprime il fatto che $%
\sum_{i=1}^{n}D_{x_{i}}u_{v_{i}}=u_{f}$. Potremmo quindi dare la seguente
definizione di soluzione debole dell'equazione $Lu=f$:

\textbf{Def (soluzione debole, primo tentativo)} Si dice che $u$ \`{e}
soluzione debole di $Lu=f$, con $a\in L^{\infty }(%
%TCIMACRO{\U{3a9} }%
%BeginExpansion
\Omega
%EndExpansion
)$ e $f\in L_{loc}^{1}(%
%TCIMACRO{\U{3a9} }%
%BeginExpansion
\Omega
%EndExpansion
)$, se $u\in C^{1}(%
%TCIMACRO{\U{3a9} }%
%BeginExpansion
\Omega
%EndExpansion
)$ e vale $-\int_{\Omega }\sum_{i=1}^{n}a\left( \mathbf{x}\right)
u_{x_{i}}\phi _{x_{i}}d\mathbf{x}=\int_{\Omega }f\phi d\mathbf{x}$ $\forall $
$\phi \in \mathcal{D}(%
%TCIMACRO{\U{3a9} }%
%BeginExpansion
\Omega
%EndExpansion
)$.

Questa definizione ha il pregio di dar senso all'equazione anche nel caso in
cui $a$ \`{e} discontinuo (purch\'{e} limitato). Il difetto della
definizione \`{e} il quadro funzionale: lo spazio $C^{1}(%
%TCIMACRO{\U{3a9} }%
%BeginExpansion
\Omega
%EndExpansion
)$ si accorda male con la definizione di soluzione data mediante un'identit%
\`{a} integrale. Infatti gli sviluppi dell'analisi del '900 hanno mostrato
che difficilmente si riesce a dimostrare teoremi di esistenza se non si
ambienta la ricerca della soluzione $u$ in un opportuno spazio di funzioni,
vettoriale, normato e completo. Se un'uguaglianza integrale entra in modo
determinante nella definizione di soluzione, lo spazio funzionale in cui
cercare la soluzione dovr\`{a} essere uno spazio vettoriale munito di una
norma di tipo integrale, ma lo spazio $C^{1}(%
%TCIMACRO{\U{3a9} }%
%BeginExpansion
\Omega
%EndExpansion
)$ munito di una norma integrale non \`{e} completo.

Occorre allargare questo spazio $C^{1}(%
%TCIMACRO{\U{3a9} }%
%BeginExpansion
\Omega
%EndExpansion
)$ a uno spazio pi\`{u} stabile rispetto alle approssimazioni in norma
integrale; ma questo significa che anche la definizione di derivata di $u$
va indebolita, nella direzione di una definizione di tipo integrale. Questo
porta a introdurre la definizione di derivata debole e di spazi di funzioni
derivabili in senso debole, gli spazi di Sobolev, che come vedremo sono
spazi di Hilbert di funzioni derivabili in senso debole.

\section{Spazi di Sobolev (1935)}

\textbf{Def} Dato $I\subseteq 
%TCIMACRO{\U{211d} }%
%BeginExpansion
\mathbb{R}
%EndExpansion
$ intervallo eventualmente illimitato e $f\in L_{loc}^{1}\left( I\right) $,
si dice che $f$ \`{e} derivabile in senso debole in $I$ se esiste $g\in
L_{loc}^{1}\left( I\right) :-\int_{I}f\left( x\right) \phi ^{\prime }\left(
x\right) dx=\int_{I}g\left( x\right) \phi \left( x\right) dx$ $\forall $ $%
\phi \in C_{0}^{1}\left( I\right) $. In tal caso $g$ si dice derivata debole
di $f$ in $I$.

Se vale l'uguaglianza nella definizione $\forall $ $\phi \in C_{0}^{1}\left(
I\right) $, allora vale anche $\forall $ $\phi \in C_{0}^{\infty }\left(
I\right) $. La sostanza della definizione quando $I=%
%TCIMACRO{\U{211d} }%
%BeginExpansion
\mathbb{R}
%EndExpansion
$ \`{e} la seguente: si considera la distribuzione $u_{f}$ e se ne calcola
la derivata distribuzionale $Du_{f}$; se $Du_{f}$ \`{e} rappresentabile come
distribuzione associata a una funzione $g$, quest'ultima funzione si dice
derivata debole di $f$.

Si noti che la derivabilit\`{a} in senso debole \`{e} un concetto globale,
perch\'{e} l'uguaglianza richiesta \`{e} globale: o $f$ \`{e} derivabile in
senso debole in $I$, oppure non lo \`{e}; non esiste la derivabilit\`{a}
debole in un punto.

Se \`{e} $f$ derivabile in senso debole in $I$, lo \`{e} anche in ogni
intervallo $I^{\prime }$ contenuto in $I$ (basta considerare solo le $\phi $
con supporto in $I^{\prime }\subseteq I$).

Se $f\in C^{1}\left( 
%TCIMACRO{\U{211d} }%
%BeginExpansion
\mathbb{R}
%EndExpansion
\right) $, integrando per parti si mostra che $f$ \`{e} derivabile in senso
debole con derivata debole $g=f^{\prime }$.

\begin{enumerate}
\item $f\left( x\right) =\left\vert x\right\vert $ non \`{e} derivabile in
senso classico in $%
%TCIMACRO{\U{211d} }%
%BeginExpansion
\mathbb{R}
%EndExpansion
$; lo \`{e} in senso debole? $-Du_{f}\left( \phi \right) =\int_{%
%TCIMACRO{\U{211d} }%
%BeginExpansion
\mathbb{R}
%EndExpansion
}f\left( x\right) \phi ^{\prime }\left( x\right) =-\int_{-\infty }^{0}x\phi
^{\prime }\left( x\right) dx+\int_{0}^{+\infty }x\phi ^{\prime }\left(
x\right) dx=$ $-\left( \left[ x\phi \left( x\right) \right] _{-\infty
}^{0}-\int_{-\infty }^{0}\phi \left( x\right) dx\right) +\left[ x\phi \left(
x\right) \right] _{0}^{+\infty }-\int_{0}^{+\infty }\phi \left( x\right)
dx=-\int_{%
%TCIMACRO{\U{211d} }%
%BeginExpansion
\mathbb{R}
%EndExpansion
}g\left( x\right) \phi \left( x\right) dx$, con $g\left( x\right) =\left\{ 
\begin{array}{c}
1\text{ se }x>0 \\ 
-1\text{ se }x<0%
\end{array}%
\right. \in L_{loc}^{1}\left( 
%TCIMACRO{\U{211d} }%
%BeginExpansion
\mathbb{R}
%EndExpansion
\right) $ (non occorre definirla in $0$: comunque \`{e} definita quasi
ovunque). Dunque $f$ \`{e} derivabile in senso debole in $%
%TCIMACRO{\U{211d} }%
%BeginExpansion
\mathbb{R}
%EndExpansion
$ con derivata debole $g$, che \`{e} la funzione segno (in effetti questo
poteva essere dedotto immediatamente sapendo che la derivata distribuzionale
di $u_{\left\vert x\right\vert }$ \`{e} associata alla funzione segno).
Questo esempio potrebbe far pensare che la derivata debole, quando esiste,
coincida con la derivata classica. Il seguente esempio confuta (?) questa
ipotesi.

\item La funzione gradino $H\left( x\right) =\left\{ 
\begin{array}{c}
1\text{ se }x>0 \\ 
0\text{ se }x<0%
\end{array}%
\right. $ \`{e} derivabile in senso debole in $%
%TCIMACRO{\U{211d} }%
%BeginExpansion
\mathbb{R}
%EndExpansion
$? $-Du_{H}\left( \phi \right) =\int_{%
%TCIMACRO{\U{211d} }%
%BeginExpansion
\mathbb{R}
%EndExpansion
}H\left( x\right) \phi ^{\prime }\left( x\right) dx=\int_{0}^{+\infty }\phi
^{\prime }\left( x\right) dx=-\phi \left( 0\right) $, che non pu\`{o} essere
uguale a $-\int_{%
%TCIMACRO{\U{211d} }%
%BeginExpansion
\mathbb{R}
%EndExpansion
}g\left( x\right) \phi \left( x\right) dx$ per alcuna $g\in
L_{loc}^{1}\left( 
%TCIMACRO{\U{211d} }%
%BeginExpansion
\mathbb{R}
%EndExpansion
\right) $: il valore dell'integrale non pu\`{o} essere univocamente
determinato dal valore di $g$ in un solo punto, dato che questa \`{e}
definita in un solo punto. Dunque $H$ non \`{e} derivabile in senso debole
in $%
%TCIMACRO{\U{211d} }%
%BeginExpansion
\mathbb{R}
%EndExpansion
$ (mentre lo \`{e} in qualsiasi intervallo contenuto in $\left( -\infty
,0\right) $, e ha derivata debole $0$). In effetti questo poteva essere
dedotto immediatamente sapendo che $Du_{H}$ \`{e} la delta. Il risultato pu%
\`{o} essere esteso a qualsiasi funzione con discontinuit\`{a} a salto.
\end{enumerate}

Si estende la definizione a $%
%TCIMACRO{\U{211d} }%
%BeginExpansion
\mathbb{R}
%EndExpansion
^{n}$.

\textbf{Def} Dato $\Omega \subseteq 
%TCIMACRO{\U{211d} }%
%BeginExpansion
\mathbb{R}
%EndExpansion
^{n}$ dominio eventualmente illimitato e $f\in L_{loc}^{1}\left( \Omega
\right) $, si dice che $f$ \`{e} derivabile in senso debole in $\Omega $
rispetto alla variabile $x_{j}$ se esiste $g\in L_{loc}^{1}\left( \Omega
\right) :-\int_{\Omega }f\left( \mathbf{x}\right) \frac{\partial \phi \left( 
\mathbf{x}\right) }{\partial x_{j}}d\mathbf{x}=\int_{\Omega }g\left( \mathbf{%
x}\right) \phi \left( \mathbf{x}\right) d\mathbf{x}$ $\forall $ $\phi \in
C_{0}^{1}\left( \Omega \right) $. In tal caso $g$ si dice derivata debole di 
$f$ rispetto a $x_{j}$ in $\Omega $. Se $f$ \`{e} derivabile in senso debole
rispetto a $x_{j}$ $\forall $ $j$, si dice che $f$ \`{e} derivabile in senso
debole in $\Omega $.

Se $f$ \`{e} derivabile in senso debole rispetto a $x_{j}$, la sua derivata
debole \`{e} unica in $L_{loc}^{1}\left( \Omega \right) $: se esistono $%
g_{1},g_{2}\in L_{loc}^{1}\left( \Omega \right) :-\int_{\Omega }f\left( 
\mathbf{x}\right) \frac{\partial \phi \left( \mathbf{x}\right) }{\partial
x_{j}}d\mathbf{x}=\int_{\Omega }g_{1}\left( \mathbf{x}\right) \phi \left( 
\mathbf{x}\right) d\mathbf{x=}\int_{\Omega }g_{2}\left( \mathbf{x}\right)
\phi \left( \mathbf{x}\right) d\mathbf{x}$ $\forall $ $\phi \in
C_{0}^{1}\left( \Omega \right) $, allora si ha in particolare $\int_{\Omega
}\left( g_{1}-g_{2}\right) \phi d\mathbf{x}=0$ $\forall $ $\phi \in
C_{0}^{1}\left( \Omega \right) $, e per il lemma di annullamento $%
g_{1}=g_{2} $ q. o.\footnote{%
alla fine, \`{e} la stessa cosa che dire che se una distribuzione \`{e}
associata a una funzione, allora tale funzione \`{e} unica}

Per le derivate deboli valgono vari teoremi ben noti per le derivate
classiche.

\textbf{Teo (derivata nulla implica costante)}%
\begin{gather*}
\text{Hp: }\Omega \subseteq 
%TCIMACRO{\U{211d} }%
%BeginExpansion
\mathbb{R}
%EndExpansion
^{n}\text{ \`{e} aperto e connesso, }f\in L_{loc}^{1}\left( \Omega \right) 
\text{ \`{e} derivabile in } \\
\text{senso debole in }\Omega \text{, }\frac{\partial f}{\partial x_{j}}=0%
\text{ q. o. in }\Omega \text{ }\forall \text{ }j=1,...,n \\
\text{Ts: }f\text{ \`{e} costante q. o. in }\Omega
\end{gather*}

Si noti l'usuale ipotesi di connessione.


\textbf{Def} Dato $\Omega \subseteq 
%TCIMACRO{\U{211d} }%
%BeginExpansion
\mathbb{R}
%EndExpansion
^{n}$ aperto e connesso e $p\in \left[ 1,+\infty \right] $, si dice che $%
f\in W^{1,p}\left( \Omega \right) $ se $f\in L^{p}\left( \Omega \right) $,
esistono le sue derivate deboli $\frac{\partial f}{\partial x_{1}},...,\frac{%
\partial f}{\partial x_{n}}$ e tutte appartengono a $L^{p}\left( \Omega
\right) $.

Allora \`{e} ben definita in $W^{1,p}\left( \Omega \right) $, se $p<+\infty $%
, la norma%
\begin{equation*}
\left\vert \left\vert u\right\vert \right\vert _{W^{1,p}\left( \Omega
\right) }:=\left( \left\vert \left\vert u\right\vert \right\vert
_{L^{p}\left( \Omega \right) }^{p}+\left\vert \left\vert \nabla u\right\vert
\right\vert _{L^{p}\left( \Omega \right) }^{p}\right) ^{\frac{1}{p}}
\end{equation*}

Invece $\left\vert \left\vert u\right\vert \right\vert _{W^{1,\infty }\left(
\Omega \right) }:=\left\vert \left\vert u\right\vert \right\vert _{L^{\infty
}\left( \Omega \right) }+\left\vert \left\vert \nabla u\right\vert
\right\vert _{L^{\infty }\left( \Omega \right) }$ (si pu\`{o} mostrare che
sono effettivamente norme). Il significato centrale di queste norme \`{e}
che tengono conto non solo dei valori di $u$, ma anche delle sue derivate, a
differenza dell'usuale norma $L^{p}$: una successione di funzioni che
converga in norma $W^{1,p}$ a $u$ deve avere non solo "integrale simile", ma
anche "integrale del gradiente simile".

Si definisce $H^{1}\left( \Omega \right) :=W^{1,2}\left( \Omega \right) $,
in cui \`{e} naturale definire un prodotto scalare: poich\'{e} $\left\langle
u,u\right\rangle _{H^{1}\left( \Omega \right) }=\left\vert \left\vert
u\right\vert \right\vert _{L^{2}\left( \Omega \right) }^{2}+\left\vert
\left\vert \nabla u\right\vert \right\vert _{L^{2}\left( \Omega \right)
}^{2} $, si pone 
\begin{equation*}
\left\langle f,g\right\rangle _{H^{1}\left( \Omega \right) }:=\int_{\Omega
}fgd\mathbf{x}+\int_{\Omega }\left\langle \nabla f,\nabla g\right\rangle d%
\mathbf{x}
\end{equation*}

Effettivamente \`{e} bilineare e commutativo, $\left\langle u,u\right\rangle
_{H^{1}\left( \Omega \right) }=\int_{\Omega }u^{2}d\mathbf{x+}\int_{\Omega
}\sum_{i=1}^{n}u_{x_{i}}^{2}d\mathbf{x}\geq 0$ e $\left\langle
u,u\right\rangle _{H^{1}\left( \Omega \right) }=0\Longleftrightarrow u=0$
q.o. Peraltro $\left\vert \left\vert \nabla u\right\vert \right\vert
_{L^{2}\left( \Omega \right) }^{2}$ ha il significato fisico di energia: un
segnale $u$ tale che $\left\vert \left\vert \nabla u\right\vert \right\vert
_{L^{2}\left( \Omega \right) }^{2}<+\infty $ si dice a energia finita (da
approfondire).

\textbf{Teo (completezza degli spazi di Sobolev)}%
\begin{eqnarray*}
\text{Hp}\text{: } &&\Omega \subseteq 
%TCIMACRO{\U{211d} }%
%BeginExpansion
\mathbb{R}
%EndExpansion
^{n}\text{ \`{e} un dominio, }p\in \left[ 1,+\infty \right] \\
\text{Ts}\text{: } &&W^{1,p}\left( \Omega \right) \text{ \`{e} uno spazio di
Banach, }H^{1}\left( \Omega \right) \text{ \`{e} uno spazio di Hilbert }
\end{eqnarray*}

$H^{1}\left( \Omega \right) $ \`{e} uno spazio di Hilbert sostanzialmente
perch\'{e} $L^{2}\left( \Omega \right) $ lo \`{e}.

\textbf{Dim} Ovviamente i $W^{1,p}\left( \Omega \right) $ sono spazi
vettoriali normati e $H^{1}\left( \Omega \right) $ \`{e} uno spazio
prehilbertiano.

Dimostriamo solo che $H^{1}\left( \Omega \right) $ \`{e} di Hilbert, quindi
che \`{e} completo rispetto alla norma indotta dal prodotto scalare, cio\`{e}
che se $\left\{ f_{k}\right\} _{k=1}^{+\infty }$ \`{e} una successione di
Cauchy in $H^{1}\left( \Omega \right) $ allora $\left\{ f_{k}\right\} $ \`{e}
convergente in $H^{1}\left( \Omega \right) $. Suppongo quindi che $%
\left\vert \left\vert f_{k}-f_{h}\right\vert \right\vert _{H^{1}\left(
\Omega \right) }^{2}=\left\vert \left\vert f_{k}-f_{h}\right\vert
\right\vert _{L^{2}\left( \Omega \right) }^{2}+\left\vert \left\vert \nabla
\left( f_{k}-f_{h}\right) \right\vert \right\vert _{L^{2}\left( \Omega
\right) }^{2}\rightarrow ^{k,h\rightarrow +\infty }0$: allora anche $%
\left\vert \left\vert f_{k}-f_{h}\right\vert \right\vert _{L^{2}\left(
\Omega \right) },\left\vert \left\vert \nabla \left( f_{k}-f_{h}\right)
\right\vert \right\vert _{L^{2}\left( \Omega \right) }\rightarrow
^{k,h\rightarrow +\infty }0$ e in particolare $\left\vert \left\vert \frac{%
\partial f_{k}}{\partial x_{j}}-\frac{\partial f_{h}}{\partial x_{j}}%
\right\vert \right\vert _{L^{2}\left( \Omega \right) }\rightarrow
^{k,h\rightarrow +\infty }0$. Ma allora $\left\{ f_{k}\right\} _{k},\left\{ 
\frac{\partial f_{k}}{\partial x_{j}}\right\} _{k}$ $\forall $ $j=1,...,n$
sono successioni di Cauchy in $L^{2}\left( \Omega \right) $, il quale \`{e}
completo: quindi $\exists $ $g,g_{1},...,g_{n}\in L^{2}\left( \Omega \right)
:f_{k}\rightarrow ^{L^{2}\left( \Omega \right) }g,\frac{\partial f_{k}}{%
\partial x_{j}}\rightarrow ^{L^{2}\left( \Omega \right) }g_{j}$. Vorrei dire
che $f_{k}$ converge in norma $H^{1}$ a $g$, quindi mostro che le $g_{j}$
sono le derivate deboli di $g$. Poich\'{e} $f_{k}\in H^{1}\left( \Omega
\right) $, le $f_{k}$ sono derivabili in senso debole, cio\`{e} $%
\int_{\Omega }f_{k}\frac{\partial \phi }{\partial x_{j}}d\mathbf{x}%
=-\int_{\Omega }\frac{\partial f_{k}}{\partial x_{j}}\phi d\mathbf{x}$ $%
\forall $ $\phi \in C_{0}^{1}\left( \Omega \right) $. Ma $f_{k}\rightarrow
^{L^{2}\left( \Omega \right) }g,\frac{\partial f_{k}}{\partial x_{j}}%
\rightarrow ^{L^{2}\left( \Omega \right) }g_{j}$, quindi per continuit\`{a}
del prodotto scalare in $L^{2}\left( \Omega \right) $ il lato sinistro per $%
k\rightarrow +\infty $ tende a $\int_{\Omega }g\frac{\partial \phi }{%
\partial x_{j}}d\mathbf{x}$, mentre il lato destro tende a $-\int_{\Omega
}g_{j}\phi d\mathbf{x}$. Allora vale $\int_{\Omega }g\frac{\partial \phi }{%
\partial x_{j}}d\mathbf{x}=-\int_{\Omega }g_{j}\phi d\mathbf{x}$ $\forall $ $%
\phi \in C_{0}^{1}\left( \Omega \right) $, cio\`{e} $g\in H^{1}\left( \Omega
\right) $ con derivate deboli $g_{j}$. Allora $f_{k}\rightarrow
^{L^{2}\left( \Omega \right) }g,\frac{\partial f_{k}}{\partial x_{j}}%
\rightarrow ^{L^{2}\left( \Omega \right) }g_{j}$ e dunque $\left\vert
\left\vert f_{k}-g\right\vert \right\vert _{H^{1}\left( \Omega \right) }=%
\sqrt{\left\vert \left\vert f_{k}-g\right\vert \right\vert _{L^{2}\left(
\Omega \right) }^{2}+\left\vert \left\vert \nabla \left( f_{k}-g\right)
\right\vert \right\vert _{L^{2}\left( \Omega \right) }^{2}}\rightarrow
^{k\rightarrow +\infty }0$, cio\`{e} $f_{k}\rightarrow ^{H^{1}\left( \Omega
\right) }g$. $\blacksquare $

L'idea \`{e} che in $H^{1}\left( \Omega \right) $ cercheremo le soluzioni
delle EDP, quindi interessa sapere quanto le funzioni di $H^{1}\left( \Omega
\right) $ possono essere regolari o irregolari. Dagli esempi visti, abbiamo $%
\left\vert x\right\vert \in H^{1}\left( -1,1\right) ,H\left( x\right)
\not\in H^{1}\left( -1,1\right) $.

\textbf{Teo (caratterizzazione di }$H^{1}\left( a,b\right) $\textbf{\ per }$%
n=1$\textbf{)}%
\begin{gather*}
f\in H^{1}\left( a,b\right) \Longleftrightarrow \text{(i) }f\in C^{0}\left( %
\left[ a,b\right] \right) \\
\text{(ii) }\exists \text{ }f^{\prime }\text{ derivata classica q. o. in }%
\left( a,b\right) \text{ e }f^{\prime }\in L^{2}\left( a,b\right) \\
\text{(iii) }\forall \text{ }x_{1},x_{2}\in \left[ a,b\right] \text{ vale }%
f\left( x_{2}\right) -f\left( x_{1}\right) =\int_{x_{1}}^{x_{2}}f^{\prime
}\left( t\right) dt
\end{gather*}

(i) potrebbe apparire nebulosa: $f\in H^{1}\left( a,b\right) $ \`{e} in realt%
\`{a} una classe di equivalenza di funzioni. (i) sarebbe quindi, pi\`{u}
precisamente, "esiste almeno una funzione continua nella classe di
equivalenza di $f$", e a tale funzione continua si riferisce (iii) (? da
accertare). La verit\`{a} di (iii) \`{e} invece legata all'assoluta continuit%
\`{a} di $f$ (concetto che non trattiamo).

Il teorema \`{e} coerente con gli esempi visti: per $n=1$, le funzioni con
discontinuit\`{a} a salto non soddisfano (i).

\begin{enumerate}
\item Per quali $\alpha $ $x^{\alpha }\in H^{1}\left( 0,1\right) $? $%
x^{\alpha }\in C^{0}\left( \left[ 0,1\right] \right) \Longleftrightarrow
\alpha \geq 0$, e in tal caso \`{e} anche in $L^{2}\left( 0,1\right) $. La
derivata classica $\alpha x^{\alpha -1}$ \`{e} definita q. o. in $\left(
0,1\right) $ $\forall $ $\alpha $; $\alpha x^{\alpha -1}\in L^{2}\left(
0,1\right) \Longleftrightarrow \alpha \int_{0}^{1}x^{2\alpha -2}dx$, cio\`{e}
$\alpha >\frac{1}{2}$. Quindi se $\alpha >\frac{1}{2}$ $x^{\alpha }\in
H^{1}\left( 0,1\right) $. Si noti che se $\alpha \in \left( \frac{1}{2}%
,1\right) $ $x^{\alpha }\in H^{1}\left( 0,1\right) $, ma non \`{e} in $%
C^{1}\left( \left[ 0,1\right] \right) $: $H^{1}\left( 0,1\right) $ \`{e} uno
spazio pi\`{u} grande.
\end{enumerate}

Quindi se $n=1$ in realt\`{a} le funzioni di $H^{1}$ non possono essere
troppo irregolari: devono essere comunque almeno continue. Se invece $n>1$, 
\`{e} possibile che ci siano funzioni $H^{1}\left( \Omega \right) $ ma
discontinue?

\begin{enumerate}
\item Per quali $\alpha $ $f\left( \mathbf{x}\right) =\frac{1}{\left\vert
\left\vert \mathbf{x}\right\vert \right\vert ^{\alpha }}\in H^{1}\left(
B_{1}\left( \mathbf{0}\right) \right) $? $f\in L^{2}\left( B_{1}\left( 
\mathbf{0}\right) \right) \Longleftrightarrow \int_{B_{1}\left( \mathbf{0}%
\right) }\frac{1}{\left\vert \left\vert \mathbf{x}\right\vert \right\vert
^{2\alpha }}d\mathbf{x}=c\int_{0}^{1}\frac{1}{\rho ^{2\alpha }}\rho
^{n-1}d\rho <+\infty $, cio\`{e} $\alpha <\frac{n}{2}$. Invece $\left\vert 
\frac{\partial f}{\partial x_{j}}\right\vert =\left\vert \frac{-\alpha x_{j}%
}{\left\vert \left\vert \mathbf{x}\right\vert \right\vert ^{\alpha +1}}%
\right\vert =\frac{c}{\left\vert \left\vert \mathbf{x}\right\vert
\right\vert ^{\alpha +1}}$, che \`{e} in $%
L^{2}$ se e solo se $\int_{B_{1}\left( \mathbf{0}\right) }\frac{c}{%
\left\vert \left\vert \mathbf{x}\right\vert \right\vert ^{2\alpha +2}}d%
\mathbf{x}=\int_{B_{1}\left( \mathbf{0}\right) }\frac{c}{\rho ^{2\alpha +2}}%
\rho ^{n-1}d\mathbf{x<+\infty }$, cio\`{e} $\alpha <\frac{n-2}{2}$.

La condizione per la discontinuit\`{a} \`{e} $\alpha >0$; se $n=2$ la
richiesta per l'appartenenza a $H^{1}$ \`{e} $\alpha <0$, quindi da questo
esempio non si ricavano casi di funzioni in $H^{1}\left( B_{1}\left( \mathbf{%
0}\right) \right) $ ma discontinue. Si hanno invece per $n\geq 3$: basta
prendere $f$ con $\alpha \in \left( 0,\frac{n-2}{2}\right) $. E. g. per $n=3$
$\alpha =\frac{1}{4}$.

Per $n=2$ si considera $f\left( \mathbf{x}\right) =\left\vert \log \left(
\left\vert \left\vert \mathbf{x}\right\vert \right\vert \right) \right\vert
^{\alpha }$, che se $\alpha >0$ \`{e} discontinua. $f\in H^{1}\left( B_{%
\frac{1}{2}}\left( \mathbf{0}\right) \right) $? $f\in L^{2}\left( B_{\frac{1%
}{2}}\left( \mathbf{0}\right) \right) $ perch\'{e} $\int_{B_{\frac{1}{2}%
}\left( \mathbf{0}\right) }\left\vert \log \left( \left\vert \left\vert 
\mathbf{x}\right\vert \right\vert \right) \right\vert ^{2\alpha }d\mathbf{x}%
=\int_{0}^{1/2}\left\vert \log \rho \right\vert ^{2\alpha }\rho d\rho
<+\infty $ $\forall $ $\alpha $; $\left\vert \frac{\partial f}{\partial x_{j}%
}\right\vert =\left\vert \alpha \left\vert \log \left( \left\vert \left\vert 
\mathbf{x}\right\vert \right\vert \right) \right\vert ^{\alpha -1}\frac{x_{j}%
}{\left\vert \left\vert \mathbf{x}\right\vert \right\vert }\right\vert \in
L^{2}\Longleftrightarrow ^{}\int_{0}^{1/2}\rho \frac{\left\vert \log
\rho \right\vert ^{2\alpha -2}}{\rho ^{2}}d\rho <+\infty $, cio\`{e} $%
2-2\alpha >1$, da cui $\alpha <\frac{1}{2}$. Dunque scegliendo $\alpha \in
\left( 0,\frac{1}{2}\right) $ si hanno esempi di funzioni $H^{1}$ e
discontinue.

\item $f_{1}\left( x\right) =\sin \left\vert x\right\vert \in H^{1}\left(
-1,1\right) $ in quanto continua e derivabile classicamente q. o. con
derivata $\left\{ 
\begin{array}{c}
\cos x\text{ se }x>0 \\ 
-\cos \left( -x\right) \text{ se }x<0%
\end{array}%
\right. $ $L^{2}\left( -1,1\right) $. $f_{2}\left( x\right) =\frac{\sin
\left\vert x\right\vert }{x}\not\in H^{1}\left( -1,1\right) $ perch\'{e} non 
\`{e} continua in $0$. $f_{3}\left( x\right) =\frac{\sin x}{\left\vert
x\right\vert ^{\frac{3}{4}}}=\frac{\sin x}{\left\vert x\right\vert }%
\left\vert x\right\vert ^{\frac{1}{4}}\in C^{0}\left( \left[ -1,1\right]
\right) $ e $f_{3}^{\prime }\left( x\right) =\left\{ 
\begin{array}{c}
\frac{x^{\frac{3}{4}}\cos x-\frac{3}{4}x^{-\frac{1}{4}}\sin x}{x^{\frac{3}{2}%
}}\text{ se }x>0 \\ 
\frac{\left( -x\right) ^{\frac{3}{4}}\cos x+\frac{3}{4}x^{-\frac{1}{4}}\sin x%
}{\left( -x\right) ^{\frac{3}{2}}}\text{ se }x<0%
\end{array}%
\right. $, che non \`{e} $L^{2}$ (controlla): quindi $f_{3}\not\in
H^{1}\left( -1,1\right) $.
\end{enumerate}

Tuttavia anche se $n>1$ le funzioni con discontinuit\`{a} a salto non
possono essere $H^{1}$, e. g. $f\left( \mathbf{x}\right) =I_{B_{R}}\left( 
\mathbf{0}\right) $: la derivata distribuzionale su $u_{f}$ sarebbe una
delta sul cerchio.

\textbf{Teo (Rellich)}%
\begin{gather*}
\text{Hp: }f\text{ \`{e} lipschitziana in }\Omega \subseteq 
%TCIMACRO{\U{211d} }%
%BeginExpansion
\mathbb{R}
%EndExpansion
^{n} \\
\text{Ts: }f\text{ \`{e} derivabile in senso classico q. o., }\frac{\partial
f}{\partial x_{j}}\in L^{\infty }\left( \Omega \right) \text{ }\forall \text{
}j\text{ } \\
\text{e le sue derivate classiche sono anche derivate deboli}
\end{gather*}

Questo teorema istituisce un collegamento tra derivate classiche e derivate
deboli. E' coerente con quanto visto: $\left\vert x\right\vert $ \`{e}
lipschitziana.

Dunque se $\Omega $ \`{e} limitato allora $Lip\left( \Omega \right)
\subseteq W^{1,\infty }\left( \Omega \right) \subseteq H^{1}\left( \Omega
\right) $ (ricordando che una funzione lipschitziana su un dominio limitato 
\`{e} limitata). In generale, $Lip\left( \Omega \right) $ \`{e} l'analogo
soboleviano delle funzioni "regolari": se $\Omega $ \`{e} limitato, $%
Lip\left( \Omega \right) $ sta in qualsiasi spazio di Sobolev $W^{1,p}\left(
\Omega \right) $.

\textbf{Teo (derivata del prodotto)}%
\begin{eqnarray*}
\text{Hp}\text{: } &&\Omega \subseteq 
%TCIMACRO{\U{211d} }%
%BeginExpansion
\mathbb{R}
%EndExpansion
^{n}\text{, }f\in H^{1}\left( \Omega \right) \text{, }g\in Lip\left( \Omega
\right) \text{ \`{e} limitata} \\
\text{Ts}\text{: } &&fg\in H^{1}\left( \Omega \right) \text{ e }\frac{%
\partial }{\partial x_{j}}\left( fg\right) =\frac{\partial f}{\partial x_{j}}%
g+f\frac{\partial g}{\partial x_{j}}\text{ }
\end{eqnarray*}

Gli spazi di appartenenza delle funzioni al lato destro sono coerenti col
fatto che $fg\in H^{1}$: $\frac{\partial f}{\partial x_{j}}\in L^{2},g\in
L^{\infty }$, quindi il prodotto \`{e} $L^{2}$, e lo stesso vale per il
secondo addendo grazie a Rellich.

\subsection{Spazi di Sobolev di funzioni nulle sul bordo del dominio}

\textbf{Def} Dato $\Omega \subseteq 
%TCIMACRO{\U{211d} }%
%BeginExpansion
\mathbb{R}
%EndExpansion
^{n}$, si dice che $f\in H^{1}\left( \Omega \right) $ vale zero sul bordo, e
si scrive $f\in H_{0}^{1}\left( \Omega \right) $, se $\exists $ $\left\{
\phi _{k}\right\} _{k=1}^{+\infty }\subseteq C_{0}^{1}\left( \Omega \right)
:\phi _{k}\rightarrow ^{H^{1}\left( \Omega \right) }f$.

La definizione \`{e} ben posta perch\'{e} $C_{0}^{1}\left( \Omega \right)
\subseteq H^{1}\left( \Omega \right) $.

La definizione significa che $H_{0}^{1}\left( \Omega \right) $ \`{e} la
chiusura di $C_{0}^{1}\left( \Omega \right) $ in norma $H_{1}\left( \Omega
\right) $; equivalentemente, che $C_{0}^{1}\left( \Omega \right) $ \`{e}
denso in $H_{0}^{1}\left( \Omega \right) $ rispetto alla norma $H^{1}\left(
\Omega \right) $. La definizione data cattura bene l'idea di funzione nulla
al bordo perch\'{e} si chiede la convergenza in norma $H^{1}\left( \Omega
\right) $, che \`{e} una convergenza in $L^{2}$ delle $\phi _{k}$\textit{\ e
delle sue derivate}: per una funzione su $I\subseteq 
%TCIMACRO{\U{211d} }%
%BeginExpansion
\mathbb{R}
%EndExpansion
$ non nulla al bordo esisterebbe $\left\{ \phi _{k}\right\} _{k=1}^{+\infty
}\subseteq C_{0}^{1}\left( \Omega \right) :\phi _{k}\rightarrow
^{L^{2}\left( \Omega \right) }f$, ma non tale che $\phi _{k}\rightarrow
^{H^{1}\left( \Omega \right) }f$, perch\'{e} le derivate delle $\phi _{k}$
agli estremi dell'intervallo sarebbero ripidissime, dovendo raggiungere la $%
f $ che non \`{e} nulla agli estremi, e dunque le derivate di $\phi _{k}$
non convergerebbero alle derivate di $f$.

Si pu\`{o} mostrare che $H_{0}^{1}\left( \Omega \right) $ \`{e} chiuso in
norma $H^{1}\left( \Omega \right) $ (cio\`{e} ogni successione di $%
H_{0}^{1}\left( \Omega \right) $ convergente in norma $H^{1}\left( \Omega
\right) $ converge a una funzione $H_{0}^{1}\left( \Omega \right) $) e
quindi \`{e} completo in tale norma, essendo un sottospazio vettoriale di $%
H^{1}\left( \Omega \right) $ completo.

\textbf{Teo (funzioni nulle al bordo in una dimensione)}%
\begin{eqnarray*}
\text{Hp} &\text{: }&\left( a,b\right) \subseteq 
%TCIMACRO{\U{211d} }%
%BeginExpansion
\mathbb{R}
%EndExpansion
\\
\text{Ts} &\text{: }&f\in H_{0}^{1}\left( a,b\right) \Longleftrightarrow
f\in H^{1}\left( a,b\right) \text{ e }f\left( a\right) =f\left( b\right) =0
\end{eqnarray*}

Quindi la definizione data sembra essere un'estensione sensata del caso ben
noto in una dimensione. Come al solito, la condizione $f\left( a\right)
=f\left( b\right) =0$ \`{e} richiesta in realt\`{a} alla funzione continua
appartenente alla classe di equivalenza di $f\in H^{1}\left( a,b\right) $.

\textbf{Teo (integrazione per parti)}%
\begin{eqnarray*}
\text{Hp}\text{: } &&\Omega \subseteq 
%TCIMACRO{\U{211d} }%
%BeginExpansion
\mathbb{R}
%EndExpansion
^{n},f\in H^{1}\left( \Omega \right) ,g\in H_{0}^{1}\left( \Omega \right) \\
\text{Ts}\text{: } &&\int_{\Omega }f\frac{\partial g}{\partial x_{j}}%
dx=-\int_{\Omega }\frac{\partial f}{\partial x_{j}}gdx
\end{eqnarray*}

Si noti che gli integrali sono ben posti: le funzioni integrande sono
prodotto di due funzioni $L^{2}$. Quindi le funzioni $H_{0}^{1}\left( \Omega
\right) $ si comportano proprio come le funzioni nulle al bordo, dato che
soddisfano una formula di integrazione per parti identica a quella vista per
le $\phi $ a supporto compatto.

\textbf{Dim} E' una conseguenza immediata della definizione di derivata
debole e della densit\`{a} di $C_{0}^{1}\left( \Omega \right) $ in $%
H_{0}^{1}\left( \Omega \right) $.

Infatti, per ipotesi $\exists $ $\left\{ \phi _{k}\right\} _{k=1}^{+\infty
}\subseteq C_{0}^{1}\left( \Omega \right) :\phi _{k}\rightarrow
^{H^{1}\left( \Omega \right) }g$. Inoltre, per definizione di derivata
debole, $\forall $ $k$ vale $\int_{\Omega }f\frac{\partial \phi _{k}}{%
\partial x_{j}}dx=-\int_{\Omega }\frac{\partial f}{\partial x_{j}}\phi _{k}$%
. Per $k\rightarrow +\infty $ $\phi _{k}\rightarrow ^{L^{2}\left( \Omega
\right) }g$ e $\frac{\partial \phi _{k}}{\partial x_{j}}\rightarrow
^{L^{2}\left( \Omega \right) }\frac{\partial g}{\partial x_{j}}$, quindi,
per continuit\`{a} del prodotto scalare in $L^{2}$, si ha che $\int_{\Omega
}f\frac{\partial \phi _{k}}{\partial x_{j}}dx\rightarrow ^{k\rightarrow
+\infty }\int_{\Omega }f\frac{\partial g}{\partial x_{j}}dx$ e $%
-\int_{\Omega }\frac{\partial f}{\partial x_{j}}\phi _{k}dx\rightarrow
^{k\rightarrow +\infty }-\int_{\Omega }\frac{\partial f}{\partial x_{j}}gdx$%
: allora $\int_{\Omega }f\frac{\partial g}{\partial x_{j}}dx=-\int_{\Omega }%
\frac{\partial f}{\partial x_{j}}gdx$. $\blacksquare $

\textbf{Teo (disuguaglianza di Poincar\'{e})}%
\begin{eqnarray*}
\text{Hp}\text{: } &&\Omega \subseteq 
%TCIMACRO{\U{211d} }%
%BeginExpansion
\mathbb{R}
%EndExpansion
^{n}\text{ \`{e} un dominio limitato, }u\in H_{0}^{1}\left( \Omega \right) \\
\text{Ts} &\text{:}&\text{ }\exists \text{ }c=c\left( \Omega \right)
>0:\left\vert \left\vert u\right\vert \right\vert _{L^{2}\left( \Omega
\right) }\leq c\left\vert \left\vert \nabla u\right\vert \right\vert
_{L^{2}\left( \Omega \right) }
\end{eqnarray*}

Questa maggiorazione potrebbe apparire strana: tipicamente accade il
contrario, cio\`{e} una funzione si maggiora con le sue derivate, e in
effetti per funzioni qualsiasi la disuguaglianza non ha senso (si pensi alle
funzioni costanti e positive). La disuguaglianza potrebbe avere senso per%
\`{o} in $H_{0}^{1}\left( \Omega \right) $, per funzioni che si annullano al
bordo.

\textbf{Dim} Si dimostra il risultato supponendo $u\in C_{0}^{1}\left(
\Omega \right) $ e poi si estende a $H_{0}^{1}\left( \Omega \right) $, per
densit\`{a}.

Essendo $\Omega $ limitato, esiste $R:\Omega \subseteq B_{R}\left( \mathbf{0}%
\right) $, e quindi si pu\`{o} applicare il teorema della divergenza: $%
\int_{\Omega }\func{div}\left( \mathbf{x}u^{2}\left( \mathbf{x}\right)
\right) d\mathbf{x}=\int_{B_{R}\left( \mathbf{0}\right) }\func{div}\left( 
\mathbf{x}u^{2}\left( \mathbf{x}\right) \right) d\mathbf{x=}\int_{\partial
B_{R}\left( \mathbf{0}\right) }\left\langle \mathbf{x}u^{2}\left( \mathbf{x}%
\right) ,n\right\rangle dS=0$ perch\'{e} $u$ \`{e} a supporto compatto in $%
\Omega $. In questo modo si \`{e} applicato il teorema della divergenza
senza chiedere alcuna regolarit\`{a} a $\Omega $. Ma $\func{div}\left( 
\mathbf{x}u^{2}\left( \mathbf{x}\right) \right) =\sum_{i=1}^{n}\frac{%
\partial }{\partial x_{i}}\left( x_{i}u^{2}\left( \mathbf{x}\right) \right)
=\sum_{i=1}^{n}\left( u^{2}\left( \mathbf{x}\right) +2x_{i}\frac{\partial u}{%
\partial x_{i}}u\left( \mathbf{x}\right) \right) =nu^{2}+2u\left\langle 
\mathbf{x},\nabla u\right\rangle $, quindi si \`{e} scritto $\int_{\Omega
}\left( nu^{2}+2u\left\langle \mathbf{x},\nabla u\right\rangle \right) d%
\mathbf{x}=0$. Allora $\left\vert n\int_{\Omega }u^{2}\left( \mathbf{x}%
\right) d\mathbf{x}\right\vert =\left\vert 2\int_{\Omega }u\left\langle 
\mathbf{x},\nabla u\right\rangle d\mathbf{x}\right\vert \leq 2\int_{\Omega
}\left\vert u\right\vert \left\vert \left\langle \mathbf{x},\nabla
u\right\rangle \right\vert d\mathbf{x}$: poich\'{e} $\left\vert \left\langle 
\mathbf{x},\nabla u\right\rangle \right\vert \leq \left\vert \left\vert 
\mathbf{x}\right\vert \right\vert \left\vert \left\vert \nabla u\right\vert
\right\vert \leq R\left\vert \left\vert \nabla u\right\vert \right\vert $,
si ha $\left\vert n\int_{\Omega }u^{2}\left( \mathbf{x}\right) d\mathbf{x}%
\right\vert \leq 2R\int_{\Omega }\left\vert u\right\vert \left\vert
\left\vert \nabla u\right\vert \right\vert d\mathbf{x}\leq 2R\left\vert
\left\vert u\right\vert \right\vert _{L^{2}\left( \Omega \right) }\left\vert
\left\vert \nabla u\right\vert \right\vert _{L^{2}\left( \Omega \right) }$,
per la disuguaglianza di Hoelder. Dunque $n\left\vert \left\vert
u\right\vert \right\vert _{L^{2}\left( \Omega \right) }^{2}\leq 2R\left\vert
\left\vert u\right\vert \right\vert _{L^{2}\left( \Omega \right) }\left\vert
\left\vert \nabla u\right\vert \right\vert _{L^{2}\left( \Omega \right) }$,
e si ha la tesi con $c=\frac{2R}{n}$.

Ora suppongo che $u\in H_{0}^{1}\left( \Omega \right) $: per ipotesi esiste $%
\left\{ u_{k}\right\} \subseteq C_{0}^{1}\left( \Omega \right)
:u_{k}\rightarrow ^{H_{1}\left( \Omega \right) }u$, e $\forall $ $k$ vale $%
\left\vert \left\vert u_{k}\right\vert \right\vert _{L^{2}\left( \Omega
\right) }\leq c\left\vert \left\vert \nabla u_{k}\right\vert \right\vert
_{L^{2}\left( \Omega \right) }$: per $k\rightarrow +\infty $ il lato
sinistro tende a $\left\vert \left\vert u\right\vert \right\vert
_{L^{2}\left( \Omega \right) }$, il lato destro a $c\left\vert \left\vert
\nabla u\right\vert \right\vert _{L^{2}\left( \Omega \right) }$ (perch\'{e}
la convergenza in norma implica la convergenza delle norme $H^{1}$ e quindi
di ciascuna delle norme $L^{2}$), che \`{e} la tesi. $\blacksquare $

Raffinando un poco la dimostrazione si pu\`{o} ottenere la tesi con $%
c_{\Omega }=\frac{diam\left( \Omega \right) }{n}$. La disuguaglianza perde
ovviamente di significato se $\Omega $ \`{e} illimitato: $diam\left( \Omega
\right) =+\infty $.

\textbf{Corollario (equivalenza tra norme)}%
\begin{gather*}
\text{Hp}\text{: }\Omega \subseteq 
%TCIMACRO{\U{211d} }%
%BeginExpansion
\mathbb{R}
%EndExpansion
^{n}\text{ \`{e} un dominio limitato} \\
\text{Ts}\text{: in }H_{0}^{1}\left( \Omega \right) \text{ la norma }%
H^{1}\left( \Omega \right) \text{ \`{e} equivalente alla norma }\left\vert
\left\vert \cdot \right\vert \right\vert _{H_{0}^{1}\left( \Omega \right)
}=\left\vert \left\vert \nabla \cdot \right\vert \right\vert _{L^{2}\left(
\Omega \right) }
\end{gather*}

Si ricorda che in uno spazio $X$ una norma $\left\vert \left\vert \cdot
\right\vert \right\vert _{1}$ su $X$ si dice equivalente a un'altra norma $%
\left\vert \left\vert \cdot \right\vert \right\vert _{2}$ su $X$ se $\exists 
$ $c_{1},c_{2}\in 
%TCIMACRO{\U{211d} }%
%BeginExpansion
\mathbb{R}
%EndExpansion
:c_{1}\left\vert \left\vert x\right\vert \right\vert _{2}\leq \left\vert
\left\vert x\right\vert \right\vert _{1}\leq c_{2}\left\vert \left\vert
x\right\vert \right\vert _{2}$ $\forall $ $x\in X$.

E' proprio un'applicazione del teorema sopra: la convergenza in $H^{1}$, che
sarebbe una convergenza in $L^{2}$ delle funzioni e delle derivate, si
riduce a una convergenza in $L^{2}$ delle derivate grazie alla
disuguaglianza sopra.

\textbf{Dim} Ovviamente $\left\vert \left\vert u\right\vert \right\vert
_{H^{1}\left( \Omega \right) }=\left( \left\vert \left\vert u\right\vert
\right\vert _{L^{2}\left( \Omega \right) }^{2}+\left\vert \left\vert \nabla
u\right\vert \right\vert _{L^{2}\left( \Omega \right) }^{2}\right) ^{\frac{1%
}{2}}\geq \left\vert \left\vert \nabla u\right\vert \right\vert
_{L^{2}\left( \Omega \right) }$; d'altra parte, per la disuguaglianza sopra, 
$\left\vert \left\vert u\right\vert \right\vert _{H^{1}\left( \Omega \right)
}\leq \left( \left( c_{\Omega }^{2}+1\right) \left\vert \left\vert \nabla
u\right\vert \right\vert _{L^{2}\left( \Omega \right) }^{2}\right) ^{\frac{1%
}{2}}=\sqrt{1+c^{2}}\left\vert \left\vert \nabla u\right\vert \right\vert
_{L^{2}\left( \Omega \right) }$. $\blacksquare $

\subsection{Duale di $H_{0}^{1}\left( \Omega \right) $}


Considero ora $T\in H_{0}^{1}\left( \Omega \right) ^{\ast }$, cio\`{e} $T$
funzionale lineare e continuo su $H_{0}^{1}\left( \Omega \right) $, che si 
\`{e} detto essere uno spazio di Hilbert. Voglio capire se $T$ può essere in generale associato a un'espressione specifica. \\
Per il teorema di Riesz, $%
\exists $ $!$ $f\in H_{0}^{1}\left( \Omega \right) :\forall $ $g\in
H_{0}^{1}\left( \Omega \right) $ vale $T\left( g\right) =\left\langle
f,g\right\rangle _{H_{0}^{1}\left( \Omega \right) }=\int_{\Omega }\left(
fg+\left\langle \nabla f,\nabla g\right\rangle \right) d\mathbf{x}$, e la
norma del funzionale \`{e} $\left\vert \left\vert T\right\vert \right\vert
_{H_{0}^{1}\left( \Omega \right) ^{\ast }}=\left\vert \left\vert
f\right\vert \right\vert _{H_{0}^{1}\left( \Omega \right) }$.

Se in particolare $g\in \mathcal{D}\left( \Omega \right) $, $\int_{\Omega
}\left( fg+\left\langle \nabla f,\nabla g\right\rangle \right) d\mathbf{x}%
=\int_{\Omega }\left( fg+\sum_{i=1}^{n}f_{x_{i}}g_{x_{i}}\right) d\mathbf{x}$
pu\`{o} essere vista come azione della distribuzione $u_{f}-%
\sum_{i=1}^{n}D_{x_{i}}u_{f_{x_{i}}}$ su $g$ (B dice, con abuso di
notazione, $f-\sum_{i=1}^{n}\left( f_{x_{i}}\right) _{x_{i}}$). Quindi, se $%
T\in H_{0}^{1}\left( \Omega \right) ^{\ast }$, la restrizione di $T$ a $%
\mathcal{D}\left( \Omega \right) $ \`{e} la distribuzione $%
u_{f}-\sum_{i=1}^{n}D_{x_{i}}u_{f_{x_{i}}}$, (B dice, con abuso, la
distribuzione $f-\sum_{i=1}^{n}\left( f_{x_{i}}\right) _{x_{i}}$), con $%
f,f_{x_{i}}\in L^{2}\left( \Omega \right) $, e in generale non \`{e}
associata ad alcuna funzione.

Viceversa, prese $f_{0},...,f_{n}\in L^{2}\left( \Omega \right) $, considero
la distribuzione $u_{f_{0}}-\sum_{i=1}^{n}D_{x_{i}}u_{f_{i}}$ (B dice la
distribuzione $f_{0}-\sum_{i=1}^{n}\left( f_{i}\right) _{x_{i}}$): $T\left(
g\right) =$($^{\text{abuso}}\int_{\Omega }\left( f_{0}-\sum_{i=1}^{n}\left(
f_{i}\right) _{x_{i}}\right) gd\mathbf{x}$)$=\int_{\Omega }\left(
f_{0}g+\sum_{i=1}^{n}f_{i}g_{x_{i}}\right) d\mathbf{x}$ $\forall $ $g\in 
\mathcal{D}\left( \Omega \right) $. Quest'ultimo integrale sarebbe ben
definito anche se $g$ non fosse una funzione test, ma solo $H_{0}^{1}\left(
\Omega \right) $: si avrebbe sempre il prodotto tra due funzioni $L^{2}$. Ma
allora l'estensione di $T$ da distribuzione a elemento di $H_{0}^{1}\left(
\Omega \right) ^{\ast }$ \`{e} univocamente determinata, grazie alla densit%
\`{a} di $C_{0}^{1}\left( \Omega \right) $ in $H_{0}^{1}\left( \Omega
\right) $: sapere quanto vale $T$ sulle funzioni test \`{e} sufficiente per
sapere quanto vale su tutte le funzioni $H_{0}^{1}\left( \Omega \right) $.

Questo dimostra il seguente

\textbf{Teo (caratterizzazione distribuzionale di }$H_{0}^{1}\left( \Omega
\right) ^{\ast }$)\textbf{\ } 
\begin{gather*}
H_{0}^{1}\left( \Omega \right) ^{\ast }\text{ si pu\`{o} identificare con lo
spazio delle distribuzioni su }\Omega \text{ } \\
\text{del tipo }u_{f_{0}}-\sum_{i=1}^{n}D_{x_{i}}u_{f_{i}}\text{, con }%
f_{i}\in L^{2}\left( \Omega \right) \text{ }\forall \text{ }i=0,...,n
\end{gather*}

B dice: si identifica con le distribuzioni del tipo $\left\langle
f_{0}-\sum_{i=1}^{n}\left( f_{i}\right) _{x_{i}},g\right\rangle $

Si noti che questo non \`{e} vero per $H^{1}\left( \Omega \right) ^{\ast }$:
l'estensione di $T$ da $\mathcal{D}\left( \Omega \right) $ a $H^{1}\left(
\Omega \right) $ non \`{e} unica, perch\'{e} $C_{0}^{1}\left( \Omega \right) 
$ non \`{e} denso in $H^{1}\left( \Omega \right) $.

Talvolta si indica $H_{0}^{1}\left( \Omega \right) ^{\ast }$ con $%
H^{-1}\left( \Omega \right) $: pensando che $H^{1}\left( \Omega \right) $ 
\`{e} lo spazio delle funzioni $L^{2}$ con derivata $L^{2}$ e $H^{0}\left(
\Omega \right) $ \`{e} usato per le funzioni solo $L^{2}$, $H^{-1}\left(
\Omega \right) $ \`{e} lo spazio delle funzioni che sono "derivata" (B dice
debole, ma in realt\`{a} distribuzionale?) di una funzione (in realt\`{a} di
una distribuzione associata a una funzione) $L^{2}$.

\begin{enumerate}
\item Un elemento di $H^{-1}\left( -1,1\right) $ \`{e} la distribuzione $%
\delta _{0}$, poich\'{e} pu\`{o} essere scritta come $Du_{H}$ e la funzione
gradino $H$ \`{e} $L^{2}\left( -1,1\right) $.

Se $n>1$ la delta non pu\`{o} essere un funzionale lineare continuo su $%
H_{0}^{1}\left( \Omega \right) $ perch\'{e} sappiamo che le funzioni di $%
H^{1}\left( \Omega \right) $ possono essere discontinue, quindi non
necessariamente nella loro classe di equivalenza c'\`{e} una funzione
continua e parlare del loro valore in un punto \`{e} privo di significato
[ma quindi perch\'{e} la delta in pi\`{u} variabili non si pu\`{o}
rappresentare nella forma $u_{f_{0}}-\sum_{i=1}^{n}D_{x_{i}}u_{f_{i}}$?).

\item Dato $\Omega \subseteq 
%TCIMACRO{\U{211d} }%
%BeginExpansion
\mathbb{R}
%EndExpansion
^{n},n\geq 2$, con $B_{R}\left( \mathbf{0}\right) \subset \Omega $, si
considera $f\left( \mathbf{x}\right) =I_{B_{R}\left( \mathbf{0}\right)
}\left( \mathbf{x}\right) $. $f\in L^{2}\left( \Omega \right) $; le derivate
distribuzionali di $u_{f}$ non sono associabili ad alcuna funzione e sono in 
$H^{-1}\left( \Omega \right) $. Infatti, se $g\in H_{0}^{1}\left( \Omega
\right) $, $D_{x_{i}}u_{f}\left( g\right) =-u_{f}\left( g_{x_{i}}\right)
=-\int_{B_{R}\left( \mathbf{0}\right) }g_{x_{i}}d\mathbf{x}$, che ha
perfettamente senso. Se $g$ fosse abbastanza regolare da poter applicare il
teorema della divergenza, l'integrale si riscriverebbe come integrale di
superficie: il valore del funzionale dipenderebbe solo dai valori di $g$
sulla superficie della sfera, ed \`{e} naturale che non possa essere
rappresentato come funzione (\`{e} pi\`{u} una distribuzione del tipo
misura).
\end{enumerate}

\subsection{Spazi di Sobolev di funzioni derivabili $m$ volte}

Si definisce $W^{2,p}\left( \Omega \right) :=\left\{ f\in W^{1,p}\left(
\Omega \right) :\forall \text{ }i=1,...,n\text{ esistono tutte\footnote{$%
\forall $ $i=1,...,n$} le derivate deboli di }\frac{\partial f}{\partial
x_{i}}\text{ e appartengono a }L^{p}\left( \Omega \right) \right\} $.
Iterando si ottiene la definizione naturale 
\begin{equation*}
W^{m,p}\left( \Omega \right)
:=\left\{ f\in W^{m-1,p}\left( \Omega \right) :\forall \text{ }\alpha
:\left\vert \alpha \right\vert =m-1\text{ esistono tutte le derivate deboli
di }D^{\alpha }f\text{ e appartengono a }L^{p}\left( \Omega \right) \right\} 
\end{equation*}

$W^{m,p}\left( \Omega \right) $ \`{e} uno spazio di Banach con la norma $%
\left\vert \left\vert u\right\vert \right\vert _{W^{m,p}\left( \Omega
\right) }:=\left( \left\vert \left\vert u\right\vert \right\vert
_{L^{p}\left( \Omega \right) }^{p}+\sum_{\left\vert \alpha \right\vert \leq
m}\left\vert \left\vert D^{\alpha }u\right\vert \right\vert _{L^{p}\left(
\Omega \right) }^{p}\right) ^{1/p}$. Si definisce allora $H^{m}\left( \Omega
\right) :=W^{m,2}\left( \Omega \right) $, che \`{e} uno spazio di Hilbert
con il prodotto scalare $\left\langle f,g\right\rangle _{H^{m}\left( \Omega
\right) }=\int_{\Omega }\left( fg+\sum_{\left\vert \alpha \right\vert \leq
m}D^{\alpha }fD^{\alpha }g\right) d\mathbf{x}$.

\subsection{Approssimazione di funzioni $H^{1}\left( \Omega \right) $}

I teoremi di approssimazione di funzioni $H^{1}\left( \Omega \right) $ con
funzioni pi\`{u} regolari sono estremamente utili per dimostrare teoremi: si
dimostra la tesi desiderata sulle funzioni pi\`{u} regolari e poi per densit%
\`{a} ci si estende alle funzioni $H^{1}\left( \Omega \right) $, proprio
come si \`{e} fatto nella dimostrazione dell'uguaglianza di Poincar\'{e}.

\textbf{Teo (approssimazione locale)}%
\begin{gather*}
\text{Hp}\text{: }\Omega \subseteq 
%TCIMACRO{\U{211d} }%
%BeginExpansion
\mathbb{R}
%EndExpansion
^{n}\text{ dominio, }u\in H^{1}\left( \Omega \right) \text{, }\Omega
^{\prime }\subset \subset \Omega \\
\text{Ts}\text{: }\exists \text{ }\left\{ u_{k}\right\} _{k=1}^{+\infty
}\subseteq \mathcal{D}\left( \Omega \right) :\left\vert \left\vert
u_{k}-u\right\vert \right\vert _{H^{1}\left( \Omega ^{\prime }\right)
}\rightarrow ^{k\rightarrow +\infty }0
\end{gather*}

$\Omega ^{\prime }\subset \subset \Omega $ significa che $\Omega ^{\prime }$ 
\`{e} contenuto con compattezza in $\Omega $, cio\`{e} $\Omega ^{\prime }$ 
\`{e} limitato e $\bar{\Omega}^{\prime }\subseteq \Omega $.

La convergenza c'\`{e} solo in $H^{1}\left( \Omega ^{\prime }\right) $, non
in $H^{1}\left( \Omega \right) $, quindi \`{e} solo l\`{\i} che
l'approssimazione \`{e} buona: vicino al bordo le derivate delle $u_{k}$
saranno molto grandi, perch\'{e} $u_{k}$ deve essere a supporto compatto e
al contempo approssimare $u$, che in generale non \`{e} nulla sul bordo.

\textbf{Teo (approssimazione globale in }$%
%TCIMACRO{\U{211d} }%
%BeginExpansion
\mathbb{R}
%EndExpansion
^{n}$\textbf{)}%
\begin{gather*}
\text{Hp}\text{: }u\in H^{1}\left( 
%TCIMACRO{\U{211d} }%
%BeginExpansion
\mathbb{R}
%EndExpansion
^{n}\right) \\
\text{Ts}\text{: }\exists \text{ }\left\{ u_{k}\right\} _{k=1}^{+\infty
}\subseteq \mathcal{D}\left( 
%TCIMACRO{\U{211d} }%
%BeginExpansion
\mathbb{R}
%EndExpansion
^{n}\right) :\left\vert \left\vert u_{k}-u\right\vert \right\vert
_{H^{1}\left( 
%TCIMACRO{\U{211d} }%
%BeginExpansion
\mathbb{R}
%EndExpansion
^{n}\right) }\rightarrow ^{k\rightarrow +\infty }0
\end{gather*}

cio\`{e} $\mathcal{D}\left( 
%TCIMACRO{\U{211d} }%
%BeginExpansion
\mathbb{R}
%EndExpansion
^{n}\right) $ \`{e} denso in $H^{1}\left( 
%TCIMACRO{\U{211d} }%
%BeginExpansion
\mathbb{R}
%EndExpansion
^{n}\right) $. Questo implica che ogni $u\in H^{1}\left( 
%TCIMACRO{\U{211d} }%
%BeginExpansion
\mathbb{R}
%EndExpansion
^{n}\right) $ in realt\`{a} soddisfa la definizione di $H_{0}^{1}\left( 
%TCIMACRO{\U{211d} }%
%BeginExpansion
\mathbb{R}
%EndExpansion
^{n}\right) $, cio\`{e} $H_{0}^{1}\left( 
%TCIMACRO{\U{211d} }%
%BeginExpansion
\mathbb{R}
%EndExpansion
^{n}\right) =H^{1}\left( 
%TCIMACRO{\U{211d} }%
%BeginExpansion
\mathbb{R}
%EndExpansion
^{n}\right) $ (che intuitivamente ha senso, pensando al fatto che $%
%TCIMACRO{\U{211d} }%
%BeginExpansion
\mathbb{R}
%EndExpansion
^{n}$ non ha bordo).

\textbf{Teo (approssimazione globale in un dominio con frontiera regolare)}%
\begin{gather*}
\text{Hp}\text{: }\Omega \subseteq 
%TCIMACRO{\U{211d} }%
%BeginExpansion
\mathbb{R}
%EndExpansion
^{n}\text{ \`{e} un dominio limitato e lipschitziano oppure un semispazio, }%
u\in H^{1}\left( \Omega \right) \\
\text{Ts}\text{: }\exists \text{ }\left\{ u_{k}\right\} _{k=1}^{+\infty
}\subseteq \mathcal{D}\left( 
%TCIMACRO{\U{211d} }%
%BeginExpansion
\mathbb{R}
%EndExpansion
^{n}\right) :\left\vert \left\vert u_{k}-u\right\vert \right\vert
_{H^{1}\left( \Omega \right) }\rightarrow ^{k\rightarrow +\infty }0
\end{gather*}

Quindi in particolare $\exists $ $\left\{ u_{k}\right\} _{k=1}^{+\infty
}\subseteq C^{1}\left( \bar{\Omega}\right) :\left\vert \left\vert
u_{k}-u\right\vert \right\vert _{H^{1}\left( 
%TCIMACRO{\U{211d} }%
%BeginExpansion
\mathbb{R}
%EndExpansion
^{n}\right) }\rightarrow ^{k\rightarrow +\infty }0$.

Il terzo risultato permette di dimostrare il seguente

\textbf{Teo (derivazione del prodotto)}%
\begin{eqnarray*}
\text{Hp}\text{: } &&\Omega \subseteq 
%TCIMACRO{\U{211d} }%
%BeginExpansion
\mathbb{R}
%EndExpansion
^{n}\text{ \`{e} un dominio limitato e lipschitziano, }f,g\in H^{1}\left(
\Omega \right) \\
\text{Ts}\text{: } &&fg\in W^{1,1}\left( \Omega \right) \text{ e }\left(
fg\right) _{x_{i}}=f_{x_{i}}g+fg_{x_{i}}\text{ }\forall \text{ }i=1,...,n
\end{eqnarray*}

Si noti che in generale il prodotto di due funzioni $H^{1}$ non \`{e} $H^{1}$%
, perch\'{e} il prodotto di funzioni $L^{2}$ \`{e} solo $L^{1}$.

\textbf{Dim} $fg\in L^{1}\left( \Omega \right) $ perch\'{e} $f,g\in
L^{2}\left( \Omega \right) $. Si applica il teorema immediatamente sopra: $%
\exists $ $\left\{ f_{k}\right\} _{k},\left\{ g_{k}\right\} _{k}\subseteq 
\mathcal{D}\left( 
%TCIMACRO{\U{211d} }%
%BeginExpansion
\mathbb{R}
%EndExpansion
^{n}\right) :\left\vert \left\vert f_{k}-f\right\vert \right\vert
,\left\vert \left\vert g_{k}-g\right\vert \right\vert _{H^{1}\left( \Omega
\right) }\rightarrow ^{k\rightarrow +\infty }0$. Essendo $f_{k},g_{k}\in
C^{1}$, vale la regola di derivazione: $\left( f_{k}g_{k}\right)
_{x_{i}}=\left( f_{k}\right) _{x_{i}}g_{k}+f_{k}\left( g_{k}\right) _{x_{i}}$%
, e $f_{k}\rightarrow ^{L^{2}\left( \Omega \right) }f,g_{k}\rightarrow
^{L^{2}\left( \Omega \right) }g,\left( f_{k}\right) _{x_{i}}\rightarrow
^{L^{2}\left( \Omega \right) }f_{x_{i}},\left( g_{k}\right)
_{x_{i}}\rightarrow ^{L^{2}\left( \Omega \right) }g_{x_{i}}$.

Osservo la convergenza del lato destro dell'uguaglianza. Per continuit\`{a}
del prodotto scalare $\int_{\Omega }\left( f_{k}\right) _{x_{i}}g_{k}d%
\mathbf{x}\rightarrow \int_{\Omega }f_{x_{i}}gd\mathbf{x}$, cio\`{e} $\left(
f_{k}\right) _{x_{i}}g_{k}\rightarrow ^{L^{1}\left( \Omega \right)
}f_{x_{i}}g$, e analogamente $f_{k}\left( g_{k}\right) _{x_{i}}\rightarrow
^{L^{1}\left( \Omega \right) }fg_{x_{i}}$: quindi $\left( f_{k}\right)
_{x_{i}}g_{k}+f_{k}\left( g_{k}\right) _{x_{i}}\rightarrow
^{L^{1}}f_{x_{i}}g+fg_{x_{i}}$. Per quanto riguarda il lato sinistro, so
(per la formula di integrazione per parti, essendo $f_{k}g_{k}\in
H^{1}\left( \Omega \right) $) che $\forall $ $\phi \in C_{0}^{1}\left(
\Omega \right) $ (e quindi $H_{0}^{1}\left( \Omega \right) $) vale $%
-\int_{\Omega }\left( f_{k}g_{k}\right) \phi _{x_{i}}d\mathbf{x}%
=\int_{\Omega }\left( f_{k}g_{k}\right) _{x_{i}}\phi d\mathbf{x}$. Quindi,
poich\'{e} $f_{k}\rightarrow ^{L^{2}}f,g_{k}\rightarrow ^{L^{2}}g$, per
continuit\`{a} del prodotto scalare $f_{k}g_{k}\rightarrow ^{L^{1}}fg$ e
quindi il lato sinistro tende a $-\int_{\Omega }fg\phi _{x_{i}}d\mathbf{x}$
(moltiplicare per $\phi $ non cambia nulla); il lato destro, per quanto
mostrato sopra, tende a $\int_{\Omega }\left( f_{x_{i}}g+fg_{x_{i}}\right)
\phi d\mathbf{x}$. Si ha quindi $-\int_{\Omega }fg\phi _{x_{i}}d\mathbf{x=}%
\int_{\Omega }\left( f_{x_{i}}g+fg_{x_{i}}\right) \phi d\mathbf{x}$, cio\`{e}
$\exists $ la derivata debole di $fg$ ed \`{e} proprio $%
f_{x_{i}}g+fg_{x_{i}} $. $\blacksquare $

\subsection{Traccia di una funzione $H^{1}\left( \Omega \right) $}

Abbiamo dato una definizione di funzioni $H^{1}\left( \Omega \right) $ che
sono nulle sul bordo; se non sono nulle, possiamo comunque dare una
definizione del loro comportamento sul bordo? Se $u\in H^{1}\left( \Omega
\right) $, non ha senso - perlomeno in senso stretto - parlare della sua
restrizione a $\partial \Omega $, perch\'{e} se $\Omega $ \`{e} aperto $\partial \Omega $ ha misura nulla, $u$ \`{e}
definita solo quasi ovunque e non necessariamente nella sua classe di
equivalenza c'\`{e} una funzione continua.

Allora si introduce il concetto di traccia, che generalizza quello di
restrizione: vorremmo poter assegnare a ogni $u\in H^{1}\left( \Omega
\right) $ una sua funzione "traccia" appartenente a $L^{2}\left( \partial
\Omega \right) $ in modo che:

\begin{description}
\item[i] sia coerente con il significato classico di restrizione, cio\`{e}
se $u\in C^{1}\left( \bar{\Omega}\right) $ allora la
traccia di $u$ coincide con $u|_{\partial \Omega }$;

\item[ii] sia coerente con il caso gi\`{a} definito, cio\`{e} se $u\in
H_{0}^{1}\left( \Omega \right) $ allora la traccia di $u$ \`{e} la funzione
nulla.
\end{description}

Si noti che essere $L^{2}\left( \partial \Omega \right) $, se $\partial
\Omega $ \`{e} abbastanza regolare (cio\`{e} \`{e} grafico di una funzione $%
f $ abbastanza regolare; tipicamente si chiede $\Omega $ dominio
lipschitziano), significa essere $L^{2}$ rispetto alla misura di superficie
del grafico di $f$ (e tale misura si scriver\`{a} quindi, come al solito, $%
d\sigma =\sqrt{1+\left\vert \left\vert \nabla f\right\vert \right\vert ^{2}}d%
\mathbf{y}$).

\begin{enumerate}
\item Se e. g. $\Omega =%
%TCIMACRO{\U{211d} }%
%BeginExpansion
\mathbb{R}
%EndExpansion
\times \lbrack 0,+\infty )$ e $u\in H^{1}\left( \Omega \right) $, $\partial
\Omega =\left\{ \left( x,0\right) :x\in 
%TCIMACRO{\U{211d} }%
%BeginExpansion
\mathbb{R}
%EndExpansion
\right\} $; si sta cercando $tr\left( u\right) $ tale che se $u\in
C^{1}\left( \bar{\Omega}\right) $ allora $tr\left( u\right) =u|_{\partial
\Omega }=u\left( x,0\right) $, e $tr\left( u\right) \in L^{2}\left( \partial
\Omega \right) $ se $\left\vert \left\vert tr\left( u\right) \right\vert
\right\vert _{L^{2}\left( \partial \Omega \right) }=\int_{%
%TCIMACRO{\U{211d} }%
%BeginExpansion
\mathbb{R}
%EndExpansion
}u^{2}\left( x,0\right) dx<+\infty $. Se $\partial \Omega $ fosse stata una
curva, si sarebbe calcolato un integrale curvilineo.
\end{enumerate}

\textbf{Teo (esistenza della traccia)}%
\begin{gather*}
\text{Hp: }\Omega \subseteq 
%TCIMACRO{\U{211d} }%
%BeginExpansion
\mathbb{R}
%EndExpansion
^{n}\text{ \`{e} un dominio limitato e lipschitziano oppure un semispazio} \\
\text{Ts: }\exists \text{ }\tau _{0}:H^{1}\left( \Omega \right) \rightarrow
L^{2}\left( \partial \Omega \right) \text{ lineare e continuo tale che} \\
\text{(i) }\forall \text{ }u\in C^{1}\left( \bar{\Omega}\right) \text{ }\tau
_{0}u=u|_{\partial \Omega } \\
\text{(ii) }\ker \tau _{0}=H_{0}^{1}\left( \Omega \right)
\end{gather*}

La dimostrazione si fa tipicamente sul semipiano, usando il terzo teorema di
approssimazione (infatti le ipotesi sono le stesse).

Abbiamo gi\`{a} visto che se $f\in H^{1}\left( \Omega \right) ,g\in
H_{0}^{1}\left( \Omega \right) $ allora vale la formula di integrazione per
parti $\int_{\Omega }f\frac{\partial g}{\partial x_{j}}dx=-\int_{\Omega }%
\frac{\partial f}{\partial x_{j}}gdx$; l'esistenza di un operatore di
traccia permette di generalizzare la formula a $f,g\in H^{1}\left( \Omega
\right) $, poich\'{e} ora \`{e} ben definito un termine di bordo.

\textbf{Teo (integrazione per parti di funzioni }$H^{1}\left( \Omega \right) 
$)%
\begin{gather*}
\text{Hp: }\Omega \subseteq 
%TCIMACRO{\U{211d} }%
%BeginExpansion
\mathbb{R}
%EndExpansion
^{n}\text{ \`{e} un dominio limitato e lipschitziano oppure un semispazio, }%
u,v\in H^{1}\left( \Omega \right) \\
\text{Ts: }\int_{\Omega }u_{x_{i}}vd\mathbf{x}=\int_{\partial \Omega }\left(
\tau _{0}u\right) \left( \tau _{0}v\right) n_{i}dS-\int_{\Omega }uv_{x_{i}}d%
\mathbf{x}
\end{gather*}

Le ipotesi sono quelle necessarie per l'esistenza della traccia.

\textbf{Dim} Per il terzo teorema di approssimazione $\exists $ $\left\{
u_{k}\right\} _{k},\left\{ v_{k}\right\} _{k}\subseteq C^{1}\left( \bar{%
\Omega}\right) :u_{k}\rightarrow ^{H^{1}\left( \Omega \right)
}u,v_{k}\rightarrow ^{H^{1}\left( \Omega \right) }v$. Per il teorema della
divergenza\footnote{%
sarebbe la formula di integrazione per parti generalizzata grazie al teorema
della divergenza, ma non abbiamo mai visto prima questa versione} $%
\int_{\Omega }\left( u_{k}\right) _{x_{i}}v_{k}d\mathbf{x}=\int_{\partial
\Omega }u_{k}v_{k}n_{i}dS-\int_{\Omega }u_{k}\left( v_{k}\right) _{x_{i}}d%
\mathbf{x}$. Per $k\rightarrow +\infty $ il lato sinistro converge a $%
\int_{\Omega }u_{x_{i}}vd\mathbf{x}$ per continuit\`{a} del prodotto
scalare, il secondo addendo del lato destro a $-\int_{\Omega }uv_{x_{i}}d%
\mathbf{x}$. Invece, poich\'{e} $\tau _{0}$ \`{e} continuo, se $%
u_{k}\rightarrow ^{H^{1}\left( \Omega \right) }u$ allora $\tau
_{0}u_{k}\rightarrow ^{L^{2}\left( \partial \Omega \right) }\tau _{0}u$; ma $%
\tau _{0}u_{k}=u_{k}|_{\partial \Omega }$, e lo stesso vale per le $v_{k}$.
Quindi, sempre per continuit\`{a} del prodotto scalare su $L^{2}\left(
\partial \Omega \right) $, il primo addendo del lato destro converge a $%
\int_{\partial \Omega }\left( \tau _{0}u\right) \left( \tau _{0}v\right)
n_{i}dS$. Si ha quindi l'uguaglianza della tesi. $\blacksquare $

\begin{enumerate}
\item Vediamo un esempio di funzionale lineare e continuo su $H^{1}\left(
\Omega \right) $ definito con la traccia. $T:H^{1}\left( \Omega \right)
\rightarrow 
%TCIMACRO{\U{211d} }%
%BeginExpansion
\mathbb{R}
%EndExpansion
,T\left( f\right) =\int_{\partial \Omega }\tau _{0}fdS$: \`{e} lineare per
linearit\`{a} di integrale e traccia, e $\left\vert T\left( f\right)
\right\vert \leq ^{}\left\vert \partial \Omega \right\vert ^{\frac{1%
}{2}}\left\vert \left\vert \tau _{0}f\right\vert \right\vert _{L^{2}\left(
\partial \Omega \right) }\leq c\left\vert \left\vert f\right\vert
\right\vert _{H^{1}\left( \Omega \right) }$ (l'ultima disuguaglianza vale
per continuit\`{a} della traccia), quindi effettivamente $T$ \`{e} continuo.
Si nota che $T|_{\mathcal{D}\left( \Omega \right) }$ \`{e} il funzionale
nullo perch\'{e} $\mathcal{D}\left( \Omega \right) \subseteq H_{0}^{1}\left(
\Omega \right) $. Da questo \`{e} evidente che un elemento di $H^{1}\left(
\Omega \right) ^{\ast }$ non \`{e} univocamente determinato dai valori che
assume su $\mathcal{D}\left( \Omega \right) $, a differenza degli elementi
di $H_{0}^{1}\left( \Omega \right) ^{\ast }$.
\end{enumerate}

Ovviamente nel contesto delle EDP si user\`{a} l'operatore di traccia per
assegnare i dati al bordo. Ha quindi senso chiedersi se qualsiasi funzione
in $L^{2}\left( \partial \Omega \right) $ possa essere vista come traccia di
una funzione in $H^{1}\left( \Omega \right) $, cio\`{e} se data $f\in
L^{2}\left( \partial \Omega \right) $ qualsiasi, $\exists $ $u\in
H^{1}\left( \Omega \right) :\tau _{0}u=f$. In generale la risposta \`{e} no:
l'operatore traccia non \`{e} suriettivo, cio\`{e} $\func{Im}\tau
_{0}=\tau _{0}\left( H^{1}\left( \Omega \right) \right) \subsetneq
L^{2}\left( \partial \Omega \right) $. L'immagine dell'operatore $\func{Im}%
\tau _{0}=\left\{ f\in L^{2}\left( \partial \Omega \right) :\exists \text{ }%
u\in H^{1}\left( \Omega \right) :\tau _{0}u=f\right\} $ si indica con $%
H^{1/2}\left( \partial \Omega \right) $. Il significato del simbolo \`{e}
che queste funzioni sono "mezza volta derivabili": un po' meglio che $%
L^{2}\left( \partial \Omega \right) \footnote{%
questa intuizione \`{e} formalizzata dalla teoria degli spazi di Sobolev con
esponente reale}$.

Una $u\in H^{1}\left( \Omega \right) :\tau _{0}u=f$ si dice un rilevamento
di $f$. Quindi se $f\in H^{1/2}\left( \partial \Omega \right) $ vale $%
\left\vert \left\vert f\right\vert \right\vert _{L^{2}\left( \partial \Omega
\right) }\leq c\left\vert \left\vert u\right\vert \right\vert
_{H_{0}^{1}\left( \Omega \right) }$ per ogni $u$ rilevamento di $f$.

Allora si definisce $\left\vert \left\vert f\right\vert \right\vert
_{H^{1/2}\left( \Omega \right) }=\inf \left\{ \left\vert \left\vert
u\right\vert \right\vert _{H_{0}^{1}\left( \Omega \right) }\text{ al variare
di }u\in H_{0}^{1}\left( \Omega \right) :\tau _{0}u=f\right\} $. L'idea \`{e}
considerare tutte le funzioni $u$ che hanno come bordo $f$, calcolarne la
norma e prendere tra queste la norma minima.

\section{Formulazione debole\protect\footnote{\textit{l'ultimo miglio, per
cos\`{\i} dire}}}

\subsection{Complementi su spazi di Hilbert, problemi variazionali astratti,
minimi di funzionali}

\textbf{Def} Dato $V$ spazio prehilbertiano su $%
%TCIMACRO{\U{211d} }%
%BeginExpansion
\mathbb{R}
%EndExpansion
$, si dice forma bilineare su $V$ una funzione $a:V\times V\rightarrow 
%TCIMACRO{\U{211d} }%
%BeginExpansion
\mathbb{R}
%EndExpansion
$ tale che $\forall $ $\lambda ,\mu \in 
%TCIMACRO{\U{211d} }%
%BeginExpansion
\mathbb{R}
%EndExpansion
$ e $\forall $ $u,v,w\in V$ vale $a\left( \lambda u+\mu v,w\right) =\lambda
a\left( u,w\right) +\mu a\left( v,w\right) $ e $a\left( u,\lambda v+\mu
w\right) =\lambda a\left( u,v\right) +\mu a\left( u,w\right) $.

\textbf{Def} Data $a$ forma bilineare su $V$, si dice che $a$ \`{e}:

\begin{description}
\item[a] simmetrica se $a\left( u,v\right) =a\left( v,u\right) $ $\forall $ $%
u,v\in V$

\item[b] continua se $\exists $ $M>0:\left\vert a\left( u,v\right)
\right\vert \leq M\left\vert \left\vert u\right\vert \right\vert \left\vert
\left\vert v\right\vert \right\vert $ $\forall $ $u,v\in V$

\item[c] coerciva se $\exists $ $\lambda >0:a\left( u,u\right) \geq \lambda
\left\vert \left\vert u\right\vert \right\vert ^{2}$ $\forall $ $u\in V$

\item[d] nonnegativa se $a\left( u,u\right) \geq 0$ $\forall $ $u\in V$
\end{description}

a e d sono le definizioni consuete. b ricorda la definizione di continuit%
\`{a} di un funzionale; in questo caso per\`{o}, essendo due gli elementi in
ingresso di $a$, al lato destro compaiono due norme. c significa che $a$ 
\`{e} discosta da $0$ quando $u=v$; si noti che non avrebbe senso richiedere
una disuguaglianza del genere su $a\left( u,v\right) $, perch\'{e} $a$
cambierebbe segno considerando $a\left( u,-v\right) $. Ovviamente c implica
d.

\begin{enumerate}
\item Il prodotto scalare definito su $V$ \`{e} una forma bilineare
simmetrica per definizione, continua con $M=1$ per la disuguaglianza di
Cauchy-Schwarz, coerciva con uguaglianza e $\lambda =1$ per definizione di
norma in uno spazio prehilbertiano.

\item Sia $V=C_{0}^{1}\left( \left[ 0,1\right] \right) $, dotato del
prodotto scalare $\left\langle u,v\right\rangle _{V}=\int_{0}^{1}\left(
uv+u^{\prime }v^{\prime }\right) dx$ ($V$ \`{e} prehilbertiano, ma non di
Hilbert); si definisce $a\left( u,v\right) =\int_{0}^{1}\left( \alpha \left(
x\right) u^{\prime }\left( x\right) v^{\prime }\left( x\right) +\beta \left(
x\right) u^{\prime }\left( x\right) v\left( x\right) +\gamma \left( x\right)
u\left( x\right) v\left( x\right) \right) dx$. Se $\alpha ,\beta ,\gamma \in
L_{loc}^{1}\left( \left[ 0,1\right] \right) $, $a$ \`{e} ben definita. Quali
propriet\`{a} devono avere $\alpha ,\beta ,\gamma $ affinch\'{e} $a$ sia
simmetrica, continua, coerciva, nonnegativa?

Se $\beta =0$ $a$ \`{e} simmetrica. Se $\alpha ,\beta ,\gamma \in L^{\infty
} $, per la disuguaglianza di Hoelder $\left\vert a\left( u,v\right)
\right\vert \leq \left\vert \left\vert \alpha \right\vert \right\vert
_{L^{\infty }}\left\vert \left\vert u^{\prime }\right\vert \right\vert
_{L^{2}}\left\vert \left\vert v^{\prime }\right\vert \right\vert
_{L^{2}}+\left\vert \left\vert \beta \right\vert \right\vert _{L^{\infty
}}\left\vert \left\vert u^{\prime }\right\vert \right\vert
_{L^{2}}\left\vert \left\vert v\right\vert \right\vert _{L^{2}}+\left\vert
\left\vert \gamma \right\vert \right\vert _{L^{\infty }}\left\vert
\left\vert u\right\vert \right\vert _{L^{2}}\left\vert \left\vert
v\right\vert \right\vert _{L^{2}}$. Poich\'{e} $\left\vert \left\vert
u\right\vert \right\vert _{V}^{2}=\left\vert \left\vert u\right\vert
\right\vert _{L^{2}}^{2}+\left\vert \left\vert u^{\prime }\right\vert
\right\vert _{L^{2}}^{2}$, $\left\vert \left\vert u^{\prime }\right\vert
\right\vert _{L^{2}}\leq \left\vert \left\vert u\right\vert \right\vert _{V}$
(lo stesso vale per $v$) e il lato destro \`{e} maggiorato da $\left(
\left\vert \left\vert \alpha \right\vert \right\vert _{L^{\infty
}}+\left\vert \left\vert \beta \right\vert \right\vert _{L^{\infty
}}+\left\vert \left\vert \gamma \right\vert \right\vert _{L^{\infty
}}\right) \left\vert \left\vert u\right\vert \right\vert _{V}\left\vert
\left\vert v\right\vert \right\vert _{V}$, dunque $a$ \`{e} continua.

Poich\'{e} $a\left( u,u\right) =\int_{0}^{1}\left( \alpha \left( x\right)
u^{\prime }\left( x\right) ^{2}+\beta \left( x\right) u^{\prime }\left(
x\right) u\left( x\right) +\gamma \left( x\right) u^{2}\left( x\right)
\right) dx$, vale $a\left( u,u\right) \geq \lambda \left\vert \left\vert
u\right\vert \right\vert _{V}^{2}$ se $\beta =0$ e $\alpha ,\beta \geq
\lambda $: in tal caso $a$ \`{e} coerciva. Se $\beta =0$ e $\alpha ,\beta
\geq 0$, $a$ \`{e} nonnegativa.
\end{enumerate}

\textbf{Def} Dato $H$ spazio di Hilbert e $a$ forma bilineare su $H$, si
dice problema variazionale astratto (PVA) il problema: "dato $T\in H^{\ast }$%
, determinare $u\in H:T\left( v\right) =a\left( u,v\right) $ $\forall $ $%
v\in H$".

Un problema variazionale \`{e} quindi il problema di determinare $u$ tale
che un dato funzionale lineare e continuo su $H$ possa essere rappresentato
come azione di una forma bilineare data, con un ingresso fissato a $u$.

\textbf{Teo (esistenza della soluzione di un PVA)}%
\begin{gather*}
\text{Hp: }H\text{ \`{e} uno spazio di Hilbert, }a\text{ \`{e} una forma
bilineare su }H\text{ simmetrica,} \\
\text{continua con costante }M\text{ e coerciva con costante }\lambda \text{%
; }T\in H^{\ast } \\
\text{Ts: }\exists \text{ }!\text{ }u\in H:T\left( v\right) =a\left(
u,v\right) \text{ }\forall \text{ }v\in H\text{ e }\left\vert \left\vert
u\right\vert \right\vert _{H}\leq \frac{1}{\lambda }\left\vert \left\vert
T\right\vert \right\vert _{H^{\ast }}
\end{gather*}

La disuguaglianza nella tesi pu\`{o} essere interpretata come continuit\`{a}
dell'operatore $S_{a}:H^{\ast }\rightarrow H$ che a ogni funzionale $T$
associa la soluzione del PVA associato a $T$ e $a$.

La dimostrazione \`{e} di fatto un'applicazione del teorema di
rappresentazione di Riesz.

\textbf{Dim} Sia $\left\langle \_,\_\right\rangle _{H}$ il prodotto scalare
di $H$, che induce la norma $\left\vert \left\vert \cdot \right\vert
\right\vert _{H}$. Si definisce il prodotto scalare $\left\langle
u,v\right\rangle _{H_{a}}=a\left( u,v\right) $.

(1) Mostriamo che \`{e} effettivamente un prodotto scalare e che induce su $%
H $ una norma equivalente a $\left\vert \left\vert \cdot \right\vert
\right\vert _{H}$. E' bilineare e commutativo perch\'{e} $a$ \`{e} una forma
bilineare e simmetrica; \`{e} nonnegativo perch\'{e} $a$ \`{e} coerciva: $%
a\left( u,u\right) \geq \lambda \left\vert \left\vert u\right\vert
\right\vert _{H}^{2}\geq 0$; $a\left( u,u\right) =0\Longleftrightarrow
\left\vert \left\vert u\right\vert \right\vert _{H}=0\Longleftrightarrow
u=0_{H}$, per continuit\`{a} e coercivit\`{a}. Inoltre $\left\vert
\left\vert u\right\vert \right\vert _{H_{a}}^{2}\geq \lambda \left\vert
\left\vert u\right\vert \right\vert _{H}^{2}$ e d'altra parte $a\left(
u,u\right) \leq M\left\vert \left\vert u\right\vert \right\vert _{H}^{2}$
per continuit\`{a}: quindi $\lambda \left\vert \left\vert u\right\vert
\right\vert _{H}^{2}\leq \left\vert \left\vert u\right\vert \right\vert
_{H_{a}}^{2}\leq M\left\vert \left\vert u\right\vert \right\vert _{H}^{2}$,
cio\`{e} le norme $H_{a}$ e $H$ sono equivalenti.

(2) Mostriamo che $H$ dotato del prodotto scalare $\left\langle
\_,\_\right\rangle _{H_{a}}$ \`{e} uno spazio di Hilbert. E' noto che se $%
\left\{ u_{k}\right\} \subseteq H$ \`{e} una successione di Cauchy secondo
la norma $H$, allora converge in $H$; mostriamo che se $\left\{
u_{k}\right\} $ \`{e} di Cauchy secondo la norma $H_{a}$, allora converge in 
$H_{a}$. Si suppone quindi che $\left\vert \left\vert u_{k}-u_{h}\right\vert
\right\vert _{H_{a}}\rightarrow ^{k,h+\infty }0$: ma poich\'{e} $\sqrt{%
\lambda }\left\vert \left\vert u_{k}-u_{h}\right\vert \right\vert _{H}\leq
\left\vert \left\vert u_{k}-u_{h}\right\vert \right\vert _{H_{a}}$, allora $%
\left\vert \left\vert u_{k}-u_{h}\right\vert \right\vert _{H}\rightarrow
^{k,h+\infty }0$: poich\'{e} $H$ \`{e} di Hilbert, $\exists $ $u:\left\vert
\left\vert u_{k}-u\right\vert \right\vert _{H}\rightarrow ^{k\rightarrow
+\infty }0$. Ma $\left\vert \left\vert u_{k}-u\right\vert \right\vert
_{H_{a}}\leq \sqrt{M}\left\vert \left\vert u_{k}-u\right\vert \right\vert
_{H}$, quindi vale anche $\left\vert \left\vert u_{k}-u\right\vert
\right\vert _{H_{a}}\rightarrow ^{k\rightarrow +\infty }0$ e $H$ \`{e}
completo con la norma $H_{a}$.

(3) Consideriamo il PVA "dato $T\in H^{\ast }$, determinare $u\in H:T\left(
v\right) =a\left( u,v\right) $ $\forall $ $v\in H$". Mostriamo che $T$ \`{e}
un funzionale lineare e continuo su $H$ anche secondo la norma $H_{a}$. La
linearit\`{a} \`{e} ovvia; $\left\vert T\left( v\right) \right\vert \leq
\left\vert \left\vert T\right\vert \right\vert _{H^{\ast }}\left\vert
\left\vert v\right\vert \right\vert _{H}$, ma $\left\vert \left\vert
v\right\vert \right\vert _{H}\leq \frac{\left\vert \left\vert u\right\vert
\right\vert _{H_{a}}}{\sqrt{\lambda }}$, quindi $\left\vert T\left( v\right)
\right\vert \leq \left\vert \left\vert T\right\vert \right\vert _{H^{\ast }}%
\frac{\left\vert \left\vert u\right\vert \right\vert _{H_{a}}}{\sqrt{\lambda 
}}$, quindi $T$ \`{e} continuo anche secondo $H_{a}$ con $\left\vert
\left\vert T\right\vert \right\vert _{H_{a}^{\ast }}\leq \frac{\left\vert
\left\vert T\right\vert \right\vert _{H^{\ast }}}{\sqrt{\lambda }}$. Allora,
per il teorema di rappresentazione di Riesz, $\exists $ $!$ $u\in H:T\left(
v\right) =\left\langle u,v\right\rangle _{H_{a}}$ $\forall $ $v\in H$ e
inoltre $\left\vert \left\vert u\right\vert \right\vert _{H_{a}}=\left\vert
\left\vert T\right\vert \right\vert _{H_{a}^{\ast }}$. Traducendo tutto in $%
H $, $\exists $ $!$ $u\in H:T\left( v\right) =a\left( u,v\right) $ $\forall $
$v\in H$ e inoltre $\left\vert \left\vert u\right\vert \right\vert _{H}\leq 
\frac{1}{\lambda }\left\vert \left\vert T\right\vert \right\vert _{H^{\ast
}} $. $\blacksquare $

Si osservi che in (2) si \`{e} in realt\`{a} mostrato in generale che se $%
\left( H,\left\langle \_,\_\right\rangle _{1}\right) $ \`{e} uno spazio di
Hilbert e $\left\langle \_,\_\right\rangle _{2}$ \`{e} un prodotto scalare
che induce una norma equivalente a quella indotta da $\left\langle
\_,\_\right\rangle _{1}$, allora anche $\left( H,\left\langle
\_,\_\right\rangle _{2}\right) $ \`{e} uno spazio di Hilbert.

Inoltre le ipotesi di continuit\`{a} e coercivit\`{a} si sono usate in modo 
\textit{quantitativo}, cio\`{e} per scrivere disuguaglianze essenziali alla
dimostrazione, mentre la simmetria si \`{e} usata in modo solo qualitativo,
cio\`{e} per affermare che $a\left( u,v\right) $ \`{e} un prodotto scalare.
Esiste infatti una versione pi\`{u} generale del teorema, che rimuove
quest'ipotesi, enunciata di seguito.

\textbf{Teo (Lax-Milgram)}%
\begin{gather*}
\text{Hp: }H\text{ \`{e} uno spazio di Hilbert, }a\text{ \`{e} una forma
bilineare su }H\text{ } \\
\text{continua con costante }M\text{ e coerciva con costante }\lambda \text{%
; }T\in H^{\ast } \\
\text{Ts: }\exists \text{ }!\text{ }u\in H:T\left( v\right) =a\left(
u,v\right) \text{ }\forall \text{ }v\in H\text{ e }\left\vert \left\vert
u\right\vert \right\vert _{H}\leq \frac{1}{\lambda }\left\vert \left\vert
T\right\vert \right\vert _{H^{\ast }}
\end{gather*}

Noi sopra abbiamo dimostrato il teorema di Lax-Milgram per forme simmetriche.

La risoluzione di un PVA \`{e} strettamente connessa alla risoluzione di
problemi di minimo di funzionali.

Dato il PVA "dato $T\in H^{\ast }$, determinare $u\in H:T\left( v\right)
=a\left( u,v\right) $ $\forall $ $v\in H$", si considera il funzionale $%
J:H\rightarrow 
%TCIMACRO{\U{211d} }%
%BeginExpansion
\mathbb{R}
%EndExpansion
,J\left( u\right) =\frac{1}{2}a\left( u,u\right) -T\left( u\right) $. Esso
ha spesso il significato fisico di energia; e. g. la soluzione del problema
di Dirichlet \`{e} quella che minimizza l'energia...?.

\textbf{Teo (equivalenza tra PVA e minimo di un funzionale)}%
\begin{gather*}
\text{Hp: }H\text{ \`{e} uno spazio di Hilbert, }a\text{ \`{e} una forma
bilineare su }H\text{ simmetrica e nonnegativa; }T\in H^{\ast } \\
\text{Ts: }u\in H\text{ risolve il PVA }T\left( v\right) =a\left( u,v\right) 
\text{ }\forall \text{ }v\in H\text{ }\Longleftrightarrow u=\arg \min_{v\in
H}J\left( v\right)
\end{gather*}

Quindi $u$ risolve il problema variazionale astratto se e solo se \`{e} il
punto di minimo di un opportuno funzionale legato a $a$ e $T$.

\textbf{Dim} $u\in H$ \`{e} tale che $J\left( u\right) \leq J\left( v\right) 
$ $\forall $ $v\in H\Longleftrightarrow \footnote{%
Dalla scrittura seguente, che calcola una "piccola variazione" di $J$, nasce
il nome della disciplina del calcolo delle variazioni.}J\left( u\right) \leq
J\left( u+\varepsilon v\right) $ $\forall $ $v\in H,\forall $ $\varepsilon
\in 
%TCIMACRO{\U{211d} }%
%BeginExpansion
\mathbb{R}
%EndExpansion
$. Sostituendo la definizione di $J$ e sfruttando la bilinearit\`{a} di $a$
si ha $J\left( u\right) \leq J\left( u+\varepsilon v\right)
\Longleftrightarrow 0\leq \frac{1}{2}\varepsilon ^{2}a\left( v,v\right)
+\varepsilon a\left( u,v\right) -\varepsilon T\left( v\right) $: quindi $%
J\left( u\right) \leq J\left( v\right) $ $\forall $ $v\in
H\Longleftrightarrow $ ($\ast $) $\varepsilon \left( a\left( u,v\right)
-T\left( v\right) \right) +\frac{1}{2}\varepsilon ^{2}a\left( v,v\right)
\geq 0$ $\forall $ $v\in H,\forall $ $\varepsilon \in 
%TCIMACRO{\U{211d} }%
%BeginExpansion
\mathbb{R}
%EndExpansion
$.

Mostriamo che l'ultima proposizione scritta \`{e} equivalente a dire che $u$
risolve il PVA. Se $u$ risolve il PVA, allora $T\left( u\right) =a\left(
u,v\right) $ $\forall $ $v\in H$, quindi $\varepsilon \left( a\left(
u,v\right) -T\left( v\right) \right) +\frac{1}{2}\varepsilon ^{2}a\left(
v,v\right) =\frac{1}{2}\varepsilon ^{2}a\left( v,v\right) \geq 0$ $\forall $ 
$v\in H,\forall $ $\varepsilon \in 
%TCIMACRO{\U{211d} }%
%BeginExpansion
\mathbb{R}
%EndExpansion
$, per nonnegativit\`{a} di $a$. Viceversa, se $u$ soddisfa $\ast $ e per
assurdo $\exists $ $v\in H:T\left( v\right) \neq a\left( u,v\right) $,
allora $\exists $ $\varepsilon \in 
%TCIMACRO{\U{211d} }%
%BeginExpansion
\mathbb{R}
%EndExpansion
:\varepsilon \left( a\left( u,v\right) -T\left( v\right) \right) <0$ ($%
a\left( u,v\right) -T\left( v\right) $ \`{e} un numero con segno, basta
prendere $\varepsilon $ discorde), e se $\varepsilon $ \`{e} abbastanza
piccolo si ha $\varepsilon \left( a\left( u,v\right) -T\left( v\right)
\right) +\frac{1}{2}\varepsilon ^{2}a\left( v,v\right) <0$ (perch\'{e} per $%
\varepsilon \rightarrow 0$ il termine quadratico \`{e} trascurabile rispetto
a quello lineare), che \`{e} assurdo. $\blacksquare $

\subsection{Formulazione debole di problemi ai limiti per equazioni
uniformemente ellittiche in forma di divergenza}

\subsubsection{Problema di Dirichlet}

Consideriamo il problema di Dirichlet $\left\{ 
\begin{array}{c}
-\func{div}\left( a\left( \mathbf{x}\right) \nabla u\left( \mathbf{x}\right)
\right) +d\left( \mathbf{x}\right) u\left( \mathbf{x}\right) =f\left( 
\mathbf{x}\right) \text{ in }\Omega \\ 
u=0\text{ su }\partial \Omega%
\end{array}%
\right. $. L'equazione \`{e} quella di diffusione con anche il termine di
reazione, ma senza il termine del prim'ordine. $a$ \`{e} il coefficiente di
conducibilit\`{a}, eventualmente discontinuo grazie alla formulazione debole
che definiremo; in base al significato fisico dovrebbe essere $a,d\geq 0$.
L'obiettivo \`{e} dimostrare, con l'apparato teorico appena presentato, un
risultato di \textit{buona posizione} in senso debole per tale problema: cio%
\`{e} l'esistenza e unicit\`{a} della soluzione debole e la sua dipendenza
continua dai dati.

Ricaviamo come al solito la definizione di soluzione debole: supponiamo $u$
soluzione classica del problema, per cui $\Omega \subseteq 
%TCIMACRO{\U{211d} }%
%BeginExpansion
\mathbb{R}
%EndExpansion
^{n}$ dominio qualsiasi, $u\in C^{2}\left( \Omega \right) ,u\in C^{0}\left( 
\bar{\Omega}\right) ,a\in C^{1}\left( \Omega \right) ,d,f\in C^{0}\left(
\Omega \right) $. Moltiplicando per $\phi \in C_{0}^{1}\left( \Omega \right) 
$ e integrando su $\Omega $ si ottiene $\int_{\Omega }\left( -\func{div}%
\left( a\left( \mathbf{x}\right) \nabla u\left( \mathbf{x}\right) \right)
\phi \left( \mathbf{x}\right) +d\left( \mathbf{x}\right) u\left( \mathbf{x}%
\right) \phi \left( \mathbf{x}\right) \right) d\mathbf{x}=\int_{\Omega
}f\left( \mathbf{x}\right) \phi \left( \mathbf{x}\right) d\mathbf{x}$. Si pu%
\`{o} pensare in realt\`{a} di star integrando su un qualsiasi dominio
limitato lipschitziano contenente $\Omega $ (dato che $\phi $ ha supporto
contenuto in $\Omega $), per cui si pu\`{o} applicare la formula di
integrazione per parti generalizzata, basata sul teorema della divergenza,
senza ulteriori ipotesi su $\Omega $: si ottiene $\int_{\Omega }\left(
\left\langle a\nabla u,\nabla \phi \right\rangle +du\phi \right) d\mathbf{x}%
=\int_{\Omega }f\phi d\mathbf{x}$ $\forall $ $\phi \in C_{0}^{1}\left(
\Omega \right) $ (il termine di bordo \`{e} nullo perch\'{e} $\phi \in
C_{0}^{1}\left( \Omega \right) $). Quali sono le ipotesi minime per cui
entrambi gli integrali sono ben definiti? Poich\'{e} $\nabla \phi \in
L^{2}\left( \Omega \right) $, \`{e} sufficiente che $u\in H^{1}\left( \Omega
\right) ,a,d\in L^{\infty }\left( \Omega \right) ,f\in L^{2}\left( \Omega
\right) $, e in tal caso l'uguaglianza scritta vale $\phi \in
H_{0}^{1}\left( \Omega \right) $ (data la densit\`{a} di $C_{0}^{1}\left(
\Omega \right) $ in $H_{0}^{1}\left( \Omega \right) $). Si noti che non si 
\`{e} ancora imposta la condizione al bordo.

\textbf{Def} Dati $\Omega \subseteq 
%TCIMACRO{\U{211d} }%
%BeginExpansion
\mathbb{R}
%EndExpansion
^{n}$, $a,d\in L^{\infty }\left( \Omega \right) ,f\in L^{2}\left( \Omega
\right) $, si dice che $u\in H_{0}^{1}\left( \Omega \right) $ \`{e}
soluzione debole del problema $\left\{ 
\begin{array}{c}
-\func{div}\left( a\left( \mathbf{x}\right) \nabla u\left( \mathbf{x}\right)
\right) +d\left( \mathbf{x}\right) u\left( \mathbf{x}\right) =f\left( 
\mathbf{x}\right) \text{ in }\Omega \\ 
u=0\text{ su }\partial \Omega%
\end{array}%
\right. $ se vale $\int_{\Omega }\left( \left\langle a\nabla u,\nabla \phi
\right\rangle +du\phi \right) d\mathbf{x}=\int_{\Omega }f\phi d\mathbf{x}$ $%
\forall $ $\phi \in H_{0}^{1}\left( \Omega \right) $.

Ovviamente dal punto di vista matematico non \`{e} necessaria alcuna
richieste su $a,d$, sebbene non abbiano significato fisico qualora negativi.

I calcoli fatti sopra mostrano che se $u\in C^{2}\left( \Omega \right) ,a\in
C^{1}\left( \Omega \right) ,d,f\in C^{0}\left( \Omega \right) $ e $u$ \`{e}
soluzione classica del problema, allora $u$ \`{e} soluzione debole \textit{%
dell'equazione}. Per concludere che $u\in H_{0}^{1}\left( \Omega \right) $
occorre richiedere $\Omega $ dominio limitato lipschitziano e $u\in
C^{1}\left( \bar{\Omega}\right) $, che sono le ipotesi per applicare il
teorema di esistenza della traccia. In tal caso infatti, poich\'{e} $u=0$ su 
$\partial \Omega $ (nel senso classico), si ha $\tau _{0}u=u|_{\partial
\Omega }=0$, e $\ker \tau _{0}=H_{0}^{1}\left( \Omega \right) $. Allora
abbiamo dimostrato

\textbf{Teo (soluzione classica implica debole)}%
\begin{gather*}
\text{Hp}\text{: }\Omega \text{ \`{e} un dominio limitato lipschitziano, }%
u\in C^{2}\left( \Omega \right) \cap C^{1}\left( \bar{\Omega}\right) , \\
a\in C^{1}\left( \Omega \right) ,d,f\in C^{0}\left( \Omega \right) \text{, }u%
\text{ risolve il problema in senso classico} \\
\text{Ts}\text{: }u\text{ \`{e} soluzione debole del problema}
\end{gather*}

Viceversa, se $u$ \`{e} soluzione debole e valgono tutte le ipotesi del
teorema sopra, $u$ \`{e} soluzione classica. Infatti, sapendo che $u\in
H_{0}^{1}\left( \Omega \right) $, si ha $\tau _{0}u=0=u|_{\partial \Omega }$%
, quindi $u=0$ su $\partial \Omega $ in senso classico. Si sa inoltre che $%
\int_{\Omega }\left( \left\langle a\nabla u,\nabla \phi \right\rangle
+du\phi \right) d\mathbf{x}=\int_{\Omega }f\phi d\mathbf{x}$ $\forall $ $%
\phi \in H_{0}^{1}\left( \Omega \right) $; allora, considerando in
particolare $\phi \in C_{0}^{1}\left( \Omega \right) $ e integrando per
parti con il teorema della divergenza, si trova $\int_{\Omega }\left( -\func{%
div}\left( a\nabla u\right) \phi +du\phi \right) d\mathbf{x}=\int_{\Omega
}f\phi d\mathbf{x}$ $\forall $ $\phi \in C_{0}^{1}\left( \Omega \right) $,
per cui l'integranda \`{e} nulla e $u$ risolve l'equazione in senso classico.

Per ottenere dei risultati di esistenza della soluzione debole vogliamo
usare il teorema di Lax-Milgram, e dobbiamo dunque riformulare il problema
di trovare $u\in H_{0}^{1}\left( \Omega \right) :\int_{\Omega }\left(
\left\langle a\nabla u,\nabla \phi \right\rangle +du\phi \right) d\mathbf{x}%
=\int_{\Omega }f\phi d\mathbf{x}$ $\forall $ $\phi \in H_{0}^{1}\left(
\Omega \right) $ come PVA. Ci\`{o} \`{e} molto semplice: dato lo spazio di
Hilbert $H=H_{0}^{1}\left( \Omega \right) $, si definisce su $H$ la forma
bilineare $a\left( u,v\right) =\int_{\Omega }\left( \left\langle a\nabla
u,\nabla v\right\rangle +duv\right) d\mathbf{x}$. Si definisce inoltre, data $f\in
L^{2}\left( \Omega \right) $, il funzionale $T:H_{0}^{1}\left( \Omega
\right) \rightarrow 
%TCIMACRO{\U{211d} }%
%BeginExpansion
\mathbb{R}
%EndExpansion
$, $T\left( v\right) =\int_{\Omega }fvd\mathbf{x}$: la linearit\`{a} \`{e}
ovvia, e $T$ \`{e} un funzionale continuo perch\'{e} (si ricordi quanto
detto sulla rappresentazione degli elementi di $H_{0}^{1}\left( \Omega
\right) ^{\ast }$), ristretto a $\mathcal{D}\left( \Omega \right) $, \`{e}
una distribuzione associata a una funzione $L^{2}$; volendo comunque
verificare direttamente la continuit\`{a} si ha $\left\vert T\left( v\right)
\right\vert \leq \left\vert \left\vert v\right\vert \right\vert
_{L^{2}\left( \Omega \right) }\left\vert \left\vert f\right\vert \right\vert
_{L^{2}\left( \Omega \right) }\leq \left\vert \left\vert v\right\vert
\right\vert _{H_{0}^{1}\left( \Omega \right) }\left\vert \left\vert
f\right\vert \right\vert _{L^{2}\left( \Omega \right) }$, per cui $%
\left\vert \left\vert T\right\vert \right\vert _{H^{\ast }}\leq \left\vert
\left\vert f\right\vert \right\vert _{L^{2}\left( \Omega \right) }$. Dunque $%
T\in H_{0}^{1}\left( \Omega \right) ^{\ast }$.

Il PVA \`{e} dunque trovare $u\in H_{0}^{1}\left( \Omega \right) :a\left(
u,v\right) =T\left( v\right) $ $\forall $ $v\in H_{0}^{1}\left( \Omega
\right) $. Per applicare Lax-Milgram (o la sua versione pi\`{u} debole)
occorre verificare sotto quali ipotesi la forma bilineare $a$ \`{e}
simmetrica, continua e coerciva. Da tali ipotesi seguir\`{a} la buona
posizione del problema.

$a$ \`{e} ovviamente simmetrica. $\left\vert a\left( u,v\right) \right\vert
=\left\vert \int_{\Omega }\left( \left\langle a\nabla u,\nabla
v\right\rangle +duv\right) d\mathbf{x}\right\vert \leq $ $\left\vert
\left\vert a\right\vert \right\vert _{L^{\infty }}\left\vert \left\vert
\nabla u\right\vert \right\vert _{L^{2}}\left\vert \left\vert \nabla
v\right\vert \right\vert _{L^{2}}+\left\vert \left\vert d\right\vert
\right\vert _{L^{\infty }}\left\vert \left\vert u\right\vert \right\vert
_{L^{2}}\left\vert \left\vert v\right\vert \right\vert _{L^{2}}\leq \left(
\left\vert \left\vert a\right\vert \right\vert _{L^{\infty }}+\left\vert
\left\vert d\right\vert \right\vert _{L^{\infty }}\right) \left\vert
\left\vert u\right\vert \right\vert _{H^{1}}\left\vert \left\vert
v\right\vert \right\vert _{H^{1}}$: quindi $a$ \`{e} continua se i
coefficienti $a,d\in L^{\infty }$.

$a\left( u,u\right) =\int_{\Omega }\left( a\left\vert \left\vert \nabla
u\right\vert \right\vert ^{2}+du^{2}\right) d\mathbf{x}$: se si fanno le
ipotesi di uniforme ellitticit\`{a} $a\left( \mathbf{x}\right) \geq \lambda
>0$ $\forall $ $\mathbf{x}\in \Omega $ e $d\geq 0$, si ha $a\left(
u,u\right) \geq \lambda \left\vert \left\vert \nabla u\right\vert
\right\vert _{L^{2}}^{2}$. Per la disuguaglianza di Poincar\'{e}, se $\Omega 
$ \`{e} un dominio limitato allora $\left\vert \left\vert u\right\vert
\right\vert _{L^{2}}\leq c_{\Omega }\left\vert \left\vert \nabla
u\right\vert \right\vert _{L^{2}}$: allora $\left\vert \left\vert
u\right\vert \right\vert _{H^{1}\left( \Omega \right) }^{2}\leq \left(
1+c_{\Omega }^{2}\right) \left\vert \left\vert \nabla u\right\vert
\right\vert _{L^{2}\left( \Omega \right) }^{2}$ e quindi $a\left( u,u\right)
\geq \frac{\lambda }{1+c_{\Omega }^{2}}\left\vert \left\vert u\right\vert
\right\vert _{H^{1}\left( \Omega \right) }^{2}$. Perci\`{o} $a$ \`{e}
coerciva.

Sotto tutte queste ipotesi, per il teorema di buona posizione si ha che $%
\exists $ $!$ $u\in H_{0}^{1}\left( \Omega \right) :a\left( u,\phi \right)
=T\left( \phi \right) $ $\forall $ $\phi \in H_{0}^{1}\left( \Omega \right) $%
, e $\left\vert \left\vert u\right\vert \right\vert _{H^{1}\left( \Omega
\right) }\leq \frac{c_{\Omega }^{2}+1}{\lambda }\left\vert \left\vert
f\right\vert \right\vert _{L^{2}\left( \Omega \right) }$. Abbiamo dimostrato
il seguente

\textbf{Teo (buona posizione del problema di Dirichlet)}%
\begin{gather*}
\text{Hp: }\Omega \subseteq 
%TCIMACRO{\U{211d} }%
%BeginExpansion
\mathbb{R}
%EndExpansion
^{n}\text{ \`{e} un dominio limitato, }a,d\in L^{\infty }\left( \Omega
\right) \text{, }\lambda >0:a\left( \mathbf{x}\right) \geq \lambda \text{ }%
\forall \text{ }\mathbf{x},d\geq 0,f\in L^{2}\left( \Omega \right) \\
\text{Ts: }\exists \text{ }!\text{ soluzione debole di }\left\{ 
\begin{array}{c}
-\func{div}\left( a\left( \mathbf{x}\right) \nabla u\left( \mathbf{x}\right)
\right) +d\left( \mathbf{x}\right) u\left( \mathbf{x}\right) =f\left( 
\mathbf{x}\right) \text{ in }\Omega \\ 
u=0\text{ su }\partial \Omega%
\end{array}%
\right. \text{ e }\left\vert \left\vert u\right\vert \right\vert
_{H^{1}\left( \Omega \right) }\leq \frac{c_{\Omega }^{2}+1}{\lambda }%
\left\vert \left\vert f\right\vert \right\vert _{L^{2}\left( \Omega \right) }
\end{gather*}

Vale quindi esistenza e unicit\`{a} della soluzione, che dipende in maniera
continua dal termine noto $f$. La disuguaglianza perde di significato se $%
\Omega $ \`{e} illimitato.

La costante $\frac{c_{\Omega }^{2}+1}{\lambda }$ \`{e} una traduzione
quantitativa dell'ipotesi di uniforme ellitticit\`{a} e della disuguaglianza
di Poincar\'{e}.

\textbf{Corollario}%
\begin{gather*}
\text{Hp: le medesime del teorema sopra} \\
\text{Ts: la soluzione debole del problema \`{e} punto di minimo in }%
H_{0}^{1}\left( \Omega \right) \text{ di }J\left( u\right) =\frac{1}{2}%
a\left( u,u\right) -T\left( u\right)
\end{gather*}

dove $a$ e $T$ sono definite come sopra. Pi\`{u} esplicitamente, $J\left(
u\right) =\frac{1}{2}\int_{\Omega }\left( a\left\vert \left\vert \nabla
u\right\vert \right\vert ^{2}+du^{2}\right) d\mathbf{x}$ $-\int_{\Omega }fud%
\mathbf{x}$.

\textbf{Dim} Poich\'{e} sotto le ipotesi date $a$ \`{e} coerciva e quindi
nonnegativa, la tesi segue immediatamente dal teorema visto sull'equivalenza
tra soluzioni di PVA\ e punti di minimo di funzionali. $\blacksquare $

\subsubsection{Problema di Neumann}

Vediamo ora, per la stessa equazione, il problema di Neumann con condizione
al bordo omogenea: 
\begin{equation*}
\left\{ 
\begin{array}{c}
-\func{div}\left( a\left( \mathbf{x}\right) \nabla u\left( \mathbf{x}\right)
\right) +d\left( \mathbf{x}\right) u\left( \mathbf{x}\right) =f\left( 
\mathbf{x}\right) \text{ in }\Omega \\ 
\frac{\partial u}{\partial n}=0\text{ su }\partial \Omega%
\end{array}%
\right. 
\end{equation*}
L'obiettivo \`{e} di nuovo dimostrare un risultato di buona
posizione in senso debole per tale problema.

Ricaviamo come al solito la definizione di soluzione debole: supponiamo $u$
soluzione classica del problema, per cui $\Omega \subseteq 
%TCIMACRO{\U{211d} }%
%BeginExpansion
\mathbb{R}
%EndExpansion
^{n}$ dominio qualsiasi $u\in C^{2}\left( \Omega \right) ,u\in C^{1}\left( 
\bar{\Omega}\right) $ (la condizione al bordo deve avere significato in
senso classico!), $a\in C^{1}\left( \Omega \right) ,d,f\in C^{0}\left(
\Omega \right) $. Moltiplicando per $\phi \in C^{1}\left( \bar{\Omega}%
\right) $ e integrando su $%
\Omega $ si ottiene $\int_{\Omega }\left( -\func{div}\left( a\nabla u\right)
\phi +du\phi \right) d\mathbf{x}=\int_{\Omega }f\phi d\mathbf{x}$. Stavolta,
non essendo $\phi $ a supporto compatto, non si pu\`{o} evitare l'ipotesi di 
$\Omega $ dominio limitato lipschitziano per applicare il teorema della
divergenza: in tal caso si ottiene $\int_{\Omega }\left( \left\langle
a\nabla u,\nabla \phi \right\rangle +du\phi \right) d\mathbf{x-}%
\int_{\partial \Omega }\left\langle a\nabla u,n\right\rangle \phi
dS=\int_{\Omega }f\phi d\mathbf{x}$; il termine di bordo \`{e} nullo perch%
\'{e} $\frac{\partial u}{\partial n}=\left\langle \nabla u,n\right\rangle =0$
su $\partial \Omega $).

Si ha quindi $\int_{\Omega }\left( \left\langle a\nabla u,\nabla \phi
\right\rangle +du\phi \right) d\mathbf{x}=\int_{\Omega }f\phi d\mathbf{x}$ $%
\forall $ $\phi \in C^{1}\left( \bar{\Omega}\right) $. Quali sono le ipotesi
minime per cui entrambi gli integrali sono ben definiti? E' sufficiente che $%
u\in H^{1}\left( \Omega \right) ,a,d\in L^{\infty }\left( \Omega \right)
,f\in L^{2}\left( \Omega \right) $, e in tal caso l'uguaglianza vale non
solo $\forall $ $\phi \in C^{1}\left( \bar{\Omega}\right) $, ma $\forall $ $%
\phi \in H^{1}\left( \Omega \right) $, data la densit\`{a} di $C^{1}\left( 
\bar{\Omega}\right) $ in $H^{1}\left( \Omega \right) $ (per il terzo teorema
di approssimazione).

\textbf{Def} Dati $\Omega \subseteq 
%TCIMACRO{\U{211d} }%
%BeginExpansion
\mathbb{R}
%EndExpansion
^{n}$ dominio, $a,d\in L^{\infty }\left( \Omega \right) ,f\in L^{2}\left(
\Omega \right) $, si dice che $u\in H^{1}\left( \Omega \right) $ \`{e}
soluzione debole del problema $\left\{ 
\begin{array}{c}
-\func{div}\left( a\left( \mathbf{x}\right) \nabla u\left( \mathbf{x}\right)
\right) +d\left( \mathbf{x}\right) u\left( \mathbf{x}\right) =f\left( 
\mathbf{x}\right) \text{ in }\Omega \\ 
\frac{\partial u}{\partial n}=0\text{ su }\partial \Omega%
\end{array}%
\right. $ se vale $\int_{\Omega }\left( \left\langle a\nabla u,\nabla \phi
\right\rangle +du\phi \right) d\mathbf{x}=\int_{\Omega }f\phi d\mathbf{x}$ $%
\forall $ $\phi \in H^{1}\left( \Omega \right) $.

I calcoli fatti sopra mostrano che se $\Omega $ \`{e} un dominio limitato e
lipschitziano, $u\in C^{2}\left( \Omega \right) \cap C^{1}\left( \bar{\Omega}%
\right) ,a\in C^{1}\left( \Omega \right) ,d,f\in C^{0}\left( \Omega \right) $
e $u$ \`{e} soluzione classica del problema, allora $u$ \`{e} soluzione
debole del problema.

Viceversa:

\textbf{Teo (una soluzione debole con ingredienti regolari \`{e} classica)}%
\begin{gather*}
\text{Hp}\text{:}\text{ }\Omega \text{ \`{e} un dominio limitato e
lipschitziano, }u\in C^{2}\left( \Omega \right) \cap C^{1}\left( \bar{\Omega}%
\right) \\
\text{ }u\in H^{1}\left( \Omega \right) ,a\in C^{1}\left( \Omega \right)
,d,f\in C^{0}\left( \Omega \right) \text{, }u\text{ \`{e} soluzione debole
del problema} \\
\text{Ts}\text{: }u\text{ \`{e} soluzione classica del problema}
\end{gather*}

\textbf{Dim} Applicando la definizione di soluzione debole in particolare
alle $\phi \in C_{0}^{1}\left( \Omega \right) $ e poi il teorema di
divergenza all'indietro si trova $\int_{\Omega }\left( -\func{div}\left(
a\nabla u\right) \phi +du\phi \right) d\mathbf{x}=\int_{\Omega }f\phi d%
\mathbf{x}$ $\forall $ $\phi \in C_{0}^{1}\left( \Omega \right) $, da cui $-%
\func{div}\left( a\nabla u\right) +du-f=0$ punto per punto in $\Omega $:
quindi $u$ risolve l'equazione in senso classico.

Applicando la definizione di soluzione debole in particolare alle $\phi \in
C^{1}\left( \bar{\Omega}\right) $ (in modo che ci sia anche il termine di
bordo) e poi il teorema di divergenza all'indietro si trova $\int_{\Omega
}\left( -\func{div}\left( a\nabla u\right) +du-f\right) \phi d\mathbf{x}%
+\int_{\partial \Omega }\frac{\partial u}{\partial n}\phi d\mathbf{x}=0$ $%
\forall $ $\phi \in C^{1}\left( \bar{\Omega}\right) $: il primo addendo \`{e}
nullo per il punto precedente della dimostrazione, quindi $\frac{\partial u}{%
\partial n}=0$ in $\partial \Omega $, e $u$ soddisfa la condizione al bordo
in senso classico. $\blacksquare $

Per ottenere dei risultati di esistenza della soluzione debole vogliamo
usare il teorema di Lax-Milgram, e dobbiamo dunque riformulare il problema
di trovare $u\in H^{1}\left( \Omega \right) :\int_{\Omega }\left(
\left\langle a\nabla u,\nabla \phi \right\rangle +du\phi \right) d\mathbf{x}%
=\int_{\Omega }f\phi d\mathbf{x}$ $\forall $ $\phi \in H^{1}\left( \Omega
\right) $ come PVA. Dato lo spazio di Hilbert $H=H^{1}\left( \Omega \right) $%
, si definisce su $H$ la forma bilineare $a\left( u,v\right) =\int_{\Omega
}\left( \left\langle a\nabla u,\nabla v\right\rangle +duv\right) d\mathbf{x}$%
, con $a,d\in L^{\infty }$. Si definisce inoltre, data $f\in L^{2}\left(
\Omega \right) $, il funzionale $T:H^{1}\left( \Omega \right) \rightarrow 
%TCIMACRO{\U{211d} }%
%BeginExpansion
\mathbb{R}
%EndExpansion
$, $T\left( v\right) =\int_{\Omega }fvd\mathbf{x}$: la linearit\`{a} \`{e}
ovvia, e $T$ \`{e} un funzionale continuo perch\'{e} come sopra $\left\vert
T\left( \phi \right) \right\vert \leq \left\vert \left\vert \phi \right\vert
\right\vert _{H^{1}\left( \Omega \right) }\left\vert \left\vert f\right\vert
\right\vert _{L^{2}\left( \Omega \right) }$, per cui $\left\vert \left\vert
T\right\vert \right\vert _{H^{\ast }}\leq \left\vert \left\vert f\right\vert
\right\vert _{L^{2}\left( \Omega \right) }$.

Il PVA \`{e} dunque trovare $u\in H^{1}\left( \Omega \right) :a\left(
u,v\right) =T\left( v\right) $ $\forall $ $v\in H^{1}\left( \Omega \right) $%
. Per applicare Lax-Milgram (o la sua versione pi\`{u} debole) occorre
verificare sotto quali ipotesi la forma bilineare $a$ \`{e} simmetrica,
continua e coerciva. $a$ \`{e} simmetrica e continua se $a,d\in L^{\infty }$%
, grazie agli stessi passaggi gi\`{a} visti.

$a\left( u,u\right) =\int_{\Omega }\left( a\left\vert \left\vert \nabla
u\right\vert \right\vert ^{2}+du^{2}\right) d\mathbf{x}$: se anche si
suppone $a\left( \mathbf{x}\right) \geq \lambda >0$ $\forall $ $\mathbf{x}%
\in \Omega $, si ottiene $a\left( u,u\right) \geq \lambda \left\vert
\left\vert \nabla u\right\vert \right\vert _{L^{2}\left( \Omega \right)
}^{2}+d\int_{\Omega }u^{2}d\mathbf{x}$, ma non si pu\`{o} applicare la
disuguaglianza di Poincar\'{e} perch\'{e} non si lavora in $H_{0}^{1}\left(
\Omega \right) $. Occorre richiedere anche $d\left( \mathbf{x}\right) \geq
c_{0}>0$ $\forall $ $\mathbf{x}\in \Omega $: in tal caso $a\left( u,u\right)
\geq \lambda \left\vert \left\vert \nabla u\right\vert \right\vert
_{L^{2}\left( \Omega \right) }^{2}+c_{0}\left\vert \left\vert u\right\vert
\right\vert _{L^{2}}^{2}\geq \min \left\{ \lambda ,c_{0}\right\} \left\vert
\left\vert u\right\vert \right\vert _{H^{1}\left( \Omega \right) }^{2}$.
Abbiamo dimostrato il

\textbf{Teo (buona posizione del problema di Neumann)}%
\begin{gather*}
\text{Hp: }\Omega \subseteq 
%TCIMACRO{\U{211d} }%
%BeginExpansion
\mathbb{R}
%EndExpansion
^{n}\text{ \`{e} un dominio, }a,d\in L^{\infty }\left( \Omega \right) \text{%
, }\lambda ,c_{0}>0:a\left( \mathbf{x}\right) \geq \lambda ,d\left( \mathbf{x%
}\right) \geq c_{0}>0\text{ }\forall \text{ }\mathbf{x},f\in L^{2}\left(
\Omega \right) \\
\text{Ts: }\exists \text{ }!\text{ soluzione debole di }\left\{ 
\begin{array}{c}
-\func{div}\left( a\left( \mathbf{x}\right) \nabla u\left( \mathbf{x}\right)
\right) +d\left( \mathbf{x}\right) u\left( \mathbf{x}\right) =f\left( 
\mathbf{x}\right) \text{ in }\Omega \\ 
\frac{\partial u}{\partial n}=0\text{ su }\partial \Omega%
\end{array}%
\right. \text{ e }\left\vert \left\vert u\right\vert \right\vert
_{H^{1}\left( \Omega \right) }\leq \frac{1}{\min \left\{ \lambda
,c_{0}\right\} }\left\vert \left\vert f\right\vert \right\vert _{L^{2}\left(
\Omega \right) }
\end{gather*}

Si noti che logicamente non \`{e} pi\`{u} richiesto che $\Omega $ sia un
dominio limitato (anche se da un punto di vista sostanziale \`{e} comunque
un'ipotesi da fare, insieme alla lipschitzianit\`{a}).

\textbf{Corollario}%
\begin{gather*}
\text{Hp: le medesime del teorema sopra} \\
\text{Ts: la soluzione debole del problema \`{e} punto di minimo in }%
H^{1}\left( \Omega \right) \text{ di }J\left( u\right) =\frac{1}{2}a\left(
u,u\right) -T\left( u\right)
\end{gather*}

dove $a$ e $T$ sono definite come sopra. Pi\`{u} esplicitamente, $J\left(
u\right) =\frac{1}{2}\int_{\Omega }\left( a\left\vert \left\vert \nabla
u\right\vert \right\vert ^{2}+du^{2}\right) d\mathbf{x}$ $-\int_{\Omega }fud%
\mathbf{x}$.

\textbf{Dim} Poich\'{e} sotto le ipotesi date $a$ \`{e} coerciva e quindi
nonnegativa, la tesi segue immediatamente dal teorema visto sull'equivalenza
tra soluzioni di PVA\ e punti di minimo di funzionali. $\blacksquare $

Se $\Omega $ \`{e} un dominio limitato e lipschitziano valgono le relazioni
gi\`{a} viste: $u$ soluzione classica e $a,d,f$ regolari $\Longrightarrow $ $%
u$ soluzione debole, $u$ soluzione debole e $a,d,f$ regolari $%
\Longrightarrow u$ soluzione classica. 

Si noti che se $d=0$ il teorema di buona posizione non vale: viene a mancare
l'ipotesi $d\geq c_{0}>0$. Infatti in tal caso il problema diventa $\left\{ 
\begin{array}{c}
-\func{div}\left( a\nabla u\right) =f\text{ in }\Omega \\ 
\frac{\partial u}{\partial n}=0\text{ in }\partial \Omega%
\end{array}%
\right. $: non pu\`{o} valere l'unicit\`{a}, perch\'{e} se $u$ \`{e}
soluzione classica anche $u+k$ lo \`{e}, e se $u$ \`{e} soluzione debole
anche $u+k$ lo \`{e}. Inoltre, se $u$ \`{e} soluzione debole, $u$ \`{e} tale
che $\int_{\Omega }\left\langle a\nabla u,\nabla \phi \right\rangle d\mathbf{%
x}=\int_{\Omega }f\phi d\mathbf{x}$ $\forall $ $\phi \in H^{1}\left( \Omega
\right) $: se in particolare $\phi =1$, si
ottiene $0=\int_{\Omega }fd\mathbf{x}$, cio\`{e} c'\`{e} una condizione
necessaria che il termine noto deve soddisfare per poter parlare di
soluzioni deboli. Quindi sicuramente non si pu\`{o} neanche pi\`{u}
affermare l'esistenza di una soluzione debole "$\forall $ $f\in L^{2}$".

Con una teoria pi\`{u} raffinata, che fa uso dell'alternativa di Fredholm,
si pu\`{o} dimostrare che, se $d=0$ e $a\geq \lambda >0$ e $f\in L^{2}$
soddisfa la condizione necessaria, allora il problema ha almeno una
soluzione, che \`{e} unica a meno di una costante additiva.

\subsubsection{Equazione ellittica pi\`{u} generale}

\paragraph{Problema di Dirichlet}

Adesso si occupiamo del problema di Dirichlet per un'equazione pi\`{u}
generale, con condizione al bordo non omogenea: $\left\{ 
\begin{array}{c}
-\func{div}\left( A\left( \mathbf{x}\right) \nabla u\left( \mathbf{x}\right)
\right) +\left\langle \mathbf{b}\left( \mathbf{x}\right) ,\nabla u\left( 
\mathbf{x}\right) \right\rangle +d\left( \mathbf{x}\right) u\left( \mathbf{x}%
\right) =f\left( \mathbf{x}\right) \text{ in }\Omega \\ 
u=g\text{ su }\partial \Omega%
\end{array}%
\right. $. L'equazione descrive la diffusione (con trasporto e reazione) in
un mezzo che non solo non \`{e} omogeneo, ma neanche isotropo: $A:\Omega
\rightarrow M_{%
%TCIMACRO{\U{211d} }%
%BeginExpansion
\mathbb{R}
%EndExpansion
}\left( n,n\right) $ \`{e} simmetrica. Esplicitamente
l'equazione \`{e} $-\sum_{i,j=1}^{n}\left( a_{ij}u_{x_{i}}\right)
_{x_{j}}+\sum_{i=1}^{n}b_{i}u_{x_{i}}+du=f$.

Iniziamo da $g=0$.

Ricaviamo come al solito la definizione di soluzione debole: supponiamo $u$
soluzione classica del problema, per cui $\Omega \subseteq 
%TCIMACRO{\U{211d} }%
%BeginExpansion
\mathbb{R}
%EndExpansion
^{n}$ dominio qualsiasi $u\in C^{2}\left( \Omega \right) ,a_{ij}\in
C^{1}\left( \Omega \right) ,b_{i},d,f\in C^{0}\left( \Omega \right) $.
Moltiplicando per $\phi \in C_{0}^{1}\left( \Omega \right) $ e integrando su 
$\Omega $ si ottiene $\int_{\Omega }\left( -\func{div}\left( A\left( \mathbf{%
x}\right) \nabla u\left( \mathbf{x}\right) \right) \phi \left( \mathbf{x}%
\right) +\left\langle b\left( \mathbf{x}\right) ,\nabla u\left( \mathbf{x}%
\right) \right\rangle \phi \left( \mathbf{x}\right) +d\left( \mathbf{x}%
\right) u\left( \mathbf{x}\right) \phi \left( \mathbf{x}\right) \right) d%
\mathbf{x}=\int_{\Omega }f\left( \mathbf{x}\right) \phi \left( \mathbf{x}%
\right) d\mathbf{x}$. Si suppone in realt\`{a} di star integrando su un
dominio limitato lipschitziano contenente supp$\left( \phi \right) $, per
cui si pu\`{o} applicare la formula di integrazione per parti generalizzata,
basata sul teorema della divergenza, senza ulteriori ipotesi su $\Omega $:
si ottiene $\int_{\Omega }\left( \left\langle A\nabla u,\nabla \phi
\right\rangle +\left\langle b,\nabla u\right\rangle \phi +du\phi \right) d%
\mathbf{x}=\int_{\Omega }f\phi d\mathbf{x}$ (il termine di bordo \`{e} nullo
perch\'{e} $\phi \in C_{0}^{1}\left( \Omega \right) $), cio\`{e} $%
\int_{\Omega }\left( \sum_{i,j=1}^{n}a_{ij}u_{x_{i}}\phi
_{x_{j}}+\sum_{i=1}^{n}b_{i}u_{x_{i}}\phi +du\phi \right) d\mathbf{x}%
=\int_{\Omega }f\phi d\mathbf{x}$ $\forall $ $\phi \in C_{0}^{1}\left(
\Omega \right) $. Quali sono le ipotesi minime per cui entrambi gli
integrali sono ben definiti? E' sufficiente che $u\in H^{1}\left( \Omega
\right) ,a_{ij},d,b_{i}\in L^{\infty }\left( \Omega \right) ,f\in
L^{2}\left( \Omega \right) $, e in tal caso l'uguaglianza vale anche $%
\forall $ $\phi \in H_{0}^{1}\left( \Omega \right) $ (data la densit\`{a} di 
$C_{0}^{1}\left( \Omega \right) $ in $H_{0}^{1}\left( \Omega \right) $). Si
noti che non si \`{e} ancora imposta la condizione al bordo: si chiede $u\in
H_{0}^{1}\left( \Omega \right) $.

\textbf{Def} Dati $\Omega \subseteq 
%TCIMACRO{\U{211d} }%
%BeginExpansion
\mathbb{R}
%EndExpansion
^{n}$ dominio, $a_{ij},b_{i},d\in L^{\infty }\left( \Omega \right) ,f\in
L^{2}\left( \Omega \right) $, si dice che $u\in H_{0}^{1}\left( \Omega
\right) $ \`{e} soluzione debole del problema $\left\{ 
\begin{array}{c}
-\func{div}\left( A\left( \mathbf{x}\right) \nabla u\left( \mathbf{x}\right)
\right) +\left\langle b\left( \mathbf{x}\right) ,\nabla u\left( \mathbf{x}%
\right) \right\rangle +d\left( \mathbf{x}\right) u\left( \mathbf{x}\right)
=f\left( \mathbf{x}\right) \text{ in }\Omega \\ 
u=0\text{ su }\partial \Omega%
\end{array}%
\right. $ se vale $\int_{\Omega }\left( \sum_{i,j=1}^{n}a_{ij}u_{x_{i}}\phi
_{x_{j}}+\sum_{i=1}^{n}b_{i}u_{x_{i}}\phi +du\phi \right) d\mathbf{x}%
=\int_{\Omega }f\phi d\mathbf{x}$ $\forall $ $\phi \in H_{0}^{1}\left(
\Omega \right) $.

Per ottenere dei risultati di esistenza della soluzione debole vogliamo
usare il teorema di Lax-Milgram, e dobbiamo dunque riformulare il problema
come PVA. Dato lo spazio di Hilbert $H=H_{0}^{1}\left( \Omega \right) $, si
definisce su $H$ la forma bilineare $a\left( u,v\right) =\int_{\Omega
}\left(
\sum_{i,j=1}^{n}a_{ij}u_{x_{i}}v_{x_{j}}+\sum_{i=1}^{n}b_{i}u_{x_{i}}v+duv%
\right) d\mathbf{x}$. Si definisce inoltre, data $f\in L^{2}\left( \Omega
\right) $, il funzionale $T:H_{0}^{1}\left( \Omega \right) \rightarrow 
%TCIMACRO{\U{211d} }%
%BeginExpansion
\mathbb{R}
%EndExpansion
$, $T\left( v\right) =\int_{\Omega }fvd\mathbf{x}$, funzionale lineare e
continuo.

Il PVA \`{e} dunque trovare $u\in H_{0}^{1}\left( \Omega \right) :a\left(
u,v\right) =T\left( v\right) $ $\forall $ $v\in H_{0}^{1}\left( \Omega
\right) $. La forma bilineare $a$ non \`{e} simmetrica: verifichiamo sotto
quali ipotesi la forma bilineare $a$ \`{e} continua e coerciva per applicare
Lax-Milgram.

Per la continuit\`{a} basta $a_{ij},b_{i},d\in L^{\infty }$: infatti come al
solito, avendo posto $\alpha =\sum_{i,j=1}^{n}\left\vert \left\vert
a_{ij}\right\vert \right\vert _{L^{\infty }}$, $\beta
=\sum_{i=1}^{n}\left\vert \left\vert b_{i}\right\vert \right\vert
_{L^{\infty }}$, $\gamma =\left\vert \left\vert d\right\vert \right\vert
_{L^{\infty }}$, vale $\left\vert a\left( u,v\right) \right\vert \leq
\int_{\Omega }\left( \sum_{i,j=1}^{n}\left\vert \left\vert a_{ij}\right\vert
\right\vert _{L^{\infty }}\left\vert \left\vert \nabla u\right\vert
\right\vert \left\vert \left\vert \nabla v\right\vert \right\vert
+\sum_{i=1}^{n}\left\vert \left\vert b_{i}\right\vert \right\vert
_{L^{\infty }}\left\vert \left\vert \nabla u\right\vert \right\vert
\left\vert v\right\vert +\left\vert \left\vert d\right\vert \right\vert
_{L^{\infty }}\left\vert u\right\vert \left\vert v\right\vert \right) d%
\mathbf{x}$ $\leq \alpha \left\vert \left\vert \nabla u\right\vert
\right\vert _{L^{2}}\left\vert \left\vert \nabla v\right\vert \right\vert
_{L^{2}}+\beta \left\vert \left\vert \nabla u\right\vert \right\vert
_{L^{2}}\left\vert \left\vert v\right\vert \right\vert _{L^{2}}+\gamma
\left\vert \left\vert u\right\vert \right\vert _{L^{2}}\left\vert \left\vert
v\right\vert \right\vert _{L^{2}}\leq \left( \alpha +\beta +\gamma \right)
\left\vert \left\vert u\right\vert \right\vert _{H^{1}}\left\vert \left\vert
v\right\vert \right\vert _{H^{1}}$.

In vista della dimostrazione della coercivit\`{a} supponiamo $A$ simmetrica e uniformemente definita positiva, cio\`{e} che $\exists $ $\lambda
>0:\sum_{i,j=1}^{n}a_{ij}\left( \mathbf{x}\right) t_{i}t_{j}\geq \lambda
\left\vert \left\vert \mathbf{t}\right\vert \right\vert ^{2}$ $\forall $ $%
\mathbf{t}\in 
%TCIMACRO{\U{211d} }%
%BeginExpansion
\mathbb{R}
%EndExpansion
^{n},\forall $ $\mathbf{x}\in \Omega $ (\`{e} l'ipotesi di uniforme
ellitticit\`{a}). Allora, se inoltre $d\geq 0$, $a\left( u,u\right)
=\int_{\Omega }\left(
\sum_{i,j=1}^{n}a_{ij}u_{x_{i}}u_{x_{j}}+%
\sum_{i=1}^{n}b_{i}u_{x_{i}}u+du^{2}\right) d\mathbf{x}\geq \lambda
\left\vert \left\vert \nabla u\right\vert \right\vert
_{L^{2}}^{2}+\int_{\Omega }\sum_{i=1}^{n}b_{i}u_{x_{i}}ud\mathbf{x}$. Poich%
\'{e} $\left\vert \int_{\Omega }\sum_{i=1}^{n}b_{i}u_{x_{i}}ud\mathbf{x}%
\right\vert \leq \beta \left\vert \left\vert \nabla u\right\vert \right\vert
_{L^{2}}\left\vert \left\vert u\right\vert \right\vert _{L^{2}}$, si ha $%
a\left( u,u\right) \geq \lambda \left\vert \left\vert \nabla u\right\vert
\right\vert _{L^{2}}^{2}-\beta \left\vert \left\vert \nabla u\right\vert
\right\vert _{L^{2}}\left\vert \left\vert u\right\vert \right\vert _{L^{2}}$%
. Se $\Omega $ \`{e} un dominio limitato, con la disuguaglianza di Poincar%
\'{e} si ottiene $a\left( u,u\right) \geq \left( \lambda -c_{\Omega }\beta
\right) \left\vert \left\vert \nabla u\right\vert \right\vert _{L^{2}}^{2}$.
Se $\beta \leq \frac{\lambda }{2c_{\Omega }}$, allora $\lambda -c_{\Omega
}\beta \geq \frac{\lambda }{2}$ e infine dalla disuguaglianza di Poincar\'{e}
$a\left( u,u\right) \geq \frac{\lambda }{2}\left\vert \left\vert \nabla
u\right\vert \right\vert _{L^{2}}^{2}\geq \frac{\lambda }{2}\frac{1}{%
1+c_{\Omega }^{2}}\left\vert \left\vert u\right\vert \right\vert _{H^{1}}$.
Perci\`{o} $a$ \`{e} coerciva.

Sotto tutte queste ipotesi, per il teorema di buona posizione si ha che $%
\exists $ $!$ $u\in H_{0}^{1}\left( \Omega \right) :a\left( u,\phi \right)
=T\left( \phi \right) $ $\forall $ $\phi \in H_{0}^{1}\left( \Omega \right) $%
, e $\left\vert \left\vert u\right\vert \right\vert _{H^{1}\left( \Omega
\right) }\leq 2\frac{c_{\Omega }^{2}+1}{\lambda }\left\vert \left\vert
f\right\vert \right\vert _{L^{2}\left( \Omega \right) }$. Abbiamo dimostrato
il seguente

\textbf{Teo}%
\begin{gather*}
\text{Hp: }\Omega \subseteq 
%TCIMACRO{\U{211d} }%
%BeginExpansion
\mathbb{R}
%EndExpansion
^{n}\text{ \`{e} un dominio limitato, }a_{ij},b_{i},d\in L^{\infty }\left(
\Omega \right) \text{, }\exists \text{ }\lambda >0: \\
\sum_{i,j=1}^{n}a_{ij}\left( \mathbf{x}\right) t_{i}t_{j}\geq \lambda
\left\vert \left\vert \mathbf{t}\right\vert \right\vert ^{2}\text{ }\forall 
\text{ }\mathbf{t}\in 
%TCIMACRO{\U{211d} }%
%BeginExpansion
\mathbb{R}
%EndExpansion
^{n},\forall \text{ }\mathbf{x}\in \Omega \text{; }\beta \leq \frac{\lambda 
}{2c_{\Omega }},d\geq 0,f\in L^{2}\left( \Omega \right) \\
\text{Ts: }\exists \text{ }!\text{ soluzione debole di }\left\{ 
\begin{array}{c}
-\func{div}\left( A\left( \mathbf{x}\right) \nabla u\left( \mathbf{x}\right)
\right) +\left\langle b\left( \mathbf{x}\right) ,\nabla u\left( \mathbf{x}%
\right) \right\rangle +d\left( \mathbf{x}\right) u\left( \mathbf{x}\right)
=f\left( \mathbf{x}\right) \text{ in }\Omega \\ 
u=0\text{ su }\partial \Omega%
\end{array}%
\right. \text{ } \\
\text{e }\left\vert \left\vert u\right\vert \right\vert _{H^{1}\left( \Omega
\right) }\leq 2\frac{c_{\Omega }^{2}+1}{\lambda }\left\vert \left\vert
f\right\vert \right\vert _{L^{2}\left( \Omega \right) }
\end{gather*}

L'ipotesi $\beta \leq \frac{\lambda }{2c_{\Omega }}$ non \`{e} naturale:
dipende dalla nostra tecnica dimostrativa, e pu\`{o} essere rimossa.

Poich\'{e} la forma bilineare $a$ non \`{e} simmetrica, $u$ non pu\`{o}
essere interpretata come punto di minimo del funzionale $J$.

Ora risolviamo il problema di Dirichlet con dato al bordo non nullo: 
\begin{equation*}
\left\{ 
\begin{array}{c}
-\func{div}\left( A\left( \mathbf{x}\right) \nabla u\left( \mathbf{x}\right)
\right) +\left\langle b\left( \mathbf{x}\right) ,\nabla u\left( \mathbf{x}%
\right) \right\rangle +d\left( \mathbf{x}\right) u\left( \mathbf{x}\right)
=f\left( \mathbf{x}\right) \text{ in }\Omega \\ 
u=g\text{ su }\partial \Omega%
\end{array}%
\right.
\end{equation*}
L'operatore che appare al lato sinistro dell'equazione d'ora in
poi si indicher\`{a} con $L$. Essendo la prima volta - da quando \`{e} stata
introdotta la formulazione debole - che si affronta la situazione di $g\neq
0 $, occorre chiarire in quale senso \`{e} intesa la condizione al bordo.

Suppooniamo in prima battuta che $g$ sia definita in tutto $\Omega $ e $g\in
H^{1}\left( \Omega \right) $: si pu\`{o} allora osservare che $u=g$ su $%
\partial \Omega \Longleftrightarrow u-g=0$ su $\partial \Omega
\Longleftrightarrow ^{}u-g\in H_{0}^{1}\left( \Omega \right) $.
Allora si fa un cambio di funzione incognita per ricondursi al caso gi\`{a}
noto: data $w:=u-g$, si risolve $\left\{ 
\begin{array}{c}
Lw=f-Lg \\ 
w\in H_{0}^{1}\left( \Omega \right)%
\end{array}%
\right. $ e poi si ricava $u=w+g$. Il termine noto per\`{o} ora contiene
anche $Lg$: essendo $g$ solo $H^{1}$, occorre specificare cosa significa $Lg$%
, essendo $Lu=-\func{div}\left( A\left( \mathbf{x}\right) \nabla u\left( 
\mathbf{x}\right) \right) +\left\langle b\left( \mathbf{x}\right) ,\nabla
u\left( \mathbf{x}\right) \right\rangle +d\left( \mathbf{x}\right) u\left( 
\mathbf{x}\right) $. Si nota che, se $a_{ij},b_{i},d\in L^{\infty }$, allora 
$du,\left\langle b,\nabla u\right\rangle ,A\nabla u$ sono in $L^{2}$, e
dunque si ha una combinazione di funzioni $L^{2}$ e derivate di funzioni $%
L^{2}$: $f-Lg$ pu\`{o} essere scritta come $g_{0}-\sum_{i=1}^{n}\left(
g_{i}\right) _{x_{i}}$ con $g_{i}\in L^{2}\left( \Omega \right) $. Allora il
lato destro dell'equazione pu\`{o} essere interpretato come elemento di $%
H_{0}^{1}\left( \Omega \right) ^{\ast }=H^{-1}\left( \Omega \right) $, e
l'uguaglianza $Lw=f-Lg$ come un'uguaglianza tra due funzionali definiti su $%
H_{0}^{1}\left( \Omega \right) $.

Per un teorema visto su $H^{-1}\left( \Omega \right) $ sappiamo che, se 
$T$ \`{e} $f-Lg$, allora $\left\vert \left\vert T\right\vert \right\vert
_{H^{-1}\left( \Omega \right) }\leq \left\vert \left\vert f\right\vert
\right\vert _{L^{2}}+\left\vert \left\vert dg\right\vert \right\vert
_{L^{2}}+\left\vert \left\vert \left\langle b,\nabla g\right\rangle
\right\vert \right\vert _{L^{2}}+\left\vert \left\vert A\nabla g\right\vert
\right\vert _{L^{2}}\leq \left\vert \left\vert f\right\vert \right\vert
_{L^{2}}+\left( \left\vert \left\vert d\right\vert \right\vert _{L^{\infty
}}+\left\vert \left\vert b\right\vert \right\vert _{L^{\infty }}+\left\vert
\left\vert A\right\vert \right\vert _{L^{\infty }}\right) \left\vert
\left\vert g\right\vert \right\vert _{H^{1}\left( \Omega \right) }$: il
funzionale \`{e} lineare e continuo. Ci siamo quindi ricondotti a un
problema del tipo $\left\{ 
\begin{array}{c}
Lw=T\in H^{-1}\left( \Omega \right) \\ 
w\in H_{0}^{1}\left( \Omega \right)%
\end{array}%
\right. $: ma questo lo sappiamo gi\`{a} risolvere. Anche se abbiamo
trattato finora solo problemi del tipo $Lu=f$ con $f\in L^{2}$, in realt\`{a}
avremmo potuto subito scrivere $Lu=T\in H^{-1}\left( \Omega \right) $,
interpretando l'uguaglianza alla base della definizione di soluzione debole
come uguaglianza tra due funzionali lineari e continui. Quindi sappiamo gi%
\`{a} risolvere questo problema perch\'{e} tutti i passaggi fatti per
mostrare esistenza e unicit\`{a} della soluzione (l'ultimo teorema visto)
erano volti a mostrare le propriet\`{a} della forma bilineare al lato
sinistro; il ruolo del lato destro \`{e} stato solo quello di funzionale
lineare continuo su $H_{0}^{1}\left( \Omega \right) $, indipendentemente dal
fatto che fosse associato alla funzione $f$. Infatti noi abbiamo scritto $%
\left\vert T\left( \phi \right) \right\vert \leq \left\vert \left\vert
f\right\vert \right\vert _{L^{2}}\left\vert \left\vert \phi \right\vert
\right\vert _{L^{2}}\leq \left\vert \left\vert f\right\vert \right\vert
_{L^{2}}\left\vert \left\vert \phi \right\vert \right\vert _{H^{1}}$, ma
avremmo potuto direttamente scrivere $\left\vert T\left( \phi \right)
\right\vert \leq \left\vert \left\vert T\right\vert \right\vert
_{H^{-1}\left( \Omega \right) }\left\vert \left\vert \phi \right\vert
\right\vert _{H^{1}\left( \Omega \right) }$. Vale quindi il seguente

\textbf{Teo}%
\begin{gather*}
\text{Hp: }\Omega \subseteq 
%TCIMACRO{\U{211d} }%
%BeginExpansion
\mathbb{R}
%EndExpansion
^{n}\text{ \`{e} un dominio limitato, }a_{ij},b_{i},d\in L^{\infty }\left(
\Omega \right) \text{, }\exists \text{ }\lambda >0: \\
\sum_{i,j=1}^{n}a_{ij}\left( \mathbf{x}\right) t_{i}t_{j}\geq \lambda
\left\vert \left\vert \mathbf{t}\right\vert \right\vert ^{2}\text{ }\forall 
\text{ }t\in 
%TCIMACRO{\U{211d} }%
%BeginExpansion
\mathbb{R}
%EndExpansion
^{n},\forall \text{ }\mathbf{x}\in \Omega \text{; }\beta \leq \frac{\lambda 
}{2c_{\Omega }},d\geq 0,f\in L^{2}\left( \Omega \right) ,g\in H^{1}\left(
\Omega \right) \\
\text{Ts: }\exists \text{ }!\text{ }u\in H^{1}\left( \Omega \right)
:w=u-g\in H_{0}^{1}\left( \Omega \right) \text{ e }w\text{ \`{e} soluzione
debole di }\left\{ 
\begin{array}{c}
Lw=f-Lg\text{ in }\Omega \\ 
w\in H_{0}^{1}\left( \Omega \right)%
\end{array}%
\right. \text{;} \\
\left\vert \left\vert w\right\vert \right\vert _{H^{1}\left( \Omega \right)
}\leq c\left( \left\vert \left\vert f\right\vert \right\vert _{L^{2}\left(
\Omega \right) }+\left\vert \left\vert g\right\vert \right\vert
_{H^{1}\left( \Omega \right) }\right) \text{ e }\left\vert \left\vert
u\right\vert \right\vert _{H^{1}\left( \Omega \right) }\leq c^{\prime
}\left( \left\vert \left\vert f\right\vert \right\vert _{L^{2}\left( \Omega
\right) }+\left\vert \left\vert g\right\vert \right\vert _{H^{1}\left(
\Omega \right) }\right)
\end{gather*}

Per quanto detto sopra, le ipotesi (eccetto quella su $g$) sono le stesse
del teorema precedente!

La seconda disguaglianza nella tesi \`{e} conseguenza immediata della prima
e della definizione di $w$; la prima disuguaglianza segue dal teorema
precedente e dal fatto che $\left\vert \left\vert T\right\vert \right\vert
_{H^{-1}\left( \Omega \right) }\leq \left\vert \left\vert f\right\vert
\right\vert _{L^{2}}+\left( \left\vert \left\vert d\right\vert \right\vert
_{L^{\infty }}+\left\vert \left\vert b\right\vert \right\vert _{L^{\infty
}}+\left\vert \left\vert A\right\vert \right\vert _{L^{\infty }}\right)
\left\vert \left\vert g\right\vert \right\vert _{H^{1}\left( \Omega \right)
} $.

Vogliamo ora affrontare il caso in cui il dato al bordo $g$ \`{e} definito
solo su $\partial \Omega $, applicando il teorema della traccia. Se $\Omega $
\`{e} un dominio limitato lipschitziano e $g\in H^{1/2}\left( \Omega \right) 
$, allora $\exists $ $\tilde{g}\in H^{1}\left( \Omega \right) :\tau _{0}%
\tilde{g}=g$; $\tilde{g}$ non \`{e} unica, e - come gi\`{a} menzionato - si
dice rilevamento di $g$. Allora, visto che $\tilde{g}$ \`{e} definita in
tutto $\Omega $, si pu\`{o} risolvere il problema $\left\{ 
\begin{array}{c}
Lu=f\text{ in }\Omega \\ 
u=\tilde{g}\text{ su }\partial \Omega%
\end{array}%
\right. $, ponendo $w=u-\tilde{g}$ e risolvendo $\left\{ 
\begin{array}{c}
Lw=f-L\tilde{g} \\ 
w\in H_{0}^{1}\left( \Omega \right)%
\end{array}%
\right. $. Per\`{o}, dato che $\forall $ $g\in H^{1/2}\left( \Omega \right) $
esistono infinite $\tilde{g}\in H^{1}\left( \Omega \right) :\tau _{0}\tilde{g%
}=g$ (mentre $w$ e quindi $u$ sono univocamente determinate da $\tilde{g}$):
occorre quindi mostrare che la soluzione $u$ non dipende dal rilevamento $%
\tilde{g}$, ma solo da $g$.

\textbf{Dim} Siano $\tilde{g}_{1},\tilde{g}_{2}\in H^{1}\left( \Omega
\right) :\tau _{0}\tilde{g}_{1}=\tau _{0}\tilde{g}_{2}=g$. Allora $\tilde{g}%
_{1}-\tilde{g}_{2}\in \ker \tau _{0}=H_{0}^{1}\left( \Omega \right) $. Siano 
$w_{1},w_{2}\in H_{0}^{1}\left( \Omega \right) $ soluzioni di $\left\{ 
\begin{array}{c}
Lw_{i}=f-L\tilde{g}_{i} \\ 
w_{i}\in H_{0}^{1}\left( \Omega \right)%
\end{array}%
\right. $: considero $u_{i}=w_{i}+\tilde{g}_{i}$ e $u=u_{1}-u_{2}$. Allora $%
u $ \`{e} tale che $Lu=Lu_{1}-Lu_{2}=L\left( w_{1}-w_{2}\right) +L\tilde{g}%
_{1}-L\tilde{g}_{2}=0$. Come si comporta $u$ sul bordo? Posto $w=w_{1}-w_{2}$%
, vale $u=w+\tilde{g}_{1}-\tilde{g}_{2}$: essendo somma di funzioni $%
H_{0}^{1}$, anche $u\in H_{0}^{1}$. Dunque $u$ risolve $\left\{ 
\begin{array}{c}
Lu=0\text{ in }\Omega \\ 
u=0\text{ su }\partial \Omega%
\end{array}%
\right. $: per unicit\`{a} della soluzione del problema, $u=0$. $%
\blacksquare $

Allora, risolvendo il problema in $w$ in base al risultato sopra, si ha $%
\left\vert \left\vert u\right\vert \right\vert _{H^{1}\left( \Omega \right)
}\leq c\left( \left\vert \left\vert f\right\vert \right\vert _{L^{2}\left(
\Omega \right) }+\left\vert \left\vert \tilde{g}\right\vert \right\vert
_{H^{1}\left( \Omega \right) }\right) $, e questa disuguaglianza vale per
ogni $\tilde{g}$ rilevamento di $g$. Si pu\`{o} quindi ottenere la
maggiorazione pi\`{u} raffinata prendendo l'estremo inferiore delle norme
delle $\tilde{g}$, e si ha $\left\vert \left\vert u\right\vert \right\vert
_{H^{1}\left( \Omega \right) }\leq c\left( \left\vert \left\vert
f\right\vert \right\vert _{L^{2}\left( \Omega \right) }+\inf_{\tilde{g}\in
H^{1}\left( \Omega \right) :\tau _{0}\tilde{g}=g}\left\vert \left\vert 
\tilde{g}\right\vert \right\vert _{H^{1}\left( \Omega \right) }\right) $, cio%
\`{e} - usando la definizione di norma $H^{1/2}$ -%
\begin{equation*}
\left\vert \left\vert u\right\vert \right\vert _{H^{1}\left( \Omega \right)
}\leq c\left( \left\vert \left\vert f\right\vert \right\vert _{L^{2}\left(
\Omega \right) }+\left\vert \left\vert g\right\vert \right\vert
_{H^{1/2}\left( \Omega \right) }\right)
\end{equation*}

Abbiamo quindi dimostrato il seguente

\textbf{Teo}%
\begin{gather*}
\text{Hp: }\Omega \subseteq 
%TCIMACRO{\U{211d} }%
%BeginExpansion
\mathbb{R}
%EndExpansion
^{n}\text{ \`{e} un dominio limitato lipschitziano, }a_{ij},b_{i},d\in
L^{\infty }\left( \Omega \right) \text{, }\exists \text{ }\lambda >0: \\
\sum_{i,j=1}^{n}a_{ij}\left( \mathbf{x}\right) t_{i}t_{j}\geq \lambda
\left\vert \left\vert \mathbf{t}\right\vert \right\vert ^{2}\text{ }\forall 
\text{ }t\in 
%TCIMACRO{\U{211d} }%
%BeginExpansion
\mathbb{R}
%EndExpansion
^{n},\forall \text{ }\mathbf{x}\in \Omega \text{; }\beta \leq \frac{\lambda 
}{2c_{\Omega }},d\geq 0,f\in L^{2}\left( \Omega \right) ,g\in H^{1/2}\left(
\Omega \right) \\
\text{Ts: }\exists \text{ }!\text{ }u\in H^{1}\left( \Omega \right) \text{
soluzione debole di }\left\{ 
\begin{array}{c}
Lu=f\text{ in }\Omega \\ 
u=g\text{ su }\partial \Omega%
\end{array}%
\right. \text{ e }\left\vert \left\vert u\right\vert \right\vert
_{H^{1}\left( \Omega \right) }\leq c\left( \left\vert \left\vert
f\right\vert \right\vert _{L^{2}\left( \Omega \right) }+\left\vert
\left\vert g\right\vert \right\vert _{H^{1/2}\left( \Omega \right) }\right)
\end{gather*}

Si \`{e} aggiunta l'ipotesi di lipschitzianit\`{a} del dominio per usare il
teorema di traccia, e si \`{e} dovuto chiedere $g\in H^{1/2}\left( \Omega
\right) $. Come al solito, al lato destro della disuguaglianza si potrebbe
scrivere, pi\`{u} in generale, $\left\vert \left\vert T\right\vert
\right\vert _{H^{-1}\left( \Omega \right) }$.

\paragraph{Problema di Neumann}

Affrontiamo il problema di Neumann direttamente per l'operatore completo,
con dato al bordo non nullo: $\left\{ 
\begin{array}{c}
Lu=-\func{div}\left( A\left( \mathbf{x}\right) \nabla u\right) +\left\langle
b,\nabla u\right\rangle +du=f\text{ in }\Omega \\ 
\frac{\partial u}{\partial n_{A}}=g\text{ su }\partial \Omega%
\end{array}%
\right. $. Ricaviamo come al solito la formulazione debole, supponendo che $%
u\in C^{2}\left( \Omega \right) \cap C^{1}\left( \bar{\Omega}\right) $
risolva l'equazione in senso classico con $a_{ij}\in C^{1}\left( \Omega
\right) ,b,d,f\in C^{0}\left( \Omega \right) ,g\in C^{0}\left( \partial
\Omega \right) $, $\Omega $ dominio limitato e lipschitziano. Allora,
moltiplicando per $\phi \in C^{1}\left( \bar{\Omega}\right) $ e integrando, $%
u$ soddisfa anche $-\int_{\Omega }\func{div}\left( A\nabla u\right) \phi d%
\mathbf{x}+\int_{\Omega }\left\langle b,\nabla u\right\rangle \phi d\mathbf{x%
}+\int_{\Omega }du\phi d\mathbf{x}=\int_{\Omega }f\phi d\mathbf{x}$: per il
teorema della divergenza il primo addendo al lato sinistro \`{e} $%
\int_{\Omega }\left\langle A\nabla u,\nabla \phi \right\rangle d\mathbf{x-}%
\int_{\partial \Omega }\left\langle A\nabla u,n\right\rangle \phi
dS=\int_{\Omega }\sum_{i,j=1}^{n}a_{ji}u_{x_{i}}\phi _{x_{j}}d\mathbf{x}%
-\int_{\partial \Omega }\phi \sum_{i,j=1}^{n}a_{ji}u_{x_{i}}n_{j}dS$. $\left\langle An,\nabla u\right\rangle $ \`{e} la
derivata di $u$ non nella direzione normale, ma nella direzione effettiva in
cui avviene la diffusione (essendo il mezzo non isotropo): \`{e} quindi in
quella direzione che ha senso stabilire la condizione al bordo, che
rappresenta il flusso uscente. Per questo nella formulazione del problema si
scrive $\frac{\partial u}{\partial n_{A}}$.

Si ottiene quindi $\int_{\Omega }\left\langle A\nabla u,\nabla \phi
\right\rangle d\mathbf{x}+\int_{\Omega }\left\langle b,\nabla u\right\rangle
\phi d\mathbf{x}+\int_{\Omega }du\phi d\mathbf{x}=\int_{\partial \Omega
}g\phi d\mathbf{x}+\int_{\Omega }f\phi d\mathbf{x}$, che pu\`{o} essere
scritta come $a\left( u,\phi \right) =T\left( \phi \right) $ $\forall $ $%
\phi \in C^{1}\left( \bar{\Omega}\right) $, con le ovvie definizioni di $a$
e $T$. Per densit\`{a}, l'uguaglianza vale $\forall $ $\phi \in H^{1}\left(
\Omega \right) $.

\textbf{Def} Dati $a_{ij},b_{i},d\in L^{\infty }\left( \Omega \right) ,f\in
L^{2}\left( \Omega \right) ,g\in L^{2}\left( \partial \Omega \right) $, si
dice che $u$ \`{e} soluzione debole di $\left\{ 
\begin{array}{c}
Lu=f\text{ in }\Omega \\ 
\frac{\partial u}{\partial n_{A}}=g\text{ su }\partial \Omega%
\end{array}%
\right. $ se $u\in H^{1}\left( \Omega \right) $ \`{e} tale che $\int_{\Omega
}\left\langle A\nabla u,\nabla \phi \right\rangle d\mathbf{x}+\int_{\Omega
}\left\langle b,\nabla u\right\rangle \phi d\mathbf{x}+\int_{\Omega }du\phi d%
\mathbf{x}=\int_{\partial \Omega }g\phi d\mathbf{x}+\int_{\Omega }f\phi d%
\mathbf{x}$ $\forall $ $\phi \in H^{1}\left( \Omega \right) $.

Si riformula l'uguaglianza come problema variazionale astratto: si cerca $%
u\in H^{1}\left( \Omega \right) :a\left( u,\phi \right) =T\left( \phi
\right) $ $\forall $ $\phi \in H^{1}\left( \Omega \right) $, con $f\in
L^{2}\left( \Omega \right) $, $g\in L^{2}\left( \partial \Omega \right) $, $%
a\left( u,\phi \right) =\int_{\Omega }\left\langle A\nabla u,\nabla \phi
\right\rangle d\mathbf{x}+\int_{\Omega }\left\langle b,\nabla u\right\rangle
\phi d\mathbf{x}+\int_{\Omega }du\phi d\mathbf{x}$, $T\left( \phi \right)
=\int_{\partial \Omega }g\tau _{0}\phi d\mathbf{x}+\int_{\Omega }f\phi d%
\mathbf{x}$.

$T$ \`{e} lineare e continuo. Infatti $\left\vert T\left( \phi \right)
\right\vert \leq \left\vert \left\vert f\right\vert \right\vert
_{L^{2}\left( \Omega \right) }\left\vert \left\vert \phi \right\vert
\right\vert _{L^{2}\left( \Omega \right) }+\left\vert \left\vert
g\right\vert \right\vert _{L^{2}\left( \partial \Omega \right) }\left\vert
\left\vert \tau _{0}\phi \right\vert \right\vert _{L^{2}\left( \partial
\Omega \right) }$: poich\'{e} $\left\vert \left\vert \phi \right\vert
\right\vert _{L^{2}}\leq \left\vert \left\vert \phi \right\vert \right\vert
_{H^{1}}$ e per continuit\`{a} della traccia $\left\vert \left\vert \tau
_{0}\phi \right\vert \right\vert _{L^{2}\left( \partial \Omega \right) }\leq
c\left\vert \left\vert \phi \right\vert \right\vert _{H^{1}\left( \Omega
\right) }$, vale $\left\vert T\left( \phi \right) \right\vert \leq \left(
\left\vert \left\vert f\right\vert \right\vert _{L^{2}\left( \Omega \right)
}+c\left\vert \left\vert g\right\vert \right\vert _{L^{2}\left( \partial
\Omega \right) }\right) \left\vert \left\vert \phi \right\vert \right\vert
_{H^{1}\left( \Omega \right) }$. Dunque $\left\vert \left\vert T\right\vert
\right\vert _{H^{1}\left( \Omega \right) ^{\ast }}\leq \left\vert \left\vert
f\right\vert \right\vert _{L^{2}\left( \Omega \right) }+c\left\vert
\left\vert g\right\vert \right\vert _{L^{2}\left( \partial \Omega \right) }$.

Occorre ora dimostrare sotto quali ipotesi $a\left( u,\phi \right) $ \`{e}
continua e coerciva su $H^{1}\left( \Omega \right) $. Per $a_{ij},b_{i},d\in
L^{\infty }$ $a$ \`{e} continua (si sono gi\`{a} fatti i conti in un caso
analogo). Per valutare la coercivit\`{a} non si pu\`{o} usare la
disuguaglianza di Poincar\'{e}; si chiede, come gi\`{a} visto, che esista $%
\lambda >0:\sum_{i,j=1}^{n}a_{ij}t_{i}t_{j}\geq \lambda \left\vert
\left\vert \mathbf{t}\right\vert \right\vert ^{2}$ e $d\left( \mathbf{x}%
\right) \geq c_{0}>0$. Allora $a\left( u,u\right) \geq \lambda \left\vert
\left\vert \nabla u\right\vert \right\vert _{L^{2}\left( \Omega \right)
}^{2}+\int_{\Omega }\left\langle b,\nabla u\right\rangle ud\mathbf{x}%
+c_{0}\left\vert \left\vert u\right\vert \right\vert _{L^{2}\left( \Omega
\right) }^{2}$. $\left\vert \int_{\Omega }\left\langle b,\nabla
u\right\rangle ud\mathbf{x}\right\vert \leq \beta \int_{\Omega }\left\vert
\left\vert \nabla u\right\vert \right\vert \left\vert u\right\vert d\mathbf{x%
}\leq \beta \left\vert \left\vert \nabla u\right\vert \right\vert
_{2}\left\vert \left\vert u\right\vert \right\vert _{2}$ con $\beta =^{\text{%
}}\sqrt{\sum_{i=1}^{n}\left\vert \left\vert b_{i}\right\vert \right\vert
_{L^{\infty }}^{2}}$; poich\'{e} inoltre $\left\vert ab\right\vert \leq 
\frac{1}{2}\left( a^{2}+b^{2}\right) $, si ottiene l'ulteriore maggiorazione 
$\frac{\beta }{2}\left( \left\vert \left\vert \nabla u\right\vert
\right\vert _{2}^{2}+\left\vert \left\vert u\right\vert \right\vert
_{2}^{2}\right) $. Quindi $a\left( u,u\right) \geq \left( \lambda -\frac{%
\beta }{2}\right) \left\vert \left\vert \nabla u\right\vert \right\vert
_{L^{2}\left( \Omega \right) }^{2}+\left( c_{0}-\frac{\beta }{2}\right)
\left\vert \left\vert u\right\vert \right\vert _{L^{2}\left( \Omega \right)
}^{2}$, che \`{e} maggiore o uguale di $c\left\vert \left\vert u\right\vert
\right\vert _{H^{1}\left( \Omega \right) }^{2}$ purch\'{e} $\frac{\beta }{2}<%
\frac{\lambda }{2},\beta <c_{0}$, quindi se $\beta \leq \min \left\{ \lambda
,c_{0}\right\} $.

Abbiamo dimostrato il seguente

\textbf{Teo}%
\begin{gather*}
\text{Hp: }\Omega \subseteq 
%TCIMACRO{\U{211d} }%
%BeginExpansion
\mathbb{R}
%EndExpansion
^{n}\text{ \`{e} un dominio limitato lipschitziano, }a_{ij},b_{i},d\in
L^{\infty }\left( \Omega \right) \text{, }\exists \text{ }\lambda
>0:\sum_{i,j=1}^{n}a_{ij}\left( \mathbf{x}\right) t_{i}t_{j}\geq \lambda
\left\vert \left\vert \mathbf{t}\right\vert \right\vert ^{2}\text{ } \\
\forall \text{ }t\in 
%TCIMACRO{\U{211d} }%
%BeginExpansion
\mathbb{R}
%EndExpansion
^{n},\forall \text{ }\mathbf{x}\in \Omega \text{; }d\left( \mathbf{x}\right)
\geq c_{0}>0\text{, }\beta \text{ abbastanza piccolo in dipendenza da }%
\lambda \text{ e }c_{0},f\in L^{2}\left( \Omega \right) ,g\in L^{2}\left(
\partial \Omega \right) \\
\text{Ts: }\exists \text{ }!\text{ }u\in H^{1}\left( \Omega \right) \text{
soluzione debole di }\left\{ 
\begin{array}{c}
Lu=f\text{ in }\Omega \\ 
\frac{\partial u}{\partial n_{A}}=g\text{ su }\partial \Omega%
\end{array}%
\right. \text{ e }\left\vert \left\vert u\right\vert \right\vert
_{H^{1}\left( \Omega \right) }\leq c\left( \left\vert \left\vert
f\right\vert \right\vert _{L^{2}\left( \Omega \right) }+\left\vert
\left\vert g\right\vert \right\vert _{L^{2}\left( \partial \Omega \right)
}\right)
\end{gather*}

ripeti osservazioni gi\`{a} viste per il pb di Neumann

\textbf{Regolarizzazione} Abbiamo visto finora risultati del tipo: se $u$ 
\`{e} soluzione debole di un problema, i coefficienti, il dominio e $u$
stessa sono regolari, allora $u$ \`{e} anche soluzione classica. Questo \`{e}
un risultato generalmente facile da dimostrare, e non molto utile. Molto pi%
\`{u} interessante \`{e} invece chiedersi: se $u$ \`{e} soluzione debole e i
coefficienti e il dominio sono regolari, \`{e} vero che allora $u$ \`{e}
regolare? Risultati di questo tipo si dicono di regolarizzazione, e
permettono di concludere che $u$ \`{e} soluzione classica.

Nei casi di massima regolarit\`{a} si riescono a recuperare i teoremi di
regolarit\`{a} della teoria classica; se i coefficienti non sono $C^{\infty
} $, la regolarit\`{a} della soluzione che si riesce a dedurre \`{e} sempre
un po' inferiore a quella dei coefficienti. Questo \`{e} un segno che la
teoria debole non \`{e} quella giusta per ottenere risultati classici.

\end{document}
