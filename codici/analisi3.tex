\documentclass{article}
%%%%%%%%%%%%%%%%%%%%%%%%%%%%%%%%%%%%%%%%%%%%%%%%%%%%%%%%%%%%%%%%%%%%%%%%%%%%%%%%%%%%%%%%%%%%%%%%%%%%%%%%%%%%%%%%%%%%%%%%%%%%%%%%%%%%%%%%%%%%%%%%%%%%%%%%%%%%%%%%%%%%%%%%%%%%%%%%%%%%%%%%%%%%%%%%%%%%%%%%%%%%%%%%%%%%%%%%%%%%%%%%%%%%%%%%%%%%%%%%%%%%%%%%%%%%
\usepackage{amssymb}
\usepackage{amsfonts}
\usepackage{amsmath}

\setcounter{MaxMatrixCols}{10}
%TCIDATA{OutputFilter=LATEX.DLL}
%TCIDATA{Version=5.50.0.2960}
%TCIDATA{<META NAME="SaveForMode" CONTENT="1">}
%TCIDATA{BibliographyScheme=Manual}
%TCIDATA{Created=Sunday, August 06, 2023 18:58:22}
%TCIDATA{LastRevised=Wednesday, February 21Saturday, December 30, 20243 11:06:17:18:55}
%TCIDATA{<META NAME="GraphicsSave" CONTENT="32">}
%TCIDATA{<META NAME="DocumentShell" CONTENT="Standard LaTeX\Blank - Standard LaTeX Article">}
%TCIDATA{CSTFile=40 LaTeX article.cst}

\textwidth=6.9in
\oddsidemargin=-0.3in
\topmargin=-0.5in
\textheight=9.0in
\linespread{1.8}

\input{preamble}
\usepackage{amsfonts}
\usepackage{amsmath}


\setcounter{MaxMatrixCols}{10}


\title{Analisi Matematica III\footnote{mariachiara.menicucci@mail.polimi.it per segnalare errori, richiedere il codice LaTeX ecc.}}


\begin{document}

\maketitle
Questi appunti sono stati presi durante le lezioni dell'insegnamento "Analisi Matematica III" tenuto dal prof. Grasselli durante l'A. A. 2023/24. Non sono stati revisionati da alcun docente (potrebbero contenere errori, in forma e in sostanza, di qualsiasi tipo) e non sono in alcun modo sostitutivi della frequentazione delle lezioni. \\
Essendo il corso da 5 CFU, questi appunti aiutano a seguirlo, ma non possono considerarsi una trattazione esaustiva degli argomenti, per la quale si rimanda al libro consigliato: "Analisi 3" di G. Gilardi.
\newpage
\tableofcontents

\newpage

\section{Analisi complessa}

$\left( 
%TCIMACRO{\U{2102} }%
%BeginExpansion
\mathbb{C}
%EndExpansion
,+,\cdot \right) $ \`{e} un campo, cio\`{e} $\left( 
%TCIMACRO{\U{2102} }%
%BeginExpansion
\mathbb{C}
%EndExpansion
,+\right) $ e $\left( 
%TCIMACRO{\U{2102} }%
%BeginExpansion
\mathbb{C}
%EndExpansion
\backslash \left\{ 0\right\} ,\cdot \right) $ sono due gruppi abeliani, e
vale la propriet\`{a} distributiva della somma rispetto al prodotto. $%
%TCIMACRO{\U{2102} }%
%BeginExpansion
\mathbb{C}
%EndExpansion
$, come $%
%TCIMACRO{\U{211d} }%
%BeginExpansion
\mathbb{R}
%EndExpansion
^{2}$, non pu\`{o} essere ordinato compatibilmente con le sue operazioni:
qualsiasi relazione d'ordine $R$ si definisca, \`{e} possibile che $aRb$ e
non $\left( a+c\right) R\left( b+c\right) $.

Per ogni numero complesso sono possibili tre rappresentazioni: dato $z\in 
%TCIMACRO{\U{2102} }%
%BeginExpansion
\mathbb{C}
%EndExpansion
,z=\left\{ 
\begin{array}{c}
a+ib \\ 
\rho \left( \cos \theta +i\sin \theta \right) \\ 
\rho e^{i\theta }%
\end{array}%
\right. $ (algebrica, trigonometrica ed esponenziale). Ogni numero complesso
pu\`{o} essere rappresentato graficamente sul piano di Argand-Gauss grazie
alla corrispondenza biunivoca $a+ib\longleftrightarrow \left( a,b\right) $.
(L'identit\`{a} $i^{2}+1=0$ mostra che in $%
%TCIMACRO{\U{2102} }%
%BeginExpansion
\mathbb{C}
%EndExpansion
$ non vale il teorema di Pitagora, perch\'{e} in $%
%TCIMACRO{\U{2102} }%
%BeginExpansion
\mathbb{C}
%EndExpansion
$, a differenza di $%
%TCIMACRO{\U{211d} }%
%BeginExpansion
\mathbb{R}
%EndExpansion
^{2}$, \`{e} definito un prodotto) Vale $e^{i\theta }=\cos \theta +i\sin
\theta $, e per $\theta =\pi $ $e^{i\theta }=-1$.

In generale $z^{2}\neq \left\vert z\right\vert ^{2}$.

Oltre alla struttura algebrica, per parlare di limiti di funzioni in $%
%TCIMACRO{\U{2102} }%
%BeginExpansion
\mathbb{C}
%EndExpansion
$ occorre anche una topologia; essendo $%
%TCIMACRO{\U{2102} }%
%BeginExpansion
\mathbb{C}
%EndExpansion
$ uno spazio metrico con la distanza $d\left( z_{1},z_{2}\right) =\left\vert
z_{1}-z_{2}\right\vert $, $%
%TCIMACRO{\U{2102} }%
%BeginExpansion
\mathbb{C}
%EndExpansion
$ \`{e} uno spazio topologico con la topologia $\tau $ indotta dalla
metrica: la famiglia degli intorni \`{e} $\left\{ \left\{ z\in 
%TCIMACRO{\U{2102} }%
%BeginExpansion
\mathbb{C}
%EndExpansion
:d\left( z,z_{0}\right) <r\right\} _{\substack{ z_{0}\in 
%TCIMACRO{\U{2102} }%
%BeginExpansion
\mathbb{C}
%EndExpansion
\\ r>0}}\right\} =\left\{ B_{r}\left( z_{0}\right) _{\substack{ z_{0}\in 
%TCIMACRO{\U{2102} }%
%BeginExpansion
\mathbb{C}
%EndExpansion
\\ r>0}}\right\} $ e $\tau =\left\{ A\in 
%TCIMACRO{\U{2102} }%
%BeginExpansion
\mathbb{C}
%EndExpansion
:A\text{ \`{e} aperto rispetto a }d\right\} $.

Si pu\`{o} considerare anche $%
%TCIMACRO{\U{2102} }%
%BeginExpansion
\mathbb{C}
%EndExpansion
$ esteso, indicato con $%
%TCIMACRO{\U{2102} }%
%BeginExpansion
\mathbb{C}
%EndExpansion
^{\ast }$, che \`{e} $%
%TCIMACRO{\U{2102} }%
%BeginExpansion
\mathbb{C}
%EndExpansion
\cup \left\{ \infty \right\} $: gli intorni di $\infty $ sono del tipo $\bar{%
B}_{r}\left( 0\right) ^{c}=\left\{ z:\left\vert z\right\vert >r\right\} $,
al variare di $r>0$. $%
%TCIMACRO{\U{2102} }%
%BeginExpansion
\mathbb{C}
%EndExpansion
^{\ast }$ \`{e} uno spazio topologico con aritmetizzazione parziale.

Esiste una corrispondenza biunivoca tra $%
%TCIMACRO{\U{2102} }%
%BeginExpansion
\mathbb{C}
%EndExpansion
$ e i punti della superficie di una sfera, che si realizza tramite la
proiezione stereografica: la sfera rappresenta un "modello al finito" di un
insieme infinito. Se per semplicit\`{a} si suppone di avere la sfera
goniometrica, in cui il polo Nord \`{e} il punto $N=\left( 0,0,1\right) $,
dato un punto $P$ sulla superficie della sfera esiste un'unica retta che
passa per $\left( P,N\right) $: il punto in cui questa retta interseca il
piano $z=0$ \`{e} la proiezione del punto sul piano complesso. Pi\`{u} $P$ 
\`{e} vicino a $N$, pi\`{u} la sua proiezione \`{e} un punto del piano con
modulo grande. (p. 226)

\textbf{Def} Una funzione $f:D\subseteq 
%TCIMACRO{\U{2102} }%
%BeginExpansion
\mathbb{C}
%EndExpansion
\rightarrow 
%TCIMACRO{\U{2102} }%
%BeginExpansion
\mathbb{C}
%EndExpansion
$ si dice funzione complessa di variabile complessa.

Tali funzioni ci interessano in particolare se $D$ \`{e} aperto.

\textbf{Def} Dato $A\subseteq 
%TCIMACRO{\U{2102} }%
%BeginExpansion
\mathbb{C}
%EndExpansion
$ aperto, $f:A\rightarrow 
%TCIMACRO{\U{2102} }%
%BeginExpansion
\mathbb{C}
%EndExpansion
$, $z_{0}$ punto di accumulazione per $A$ e $l\in 
%TCIMACRO{\U{2102} }%
%BeginExpansion
\mathbb{C}
%EndExpansion
$, si dice che $\lim_{z\rightarrow z_{0}}f\left( z\right) =l$ se $\forall $ $%
\varepsilon >0$ $\exists $ $\delta =\delta \left( \varepsilon \right) :z\in
\left( B_{\delta }\left( z_{0}\right) \backslash \left\{ z_{0}\right\}
\right) \cap A\Longrightarrow \left\vert f\left( z\right) -l\right\vert
<\varepsilon $.

La definizione di limite \`{e} quella naturale di ogni spazio metrico e
topologico. Non ci sono tutte le definizioni aggiuntive di limite destro o
sinistro, dato che non esiste una relazione d'ordine in $%
%TCIMACRO{\U{2102} }%
%BeginExpansion
\mathbb{C}
%EndExpansion
$.

\begin{enumerate}
\item Si noti che $\lim_{z\rightarrow z_{0}}f\left( z\right) =0$ \`{e}
equivalente a $\lim_{z\rightarrow z_{0}}\left\vert f\left( z\right)
\right\vert =0$.
\end{enumerate}

\textbf{Def} Dato $A\subseteq 
%TCIMACRO{\U{2102} }%
%BeginExpansion
\mathbb{C}
%EndExpansion
$ aperto e $f:A\rightarrow 
%TCIMACRO{\U{2102} }%
%BeginExpansion
\mathbb{C}
%EndExpansion
$, si dice che $f$ \`{e} continua in $z_{0}\in A$ se $\lim_{z\rightarrow
z_{0}}f\left( z\right) =f\left( z_{0}\right) $, cio\`{e} $\forall $ $%
\varepsilon >0$ $\exists $ $\delta =\delta \left( \varepsilon ,z_{0}\right)
:z\in B_{\delta }\left( z_{0}\right) \cap A\Longrightarrow d\left( f\left(
z\right) ,f\left( z_{0}\right) \right) <\varepsilon $.

Si pu\`{o} sempre scrivere $f\left( z\right) =\func{Re}f\left( z\right) +i%
\func{Im}f\left( z\right) =u\left( x,y\right) +iv\left( x,y\right) $.
Viceversa, date $u,v:D\subseteq 
%TCIMACRO{\U{211d} }%
%BeginExpansion
\mathbb{R}
%EndExpansion
^{2}\rightarrow 
%TCIMACRO{\U{211d} }%
%BeginExpansion
\mathbb{R}
%EndExpansion
$, si pu\`{o} definire $f\left( z\right) =u\left( \func{Re}z,\func{Im}%
z\right) +iv\left( \func{Re}z,\func{Im}z\right) $: c'\`{e} una
corrispondenza biunivoca tra $f$ e la coppia di funzioni reali $\left(
u,v\right) $.

\begin{enumerate}
\item $f\left( z\right) =z^{2}=\left( x+iy\right) ^{2}=x^{2}-y^{2}+2xyi$: $%
u\left( x,y\right) =x^{2}-y^{2},v\left( x,y\right) =2xy$.

\item $f\left( z\right) =\func{Re}z=x$: $u\left( x,y\right) =x,v\left(
x,y\right) =0$.
\end{enumerate}

Se $u,v$ sono continue in $%
%TCIMACRO{\U{211d} }%
%BeginExpansion
\mathbb{R}
%EndExpansion
^{2}$, $f$ \`{e} continua in $%
%TCIMACRO{\U{2102} }%
%BeginExpansion
\mathbb{C}
%EndExpansion
$.

La definizione di derivata si pu\`{o} facilmente dare grazie all'esistenza
di un prodotto in $%
%TCIMACRO{\U{2102} }%
%BeginExpansion
\mathbb{C}
%EndExpansion
$, come in $%
%TCIMACRO{\U{211d} }%
%BeginExpansion
\mathbb{R}
%EndExpansion
$.

\textbf{Def }Dato $A\subseteq 
%TCIMACRO{\U{2102} }%
%BeginExpansion
\mathbb{C}
%EndExpansion
$ aperto, $f:A\rightarrow 
%TCIMACRO{\U{2102} }%
%BeginExpansion
\mathbb{C}
%EndExpansion
$ e $z_{0}\in A$, si dice che $f$ \`{e} derivabile in $z_{0}$ se esiste
finito $\lim_{z\rightarrow z_{0}}\frac{f\left( z\right) -f\left(
z_{0}\right) }{z-z_{0}}$. In tal caso, il valore di tale limite si dice
derivata prima di $f$ in $z_{0}$ e si indica con $f^{\prime }\left(
z_{0}\right) $.

$A$ aperto fa s\`{\i} che $z_{0}$ sia interno ad $A$ e si possa calcolare $%
f\left( z\right) $ per $z$ in un intorno di $z_{0}$. Il rapporto
incrementale \`{e} ben definito perch\'{e} in $%
%TCIMACRO{\U{2102} }%
%BeginExpansion
\mathbb{C}
%EndExpansion
$ \`{e} definito un prodotto, mentre in $%
%TCIMACRO{\U{211d} }%
%BeginExpansion
\mathbb{R}
%EndExpansion
^{2}$ si dovrebbe calcolare il reciproco di un vettore.

\textbf{Def }Dato $A\subseteq 
%TCIMACRO{\U{2102} }%
%BeginExpansion
\mathbb{C}
%EndExpansion
$ aperto e $f:A\rightarrow 
%TCIMACRO{\U{2102} }%
%BeginExpansion
\mathbb{C}
%EndExpansion
$, si dice che $f$ \`{e} differenziabile in $z_{0}\in A$ se $\exists $ $%
\alpha \in 
%TCIMACRO{\U{2102} }%
%BeginExpansion
\mathbb{C}
%EndExpansion
:f\left( z\right) =f\left( z_{0}\right) +\alpha \left( z-z_{0}\right)
+o\left( z-z_{0}\right) $ per $\left\vert z-z_{0}\right\vert \rightarrow 0$
(equivalentemente, $z\rightarrow z_{0}$).

\textbf{Prop 1.1 (equivalenza tra derivabilit\`{a} e differenziabilit\`{a})}%
\begin{gather*}
\text{Hp: }A\subseteq 
%TCIMACRO{\U{2102} }%
%BeginExpansion
\mathbb{C}
%EndExpansion
\text{ aperto, }f:A\rightarrow 
%TCIMACRO{\U{2102} }%
%BeginExpansion
\mathbb{C}
%EndExpansion
\text{, }z_{0}\in A \\
\text{Ts: }f\text{ \`{e} derivabile in }z_{0}\Longleftrightarrow \text{ \`{e}
differenziabile in }z_{0}\text{ e in tal caso risulta }\alpha =f^{\prime
}\left( z_{0}\right)
\end{gather*}

Valgono tutte le usuali regole di derivazione, anche quella per la
derivazione di una funzione composta.

\begin{enumerate}
\item $f\left( z\right) =z^{2}$; $f^{\prime }\left( z\right) =2z$.

\item $f\left( z\right) =z^{2}e^{z}$; $f^{\prime }\left( z\right)
=2ze^{z}+z^{2}e^{z}$.

\item $f\left( z\right) =\operatorname{Re}(z)$: per calcolare $f^{\prime }\left(
z\right) $ non posso usare regole di derivazione. Devo calcolare $%
\lim_{\left( x,y\right) \rightarrow \left( x_{0},y_{0}\right) }\frac{x-x_{0}%
}{x-x_{0}+i\left( y-y_{0}\right) }$; considero la restrizione lungo la retta 
$x=x_{0}+t,y=y_{0}$ (movimento orizzontale sul piano di Argand-Gauss) e
calcolo $\lim_{t\rightarrow 0}\frac{x_{0}+t-x_{0}}{t}=1$; considero la
restrizione lungo la retta $x=x_{0},y=y_{0}+t$ (movimento verticale) e
calcolo $\lim_{t\rightarrow 0}\frac{x_{0}-x_{0}}{it}=0$. Poich\'{e} il
valore del limite dipende dal cammino, non esiste, e $f$ non \`{e}
derivabile in alcun punto.
\end{enumerate}

Scopriremo che la condizione di differenziabilit\`{a} in $%
%TCIMACRO{\U{2102} }%
%BeginExpansion
\mathbb{C}
%EndExpansion
$ \`{e} molto pi\`{u} restrittiva che in $%
%TCIMACRO{\U{211d} }%
%BeginExpansion
\mathbb{R}
%EndExpansion
$: se $f$ \`{e} differenziabile in $%
%TCIMACRO{\U{2102} }%
%BeginExpansion
\mathbb{C}
%EndExpansion
$ in ogni punto, allora possiede molte propriet\`{a} significative (e. g. 
\`{e} somma di una serie di potenze). Infatti in $%
%TCIMACRO{\U{211d} }%
%BeginExpansion
\mathbb{R}
%EndExpansion
$ le funzioni non derivabili sono pi\`{u} complicate della semplice $f\left(
z\right) =\operatorname{Re}(z)$.

\textbf{Teo 1.2 (condizioni di Cauchy-Riemann)}%
\begin{gather*}
\text{Hp: }A\subseteq 
%TCIMACRO{\U{2102} }%
%BeginExpansion
\mathbb{C}
%EndExpansion
\text{ aperto, }f:A\rightarrow 
%TCIMACRO{\U{2102} }%
%BeginExpansion
\mathbb{C}
%EndExpansion
\text{, }z_{0}=x_{0}+iy_{0}\in A \\
\text{Ts: }f\text{ \`{e} derivabile in }z_{0}\Longleftrightarrow \text{ }u,v%
\text{ sono differenziabili in }z_{0}\text{ } \\
\text{e }\frac{\partial u}{\partial x}\left( x_{0},y_{0}\right) =\frac{%
\partial v}{\partial y}\left( x_{0},y_{0}\right) \text{, }\frac{\partial u}{%
\partial y}\left( x_{0},y_{0}\right) =-\frac{\partial v}{\partial x}\left(
x_{0},y_{0}\right)
\end{gather*}

Questo teorema fornisce una caratterizzazione delle funzioni derivabili che
rende evidente quanto siano rare. Le uguaglianze nella tesi si dicono
condizioni di Cauchy-Riemann, e sono equivalenti a dire che i campi
vettoriali $\left( u,-v\right) $ e $\left( v,u\right) $ hanno rotore nullo.

Ovviamente le condizioni non sono verificate per $\func{Re}z$.

\textbf{Dim} $\Longrightarrow $ Essendo $f$ differenziabile, $\exists $ $%
\alpha :f\left( z_{0}+h\right) -f\left( z_{0}\right) =\alpha h+o\left(
h\right) $, con $\alpha =\alpha _{1}+i\alpha _{2},h=h_{1}+ih_{2}$. Pongo $%
u:=u\left( x_{0}+h_{1},y_{0}+h_{2}\right) $, $v:=v\left(
x_{0}+h_{1},y_{0}+h_{2}\right) $, $u_{0}=u\left( x_{0},y_{0}\right)
,v_{0}=v\left( x_{0},y_{0}\right) $: essendo $f=u+iv$, vale $%
u+iv-u_{0}-iv_{0}=\left( \alpha _{1}+i\alpha _{2}\right) \left(
h_{1}+ih_{2}\right) +o\left( h_{1},h_{2}\right) +io\left( h_{1},h_{2}\right)
=\alpha _{1}h_{1}-\alpha _{2}h_{2}+o\left( h_{1},h_{2}\right) +i\left(
\alpha _{1}h_{2}+\alpha _{2}h_{1}+o\left( h_{1},h_{2}\right) \right) $:
allora, eguagliando parte reale e immaginaria dei due lati, dev'essere $%
u-u_{0}=\alpha _{1}h_{1}-\alpha _{2}h_{2}+o\left( h_{1},h_{2}\right)
,v-v_{0}=\alpha _{1}h_{2}+\alpha _{2}h_{1}+o\left( h_{1},h_{2}\right) $, cio%
\`{e} $u,v$ sono differenziabili come funzioni di due variabili e vale $%
\alpha _{1}=\frac{\partial u}{\partial x},\alpha _{2}=-\frac{\partial u}{%
\partial y},\alpha _{1}=\frac{\partial v}{\partial y},\alpha _{2}=\frac{%
\partial v}{\partial x}$.

$\Longleftarrow $ Se $u,v$ sono differenziabili, si ripete il conto
all'indietro e concludo che $f\left( z_{0}+h\right) -f\left( z_{0}\right)
=u+iv-u_{0}-iv_{0}=\left( \alpha _{1}+i\alpha _{2}\right) \left(
h_{1}+ih_{2}\right) +o\left( h_{1},h_{2}\right) +io\left( h_{1},h_{2}\right) 
$, per cui $f$ \`{e} differenziabile in $z_{0}$. $\blacksquare $

\begin{enumerate}
\item E' possibile sfruttare le condizioni di Cauchy-Riemann per verificare
la derivabilit\`{a} di $f$ in altro modo. Sia $z^{\ast }$ il coniugato di $%
z\in 
%TCIMACRO{\U{2102} }%
%BeginExpansion
\mathbb{C}
%EndExpansion
$ qualsiasi: vale $\func{Re}z=\frac{z+z^{\ast }}{2},\func{Im}z=\frac{%
z-z^{\ast }}{2i}$. Allora $f\left( z\right) =u\left( \frac{z+z^{\ast }}{2},%
\frac{z-z^{\ast }}{2i}\right) +iv\left( \frac{z+z^{\ast }}{2},\frac{%
z-z^{\ast }}{2i}\right) $: quindi $\frac{\partial f}{\partial z^{\ast }}%
=\left\langle \nabla u\left( x,y\right) ,\left( 
\begin{array}{c}
1/2 \\ 
-\frac{1}{2i}%
\end{array}%
\right) \right\rangle +i\left\langle \nabla v\left( x,y\right) ,\left( 
\begin{array}{c}
1/2 \\ 
-\frac{1}{2i}%
\end{array}%
\right) \right\rangle =\frac{1}{2}\left( \frac{\partial u}{\partial x}\left(
x,y\right) -\frac{\partial v}{\partial y}\left( x,y\right) \right) +\frac{1}{%
2}i\left( \frac{\partial u}{\partial y}\left( x,y\right) +\frac{\partial v}{%
\partial x}\left( x,y\right) \right) $. Quindi $f$ \`{e} derivabile in $%
z_{0} $ se e solo se $\frac{\partial f}{\partial z^{\ast }}\left(
z_{0}\right) =0$.
\end{enumerate}

\textbf{Def }Dato $A\subseteq 
%TCIMACRO{\U{2102} }%
%BeginExpansion
\mathbb{C}
%EndExpansion
$ aperto e $f:A\rightarrow 
%TCIMACRO{\U{2102} }%
%BeginExpansion
\mathbb{C}
%EndExpansion
$, si dice che $f$ \`{e} olomorfa in $A$ se \`{e} derivabile in ogni punto
di $A$; in tal caso si scrive $f\in \mathcal{H}\left( A\right) $. Se $f\in 
\mathcal{H}\left( 
%TCIMACRO{\U{2102} }%
%BeginExpansion
\mathbb{C}
%EndExpansion
\right) $, $f$ si dice intera.

\begin{enumerate}
\item Preso $\Omega $ connesso e $f\in \mathcal{H}\left( \Omega \right) $,
se $f$ assume solo valori reali allora $f$ \`{e} costante.

\item Se $f$ e $f^{\ast }$ sono in $\mathcal{H}\left( 
%TCIMACRO{\U{2102} }%
%BeginExpansion
\mathbb{C}
%EndExpansion
\right) $, allora $f\left( z\right) =c\in 
%TCIMACRO{\U{2102} }%
%BeginExpansion
\mathbb{C}
%EndExpansion
$ $\forall $ $z$.

\item Dato $A\subseteq 
%TCIMACRO{\U{211d} }%
%BeginExpansion
\mathbb{R}
%EndExpansion
^{2}$ aperto, $u:A\rightarrow 
%TCIMACRO{\U{211d} }%
%BeginExpansion
\mathbb{R}
%EndExpansion
$ si dice armonica se $u\in C^{2}\left( A\right) $ e $\Delta u=0$. Si pu\`{o}
inoltre dimostrare che se $u$ \`{e} armonica allora $u\in C^{\infty }\left(
A\right) $. Inoltre, se $f\left( x,y\right) =u\left( x,y\right) +iv\left(
x,y\right) $ \`{e} olomorfa in $A$ con $u,v\in C^{2}\left( A\right) $,
allora $u,v$ sono armoniche: si dimostra applicando le condizioni e poi
usando il teorema di Schwarz per le derivate miste. Questo conferma ancora
che le funzioni olomorfe sono tanto rare che le funzioni a esse associate
sono estremamente regolari.

Data $u\in C^{2}\left( A\right) $ armonica, si dice armonica coniugata di $u$
una funzione $v:\Delta v=0$ e $f\left( z\right) =u\left( x,y\right)
+iv\left( x,y\right) $ \`{e} olomorfa. Un teorema afferma che la definizione 
\`{e} ben posta se $A$ \`{e} ben fatto: se $A$ \`{e} semplicemente connesso, 
$\forall $ $u\in C^{2}\left( A\right) :\Delta u=0$ $\exists $ $v\in
C^{2}\left( A\right) :\Delta v=0$, con $v$ unica a meno di una costante, e $%
f\left( z\right) =u\left( x,y\right) +iv\left( x,y\right) $ \`{e} olomorfa.

Per trovare l'armonica coniugata di $u$ si applicano semplicemente le
condizioni di Cauchy-Riemann.
\end{enumerate}

\subsection{Serie di potenze}

\textbf{Def }La successione $\left\{ \sum_{n=0}^{N}a_{n}\left(
z-z_{0}\right) ^{n}\right\} _{N\in 
%TCIMACRO{\U{2115} }%
%BeginExpansion
\mathbb{N}
%EndExpansion
}$, che si indica con $\sum_{n\in 
%TCIMACRO{\U{2115} }%
%BeginExpansion
\mathbb{N}
%EndExpansion
}a_{n}\left( z-z_{0}\right) ^{n}$, si dice serie di potenze di coefficienti $%
a_{n}\in 
%TCIMACRO{\U{2102} }%
%BeginExpansion
\mathbb{C}
%EndExpansion
$ e centro $z_{0}\in 
%TCIMACRO{\U{2102} }%
%BeginExpansion
\mathbb{C}
%EndExpansion
$.

Una serie di potenze \`{e} quindi una successione $S_{N}$ di polinomi in $z$%
. In $z=z_{0}$ la serie converge a $0$.

Fissato $z$, se la successione converge a $l\in 
%TCIMACRO{\U{2102} }%
%BeginExpansion
\mathbb{C}
%EndExpansion
$, allora $\left\vert a_{n}\left( z-z_{0}\right) ^{n}\right\vert \rightarrow
^{n\rightarrow +\infty }0$. Infatti $S_{N}=S_{N-1}+a_{N}\left(
z-z_{0}\right) ^{N}$, cio\`{e} $a_{N}\left( z-z_{0}\right)
^{N}=S_{N}-S_{N-1} $. Poich\'{e} $\lim_{N\rightarrow +\infty }\left(
S_{N}-S_{N-1}\right) =0$, anche $\lim_{N\rightarrow +\infty }a_{N}\left(
z-z_{0}\right) ^{N}=\lim_{N\rightarrow +\infty }\left\vert a_{N}\left(
z-z_{0}\right) ^{N}\right\vert =0$: questa condizione si dice condizione
necessaria di convergenza.

Ricordiamo che $\lim \sup_{n}a_{n}=\lim_{n\rightarrow +\infty }\sup_{k\geq
n}a_{k}$, mentre $\lim \inf_{n}a_{n}=\lim_{n\rightarrow +\infty }\inf_{k\geq
n}a_{k}$; esistono sempre perch\'{e} $\sup_{k\geq n}a_{k}$ e $\inf_{k\geq
n}a_{k}$ sono successioni monotone in $n$ (rispettivamente decrescente e
crescente).

\textbf{Teo 1.3 (convergenza delle serie di potenze) }%
\begin{gather*}
\text{Hp: }a_{n}\in 
%TCIMACRO{\U{2102} }%
%BeginExpansion
\mathbb{C}
%EndExpansion
\text{ }\forall \text{ }n\text{, }z_{0}\in 
%TCIMACRO{\U{2102} }%
%BeginExpansion
\mathbb{C}
%EndExpansion
\text{, }\lim \sup_{n}\left\vert a_{n}\right\vert ^{\frac{1}{n}}=\alpha \in 
\bar{%
%TCIMACRO{\U{211d}}%
%BeginExpansion
\mathbb{R}%
%EndExpansion
}^{\ast }=\left[ 0,+\infty \right] \text{, }r=\frac{1}{\alpha } \\
\text{Ts: (i) }\sum_{n\in 
%TCIMACRO{\U{2115} }%
%BeginExpansion
\mathbb{N}
%EndExpansion
}a_{n}\left( z-z_{0}\right) ^{n}\text{ converge assolutamente }\forall \text{
}z:\left\vert z-z_{0}\right\vert <r \\
\text{(ii) }\sum_{n\in 
%TCIMACRO{\U{2115} }%
%BeginExpansion
\mathbb{N}
%EndExpansion
}a_{n}\left( z-z_{0}\right) ^{n}\text{ non converge per alcun }z:\left\vert
z-z_{0}\right\vert >r
\end{gather*}

Le ipotesi sono sempre verificate: il teorema \`{e} estremamente generale.
Resta da determinare cosa accade agli $z:\left\vert z-z_{0}\right\vert =r$.

\textbf{Dim} Senza perdita di generalit\`{a} si suppone $z_{0}=0$. Sia $z\in 
%TCIMACRO{\U{2102} }%
%BeginExpansion
\mathbb{C}
%EndExpansion
$.

(i) Sia $r>0$. Se $\left\vert z\right\vert <r$, allora $\exists $ $q\in
\left( 0,1\right) ,\exists $ $n_{0}\in 
%TCIMACRO{\U{2115} }%
%BeginExpansion
\mathbb{N}
%EndExpansion
:\left\vert a_{n}\right\vert ^{\frac{1}{n}}\left\vert z\right\vert <q$ $%
\forall $ $n>n_{0}$. Infatti, se per assurdo cos\`{\i} non fosse, si avrebbe
che $\forall $ $q\in \left( 0,1\right) ,\forall $ $n_{0}\in 
%TCIMACRO{\U{2115} }%
%BeginExpansion
\mathbb{N}
%EndExpansion
:\exists $ $n>n_{0}:\left\vert a_{n}\right\vert ^{\frac{1}{n}}\left\vert
z\right\vert \geq q$. Ma allora $\forall $ $n_{0}$ $\sup_{m\geq
n_{0}}\left\vert a_{m}\right\vert ^{\frac{1}{m}}\left\vert z\right\vert \geq
1$: dunque $\lim \sup_{n}$ $\left\vert a_{n}\right\vert ^{\frac{1}{n}%
}\left\vert z\right\vert \geq 1$, cio\`{e} $\alpha \left\vert z\right\vert
\geq 1$, cio\`{e} $\left\vert z\right\vert \geq r$, che \`{e} assurdo.

Quindi definitivamente $\left\vert a_{n}\right\vert \left\vert z\right\vert
^{n}\leq q^{n}$ e $\sum_{n=0}^{+\infty }q^{n}$, in quanto serie geometrica
con ragione $<1$, converge: quindi $\sum_{n\in 
%TCIMACRO{\U{2115} }%
%BeginExpansion
\mathbb{N}
%EndExpansion
}a_{n}z^{n}$ converge assolutamente\footnote{%
sarebbe necessario citare un criterio di convergenza assoluta per le serie
complesse}.

Se $r=0$, la serie converge solo per $z=z_{0}$ (anche alla luce di (ii)).

Se $r=+\infty $, $r$ nel ragionamento sopra pu\`{o} essere scelto grande a
piacere, quindi la serie converge assolutamente $\forall $ $z$.

(ii) Sia $\left\vert z\right\vert >r$. Allora $\exists $ $\left\{
n_{k}\right\} _{k\in 
%TCIMACRO{\U{2115} }%
%BeginExpansion
\mathbb{N}
%EndExpansion
}:\left\vert a_{n_{k}}\right\vert ^{\frac{1}{n_{k}}}\left\vert z\right\vert
>1$ $\forall $ $k$: infatti, se per assurdo non fosse cos\`{\i}, vorrebbe
dire che $\exists $ $\nu :\left\vert a_{n}\right\vert ^{\frac{1}{n}%
}\left\vert z\right\vert \leq 1$ $\forall $ $n>\nu $. Allora si avrebbe $%
\sup_{m\geq n}\left\vert a_{m}\right\vert ^{\frac{1}{m}}\left\vert
z\right\vert \leq 1$ $\forall $ $n>\nu $, da cui $\lim \sup_{n}\left\vert
a_{n}\right\vert ^{\frac{1}{n}}\left\vert z\right\vert \leq 1$, cio\`{e} $%
\alpha \left\vert z\right\vert \leq 1\Longleftrightarrow \left\vert
z\right\vert \leq r$, che \`{e} assurdo. Allora non \`{e} possibile che $%
\left\vert a_{n}\right\vert ^{1/n}\left\vert z\right\vert \rightarrow 0$ per 
$n\rightarrow +\infty $: \`{e} violata la condizione necessaria di
convergenza. $\blacksquare $

\textbf{Def} Data $\sum_{n=0}^{+\infty }a_{n}\left( z-z_{0}\right) ^{n}$ con 
$a_{n}\in 
%TCIMACRO{\U{2102} }%
%BeginExpansion
\mathbb{C}
%EndExpansion
$ $\forall $ $n$, $z_{0}\in 
%TCIMACRO{\U{2102} }%
%BeginExpansion
\mathbb{C}
%EndExpansion
$, $\lim \sup_{n}\left\vert a_{n}\right\vert ^{\frac{1}{n}}=\alpha \in \bar{%
%TCIMACRO{\U{211d}}%
%BeginExpansion
\mathbb{R}%
%EndExpansion
}^{\ast }=\left[ 0,+\infty \right] $ e $r=\frac{1}{\alpha }$, $r$ si dice
raggio di convergenza della serie; si dice centro di convergenza della
serie, e si indica con $C_{r}\left( z_{0}\right) $, $\left\{ z\in 
%TCIMACRO{\U{2102} }%
%BeginExpansion
\mathbb{C}
%EndExpansion
^{\ast }:\left\vert z-z_{0}\right\vert \leq r\right\} $.

Il centro di convergenza \`{e} un disco chiuso, anche se in generale non
sappiamo se includere il bordo; esso contiene anche $+\infty $ se $r=+\infty 
$.

Se $\left\vert z\right\vert <r$, dalla dimostrazione si ricava che $%
\left\vert a_{n}\right\vert \left\vert z\right\vert ^{n}\leq q^{n}$, dove al
lato destro si ha il termine generale di una serie convergente: quindi la
serie $\sum_{n=0}^{+\infty }a_{n}z^{n}$ \`{e} totalmente convergente nel
disco aperto di raggio $r$. Per il criterio di Weierstrass, la convergenza 
\`{e} anche uniforme in qualsiasi disco chiuso di raggio $\rho =\delta <R$ e
centrato in $z_{0}$.

Poich\'{e} la convergenza uniforme preserva la continuit\`{a}, nel disco
aperto $\mathring{C}_{r}\left( z_{0}\right) $ la somma della serie \`{e} una
funzione $f\left( z\right) =\sum_{n=0}^{+\infty }a_{n}\left( z-z_{0}\right)
^{n}$ continua.

\begin{enumerate}
\item Sul bordo del disco possono presentarsi la situazioni pi\`{u} varie
(esistono teoremi a riguardo).

$\sum_{n=0}^{+\infty }z^{n}$ ha $\alpha =R=1$; sul bordo, $\left\{ z\in 
%TCIMACRO{\U{2102} }%
%BeginExpansion
\mathbb{C}
%EndExpansion
^{\ast }:\left\vert z\right\vert =1\right\} $, non converge perch\'{e} \`{e}
violata la condizione necessaria di convergenza.

$\sum_{n=0}^{+\infty }\frac{1}{n}z^{n}$ ha $\alpha =R=1$; sul bordo, $%
\left\{ z\in 
%TCIMACRO{\U{2102} }%
%BeginExpansion
\mathbb{C}
%EndExpansion
^{\ast }:\left\vert z\right\vert =1\right\} $, converge per $z=-1$ grazie al
criterio di Leibniz, non converge per $z=1$; con un teorema si pu\`{o}
dimostrare che converge per $z\neq 1$.

$\sum_{n=0}^{+\infty }\frac{1}{n^{2}}z^{n}$ ha $\alpha =R=1$; sul bordo, $%
\left\{ z\in 
%TCIMACRO{\U{2102} }%
%BeginExpansion
\mathbb{C}
%EndExpansion
^{\ast }:\left\vert z\right\vert =1\right\} $, converge.
\end{enumerate}

\textbf{Prop 1.4 (raggio di convergenza della serie derivata) }%
\begin{gather*}
\text{Hp: }a_{n}\in 
%TCIMACRO{\U{2102} }%
%BeginExpansion
\mathbb{C}
%EndExpansion
\text{ }\forall \text{ }n\text{, }z_{0}\in 
%TCIMACRO{\U{2102} }%
%BeginExpansion
\mathbb{C}
%EndExpansion
\\
\text{Ts: }\sum_{n\in 
%TCIMACRO{\U{2115} }%
%BeginExpansion
\mathbb{N}
%EndExpansion
}a_{n}\left( z-z_{0}\right) ^{n}\text{ e }\sum_{n\in 
%TCIMACRO{\U{2115} }%
%BeginExpansion
\mathbb{N}
%EndExpansion
}na_{n}\left( z-z_{0}\right) ^{n-1}\text{ hanno lo stesso raggio di
convergenza}
\end{gather*}

\textbf{Dim} Vale $\lim \sup \left\vert na_{n}\right\vert ^{\frac{1}{n}}\leq 
\footnote{%
Si ricorda che il limsup del prodotto di due successioni \`{e} minore o
uguale del prodotto dei limsup; in generale non vale l'uguaglianza.}\lim
\sup \left\vert n\right\vert ^{\frac{1}{n}}\lim \sup \left\vert
a_{n}\right\vert ^{\frac{1}{n}}=\lim \sup \left\vert a_{n}\right\vert ^{%
\frac{1}{n}}$, ma vale anche $\lim \sup \left\vert na_{n}\right\vert ^{\frac{%
1}{n}}\geq \lim \sup \left\vert a_{n}\right\vert ^{\frac{1}{n}}$, quindi $%
\lim \sup \left\vert na_{n}\right\vert ^{\frac{1}{n}}=\lim \sup \left\vert
a_{n}\right\vert ^{\frac{1}{n}}$, e le due serie di potenze hanno lo stesso
raggio di convergenza. $\blacksquare $

\textbf{Teo 1.5 (la somma di una serie di potenze \`{e} olomorfa)}%
\begin{gather*}
\text{Hp: }\sum_{n\in 
%TCIMACRO{\U{2115} }%
%BeginExpansion
\mathbb{N}
%EndExpansion
}a_{n}\left( z-z_{0}\right) ^{n}\text{ ha raggio di convergenza }r>0\text{, }%
f\left( z\right) =\sum_{n\in 
%TCIMACRO{\U{2115} }%
%BeginExpansion
\mathbb{N}
%EndExpansion
}a_{n}\left( z-z_{0}\right) ^{n}\text{ }\forall \text{ }z:\left\vert
z-z_{0}\right\vert <r \\
\text{Ts: }f\in \mathcal{H}\left( C_{r}\left( z_{0}\right) \right) \text{ e }%
f^{\prime }\left( z\right) =\sum_{n\in 
%TCIMACRO{\U{2115} }%
%BeginExpansion
\mathbb{N}
%EndExpansion
}a_{n}n\left( z-z_{0}\right) ^{n-1}\text{ }\forall \text{ }z:\left\vert
z-z_{0}\right\vert <r
\end{gather*}

Ovviamente la somma di una serie di potenze in generale \`{e} ben definita
solo per $\left\vert z-z_{0}\right\vert <r$.

\textbf{Dim} Sia $z_{0}=0$ per
semplicit\`{a}. Considero $\tilde{z}\in C_{r}\left( z_{0}\right) :\left\vert 
\tilde{z}\right\vert <r$ e $\delta <r-\left\vert \tilde{z}\right\vert $.
Considero il rapporto incrementale $\frac{f\left( \tilde{z}+h\right)
-f\left( \tilde{z}\right) }{h}$ con $\left\vert h\right\vert <\delta $ (per
cui il rapporto \`{e} ben definito: si \`{e} all'interno del disco). $\frac{%
f\left( \tilde{z}+h\right) -f\left( \tilde{z}\right) }{h}=\frac{\sum_{n\in 
%TCIMACRO{\U{2115} }%
%BeginExpansion
\mathbb{N}
%EndExpansion
}a_{n}\left( \tilde{z}+h\right) ^{n}-\sum_{n\in 
%TCIMACRO{\U{2115} }%
%BeginExpansion
\mathbb{N}
%EndExpansion
}a_{n}\tilde{z}^{n}}{h}=\frac{\sum_{n=1}^{+\infty }a_{n}\left( \left( \tilde{%
z}+h\right) ^{n}-\tilde{z}^{n}\right) }{h}$: affinch\'{e} $f$ sia derivabile
in $\tilde{z}\in C_{r}\left( z_{0}\right) $ \`{e} necessario mostrare che la
serie al lato destro converge quando $\left\vert h\right\vert $ abbastanza
piccolo.

Infatti, poich\'{e} $a^{n}-b^{n}=\left( a-b\right)
\sum_{j=0}^{n-1}a^{j}b^{n-j-1}$, vale $\frac{\left( \tilde{z}+h\right) ^{n}-%
\tilde{z}^{n}}{h}=\sum_{j=0}^{n-1}\left( \tilde{z}+h\right) ^{j}\tilde{z}%
^{n-j-1}$, quindi $\frac{f\left( \tilde{z}+h\right) -f\left( \tilde{z}%
\right) }{h}=\sum_{n=1}^{+\infty }a_{n}\sum_{j=0}^{n-1}\left( \tilde{z}%
+h\right) ^{j}\tilde{z}^{n-j-1}$. Osservo che $\left\vert
\sum_{j=0}^{n-1}\left( \tilde{z}+h\right) ^{j}\tilde{z}^{n-j-1}\right\vert
\leq \sum_{j=0}^{n-1}\left\vert \left( \tilde{z}+h\right) ^{j}\tilde{z}%
^{n-j-1}\right\vert \leq \sum_{j=0}^{n-1}\left( \left\vert \tilde{z}%
\right\vert +\left\vert h\right\vert \right) ^{j}\left\vert \tilde{z}%
\right\vert ^{n-j-1}$. Poich\'{e} $\left\vert h\right\vert <\delta $ e $%
\left\vert \tilde{z}\right\vert \leq \left\vert \tilde{z}\right\vert +\delta 
$, si ottiene l'ulteriore maggiorazione $\sum_{j=0}^{n-1}\left( \left\vert 
\tilde{z}\right\vert +\delta \right) ^{n-1}=n\left( \left\vert \tilde{z}%
\right\vert +\delta \right) ^{n-1}$: dunque $\sum_{n=1}^{+\infty
}a_{n}\left\vert \sum_{j=0}^{n-1}\left( \tilde{z}+h\right) ^{j}\tilde{z}%
^{n-j-1}\right\vert \leq \sum_{n=1}^{+\infty }a_{n}n\left( \left\vert \tilde{%
z}\right\vert +\delta \right) ^{n-1}$: questa serie converge per $\left\vert 
\tilde{z}\right\vert +\delta <r$ perch\'{e} ci si trova nel disco di
convergenza. Inoltre la convergenza \`{e} uniforme per $\left\vert
h\right\vert \leq \frac{\delta }{2}$.

Allora, passando al limite, esiste finito $\lim_{\left\vert h\right\vert
\rightarrow 0}\frac{f\left( \tilde{z}+h\right) -f\left( \tilde{z}\right) }{h}%
=\lim_{\left\vert h\right\vert \rightarrow 0}\sum_{n=1}^{+\infty
}a_{n}\sum_{j=0}^{n-1}\left( \tilde{z}+h\right) ^{j}\tilde{z}^{n-j-1}$
(l'uniformit\`{a} consente lo scambio di limite e serie) e risulta uguale
alla serie della derivata $\sum_{n=1}^{+\infty }a_{n}n\tilde{z}^{n-1}$, per
cui $f^{\prime }\left( z\right) =\sum_{n=1}^{+\infty }a_{n}n\tilde{z}^{n-1}$%
. $\blacksquare $

\textbf{Corollario } 
\begin{gather*}
\text{Hp: }f\left( z\right) =\sum_{n\in 
%TCIMACRO{\U{2115} }%
%BeginExpansion
\mathbb{N}
%EndExpansion
}a_{n}\left( z-z_{0}\right) ^{n}\text{ ha raggio di convergenza }r>0 \\
\text{Ts: }f\text{ ha derivata continua di ogni ordine in }C_{r}\left(
z_{0}\right) \text{ e }f^{\left( k\right) }\left( z\right)
=\sum_{n=k}^{+\infty }k!\binom{n}{k}a_{n}\left( z-z_{0}\right) ^{n-k}
\end{gather*}

\textbf{Dim} La serie derivata ha lo stesso raggio di convergenza della
serie originaria, quindi vale l'argomento gi\`{a} visto. $\blacksquare $

In particolare $f^{\left( k\right) }\left( z_{0}\right) =k!a_{k}$, per cui $%
f\left( z\right) =\sum_{k=0}^{+\infty }\frac{f^{\left( k\right) }\left(
z_{0}\right) }{k!}\left( z-z_{0}\right) ^{k}$: una serie di potenze \`{e} la
serie di Taylor della sua somma.

\subsection{Estensione delle funzioni trascendenti elementari}

Le funzioni trascendenti elementari sono somma della loro serie di Taylor in 
$%
%TCIMACRO{\U{211d} }%
%BeginExpansion
\mathbb{R}
%EndExpansion
$. Si pu\`{o} estendere banalmente la cosa in campo complesso?

\textbf{Teo 1.6 (estensione delle funzioni reali in campo complesso)} 
\begin{gather*}
\text{Hp: }f:\left( a,b\right) \subseteq 
%TCIMACRO{\U{211d} }%
%BeginExpansion
\mathbb{R}
%EndExpansion
^{\ast }\rightarrow 
%TCIMACRO{\U{211d} }%
%BeginExpansion
\mathbb{R}
%EndExpansion
\text{ \`{e} derivabile in }\left( a,b\right) \text{, }\exists \text{ }%
F:A\supset \left( a,b\right) \rightarrow 
%TCIMACRO{\U{2102} }%
%BeginExpansion
\mathbb{C}
%EndExpansion
:F|_{\left( a,b\right) }=f \\
\text{Ts: }F\text{ \`{e} unica}
\end{gather*}

Quindi, dato che $e^{x}=\sum_{k=0}^{+\infty }\frac{x^{k}}{k!}$, la funzione $%
E\left( z\right) :=\sum_{k=0}^{+\infty }\frac{z^{k}}{k!}$ \`{e} l'unica
estensione di $e^{x}$: infatti $E\left( z\right) |_{%
%TCIMACRO{\U{211d} }%
%BeginExpansion
\mathbb{R}
%EndExpansion
}=e^{x}$. E' allora naturale definire $e^{z}:=\sum_{k=0}^{+\infty }\frac{%
z^{k}}{k!}$; non \`{e} banale definire in altro modo $e^{z}$. Analogamente
si definiscono le funzioni $\sin z,\cos z,\tan z,\sinh z,\cosh z,\tanh z$, e
valgono quasi tutte le propriet\`{a} gi\`{a} note: $\sin ^{2}z+\cos ^{2}z=1$%
, $e^{z_{1}+z_{2}}=e^{z_{1}z_{2}}$, dimostrabili con propriet\`{a} delle
serie di potenze.

Diventa cos\`{\i} banale l'uguaglianza $e^{z}=\cos z+i\sin z$, cio\`{e} $%
e^{x+iy}=e^{x}e^{iy}=e^{x}\left( \cos y+i\sin y\right) $: si dimostra
sommando le due serie. Vale inoltre $\sin z=\frac{1}{2i}\left(
e^{iz}-e^{-iz}\right) ,\cos z=\frac{1}{2}\left( e^{iz}+e^{-iz}\right) $:
ponendo $z=iy$ si trova $\cos z=\frac{1}{2}\left( e^{-y}+e^{y}\right) $, che 
\`{e} una funzione superiomente illimitata in modulo. Non vale quindi $%
\left\vert \sin z\right\vert \leq 1,\left\vert \cos z\right\vert \leq 1$ in
campo complesso.

\subsection{Trasformazioni conformi}

\textbf{Teo 1.7 }%
\begin{gather*}
\text{Hp: }f:A\subseteq 
%TCIMACRO{\U{2102} }%
%BeginExpansion
\mathbb{C}
%EndExpansion
\rightarrow 
%TCIMACRO{\U{2102} }%
%BeginExpansion
\mathbb{C}
%EndExpansion
\text{ \`{e} derivabile in }z_{0}\in A\text{, }\left\vert f^{\prime }\left(
z_{0}\right) \right\vert \neq 0 \\
\text{Ts: }f\text{ preserva gli angoli}
\end{gather*}

Segue una giusitificazione informale del risultato. Se $f$ \`{e} derivabile
in $z_{0}=x_{0}+iy_{0}\in A$ e $h$ coinvolto nella definizione del rapporto
incrementale \`{e} $h\in 
%TCIMACRO{\U{211d} }%
%BeginExpansion
\mathbb{R}
%EndExpansion
$, vale $f^{\prime }\left( z_{0}\right) =\frac{\partial u}{\partial x}\left(
x_{0},y_{0}\right) +i\frac{\partial v}{\partial x}\left( x_{0},y_{0}\right) $
(il valore del limite \`{e} lo stesso lungo qualsiasi direzione: infatti $%
\frac{f\left( z_{0}+h\right) -f\left( z_{0}\right) }{h}=\frac{u\left(
x_{0}+h,y_{0}\right) +iv\left( x_{0}+h,y_{0}\right) -u\left(
x_{0},y_{0}\right) -v\left( x_{0},y_{0}\right) }{h}$). Ma per Cauchy-Riemann 
$\frac{\partial u}{\partial x}\left( x_{0},y_{0}\right) +i\frac{\partial v}{%
\partial x}\left( x_{0},y_{0}\right) =\frac{\partial v}{\partial y}\left(
x_{0},y_{0}\right) -i\frac{\partial u}{\partial y}\left( x_{0},y_{0}\right) $%
, quindi $\left\vert f^{\prime }\left( z_{0}\right) \right\vert
^{2}=u_{x}^{2}\left( x_{0},y_{0}\right) +v_{x}^{2}\left( x_{0},y_{0}\right)
=u_{y}^{2}\left( x_{0},y_{0}\right) +v_{y}^{2}\left( x_{0},y_{0}\right) $.

Sia $\mathbf{F}:A\subseteq 
%TCIMACRO{\U{211d} }%
%BeginExpansion
\mathbb{R}
%EndExpansion
^{2}\rightarrow 
%TCIMACRO{\U{211d} }%
%BeginExpansion
\mathbb{R}
%EndExpansion
^{2},\mathbf{F}\left( x,y\right) =\left( 
\begin{array}{c}
u\left( x,y\right) \\ 
v\left( x,y\right)%
\end{array}%
\right) $: la sua matrice jacobiana \`{e} $J_{\mathbf{F}}\left(
x_{0},y_{0}\right) =\left[ 
\begin{array}{cc}
u_{x}\left( x_{0},y_{0}\right) & u_{y}\left( x_{0},y_{0}\right) \\ 
v_{x}\left( x_{0},y_{0}\right) & v_{y}\left( x_{0},y_{0}\right)%
\end{array}%
\right] $. Moltiplicando e dividendo per $f^{\prime }\left( z_{0}\right) $
si ottiene $\left\vert f^{\prime }\left( z_{0}\right) \right\vert \left[ 
\begin{array}{cc}
\frac{u_{x}\left( x_{0},y_{0}\right) }{\left\vert f^{\prime }\left(
z_{0}\right) \right\vert } & \frac{u_{y}\left( x_{0},y_{0}\right) }{%
\left\vert f^{\prime }\left( z_{0}\right) \right\vert } \\ 
\frac{v_{x}\left( x_{0},y_{0}\right) }{\left\vert f^{\prime }\left(
z_{0}\right) \right\vert } & \frac{v_{y}\left( x_{0},y_{0}\right) }{%
\left\vert f^{\prime }\left( z_{0}\right) \right\vert }%
\end{array}%
\right] =\left\vert f^{\prime }\left( z_{0}\right) \right\vert \Theta \left(
x_{0},y_{0}\right) $. $\Theta $ \`{e} una matrice ortogonale con $\det
\Theta \left( x_{0},y_{0}\right) =\frac{1}{\left\vert f^{\prime }\left(
z_{0}\right) \right\vert }\left( u_{x}v_{y}-v_{x}u_{y}\right) |_{\left(
x_{0},y_{0}\right) }=\frac{1}{\left\vert f^{\prime }\left( z_{0}\right)
\right\vert }\left( u_{x}^{2}+v_{x}^{2}\right) |_{\left( x_{0},y_{0}\right)
}=1$ per quanto scritto sopra. Quindi la matrice $\Theta $ rappresenta una
rotazione che preserva l'ordinamento, il fattore $\frac{1}{\left\vert
f^{\prime }\left( z_{0}\right) \right\vert }$ un'omotetia (una riduzione o
aumento di scala). Questo significa che il differenziale di $\mathbf{F}$ 
\`{e} un'applicazione lineare (rappresentata dalla matrice jacobiana) che
preserva gli angoli in $z_{0}$ (cio\`{e} gli angoli tra le rette tangenti in 
$z_{0}$ di due curve qualsiasi passanti per $z_{0}$), nel senso che $%
\left\langle \Theta \mathbf{x},\Theta \mathbf{y}\right\rangle =\left\langle 
\mathbf{x,y}\right\rangle $ e $\left\vert \left\vert \Theta \mathbf{x}%
\right\vert \right\vert =\left\vert \left\vert \mathbf{x}\right\vert
\right\vert $ $\forall $ $\mathbf{x,y}$. Dunque $f$ localmente si comporta
come composizione di una traslazione, una rotazione e un'omotetia e
anch'essa "preserva gli angoli".

\textbf{Def} Data $f:A\subseteq 
%TCIMACRO{\U{2102} }%
%BeginExpansion
\mathbb{C}
%EndExpansion
\rightarrow 
%TCIMACRO{\U{2102} }%
%BeginExpansion
\mathbb{C}
%EndExpansion
$, $f$ si dice trasformazione conforme di $A$ se $f$ preserva l'orientamento
delle curve a sostegno in $A$ e gli angoli che esse formano.

Tali trasformazioni aiutano a risolvere problemi in dominii complicati,
preservando e. g. le linee di flusso.

\subsection{Cammini, circuiti e integrali lungo cammini}

\textbf{Def} Una funzione $r:\left[ a,b\right] \rightarrow 
%TCIMACRO{\U{2102} }%
%BeginExpansion
\mathbb{C}
%EndExpansion
$ si dice di classe $C^{1}$ a tratti se \`{e} continua in $\left[ a,b\right] 
$ e $\exists $ una partizione $\left[ t_{0},t_{1}\right] ,...,\left[
t_{n-1},t_{n}\right] $ di $\left[ a,b\right] $ tale che $r|_{\left[
t_{j-1},t_{j}\right] }$ \`{e} $C^{1}\left( \left[ t_{j-1},t_{j}\right]
\right) $ $\forall $ $j=1,...,n$.

\textbf{Def} $r_{1},r_{2}$ $C^{1}$ a tratti si dicono equivalenti se $%
\exists $ $\phi :D_{r_{2}}\rightarrow D_{r_{1}}$ biunivoca e strettamente
crescente tale che $r_{2}=r_{1}\circ \phi $.

Tale relazione \`{e} di equivalenza nell'insieme delle curve $C^{1}$ a
tratti.

\textbf{Def} Si dice cammino orientato in $%
%TCIMACRO{\U{2102} }%
%BeginExpansion
\mathbb{C}
%EndExpansion
$, e si indica con $C$, una classe di equivalenza di curve $C^{1}$ a tratti,
dette parametrizzazioni.

Le parametrizzazioni nella stessa classe di equivalenza hanno tutte lo
stesso sostegno e lo stesso orientamento.

\textbf{Def} Dato $C$ cammino orientato e $r:\left[ a,b\right] \rightarrow 
%TCIMACRO{\U{2102} }%
%BeginExpansion
\mathbb{C}
%EndExpansion
$ una sua parametrizzazione qualsiasi, $r\left( a\right) ,r\left( b\right) $
si dicono primo e secondo estremo; $r\left( \left[ a,b\right] \right) =\func{%
Im}r$ si dice sostegno di $C$; se $r\left( a\right) =r\left( b\right) $ $C$
si dice chiuso e prende il nome di circuito, altrimenti si dice aperto; se $%
\gamma =r\left( \left[ a,b\right] \right) \subseteq A\subseteq 
%TCIMACRO{\U{2102} }%
%BeginExpansion
\mathbb{C}
%EndExpansion
$, $C$ si dice cammino di $A$; la classe di equivalenza delle
parametrizzazioni $r\left( b+t\left( a-b\right) \right) ,t\in \left[ 0,1%
\right] $, si dice cammino inverso di $C$ e si indica con $-C$; se $c\in
\left( a,b\right) $ e $C_{1},C_{2}$ sono parametrizzati da $r|_{\left(
a,c\right) },r|_{\left( c,b\right) }$ rispettivamente, si dice che $C$ \`{e}
somma dei cammini $C_{1},C_{2}$ e si scrive $C=C_{1}+C_{2}$.

Le nozioni definite non dipendono dalla parametrizzazione scelta.

\begin{enumerate}
\item La circonferenza di raggio $R$ e centro $z_{0}$ \`{e} parametrizzata
da $r\left( t\right) =z_{0}+Re^{it}$.

\item Cammini diversi possono avere lo stesso sostegno. $r_{1}:\left[ 0,2\pi %
\right] \rightarrow 
%TCIMACRO{\U{2102} }%
%BeginExpansion
\mathbb{C}
%EndExpansion
,r_{1}\left( t\right) =\cos t+i\sin t$ e $r_{2}:\left[ 0,3\pi \right]
\rightarrow 
%TCIMACRO{\U{2102} }%
%BeginExpansion
\mathbb{C}
%EndExpansion
,r_{2}\left( t\right) =\cos t+i\sin t$ hanno lo stesso sostegno, ma non sono
equivalenti. Generano quindi cammini diversi.
\end{enumerate}

\textbf{Def} Dato $C$ cammino di $A$ e $f:A\rightarrow 
%TCIMACRO{\U{2102} }%
%BeginExpansion
\mathbb{C}
%EndExpansion
$ continua, si dice integrale di $f$ lungo il cammino $C$ e si indica con $%
\int_{C}f\left( z\right) dz$ il numero complesso $\int_{a}^{b}f\left(
r\left( t\right) \right) r^{\prime }\left( t\right) dt$, dove $r$ \`{e} una
qualsiasi parametrizzazione di $C$. Si pone inoltre $L_{C}:=\int_{a}^{b}%
\left\vert r^{\prime }\left( t\right) \right\vert dt$.

La definizione \`{e} ben posta perch\'{e} l'integrale scritto \`{e}
invariante per parametrizzazioni equivalenti, ed esiste perch\'{e} la
funzione integranda \`{e} continua. $L_{C}$ \`{e} una possibile definizione
di lunghezza del cammino $C$.

\textbf{Propriet\`{a}}

\begin{description}
\item[-] $\int_{-C}f\left( z\right) dz=-\int_{C}f\left( z\right) dz$

\item[-] $\int_{C_{1}+C_{2}}f\left( z\right) dz=\int_{C_{1}}f\left( z\right)
dz+\int_{C_{2}}f\left( z\right) dz$

\item[-] $\left\vert \int_{C}f\left( z\right) dz\right\vert \leq
L_{C}\sup_{t\in \left[ a,b\right] }\left\vert f\left( r\left( t\right)
\right) \right\vert $
\end{description}

Le propriet\`{a} enunciate mostrano che l'operazione tra cammini $C_{1}\ast
_{f}C_{2}=\int_{C_{1}+C_{2}}f\left( z\right) dz$ ha per elemento neutro il
cammino nullo e per elemento inverso il cammino opposto.

\begin{enumerate}
\item $m\in 
%TCIMACRO{\U{2124} }%
%BeginExpansion
\mathbb{Z}
%EndExpansion
$, $f:%
%TCIMACRO{\U{2102} }%
%BeginExpansion
\mathbb{C}
%EndExpansion
\backslash \left\{ 0\right\} \rightarrow 
%TCIMACRO{\U{2102} }%
%BeginExpansion
\mathbb{C}
%EndExpansion
$, $f\left( z\right) =z^{m}$. Parametrizzo la circonferenza di raggio $R>0$
come $r\left( t\right) =Re^{it}$, $t\in \left[ 0,2\pi \right] $. $%
\int_{C}z^{m}dz=\int_{0}^{2\pi }\left( Re^{it}\right)
^{m}Rie^{it}dt=R^{m+1}i\int_{0}^{2\pi }e^{\left( m+1\right) it}dt=0$ se $%
m\neq -1$. Per $m=-1$ si ottiene $2\pi i$. Si noti che quindi l'integrale
del termine generale di una serie di potenze d\`{a} zero se $m\neq -1$.
\end{enumerate}

Questa intuizione pu\`{o} essere meglio formalizzata come segue.

\textbf{Def} Data $f:A\rightarrow 
%TCIMACRO{\U{2102} }%
%BeginExpansion
\mathbb{C}
%EndExpansion
$ con funzioni scalari associate $u,v$, si dicono forme differenziali
associate a $f$ le forme differenziali $udx-vdy$ e $vdx+udy$.

Vale allora, se $C$ \`{e} parametrizzato da $r\left( t\right) =r_{1}\left(
t\right) +ir_{2}\left( t\right) $, $\int_{C}f\left( z\right)
dz=\int_{a}^{b}f\left( r\left( t\right) \right) r^{\prime }\left( t\right)
dt=\int_{C}\left( udx-vdy\right) +i\int_{C}\left( vdx+udy\right) $, dove $C$ 
\`{e} il cammino parametrizzato da $\left( r_{1}\left( t\right) ,r_{2}\left(
t\right) \right) $. [Infatti $\int_{C}\left( udx-vdy\right) +i\int_{C}\left(
vdx+udy\right) =\int_{a}^{b}u\left( r\left( t\right) \right) r_{1}^{\prime
}\left( t\right) -v\left( r\left( t\right) \right) r_{2}^{\prime }\left(
t\right) +i\left[ v\left( r\left( t\right) \right) r_{1}^{\prime }\left(
t\right) +u\left( r\left( t\right) \right) r_{2}^{\prime }\left( t\right) %
\right] dt=$ $\int_{a}^{b}u\left( r\left( t\right) \right) \left[
r_{1}^{\prime }\left( t\right) +ir_{2}^{\prime }\left( t\right) \right]
+iv\left( r\left( t\right) \right) \left[ r_{1}^{\prime }\left( t\right)
+ir_{2}^{\prime }\left( t\right) \right] dt=\int_{a}^{b}f\left( r\left(
t\right) \right) \left( r_{1}^{\prime }\left( t\right) +ir_{2}^{\prime
}\left( t\right) \right) dt=\int_{C}f\left( z\right) dz$. ]

\begin{enumerate}
\item $f\left( z\right) =z^{2}$ ha forme differenziali associate $\left(
x^{2}-y^{2}\right) dx-2xydy$, $2xydx+\left( x^{2}-y^{2}\right) dy$.
\end{enumerate}

D'ora in poi supponiamo per semplicit\`{a} che le funzioni $u,v$ associate a 
$f$ siano $C^{1}\left( A\right) \footnote{\`{e} un'ipotesi necessaria sulle
componenti di una forma differenziale per usare l'implicazione esatte -%
\TEXTsymbol{>} chiuse}$.

\textbf{Teo 1.8 (caratterizzazione dell'olomorfia)}%
\begin{gather*}
\text{Hp: }f:A\rightarrow 
%TCIMACRO{\U{2102} }%
%BeginExpansion
\mathbb{C}
%EndExpansion
\text{ ha funzioni associate }u,v \\
\text{Ts: }f\text{ \`{e} olomorfa in }A\Longleftrightarrow \text{ le forme
differenziali a essa associate sono chiuse}
\end{gather*}

\textbf{Dim} E' noto che $f$ \`{e} olomorfa in $A$ $\Longleftrightarrow $ $%
u_{x}=v_{y}$ e $v_{x}=-u_{y}$ in $A$ per Cauchy-Riemann. Le forme
differenziale associate sono per definizione chiuse se e solo se $%
u_{y}=-v_{x}$ e $v_{y}=u_{x}$, rispettivamente, quindi le affermazioni sono
equivalenti. $\blacksquare $

\textbf{Def} Date $F,f:A\rightarrow 
%TCIMACRO{\U{2102} }%
%BeginExpansion
\mathbb{C}
%EndExpansion
$, si dice che $F$ \`{e} una primitiva per $f$ in $A$ se $F$ \`{e} olomorfa
in $A$ e $F^{\prime }\left( z\right) =f\left( z\right) $ $\forall $ $z\in A$.

\textbf{Teo 1.9}%
\begin{gather*}
\text{Hp: }f:A\rightarrow 
%TCIMACRO{\U{2102} }%
%BeginExpansion
\mathbb{C}
%EndExpansion
\text{ ha funzioni associate }u,v \\
\text{Ts: }f\text{ ha primitive in }A\Longleftrightarrow \text{le forme
differenziali a essa associate sono esatte}
\end{gather*}

\textbf{Dim} Se $U,V$ sono le funzioni associate a $F$, $F$ \`{e} una
primitiva se e solo se $\frac{\partial U}{\partial x}=u$ e $\frac{\partial V%
}{\partial x}=v$ e, per derivabilit\`{a}, $\frac{\partial U}{\partial x}=\frac{%
\partial V}{\partial y}$ e $\frac{\partial V}{\partial x}=-\frac{\partial U}{%
\partial y}$: si chiede quindi che $\nabla U=\left( 
\begin{array}{c}
u \\ 
-v%
\end{array}%
\right) ,\nabla V=\left( 
\begin{array}{c}
v \\ 
u%
\end{array}%
\right) $, cio\`{e} che le forme differenziali associate a $f$ abbiano
potenziale. $\blacksquare $

Quando $f$ ha primitive le funzioni associate a una sua primitiva sono
potenziali delle fd associate a $f$.

\textbf{Teo 1.10}%
\begin{gather*}
\text{Hp: }f:A\rightarrow 
%TCIMACRO{\U{2102} }%
%BeginExpansion
\mathbb{C}
%EndExpansion
\text{ \`{e} continua e ha primitive} \\
\text{Ts: }f\text{ \`{e} olomorfa in }A
\end{gather*}


\textbf{Dim} Se $f$ ha primitive, le forme differenziali a essa associate
sono esatte, quindi chiuse, quindi valgono le condizioni di Cauchy-Riemann. $%
\blacksquare $

\textbf{Teo 1.11}%
\begin{gather*}
\text{Hp: }f:A\rightarrow 
%TCIMACRO{\U{2102} }%
%BeginExpansion
\mathbb{C}
%EndExpansion
\text{ \`{e} continua, }A\text{ \`{e} semplicemente connesso} \\
\text{Ts: }f\text{ \`{e} olomorfa in }A\text{ $\Longleftrightarrow $ ha
primitive in }A
\end{gather*}

La propriet\`{a} topologica del dominio permette di sfruttare l'implicazione
inversa tra esattezza e chiusura delle forme differenziali.

\textbf{Corollario}%
\begin{gather*}
\text{Hp: }f:A\rightarrow 
%TCIMACRO{\U{2102} }%
%BeginExpansion
\mathbb{C}
%EndExpansion
\text{ \`{e} continua} \\
\text{Ts: }f\text{ \`{e} olomorfa in }A\text{ $\Longleftrightarrow \forall $ 
}z\in A\text{ }\exists \text{ }B_{R}\left( z\right) \subseteq A:f\text{ ha
primitive in }B_{R}\left( z\right)
\end{gather*}

\textbf{Dim} $\Longrightarrow $ Se $f$ \`{e} olomorfa, le forme
differenziali a essa associate sono chiuse; allora, fissato $z\in A$, poich%
\'{e} $A$ \`{e} aperto esiste $B_{R}\left( z\right) \subseteq A$ e tale
intorno \`{e} semplicemente connesso: dunque, per 1.11 applicato a $%
f|_{B_{R}\left( z\right) }$, $f$ ha primitive in $B_{R}\left( z\right) $.

$\Longleftarrow $ Fissato $z\in A$, $\exists $ $B_{R}\left( z\right)
\subseteq A:f$ ha primitive in $B_{R}\left( z\right) $: allora, per 1.11
applicato a $f|_{B_{R}\left( z\right) }$, $f$ \`{e} olomorfa in $B_{R}\left(
z\right) $ e in particolare in $z$. Poich\'{e} il ragionamento vale $\forall 
$ $z\in A$, $f$ \`{e} olomorfa in $A$. $\blacksquare $

\textbf{Teo 1.12}%
\begin{gather*}
\text{Hp: }f:A\rightarrow 
%TCIMACRO{\U{2102} }%
%BeginExpansion
\mathbb{C}
%EndExpansion
\text{ \`{e} continua} \\
\text{Ts: }f\text{ ha primitive in }A\Longleftrightarrow \int_{C}f\left(
z\right) dz=0\text{ }\forall \text{ }C\text{ circuito di }A
\end{gather*}

Quindi si intende lungo ogni cammino chiuso.

\textbf{Dim }$\Longrightarrow $ Se $f$ ha primitive, le forme differenziali
associate sono esatte, quindi $\int_{C}f\left( z\right) dz=\int_{C}\left(
udx-vdy\right) +i\int_{C}\left( vdx+udy\right) =0$ perch\'{e} ogni addendo 
\`{e} nullo.

$\Longleftarrow $ Se $\int_{C}\left( udx-vdy\right) +i\int_{C}\left(
vdx+udy\right) =0$ $\forall $ $C$ circuito, allora le fd associate sono
esatte e $f$ ha primitive. $\blacksquare $

\textbf{Corollario}%
\begin{gather*}
\text{Hp: }f:A\rightarrow 
%TCIMACRO{\U{2102} }%
%BeginExpansion
\mathbb{C}
%EndExpansion
\text{ \`{e} continua, }\int_{C}f\left( z\right) dz=0\text{ }\forall \text{ }%
C\text{ circuito di }A\text{, }z_{0}\in A \\
\text{Ts: }F\left( z\right) =\int_{C}f\left( w\right) dw\text{, con }C\text{
cammino da }z_{0}\text{ a }z\text{, \`{e} una primitiva di }f
\end{gather*}

\textbf{Teo 1.13 (Cauchy)}%
\begin{gather*}
\text{Hp: }f:A\rightarrow 
%TCIMACRO{\U{2102} }%
%BeginExpansion
\mathbb{C}
%EndExpansion
\text{ \`{e} olomorfa, }A\text{ \`{e} semplicemente connesso} \\
\text{Ts: }\int_{C_{1}}f\left( z\right) dz=\int_{C_{2}}f\left( z\right) dz%
\text{ }\forall \text{ }C_{1},C_{2}\text{ circuiti }A\text{-om\`{o}topi} \\
\text{/}\int_{C}f\left( z\right) dz=0\text{ }\forall \text{ }C\text{
circuito di }A
\end{gather*}

\textbf{Dim} Se $f$ \`{e} olomorfa, le forme differenziali a essa associate sono chiuse, ma
anche esatte perch\'{e} $A$ \`{e} semplicemente connesso, quindi per
entrambe ogni integrale lungo una curva chiusa \`{e} nullo. $\blacksquare $

Le due tesi sono equivalenti grazie alle propriet\`{a} topologiche di $A$
semplicemente connesso ($C_{1}$ pu\`{o} essere trasformato in $C_{2}$
mediante una deformazione continua senza uscire da $A$).

In realt\`{a} vale una versione pi\`{u} generale di tale teorema, che
rimuove l'ipotesi di semplice connessione di $A$; la dimostrazione si basa
sulla formula di Green.

\textbf{Teo (Cauchy)'}%
\begin{gather*}
\text{Hp: }f:A\rightarrow 
%TCIMACRO{\U{2102} }%
%BeginExpansion
\mathbb{C}
%EndExpansion
\text{ \`{e} olomorfa} \\
\text{Ts: }\int_{C_{1}}f\left( z\right) dz=\int_{C_{2}}f\left( z\right) dz%
\text{ }\forall \text{ }C_{1},C_{2}\text{ circuiti }A\text{-omotopi; } \\
\text{in particolare}\int_{C}f\left( z\right) dz=0\text{ }\forall \text{ }C%
\text{ circuito }A\text{-omotopo a zero}
\end{gather*}

Ci si sta ancora implicitamente avvalendo dell'ipotesi di $u,v$ $C^{1}$, che
non \`{e} necessaria (basta la differenziabilit\`{a}) ma facilita le
dimostrazioni.

\textbf{Teo 1.14 (Morera)}%
\begin{eqnarray*}
\text{Hp}\text{: } &&f:A\rightarrow 
%TCIMACRO{\U{2102} }%
%BeginExpansion
\mathbb{C}
%EndExpansion
\text{ \`{e} continua,}\int_{C}f\left( z\right) dz=0\text{ }\forall \text{ }C%
\text{ circuito di }A \\
\text{Ts}\text{: } &&f\text{ \`{e} olomorfa}
\end{eqnarray*}

\textbf{Dim} L'ipotesi implica che $f$ ammetta primitive: allora le forme
differenziali associate sono esatte, quindi chiuse, quindi vale l'olomorfia. 
$\blacksquare $

Si \`{e} quindi scoperta un'enorme differenza rispetto alle funzioni di
variabile reale: avere primitive implica avere derivate. Infatti le propriet%
\`{a} che le funzione complesse devono soddisfare per esibire tali
comportamenti sono molto forti.

\subsection{Analiticit\`{a}}

Si \`{e} dimostrato che ogni funzione somma di una serie di potenze \`{e}
olomorfa: ora si vuole mostrare anche il contrario, cio\`{e} che ogni
funzione olomorfa \`{e} localmente scrivibile come serie di potenze.

\textbf{Def} Data $f:A\subseteq 
%TCIMACRO{\U{2102} }%
%BeginExpansion
\mathbb{C}
%EndExpansion
\rightarrow 
%TCIMACRO{\U{2102} }%
%BeginExpansion
\mathbb{C}
%EndExpansion
$, $f$ si dice analitica se $\forall $ $z\in A$ $\exists $ $\delta
>0,B_{\delta }\left( z\right) \subseteq A$ e una serie di potenze di centro $%
0$ che converge a $f\left( z+h\right) $ $\forall $ $h\in 
%TCIMACRO{\U{2102} }%
%BeginExpansion
\mathbb{C}
%EndExpansion
:\left\vert h\right\vert <\delta $.

Essere analitica significa quindi essere localmente (in un intorno di $z$, $%
\forall $ $z$) sviluppabile in serie di potenze. Questa propriet\`{a} non pu%
\`{o} essere richiesta che localmente: $A$ pu\`{o} essere terribilmente
strano, mentre l'insieme di convergenza di una serie di potenze \`{e} un
disco, quindi non ha senso richiedere che $f$ sia la somma di un'unica serie
di potenze in tutto $A$.

Se $f$ \`{e} analitica, allora \`{e} olomorfa in $A$.

\begin{enumerate}
\item Sia $f\left( t\right) =\left\{ 
\begin{array}{c}
e^{-\frac{1}{2}t^{2}}\text{ se }t\neq 0 \\ 
0\text{ se }t=0%
\end{array}%
\right. $: si dimostra che $f\in C^{\infty }$ e $f^{\left( k\right) }\left(
0\right) =0$, quindi $f$ \`{e} "olomorfa", ma non pu\`{o} essere somma della
sua serie di potenze centrata in $0$, perch\'{e} essa ha coefficienti tutti
nulli. Quindi essere $C^{\infty }$ non implica l'analiticit\`{a}.
\end{enumerate}

In effetti essere somma di serie di potenze significa essere limite di
successioni di polinomi, cio\`{e} funzioni estremamente regolari: \`{e} una
propriet\`{a} molto forte, tuttavia si vedr\`{a} che in $%
%TCIMACRO{\U{2102} }%
%BeginExpansion
\mathbb{C}
%EndExpansion
$ olomorfia e analiticit\`{a} sono equivalenti.

\textbf{Teo 1.15 }(\textbf{formula di Cauchy per le derivate)}%
\begin{gather*}
\text{Hp: }\sum_{n=1}^{+\infty }c_{n}\left( z-z_{0}\right) ^{n}\text{ \`{e}
una serie di potenze con somma }f\left( z\right) \text{ e disco } \\
\text{di convergenza }B_{R}\left( z_{0}\right) \text{; }r:B_{r}\left(
z_{0}\right) \subset B_{R}\left( z_{0}\right) \\
\text{Ts: }f^{\left( k\right) }\left( z_{0}\right) =\frac{k!}{2\pi i}%
\int_{\partial B_{r}\left( z_{0}\right) }\frac{f\left( z\right) }{\left(
z-z_{0}\right) ^{k+1}}dz\text{ }\forall \text{ }k=0,1,...
\end{gather*}

Le derivate di $f$ sono ben definite perch\'{e} $f$, in quanto somma di una
serie di potenze, \`{e} $C^{\infty }$. Il teorema mostra che i valori delle
derivate di $f$ in $z_{0}$ dipendono solo dal valore di $f$ sul bordo del
disco di convergenza, e si calcolano mediante integrali.

Questo teorema potr\`{a} essere visto come un caso particolare del teorema
dei residui e di 1.25.

\textbf{Dim} Sia $f\left( z\right) =\sum_{n=1}^{+\infty }c_{n}\left(
z-z_{0}\right) ^{n}$ in $B_{r}\left( z_{0}\right) \subseteq A$. Allora $%
\frac{f\left( z\right) }{\left( z-z_{0}\right) ^{k+1}}=\sum_{n=1}^{+\infty
}c_{n}\left( z-z_{0}\right) ^{n-k-1}$. Integrando sul bordo del disco $%
\int_{\partial B_{r}\left( z_{0}\right) }\frac{f\left( z\right) }{\left(
z-z_{0}\right) ^{k+1}}dz=\int_{\partial B_{r}\left( z_{0}\right)
}\sum_{n=1}^{+\infty }c_{n}\left( z-z_{0}\right)
^{n-k-1}dz=\sum_{n=1}^{+\infty }\int_{\partial B_{r}\left( z_{0}\right)
}c_{n}\left( z-z_{0}\right) ^{n-k-1}dz$; lo scambio \`{e} lecito grazie alla
convergenza uniforme nel disco. Per quanto visto in un esempio sopra, c'\`{e}
un solo addendo nella sommatoria che non \`{e} nullo: quello per $n=k$, che
vale $\int_{\partial B_{r}\left( z_{0}\right) }c_{k}\left( z-z_{0}\right)
^{-1}dz=\int_{0}^{2\pi }c_{k}\left( Re^{it}+z_{0}-z_{0}\right) ^{-1}\left(
Rie^{it}\right) dt=\int_{0}^{2\pi }c_{k}\left( Re^{it}\right) ^{-1}\left(
Rie^{it}\right) dt=2\pi ic_{k}$. Ma \`{e} noto che $c_{k}=\frac{f^{\left(
k\right) }\left( z_{0}\right) }{k!}$, quindi si ha $\int_{\partial
B_{r}\left( z_{0}\right) }\frac{f\left( z\right) }{\left( z-z_{0}\right)
^{k+1}}dz=2\pi i\frac{f^{\left( k\right) }\left( z_{0}\right) }{k!}$. $%
\blacksquare $

La formula nel caso $k=0$ \`{e} $f\left( z_{0}\right) =\frac{1}{2\pi i}%
\int_{\partial B_{r}\left( z_{0}\right) }\frac{f\left( z\right) }{z-z_{0}}dz$%
. Il seguente teorema la generalizza a un intero intorno di $z_{0}$.

\textbf{Teo 1.16} (\textbf{formula di Cauchy: rappresentazione integrale di }%
$f$\textbf{)}%
\begin{gather*}
\text{Hp: }f:A\rightarrow 
%TCIMACRO{\U{2102} }%
%BeginExpansion
\mathbb{C}
%EndExpansion
\text{ \`{e} olomorfa, }\bar{B}_{r}\left( z_{0}\right) \subseteq A \\
\text{Ts: }\forall \text{ }z\in B_{r}\left( z_{0}\right) \text{ }f\left(
z\right) =\frac{1}{2\pi i}\int_{\partial B_{r}\left( z_{0}\right) }\frac{%
f\left( w\right) }{w-z}dw
\end{gather*}

Incredibilmente, i valori di $f$ nel disco dipendono solo dai valori di $f$
sul bordo: in $%
%TCIMACRO{\U{211d} }%
%BeginExpansion
\mathbb{R}
%EndExpansion
$ non avrebbe senso.

Anche questo teorema potr\`{a} essere visto come un caso particolare del
teorema dei residui e di 1.25.

\textbf{Dim} Sia $z\in B_{r}\left( z_{0}\right) $, $A_{z}=A\backslash
\left\{ z\right\} $ (che resta un insieme aperto). Sia $g:A_{z}\rightarrow 
%TCIMACRO{\U{2102} }%
%BeginExpansion
\mathbb{C}
%EndExpansion
,g\left( w\right) =\frac{f\left( w\right) }{w-z}$: $g$ \`{e} olomorfa in $%
A_{z}$ perch\'{e} rapporto di funzioni olomorfe. Sia $0<\varepsilon
<r-\left\vert z-z_{0}\right\vert $: allora il disco $B_{\varepsilon }\left(
z\right) $ \`{e} contenuto in $B_{r}\left( z_{0}\right) $, e i due circuiti $%
C_{r}\left( z_{0}\right) =\partial B_{r}\left( z_{0}\right) $ e $%
C_{\varepsilon }\left( z\right) =\partial B_{\varepsilon }\left( z\right) $
sono $A$-omotopi. Allora per il teorema di Cauchy (in forma generale) $%
\int_{C_{r}\left( z_{0}\right) }g\left( w\right) dw=\int_{C_{\varepsilon
}\left( z\right) }g\left( w\right) dw$. Osservo che $\int_{C_{\varepsilon
}\left( z\right) }g\left( w\right) dw=\int_{C_{\varepsilon }\left( z\right) }%
\frac{f\left( w\right) -f\left( z\right) }{w-z}dw+\int_{C_{\varepsilon
}\left( z\right) }\frac{f\left( z\right) }{w-z}dw$ e $\left\vert
\int_{C_{\varepsilon }\left( z\right) }\frac{f\left( w\right) -f\left(
z\right) }{w-z}dw\right\vert \leq 2\pi \varepsilon \sup_{w:\left\vert
w-z\right\vert \leq \varepsilon }\frac{f\left( w\right) -f\left( z\right) }{%
w-z}$: $f$ \`{e} olomorfa, quindi il rapporto incrementale \`{e} limitato.
Inoltre $\int_{C_{\varepsilon }\left( z\right) }\frac{1}{w-z}%
dw=\int_{0}^{2\pi }\left( z+\varepsilon e^{it}-z\right) ^{-1}\varepsilon
ie^{it}dt=2\pi i$. Quindi $\int_{C_{\varepsilon }\left( z\right) }g\left(
w\right) dw=\int_{C_{\varepsilon }\left( z\right) }\frac{f\left( w\right)
-f\left( z\right) }{w-z}dw+2\pi if\left( z\right) $: allora $%
\lim_{\varepsilon \rightarrow 0}\int_{C_{\varepsilon }\left( z\right)
}g\left( w\right) dw=0+2\pi if\left( z\right) $, ma $\lim_{\varepsilon
\rightarrow 0}\int_{C_{\varepsilon }\left( z\right) }g\left( w\right)
dw=\lim_{\varepsilon \rightarrow 0}\int_{C_{r}\left( z_{0}\right) }g\left(
w\right) dw=\int_{C_{r}\left( z_{0}\right) }g\left( w\right) dw=2\pi
if\left( z\right) $, che \`{e} la tesi. $\blacksquare $

\textbf{Teo 1.17 (Weierstrass)}%
\begin{gather*}
\text{Hp: }f:A\rightarrow 
%TCIMACRO{\U{2102} }%
%BeginExpansion
\mathbb{C}
%EndExpansion
\text{ \`{e} olomorfa} \\
\text{Ts: }f\text{ \`{e} analitica}
\end{gather*}

\textbf{Dim} Sia $z_{0}\in A,R>0:B_{R}\left( z_{0}\right) \subseteq A$:
mostro che esiste una serie di potenze che converge a $f$ se $\left\vert
z-z_{0}\right\vert <R$. Posto $h=z-z_{0},z\in B_{R}\left( z_{0}\right)
:\left\vert h\right\vert <r<R$, posso applicare la formula di Cauchy a $f$
in $\bar{B}_{r}\left( z_{0}\right) $, per cui vale $f\left( z\right) =\frac{1%
}{2\pi i}\int_{\partial B_{r}\left( z_{0}\right) }\frac{f\left( w\right) }{%
w-z_{0}-h}dw$. Osservo che $\frac{1}{w-z_{0}-h}=\frac{1}{w-z_{0}}\frac{1}{1-%
\frac{h}{w-z_{0}}}$: poich\'{e} $\frac{\left\vert h\right\vert }{\left\vert
w-z_{0}\right\vert }<1$ ($\left\vert h\right\vert <r,\left\vert
w-z_{0}\right\vert =r$), la serie geometrica $\sum_{n=0}^{+\infty }\left( 
\frac{h}{w-z_{0}}\right) ^{n}$ converge a $\frac{1}{1-\frac{h}{w-z_{0}}}$.
Quindi $\frac{f\left( w\right) }{w-z_{0}-h}=f\left( w\right)
\sum_{n=0}^{+\infty }\frac{h^{n}}{\left( w-z_{0}\right) ^{n+1}}$ e vale $%
f\left( z\right) =\frac{1}{2\pi i}\int_{\partial B_{r}\left( z_{0}\right) }%
\frac{f\left( w\right) }{w-z_{0}-h}dw=\frac{1}{2\pi i}\int_{\partial
B_{r}\left( z_{0}\right) }f\left( w\right) \sum_{n=0}^{+\infty }\frac{h^{n}}{%
\left( w-z_{0}\right) ^{n+1}}dw$: grazie alla convergenza uniforme si
possono scambiare serie e integrale e si ottiene $f\left( z\right)
=\sum_{n=0}^{+\infty }\left[ \frac{1}{2\pi i}\int_{\partial B_{r}\left(
z_{0}\right) }\frac{f\left( w\right) }{\left( w-z_{0}\right) ^{n+1}}dw\right]
\left( z-z_{0}\right) ^{n}$, cio\`{e} $f$ \`{e} la somma di una serie di
potenze (dalla formula di Cauchy per le derivate si ricava che i
coefficienti della serie sono $c_{k}=\frac{f^{\left( k\right) }\left(
z_{0}\right) }{k!}$). $\blacksquare $

\subsection{Singolarit\`{a} e sviluppi di Laurent}

Si \`{e} visto che $f$ \`{e} olomorfa in $A$ aperto se e solo se \`{e}
localmente sviluppabile in serie di potenze. Si vuole estendere questa
propriet\`{a} al caso in cui $f$ \`{e} olomorfa in un aperto privo di alcuni
punti: il prezzo da pagare sar\`{a} una serie di potenze in senso pi\`{u}
generale di quanto visto finora.

Sia $f:A\backslash \left\{ z_{0}\right\} \rightarrow 
%TCIMACRO{\U{2102} }%
%BeginExpansion
\mathbb{C}
%EndExpansion
,z_{0}\in A$, $f$ olomorfa. In tal caso si dice che $z_{0}$ \`{e} una
singolarit\`{a} isolata di $f$: in effetti $z_{0}$ \`{e} un punto isolato
per l'insieme dei punti di $A$ in cui $f$ non \`{e} olomorfa. Ci occuperemo
nel seguito solo di funzioni con singolarit\`{a} isolate.

\begin{enumerate}
\item $f\left( z\right) =\frac{1}{z}$: $z_{0}=0$ \`{e} una singolarit\`{a}
isolata di $f$.

\item $f\left( z\right) =\frac{1}{\sin z}$: $\left\{ k\pi :k\in 
%TCIMACRO{\U{2124} }%
%BeginExpansion
\mathbb{Z}
%EndExpansion
\right\} $ \`{e} un insieme di soli punti isolati, che sono quindi tutte
singolarit\`{a} isolate.

\item $f\left( z\right) =\frac{1}{\sin \frac{1}{z}}$: $\left\{ 0\right\}
\cup \left\{ \frac{1}{k\pi }:k\in 
%TCIMACRO{\U{2124} }%
%BeginExpansion
\mathbb{Z}
%EndExpansion
\right\} $ \`{e} un insieme di soli punti isolati eccetto $0$, che \`{e}
punto di accumulazione per l'insieme. Quindi solo gli elementi di $\left\{ 
\frac{1}{k\pi }:k\in 
%TCIMACRO{\U{2124} }%
%BeginExpansion
\mathbb{Z}
%EndExpansion
\right\} $ sono singolarit\`{a} isolate.
\end{enumerate}

\textbf{Def} Dato $z_{0}\in 
%TCIMACRO{\U{2102} }%
%BeginExpansion
\mathbb{C}
%EndExpansion
,a_{n}\in 
%TCIMACRO{\U{2102} }%
%BeginExpansion
\mathbb{C}
%EndExpansion
$ $\forall $ $n$, si dice serie bilatera di potenze, e si indica con $%
\sum_{n\in 
%TCIMACRO{\U{2124} }%
%BeginExpansion
\mathbb{Z}
%EndExpansion
}a_{n}\left( z-z_{0}\right) ^{n}$, la serie $\sum_{n=1}^{+\infty
}a_{-n}\left( z-z_{0}\right) ^{-n}+\sum_{n=0}^{+\infty }a_{n}\left(
z-z_{0}\right) ^{n}$.

La successione $a_{n}$ \`{e} qui intesa come una funzione a valori complessi
avente dominio $%
%TCIMACRO{\U{2124} }%
%BeginExpansion
\mathbb{Z}
%EndExpansion
$ e non pi\`{u} $%
%TCIMACRO{\U{2115} }%
%BeginExpansion
\mathbb{N}
%EndExpansion
$.

Indicando con $R^{\prime },R$ i raggi di convergenza delle due serie
rispettivamente, si ha che la prima converge quando $\left\vert \frac{1}{%
z-z_{0}}\right\vert <R^{\prime }$, la seconda quando $\left\vert
z-z_{0}\right\vert <R$: la serie bilatera dunque converge quando $\frac{1}{%
R^{\prime }}<\left\vert z-z_{0}\right\vert <R$; se $\rho :=\frac{1}{%
R^{\prime }}\geq R$ la serie non converge per alcun $z$.

L'insieme di convergenza menzionato \`{e} una corona circolare; come al
solito, la convergenza sui bordi non \`{e} nota. Si pone $A_{\rho ,R}\left(
z_{0}\right) :=\left\{ z\in 
%TCIMACRO{\U{2102} }%
%BeginExpansion
\mathbb{C}
%EndExpansion
:\rho <\left\vert z-z_{0}\right\vert <R\right\} $, detto campo di
convergenza. Se \`{e} non vuoto, la serie bilatera converge assolutamente in 
$A_{\rho ,R}\left( z_{0}\right) $ e uniformemente in ogni $E\subseteq
A_{\rho ,R}\left( z_{0}\right) :\bar{E}\subseteq A_{\rho ,R}\left(
z_{0}\right) $ (in tal caso si dice che $E$ \`{e} contenuto con compattezza
in $A_{\rho ,R}\left( z_{0}\right) $).

Nel caso particolare di $\rho =0$, $A_{\rho ,R}\left( z_{0}\right) $ diventa
un cerchio di raggio $R$ privato del centro.

Se $R=+\infty $, $A_{\rho ,R}\left( z_{0}\right) $ diventa l'esterno di un
cerchio di raggio $\rho $. Se inoltre $a_{n}=0$ $\forall $ $n>0$, detta $f$
la somma della serie (per semplicit\`{a} $z_{0}=0$) $f\left( z\right)
=\sum_{n=1}^{+\infty }a_{-n}\frac{1}{z^{n}}+a_{0}$, vale $f\left( \frac{1}{w}%
\right) =\sum_{n=1}^{+\infty }a_{-n}w^{n}+a_{0}$: la serie in $w=0$ converge
ad $a_{0}$, dunque $f\left( \frac{1}{w}\right) $ \`{e} olomorfa in $w=0$
(supponendo di prolungarla con olomorfia). In tal caso si dice che $f$ \`{e}
olomorfa in $z=+\infty $.

\textbf{Teo 1.18 (una funzione olomorfa in una corona \`{e} sviluppabile in
serie di Laurent)}%
\begin{gather*}
\text{Hp: }f\text{ \`{e} olomorfa in }A_{\rho ,R}\left( z_{0}\right)
=\left\{ z\in 
%TCIMACRO{\U{2102} }%
%BeginExpansion
\mathbb{C}
%EndExpansion
:\rho <\left\vert z-z_{0}\right\vert <R\right\} \\
\text{Ts: vale }f\left( z\right) =\sum_{n\in 
%TCIMACRO{\U{2124} }%
%BeginExpansion
\mathbb{Z}
%EndExpansion
}a_{n}\left( z-z_{0}\right) ^{n}\text{ in }A_{\rho ,R}\left( z_{0}\right) 
\text{, con }a_{n}=\frac{1}{2\pi i}\int_{\gamma }\frac{f\left( w\right) }{%
\left( w-z\right) ^{n+1}}dw \\
\text{e }\gamma \text{ circuito qualsiasi a sostegno in }A_{\rho ,R}\left(
z_{0}\right)
\end{gather*}

In tal caso la serie $\sum_{n\in 
%TCIMACRO{\U{2124} }%
%BeginExpansion
\mathbb{Z}
%EndExpansion
}a_{n}\left( z-z_{0}\right) ^{n}=\sum_{n=1}^{+\infty }a_{-n}\left(
z-z_{0}\right) ^{-n}+\sum_{n=0}^{+\infty }a_{n}\left( z-z_{0}\right) ^{n}$ 
\`{e} detta sviluppo di Laurent di $f$; la prima serie \`{e} detta parte
singolare, la seconda parte regolare dello sviluppo. Quindi, come si \`{e}
visto che una funzione olomorfa in un disco \`{e} analitica e quindi
sviluppabile in serie di potenze (di Taylor), una funzione olomorfa in una
corona circolare \`{e} sviluppabile in serie di Laurent, con coefficienti
che si calcolano in modo identico al primo caso (a meno del circuito). Nel
caso la funzione sia olomorfa in un aperto senza singolarit\`{a}, la parte
singolare \`{e} nulla e la serie di Laurent coincide con la serie di Taylor.

Per calcolare i coefficienti $a_{n}$ tipicamente si sceglie come $\gamma $
una circonferenza di raggio $r\in \left( \rho ,R\right) $. La dimostrazione
del teorema si basa sulla formula di Cauchy.

\begin{enumerate}
\item $f\left( z\right) =\frac{1}{1-z}$ \`{e} olomorfa in $%
%TCIMACRO{\U{2102} }%
%BeginExpansion
\mathbb{C}
%EndExpansion
\backslash \left\{ 1\right\} $. Se si considera come centro dello sviluppo $%
z_{0}=1$, $f$ \`{e} olomorfa nella corona $A_{0,+\infty }\left( z_{0}\right) 
$ ed \`{e} gi\`{a} sviluppata in serie di Laurent: $f\left( z\right) =\frac{%
-1}{z-1}$. Se invece si prende $z_{0}=0$, $f$ \`{e} olomorfa nelle due
corone $A_{0,1}\left( z_{0}\right) $ e $A_{1,+\infty }\left( z_{0}\right) $,
dunque si avranno due sviluppi separati. Nel primo caso, $f\left( z\right) =%
\frac{1}{1-z}=\sum_{k=0}^{+\infty }z^{k}$ (che ha la sola parte regolare),
nel secondo $f\left( z\right) =\frac{-1/z}{1-\frac{1}{z}}=-\sum_{k=0}^{+%
\infty }\frac{1}{z^{k+1}}$.

\item $f\left( z\right) =\frac{1}{z\left( 1-z\right) }$ \`{e} olomorfa in $%
%TCIMACRO{\U{2102} }%
%BeginExpansion
\mathbb{C}
%EndExpansion
\backslash \left\{ 0,1\right\} $. Scelto $z_{0}=0$, ci sono due corone
circolari da considerare. In primo luogo $A_{0,1}\left( z_{0}\right) $: in
tal caso $\frac{1}{z}\frac{1}{1-z}=\frac{1}{z}\sum_{n=0}^{+\infty }z^{n}$.
Quindi lo sviluppo in serie di Laurent di $f$ per $0<\left\vert z\right\vert
<1$, centrato in $0$, \`{e} $f\left( z\right) =\frac{1}{z}%
+\sum_{n=0}^{+\infty }z^{n}$, dove $\frac{1}{z}$ \`{e} la parte singolare
dello sviluppo.

In secondo luogo $A_{1,+\infty }\left( z_{0}\right) $: allora $f\left(
z\right) =\frac{1}{z\left( 1-z\right) }=\frac{1}{z}\frac{-1}{z}\frac{1}{1-%
\frac{1}{z}}=-\frac{1}{z^{2}}\sum_{n=0}^{+\infty }\frac{1}{z^{n}}%
=-\sum_{n=2}^{+\infty }\frac{1}{z^{n}}$.

Se invece si sceglie $z_{0}=1$, si hanno ancora $A_{0,1}\left( z_{0}\right) $
e $A_{1,+\infty }\left( z_{0}\right) $. Nel primo caso si ha $f\left(
z\right) =\frac{1}{1-z}\frac{1}{1-\left( 1-z\right) }=\frac{1}{1-z}%
\sum_{k=0}^{+\infty }\left( 1-z\right) ^{k}=\sum_{k=0}^{+\infty }\left(
1-z\right) ^{k-1}$. Nel secondo $f\left( z\right) =\frac{1}{1-z}\frac{1}{%
1-\left( 1-z\right) }=\frac{1}{\left( 1-z\right) ^{2}}\frac{-1}{1-\frac{1}{%
1-z}}=\frac{-1}{\left( 1-z\right) ^{2}}\sum_{k=0}^{+\infty }\frac{1}{\left(
1-z\right) ^{k}}=-\sum_{k=0}^{+\infty }\frac{1}{\left( 1-z\right) ^{k+2}}$.

\item Negli sviluppi di Laurent i coefficienti $a_{n}$ in generale non hanno
legami con le derivate di $f$ in $z_{0}=0$, a differenza di quanto accade
con gli sviluppi di Taylor. Sia $f\left( z\right) =\frac{z}{z-1}$, ben
definita per $z\neq 1$: sviluppo secondo Laurent in $z_{0}=0$. Se $%
\left\vert z\right\vert >1$ $f\left( z\right) =\frac{-z}{1-z}=\frac{1}{1-%
\frac{1}{z}}=\sum_{n=0}^{+\infty }\frac{1}{z^{n}}=\sum_{n=1}^{+\infty }\frac{%
1}{z^{n}}+1$, cio\`{e} $f$ \`{e} somma di una serie bilatera con $a_{n}=0$ $%
\forall $ $n>0,a_{0}=1$; $f$ \`{e} olomorfa in $z=+\infty $ e in tal caso
vale $1$. Tuttavia, $\left\{ 
\begin{array}{c}
\frac{f^{\left( n\right) }\left( 0\right) }{n!}=-1 \\ 
f\left( 0\right) =0%
\end{array}%
\right. $: valori che non hanno niente a che fare con i coefficienti.
\end{enumerate}

Si possono ora classificare i tipi di singolarit\`{a}, la cui definizione si
basa sulla differenza di comportamento della $f$ intorno a una singolarit%
\`{a} isolata.

\textbf{Def} Data $f:A\backslash \left\{ z_{0}\right\} \rightarrow 
%TCIMACRO{\U{2102} }%
%BeginExpansion
\mathbb{C}
%EndExpansion
,z_{0}\in A$, $f\in \mathcal{H}\left( A\backslash \left\{ z_{0}\right\}
\right) $, $z_{0}$ si dice:

\begin{description}
\item[i] singolarit\`{a} eliminabile se $\exists $ $g:A\rightarrow 
%TCIMACRO{\U{2102} }%
%BeginExpansion
\mathbb{C}
%EndExpansion
:g|_{A\backslash \left\{ z_{0}\right\} }=f$ e $g$ \`{e} olomorfa in $A$

\item[ii] singolarit\`{a} polare, o polo, se $\lim_{z\rightarrow
z_{0}}f\left( z\right) =+\infty $

\item[iii] singolarit\`{a} essenziale altrimenti.
\end{description}

(i) significa che $f$ \`{e} la restrizione di una funzione olomorfa in $A$,
e $g$ \`{e} detta prolungamento olomorfo di $f$; vale necessariamente $%
g\left( z_{0}\right) =\lim_{z\rightarrow z_{0}}f\left( z\right) \in 
%TCIMACRO{\U{2102} }%
%BeginExpansion
\mathbb{C}
%EndExpansion
$, dovendo $g$ essere continua in $z_{0}$; l'esistenza finita di tale limite 
\`{e} necessaria e sufficiente affinch\'{e} $z_{0}$ sia una singolarit\`{a}
eliminabile. (ii) \`{e}
equivalente a dire $\lim_{z\rightarrow z_{0}}\left\vert f\left( z\right)
\right\vert =+\infty $.

Data la serie di Laurent $\sum_{n=0}^{+\infty }a_{n}\left( z-z_{0}\right)
^{n}$ di $f$ centrata in $z_{0}$ e definita in un intorno di $z_{0}$,
valgono le seguenti caratterizzazioni: $z_{0}$ \`{e} una singolarit\`{a}
eliminabile $\Longleftrightarrow \lim_{z\rightarrow z_{0}}f\left( z\right)
\in 
%TCIMACRO{\U{2102} }%
%BeginExpansion
\mathbb{C}
%EndExpansion
\Longleftrightarrow a_{n}=0$ $\forall $ $n\leq -1$; $z_{0}$ \`{e} un polo $%
\Longleftrightarrow \exists $ $m<0:a_{n}=0$ $\forall $ $n<m$; $z_{0}$ \`{e}
una singolarit\`{a} essenziale $\Longleftrightarrow \lim_{z\rightarrow
z_{0}}f\left( z\right) $ $\NEG{\exists}\Longleftrightarrow \forall $ $m>0$ $%
\exists $ $n>m:a_{-n}\neq 0$. Quindi la serie di Laurent ha solo i termini
nonnegativi nel primo caso, un numero finito di termini negativi nel
secondo, un numero infinito di termini negativi nel terzo.

\begin{enumerate}
\item $f:%
%TCIMACRO{\U{2102} }%
%BeginExpansion
\mathbb{C}
%EndExpansion
\backslash \left\{ 0\right\} \rightarrow 
%TCIMACRO{\U{2102} }%
%BeginExpansion
\mathbb{C}
%EndExpansion
$, $f\left( z\right) =\frac{\sin z}{z}$. Poich\'{e} per definizione $\sin
z=\sum_{n=0}^{+\infty }\frac{\left( -1\right) ^{n}}{\left( 2n+1\right) !}%
z^{2n+1}$, $\forall $ $z\neq 0$ $\frac{\sin z}{z}=\sum_{n=0}^{+\infty }\frac{%
\left( -1\right) ^{n}}{\left( 2n+1\right) !}z^{2n}=g\left( z\right) $, che 
\`{e} olomorfa in $%
%TCIMACRO{\U{2102} }%
%BeginExpansion
\mathbb{C}
%EndExpansion
$ ed estende $f$: quindi $0$ \`{e} una singolarit\`{a} eliminabile.

\item $f:%
%TCIMACRO{\U{2102} }%
%BeginExpansion
\mathbb{C}
%EndExpansion
\backslash \left\{ 0\right\} \rightarrow 
%TCIMACRO{\U{2102} }%
%BeginExpansion
\mathbb{C}
%EndExpansion
$, $f\left( z\right) =\frac{1}{z^{m}},m\in 
%TCIMACRO{\U{2115} }%
%BeginExpansion
\mathbb{N}
%EndExpansion
$. $\forall $ $m$ $0$ \`{e} un polo. Il caso \`{e} banale: $f$ \`{e} gi\`{a}
scritta come somma di una serie di Laurent, che consta del solo termine di
grado $-m$ con $a_{-m}=1$ e coincide con la sua parte singolare.

\item $f:%
%TCIMACRO{\U{2102} }%
%BeginExpansion
\mathbb{C}
%EndExpansion
\backslash \left\{ 0\right\} \rightarrow 
%TCIMACRO{\U{2102} }%
%BeginExpansion
\mathbb{C}
%EndExpansion
$, $f\left( z\right) =e^{-\frac{1}{z}}$. $e^{-\frac{1}{z}}=\sum_{n=0}^{+%
\infty }\frac{\left( -1/z\right) ^{n}}{n!}=\sum_{n=0}^{+\infty }\frac{\left(
-1\right) ^{n}}{n!}\frac{1}{z^{n}}$: $\NEG{\exists}$ $\lim_{z\rightarrow
0}f\left( z\right) $, quindi $0$ \`{e} una singolarit\`{a} essenziale. Nel
caso di singolarit\`{a} essenziale un teorema di Picard afferma che
l'immagine di $f$ ristretta a un disco centrato in $z_{0}$ \`{e} tutto $%
%TCIMACRO{\U{2102} }%
%BeginExpansion
\mathbb{C}
%EndExpansion
$ oppure $%
%TCIMACRO{\U{2102} }%
%BeginExpansion
\mathbb{C}
%EndExpansion
\backslash \left\{ z_{0}\right\} $.
\end{enumerate}

\textbf{Teo 1.19 (rimozione delle singolarit\`{a} eliminabili)}%
\begin{gather*}
\text{Hp: }f:A\backslash \left\{ z_{0}\right\} \rightarrow 
%TCIMACRO{\U{2102} }%
%BeginExpansion
\mathbb{C}
%EndExpansion
,z_{0}\in A\text{, }f\in \mathcal{H}\left( A_{z_{0}}\right) \text{; }\exists 
\text{ }R>0:f\text{ \`{e} limitata in }B_{R}\left( z_{0}\right) \backslash
\left\{ z_{0}\right\} \subseteq A_{z_{0}} \\
\text{Ts: }z_{0}\text{ \`{e} una singolarit\`{a} eliminabile per }f
\end{gather*}

Quindi se $f$ \`{e} limitata allora $\lim_{z\rightarrow z_{0}}f\left(
z\right) \in 
%TCIMACRO{\U{2102} }%
%BeginExpansion
\mathbb{C}
%EndExpansion
$.

\subsection{Residui integrali}

Data $f$ olomorfa in $A_{\rho ,R}\left( z_{0}\right) \subseteq A_{z_{0}}$,
sia $f\left( z\right) =\sum_{n\in 
%TCIMACRO{\U{2124} }%
%BeginExpansion
\mathbb{Z}
%EndExpansion
}a_{n}\left( z-z_{0}\right) ^{n}$ il suo sviluppo di Laurent in un intorno
di $z_{0}$.

\textbf{Teo 1.20 (residui)}%
\begin{gather*}
\text{Hp: }f:A_{z_{0}}\rightarrow 
%TCIMACRO{\U{2102} }%
%BeginExpansion
\mathbb{C}
%EndExpansion
,z_{0}\in A\text{, }f\in \mathcal{H}\left( A_{\rho ,R}\left( z_{0}\right)
\right) \text{ con sviluppo di Laurent } \\
\sum_{n\in 
%TCIMACRO{\U{2124} }%
%BeginExpansion
\mathbb{Z}
%EndExpansion
}a_{n}\left( z-z_{0}\right) ^{n}\text{; }\gamma \text{ \`{e} una
circonferenza a sostegno in }A_{\rho ,R}\left( z_{0}\right) \text{ centrata
in }z_{0} \\
\text{Ts: }\int_{\gamma }f\left( z\right) dz=2\pi ia_{-1}
\end{gather*}

Si suppone che la circonferenza sia orientata in senso antiorario,
positivamente rispetto alla singolarit\`{a}. Conoscere la serie di Laurent
di $f$ rende estremamente rapido il calcolo dell'integrale.

Il teorema \`{e} un'immediata conseguenza del teorema sullo sviluppo in
serie di Laurent con $n=-1$.

[\textbf{Dim alternativa con lacune} Dato che la serie $\sum_{n\in 
%TCIMACRO{\U{2124} }%
%BeginExpansion
\mathbb{Z}
%EndExpansion
}a_{n}\left( z-z_{0}\right) ^{n}$ converge uniformemente a $f$ in $\gamma $
compatto a sostegno in $A_{\rho ,R}\left( z_{0}\right) $, \`{e} lecito
scambiare serie e integrale e $\int_{\gamma }f\left( z\right) dz=\sum_{n\in 
%TCIMACRO{\U{2124} }%
%BeginExpansion
\mathbb{Z}
%EndExpansion
}a_{n}\int_{\gamma }\left( z-z_{0}\right) ^{n}dz$. Si \`{e} gi\`{a} visto pi%
\`{u} volte che $\int_{\gamma }\left( z-z_{0}\right) ^{n}dz=0$ se $n>0$; se
invece $n<-1$, si applica la formula delle derivate di Cauchy $g^{\left(
k\right) }\left( z_{0}\right) =\frac{k!}{2\pi i}\int_{\partial B_{R}\left(
z_{0}\right) }\frac{g\left( z\right) }{\left( z-z_{0}\right) ^{k+1}}dz$ con $%
g\left( z\right) =1,k=-n-1,\partial B_{R}\left( z_{0}\right) =\gamma $:
allora $\int_{\gamma }\left( z-z_{0}\right) ^{n}dz=2\pi ik!g^{\left(
k\right) }\left( z_{0}\right) =0$. Mancano i casi di $n=0,n=-1$.]

[\textbf{Dim alternativa}: scrivo la circonferenza $\gamma $ come $\partial
B_{r}\left( z_{0}\right) ,\rho <r<R$. Parametrizzo $\gamma $ con $r:\left[
0,2\pi \right] \rightarrow 
%TCIMACRO{\U{2102} }%
%BeginExpansion
\mathbb{C}
%EndExpansion
,r\left( t\right) =z_{0}+re^{it}$ e dato che la serie $\sum_{n\in 
%TCIMACRO{\U{2124} }%
%BeginExpansion
\mathbb{Z}
%EndExpansion
}a_{n}\left( z-z_{0}\right) ^{n}$ converge uniformemente a $f$ in $\gamma $
compatto a sostegno in $A_{\rho ,R}\left( z_{0}\right) $, \`{e} lecito
scambiare serie e integrale: vale $\int_{\gamma }f\left( z\right)
dz=\sum_{n\in 
%TCIMACRO{\U{2124} }%
%BeginExpansion
\mathbb{Z}
%EndExpansion
}a_{n}\int_{\gamma }\left( z-z_{0}\right) ^{n}dz=\sum_{n\in 
%TCIMACRO{\U{2124} }%
%BeginExpansion
\mathbb{Z}
%EndExpansion
}a_{n}\int_{0}^{2\pi }r^{n}e^{int}rie^{it}dt$. L'integrale \`{e} $%
r^{n+1}i\int_{0}^{2\pi }e^{i\left( n+1\right) t}dt=\left\{ 
\begin{array}{c}
0\text{ se }n\neq -1 \\ 
2\pi i\text{ se }n=-1%
\end{array}%
\right. $, quindi $\sum_{n\in 
%TCIMACRO{\U{2124} }%
%BeginExpansion
\mathbb{Z}
%EndExpansion
}a_{n}\int_{\gamma }\left( z-z_{0}\right) ^{n}dz=2\pi ia_{-1}$. $%
\blacksquare $ ]

L'ipotesi di orientamento positivo \`{e} usata implicitamente nella scelta
della parametrizzazione $r\left( t\right) =z_{0}+re^{it}$.

\textbf{Def} Data $f:A_{z_{0}}\rightarrow 
%TCIMACRO{\U{2102} }%
%BeginExpansion
\mathbb{C}
%EndExpansion
,z_{0}\in A$, $f\in \mathcal{H}\left( A_{\rho ,R}\left( z_{0}\right) \right) 
$ con sviluppo di Laurent $\sum_{n\in 
%TCIMACRO{\U{2124} }%
%BeginExpansion
\mathbb{Z}
%EndExpansion
}a_{n}\left( z-z_{0}\right) ^{n}$, $a_{-1}$ si dice residuo integrale di $f$
in $z_{0}$ e si indica con $Res\left( f,z_{0}\right) $.

\begin{enumerate}
\item $Res\left( f,z_{0}\right) =0$ se $z_{0}$ \`{e} una singolarit\`{a}
eliminabile: in tal caso infatti la parte singolare dello sviluppo di
Laurent non c'\`{e}. L'implicazione inversa non vale.

\item $Res\left( f,z_{0}\right) =0$ se $z_{0}$ \`{e} un punto non singolare
per $f$.

\item $f\left( z\right) =\frac{1}{e^{z}-1}$ \`{e} ben definita per $z\neq 0$%
; per trovare $Res\left( f,0\right) $ \`{e} utile calcolare la serie di
Laurent di $f$. Poich\'{e} $e^{z}-1=\sum_{n=1}^{+\infty }\frac{z^{n}}{n!}$,
si dimostra che $\frac{1}{e^{z}-1}=\frac{1}{z+\frac{1}{2}z^{2}+...}=\frac{1}{%
z\left( 1+\frac{1}{2}z+\frac{1}{6}z^{2}+...\right) }=\frac{1}{z}+o\left(
1\right) $ per $z\rightarrow 0$ e quindi vale $a_{-1}=1$. Peraltro questo
mostra che $0$ \`{e} una singolarit\`{a} polare.
\end{enumerate}

Sia ora $f:A_{z_{0}}=A\backslash \left\{ z_{0}\right\} \rightarrow 
%TCIMACRO{\U{2102} }%
%BeginExpansion
\mathbb{C}
%EndExpansion
$ olomorfa in $A_{\rho ,+\infty }\left( z_{0}\right) =\left\{ z\in 
%TCIMACRO{\U{2102} }%
%BeginExpansion
\mathbb{C}
%EndExpansion
:\left\vert z-z_{0}\right\vert >\rho \right\} $: $f$ \`{e} olomorfa nel
complementare di un compatto. In tal caso si dice che $f$ ha una singolarit%
\`{a} isolata\footnote{%
Infatti, se non esiste un compatto nel complementare del quale $f$ sia
olomorfa, significa che per ogni intorno di $+\infty $ esiste una singolarit%
\`{a} che vi appartiene, cio\`{e} l'infinito \`{e} un punto di accumulazione
per le singolarit\`{a} (l'insieme delle singolarit\`{a} \`{e} illimitato) e
quindi non \`{e} una singolarit\`{a} isolata.} a $+\infty $, e si definisce
singolarit\`{a} all'infinito di $f$ la singolarit\`{a} in $0$ della funzione 
$f\left( \frac{1}{w}\right) $.

Se lo sviluppo di Laurent centrato in $z_{0}$ di tale $f$ \`{e} $\sum_{n\in 
%TCIMACRO{\U{2124} }%
%BeginExpansion
\mathbb{Z}
%EndExpansion
}a_{n}\left( z-z_{0}\right) ^{n}$, data una curva $\gamma $ a sostegno in $%
A_{\rho ,R}\left( z_{0}\right) $ orientata in senso antiorario, si dice
residuo di $f$ a $+\infty $ $\frac{\int_{-\gamma }f\left( z\right) dz}{2\pi i%
}$, o alternativamente $Res\left( \frac{-1}{z^{2}}f\left( \frac{1}{z}\right)
,0\right) $.

Si noti che vale $\int_{-\gamma }f\left( z\right) dz=-2\pi ia_{-1}$, per il
teorema immediatamente sopra, dato che si \`{e} invertito il senso di
percorrenza della curva. E' l'orientamento orario (positivo rispetto alla
singolarit\`{a}) della curva a produrre il residuo all'infinito, perch\'{e} 
\`{e} con tale orientamento che "ci si lascia a sinistra la singolarit\`{a}".

\begin{enumerate}
\item Se $\infty $ \`{e} un punto non singolare per $f$ (cio\`{e} $f$ \`{e}
olomorfa a $+\infty $), non vale necessariamente $Res\left( f,\infty \right)
=0$. Si prenda e. g. $f\left( z\right) =\frac{1}{z}$.
\end{enumerate}

\textbf{Lemma (Jordan)}%
\begin{gather*}
\text{Hp: }C\text{ \`{e} un circuito semplice} \\
\text{Ts: }\exists \text{ }A_{1},A_{2}\subseteq 
%TCIMACRO{\U{2102} }%
%BeginExpansion
\mathbb{C}
%EndExpansion
\text{ aperti tali che }A_{1}\cup C\cup A_{2}=%
%TCIMACRO{\U{2102} }%
%BeginExpansion
\mathbb{C}
%EndExpansion
\text{,} \\
A_{1}\cap A_{2}=\varnothing ,\partial A_{1}=\partial A_{2}=C
\end{gather*}

Intuitivamente, ogni curva chiusa semplice divide il piano in due regioni,
una interna alla curva e una esterna, illimitata.

\textbf{Corollario del teorema di Cauchy}%
\begin{gather*}
\text{Hp: }A\text{ aperto, }\partial A\text{ circuito semplice, }D\supseteq A%
\text{ chiuso,} \\
f:D\rightarrow 
%TCIMACRO{\U{2102} }%
%BeginExpansion
\mathbb{C}
%EndExpansion
\text{ olomorfa in }A\text{ e continua in }\bar{A} \\
\text{Ts: }\int_{\partial A}f\left( z\right) dz=0
\end{gather*}

Il teorema di Cauchy afferma che l'integrale su qualsiasi curva in $A$ \`{e}
nullo; il corollario estende il risultato anche alla frontiera di $A$,
utilizzando il teorema di Cauchy all'interno di $A$ e passando al limite.


\textbf{Teo 1.21 (residui I)}%
\begin{gather*}
\text{Hp: }A\text{ aperto, }\partial A\text{ circuito semplice, }%
f:A\rightarrow 
%TCIMACRO{\U{2102} }%
%BeginExpansion
\mathbb{C}
%EndExpansion
\text{ \`{e} olomorfa in }A\text{ e continua } \\
\text{in }\bar{A}\text{ tranne al pi\`{u} in un numero finito di singolarit%
\`{a} }z_{1},...,z_{N}\in A \\
\text{Ts: }\int_{\partial ^{+}A}f\left( z\right) dz=2\pi
i\sum_{j=1}^{N}Res\left( f,z_{j}\right)
\end{gather*}

Nel caso particolare in cui non si hanno singolarit\`{a} i residui sono
tutti nulli e l'integrale \`{e} nullo, in accordo con il corollario del
teorema di Cauchy sopra. L'idea intuitiva \`{e} che l'integrale di $f$ sul
bordo di $A$ orientato positivamente, se si hanno delle singolarit\`{a} in $%
A $, non \`{e} nullo: ci\`{o} che rimane sono appunto i residui, che si
ottengono con lo sviluppo di Laurent di $f$ nell'intorno di ogni singolarit%
\`{a}.

\textbf{Dim} Considero le singolarit\`{a} $z_{1},...,z_{N}$ e di ciascuna
considero un intorno contenuto in $A$, con bordo parametrizzato in senso
orario dalla curva $\gamma _{j},j=1,...,N$. Allora, se si considera $f$
ristretta ad $A$ privato di tutti questi intorni (con la loro frontiera), si
ha che $f$ \`{e} olomorfa in tale dominio e l'integrale lungo il bordo
parametrizzato positivamente \`{e} $\int_{\partial ^{+}A}f\left( z\right)
dz+\sum_{i=1}^{N}\int_{\gamma _{j}}f\left( z\right) dz=0$, in quanto somma
di integrali lungo il bordo di due forme differenziali chiuse, per la
formula di Green. Ma $\forall $ $j$ vale $-\int_{\gamma _{j}}f\left(
z\right) dz=\int_{-\gamma _{j}}f\left( z\right) dz=2\pi iRes\left(
f,z_{j}\right) $, da cui la tesi. $\blacksquare $

\textbf{Teo 1.22 (residui II)}%
\begin{gather*}
\text{Hp: }K\subseteq 
%TCIMACRO{\U{2102} }%
%BeginExpansion
\mathbb{C}
%EndExpansion
\text{ chiuso, }\partial K\text{ circuito semplice, }f:K^{c}\rightarrow 
%TCIMACRO{\U{2102} }%
%BeginExpansion
\mathbb{C}
%EndExpansion
\text{ \`{e} olomorfa in }K^{c}\text{ e} \\
\text{continua in }\bar{K}^{c}\text{ tranne al pi\`{u} in un numero finito
di singolarit\`{a} }z_{1},...,z_{N}\in K^{c} \\
\text{Ts: }\int_{\partial ^{-}K}f\left( z\right) dz=2\pi iRes\left( f,\infty
\right) +2\pi i\sum_{j=1}^{N}Res\left( f,z_{j}\right)
\end{gather*}

\textbf{Dim} Considero $\gamma $ circuito semplice che contenga tutte le
singolarit\`{a} e osservo che $\int_{\gamma ^{-}}f\left( z\right) dz=2\pi
iRes\left( f,\infty \right) $. Considero ora $f$ ristretta all'area
racchiusa da $\gamma $, privata di $K$ e degli intorni chiusi di tutte le
singolarit\`{a}: vale, di nuovo per la formula di Green, $\int_{\gamma
^{+}}f\left( z\right) dz+\int_{\partial K^{-}}f\left( z\right)
dz+\sum_{j=1}^{N}\int_{\gamma _{j}^{-}}f\left( z\right) dz=0$. Vale $%
\int_{\gamma ^{+}}f\left( z\right) dz=-2\pi iRes\left( f,\infty \right) $,
quindi $\int_{\gamma ^{+}}f\left( z\right) dz+\int_{\partial K^{-}}f\left(
z\right) dz+\sum_{j=1}^{N}\int_{\gamma _{j}^{-}}f\left( z\right) dz=-2\pi
iRes\left( f,\infty \right) +\int_{\partial K^{-}}f\left( z\right) dz-2\pi
i\sum_{j=1}^{N}Res\left( f,z_{j}\right) =0$, da cui la tesi. $\blacksquare $

\textbf{Corollario}%
\begin{gather*}
\text{Hp: }f\text{ \`{e} olomorfa in }%
%TCIMACRO{\U{2102} }%
%BeginExpansion
\mathbb{C}
%EndExpansion
\backslash \left\{ z_{1},...,z_{N}\right\} \\
\text{Ts: }Res\left( f,\infty \right) +\sum_{j=1}^{N}Res\left(
f,z_{j}\right) =0
\end{gather*}

Nel caso particolare in cui si abbia una sola singolarit\`{a} $z_{0}$, il
residuo all'infinito \`{e} $-a_{-1}$, l'opposto di $Res\left( f,z_{0}\right) 
$.

\textbf{Dim} Preso $\gamma $ circuito che circonda tutte le singolarit\`{a},
vale $\int_{-\gamma }f\left( z\right) dz=2\pi iRes\left( f,\infty \right) $, 
$\int_{\gamma }f\left( z\right) dz=2\pi i\sum_{j=1}^{N}Res\left(
f,z_{j}\right) $: sommando membro a membro si ha la tesi. $\blacksquare $

Si osservano ora i comportamenti degli zeri delle funzioni olomorfe.

\textbf{Teo 1.23} (\textbf{principio di identit\`{a} delle funzioni olomorfe)%
}%
\begin{gather*}
\text{Hp: }f:A\subseteq 
%TCIMACRO{\U{2102} }%
%BeginExpansion
\mathbb{C}
%EndExpansion
\rightarrow 
%TCIMACRO{\U{2102} }%
%BeginExpansion
\mathbb{C}
%EndExpansion
\text{ \`{e} olomorfa in }A\text{ connesso, }Z\left( f\right) =\left\{ z\in
A:f\left( z\right) =0\right\} \\
\text{Ts: }Z\left( f\right) =A\text{ oppure }Z\left( f\right) \text{ non ha
punti di accumulazione in }A
\end{gather*}

Gli zeri delle funzioni olomorfe non possono avere punti di accumulazione se 
$f$ non \`{e} nulla. Se $f^{\left( k\right) }=0$ $\forall $ $k=0,1,...,$,
allora $f$ \`{e} identicamente nulla nell'insieme di sviluppo.

\textbf{Teo 1.24 (unicit\`{a} del prolungamento analitico)}%
\begin{gather*}
\text{Hp: }g:E\subseteq A\subseteq 
%TCIMACRO{\U{2102} }%
%BeginExpansion
\mathbb{C}
%EndExpansion
\rightarrow 
%TCIMACRO{\U{2102} }%
%BeginExpansion
\mathbb{C}
%EndExpansion
\text{, }A\text{ connesso, }\mathcal{D}E\neq \varnothing \\
\text{Ts: }\exists \text{ al pi\`{u} una funzione }f:A\rightarrow 
%TCIMACRO{\U{2102} }%
%BeginExpansion
\mathbb{C}
%EndExpansion
\text{ olomorfa tale che }f|_{E}=g
\end{gather*}

$f$ \`{e} detta prolungamento analitico di $g$.

\textbf{Dim} Se $f_{1},f_{2}$ prolungano entrambe $g$, allora $f_{1}\left(
z\right) -f_{2}\left( z\right) =g\left( z\right) -g\left( z\right) =0$ $%
\forall $ $z\in E$, ma allora si \`{e} trovata una funzione olomorfa in $A$
connesso per cui l'insieme degli zeri \`{e} $E$, che ha punti di
accumulazione, perci\`{o} pu\`{o} essere solo $f_{1}\left( z\right)
-f_{2}\left( z\right) =0$ $\forall $ $z\in A$. $\blacksquare $

\begin{enumerate}
\item Dal teorema si deduce che anche in campo complesso vale $%
e^{z_{1}+z_{2}}=e^{z_{1}}e^{z_{2}}$ e $\sin 2z=2\sin z\cos z$, essendo
coinvolte solo funzioni olomorfe.

\item $g\left( x\right) =x\left\vert x\right\vert $ non ammette
prolungamento analitico. Infatti, se per assurdo ci fosse, per $\func{Re}z<0$
si dovrebbe avere $f\left( z\right) =-\left\vert z\right\vert ^{2}$, per $%
\func{Re}z>0$ $f\left( z\right) =\left\vert z\right\vert ^{2}$, ma sull'asse
immaginario tale $f$ non \`{e} continua e dunque non pu\`{o} essere olomorfa.
\end{enumerate}

\textbf{Corollario}%
\begin{eqnarray*}
\text{Hp} &\text{: }&f:I\subseteq 
%TCIMACRO{\U{211d} }%
%BeginExpansion
\mathbb{R}
%EndExpansion
\rightarrow 
%TCIMACRO{\U{211d} }%
%BeginExpansion
\mathbb{R}
%EndExpansion
\text{, }I\text{ \`{e} un intervallo aperto} \\
\text{Ts} &\text{: }&f\text{ \`{e} analitica in }I\Longleftrightarrow \text{
ha un prolungamento analitico in }A\subseteq 
%TCIMACRO{\U{2102}}%
%BeginExpansion
\mathbb{C}%
%EndExpansion
\end{eqnarray*}

L'implicazione inversa \`{e} ovvia. Il corollario \`{e} coerente col fatto
che $x\left\vert x\right\vert $ non \`{e} analitica, non essendo $C^{\infty
} $.

\textbf{Def} Data $f:A\rightarrow 
%TCIMACRO{\U{2102} }%
%BeginExpansion
\mathbb{C}
%EndExpansion
$ e $z_{0}$ zero isolato per $f$, si dice ordine dello zero il pi\`{u}
piccolo $n_{0}\in 
%TCIMACRO{\U{2115} }%
%BeginExpansion
\mathbb{N}
%EndExpansion
:f^{\left( n_{0}\right) }\left( z_{0}\right) \neq 0$.

Si chiede che lo zero sia isolato affinch\'{e} $f$ non sia identicamente
nulla, per il principio di identit\`{a}.

L'ordine di uno zero pu\`{o} essere caratterizzato nel seguente modo: $n_{0}$
\`{e} l'ordine dello zero $z_{0}$ per $f$ se e solo se $\lim_{z\rightarrow
z_{0}}\frac{f\left( z\right) }{\left( z-z_{0}\right) ^{n_{0}}}\in 
%TCIMACRO{\U{2102} }%
%BeginExpansion
\mathbb{C}
%EndExpansion
\backslash \left\{ 0\right\} $. Infatti, se $z_{0}$ ha ordine $n_{0}$, vale $%
f\left( z\right) =f^{\left( n_{0}\right) }\left( z_{0}\right) \left(
z-z_{0}\right) ^{n_{0}}+o\left( \left( z-z_{0}\right) ^{n_{0}}\right) $ per $%
z\rightarrow z_{0}$, e $\lim_{z\rightarrow z_{0}}\frac{f\left( z\right) }{%
\left( z-z_{0}\right) ^{n_{0}}}=\lim_{z\rightarrow z_{0}}\left( f^{\left(
n_{0}\right) }\left( z_{0}\right) +\frac{o\left( \left( z-z_{0}\right)
^{n_{0}}\right) }{\left( z-z_{0}\right) ^{n_{0}}}\right) =f^{\left(
n_{0}\right) }\left( z_{0}\right) $. Viceversa, se $\lim_{z\rightarrow z_{0}}%
\frac{f\left( z\right) }{\left( z-z_{0}\right) ^{n_{0}}}\in 
%TCIMACRO{\U{2102} }%
%BeginExpansion
\mathbb{C}
%EndExpansion
\backslash \left\{ 0\right\} $ significa che $f\left( z\right) \sim k\left(
z-z_{0}\right) ^{n_{0}}$ per $z\rightarrow z_{0}$, per cui non possono
esserci termini di grado inferiore a $n_{0}$ nello sviluppo di Taylor di $f$.

\textbf{Corollario}%
\begin{eqnarray*}
\text{Hp}\text{: } &&f:A\rightarrow 
%TCIMACRO{\U{2102} }%
%BeginExpansion
\mathbb{C}
%EndExpansion
\text{ \`{e} olomorfa in }A\text{ connesso e non identicamente nulla} \\
\text{Ts}\text{: } &&f\text{ ha al pi\`{u} zeri isolati di ordine finito}
\end{eqnarray*}

L'ordine degli zeri \`{e} ovviamente intero.

\textbf{Dim} Per il principio di identit\`{a} $Z\left( f\right) $ non ha
punti di accumulazione, quindi \`{e} vuoto oppure \`{e} formato da punti
isolati. Per definizione di ordine dello zero, $n_{0}$ \`{e} finito, perch%
\'{e} se per assurdo valesse $f^{\left( n\right) }\left( z_{0}\right) =0$ $%
\forall $ $n$ lo sviluppo di Taylor di $f$ (ben definito perch\'{e} $f$ \`{e}
olomorfa in un aperto) sarebbe nullo e $f$ sarebbe identicamente nulla,
contro l'ipotesi. $\blacksquare $

Il corollario mostra peraltro che, se $A$ \`{e} aperto, la definizione di
ordine dello zero \`{e} sempre ben posta se $z_{0}$ \`{e} uno zero isolato
per $f$ (cio\`{e} $f$ non \`{e} identicamente nulla).

Supponiamo che $f$ abbia un polo in $z_{0}$ e che $f\left( z\right) \neq 0$
in un intorno di $z_{0}$. Allora $\frac{1}{f\left( z\right) }$ \`{e} ben
definita in tale intorno a meno di $z_{0}$; ma poich\'{e} $%
\lim_{z\rightarrow z_{0}}\frac{1}{f\left( z\right) }=0$, $\frac{1}{f\left(
z\right) }$ ha una singolarit\`{a} eliminabile in $z_{0}$ e si prolunga in
modo olomorfo in $z_{0}$. Sia $g$ il prolungamento ($g\left( z_{0}\right) =0$%
): $g$, che non pu\`{o} essere identicamente nulla, soddisfa le ipotesi del
corollario e ha uno zero isolato di ordine $n$ in $z_{0}$ (quindi $\exists $ 
$\lim_{z\rightarrow z_{0}}\frac{g\left( z\right) }{\left( z-z_{0}\right) ^{n}%
}\in 
%TCIMACRO{\U{2102} }%
%BeginExpansion
\mathbb{C}
%EndExpansion
\backslash \left\{ 0\right\} $ e dunque esister\`{a} anche il limite del
reciproco $\lim_{z\rightarrow z_{0}}\left( z-z_{0}\right) ^{n}f\left(
z\right) \in 
%TCIMACRO{\U{2102} }%
%BeginExpansion
\mathbb{C}
%EndExpansion
\backslash \left\{ 0\right\} $). Viceversa, se $\lim_{z\rightarrow z_{0}}%
\frac{1}{f\left( z\right) }=0$, allora $\lim_{z\rightarrow z_{0}}f\left(
z\right) =+\infty $ e $z_{0}$ \`{e} un polo per $f$. Si \`{e} quindi detto
che $z_{0}$ \`{e} un polo per $f$ se e solo se $z_{0}$ \`{e} uno zero per
l'estensione olomorfa di $\frac{1}{f}$.

Allora la seguente definizione \`{e} ben posta.

\textbf{Def} Data $f:A\backslash \left\{ z_{0}\right\} \rightarrow 
%TCIMACRO{\U{2102} }%
%BeginExpansion
\mathbb{C}
%EndExpansion
$ olomorfa, $z_{0}$ polo per $f$ e $g$ prolungamento di $\frac{1}{f}$, si
dice ordine del polo $z_{0}$ l'ordine dello zero $z_{0}$ per $g$.

Per la caratterizzazione vista, $z_{0}$ \`{e} un polo di ordine $n$ per $f$
se e solo se $n$ \`{e} tale che $\lim_{z\rightarrow z_{0}}f\left( z\right)
\left( z-z_{0}\right) ^{n}\in 
%TCIMACRO{\U{2102} }%
%BeginExpansion
\mathbb{C}
%EndExpansion
\backslash \left\{ 0\right\} $ (o equivalentemente il pi\`{u} piccolo
naturale tale che $f\left( z\right) \left( z-z_{0}\right) ^{n}\in \mathcal{H}%
\left( A\right) $), o equivalentemente il pi\`{u} grande naturale tale che $%
a_{-n}\neq 0$. La seconda caratterizzazione si dimostra facilmente usando lo
sviluppo di Laurent.

Pi\`{u} in generale, se $f$ ha un numero finito di singolarit\`{a} $%
z_{1},...,z_{n}$ polari o eliminabili in $A$, ciascuna di ordine $n_{j}$
(per singolarit\`{a} eliminabili $n_{j}=0$), $g\left( z\right)
=\prod_{j=1}^{n}\left( z-z_{j}\right) ^{n_{j}}f\left( z\right) $ \`{e}
prolungabile a una funzione olomorfa in $A$. $f$ \`{e} in tal caso detta
meromorfa ed \`{e} scrivibile come quoziente di due funzioni olomorfe in $A$.

\begin{enumerate}
\item L'ordine $n$ del polo \`{e} dunque il numero di termini con indice
negativo presenti nello sviluppo di Laurent di $f$. Se $f\left( z\right) =%
\frac{c_{-3}}{z^{3}}+\frac{c_{-2}}{z^{2}}+\frac{c_{-1}}{z}+c_{0}+c_{1}z+...$%
, $f$ ha un polo in $0$ di ordine $3$: $z^{3}f\left( z\right)
=c_{-3}+c_{-2}z+c_{-1}z^{2}+c_{0}z^{3}+c_{1}z^{4}+...$

\item Pi\`{u} in generale, sia $f\left( z\right) =\sum_{k=\nu }^{+\infty
}c_{k}\left( z-z_{0}\right) ^{k}$, con $c_{\nu }\neq 0$. Sicuramente $z_{0}$
non \`{e} una singolarit\`{a} essenziale, altrimenti esisterebbero infiniti
termini della serie con indice negativo. Se $\nu =0$, $z_{0}$ \`{e} una
singolarit\`{a} eliminabile e il prolungamento olomorfo di $f$ vale $c_{\nu
} $ in $z_{0}$; se $\nu >0$, \`{e} eliminabile e il prolungamento ha uno
zero di ordine $\nu $ in $z_{0}$; se $\nu <0$, $z_{0}$ \`{e} un polo di
ordine $-\nu $.

Se $\nu >0$, $f^{\prime }$ ha uno zero di ordine $\nu -1$ in $z_{0}$; se $%
\nu <0$, $f^{\prime }$ ha un polo di ordine $\nu +1$ in $z_{0}$. In entrambi
i casi vale $\lim_{z\rightarrow z_{0}}\left( z-z_{0}\right) \frac{f^{\prime
}\left( z\right) }{f\left( z\right) }=\nu $. Infatti $f^{\prime }\left(
z\right) =\sum_{k=\nu }^{+\infty }c_{k}k\left( z-z_{0}\right) ^{k-1}$ (per
il corollario sulla serie di potenze della derivata; non c'\`{e} bisogno di
alcuna estensione alla serie di Laurent, perch\'{e} i termini con indice
negativo sono in numero finito), per cui $\left( z-z_{0}\right) \frac{%
f^{\prime }\left( z\right) }{f\left( z\right) }=\frac{\left( z-z_{0}\right)
\sum_{k=\nu }^{+\infty }c_{k}k\left( z-z_{0}\right) ^{k-1}}{\sum_{k=\nu
}^{+\infty }c_{k}\left( z-z_{0}\right) ^{k}}=$ $\frac{\sum_{k=\nu }^{+\infty
}c_{k}k\left( z-z_{0}\right) ^{k}}{\sum_{k=\nu }^{+\infty }c_{k}\left(
z-z_{0}\right) ^{k}}\sim \frac{c_{\nu }\nu \left( z-z_{0}\right) ^{\nu }}{%
c_{\nu }\left( z-z_{0}\right) ^{\nu }}$ per $z\rightarrow z_{0}$, per cui $%
\lim_{z\rightarrow z_{0}}\left( z-z_{0}\right) \frac{f^{\prime }\left(
z\right) }{f\left( z\right) }=\nu $.
\end{enumerate}

Vale inoltre il teorema di de L'Hopital: se $z_{0}$ \`{e} uno zero isolato o
un polo di $f$ e $g$, allora $\lim_{z\rightarrow z_{0}}\frac{f\left(
z\right) }{g\left( z\right) },\lim_{z\rightarrow z_{0}}\frac{f^{\prime
}\left( z\right) }{g^{\prime }\left( z\right) }$ esistono, finiti o
infiniti, e coincidono. Quindi, se $z_{0}$ non \`{e} un singolarit\`{a}
essenziale n\'{e} per $f$ n\'{e} per $g$, non \`{e} una singolarit\`{a}
essenziale per $\frac{f}{g}$.

\textbf{Teo 1.25 (calcolo dei residui I)} 
\begin{eqnarray*}
\text{Hp}\text{: } &&f:A\backslash \left\{ z_{0}\right\} \subseteq 
%TCIMACRO{\U{2102} }%
%BeginExpansion
\mathbb{C}
%EndExpansion
\rightarrow 
%TCIMACRO{\U{2102} }%
%BeginExpansion
\mathbb{C}
%EndExpansion
\text{ \`{e} olomorfa, }z_{0}\text{ \`{e} un polo di ordine }m\text{ per }f
\\
\text{Ts}\text{: } &&Res\left( f,z_{0}\right) =\frac{1}{\left( m-1\right) !}%
\lim_{z\rightarrow z_{0}}\frac{d^{m-1}}{dz^{m-1}}\left( f\left( z\right)
\left( z-z_{0}\right) ^{m}\right)
\end{eqnarray*}

\textbf{Dim} Poich\'{e} l'ordine di $z_{0}$ \`{e} $m$, la funzione $P\left(
z\right) =f\left( z\right) \left( z-z_{0}\right) ^{m}$ \`{e} olomorfa in $A$
e quindi analitica, cio\`{e} $P\left( z\right) =\sum_{k=0}^{+\infty }\frac{%
P^{\left( k\right) }\left( z_{0}\right) }{k!}\left( z-z_{0}\right) ^{k}$.
D'altronde $f\left( z\right) =\sum_{k=-m}^{+\infty }a_{k}\left(
z-z_{0}\right) ^{k}$, per cui vale anche $P\left( z\right)
=\sum_{k=-m}^{+\infty }a_{k}\left( z-z_{0}\right) ^{k+m}$, cio\`{e} -
traslando l'indice - $\sum_{j=0}^{+\infty }a_{j-m}\left( z-z_{0}\right) ^{j}$%
. $Res\left( f,z_{0}\right) $ \`{e} il termine $a_{-1}$: poich\'{e} $%
\sum_{k=0}^{+\infty }\frac{P^{\left( k\right) }\left( z_{0}\right) }{k!}%
\left( z-z_{0}\right) ^{k}=\sum_{j=0}^{+\infty }a_{j-m}\left( z-z_{0}\right)
^{j}$, prendendo $m=j+1$ si ottiene $a_{-1}=\frac{P^{\left( m-1\right)
}\left( z_{0}\right) }{\left( m-1\right) !}$. $\blacksquare $

\begin{enumerate}
\item $f\left( z\right) =\frac{\sin \frac{\pi }{z}}{z^{2}-1}$ ha le
singolarit\`{a} eliminabili $1,-1$ e la singolarit\`{a} essenziale $0$. $%
\infty $ \`{e} una singolarit\`{a} isolata: ne valuto la natura. Posto $w=%
\frac{1}{z},g\left( w\right) :=f\left( \frac{1}{w}\right) =\frac{w^{2}\sin
\pi w}{1-w^{2}}$. $g$ ha una singolarit\`{a} eliminabile in $0$, quindi $%
\infty $ \`{e} una singolarit\`{a} eliminabile per $f$. Voglio calcolare $%
\int_{\gamma }f\left( z\right) dz$ con $\gamma $ circuito che contiene solo $%
0$: occorre quindi $Res\left( f,0\right) $ oppure $Res\left( f,\infty
\right) ,Res\left( f,1\right) ,Res\left( f,-1\right) $. Questi sono tutti
nulli, quindi $\int_{\gamma }f\left( z\right) dz=0$.
\end{enumerate}

\textbf{Teo 1.26 (calcolo dei residui II)} 
\begin{gather*}
\text{Hp}\text{: }F:A\backslash \left\{ z_{0}\right\} \subseteq 
%TCIMACRO{\U{2102} }%
%BeginExpansion
\mathbb{C}
%EndExpansion
\rightarrow 
%TCIMACRO{\U{2102} }%
%BeginExpansion
\mathbb{C}
%EndExpansion
\text{ \`{e} olomorfa, }F\left( z\right) =\frac{f\left( z\right) }{g\left(
z\right) }\text{, }f,g\in \mathcal{H}\left( A\right) ,g\left( z_{0}\right)
=0,g^{\prime }\left( z_{0}\right) \neq 0 \\
\text{Ts}\text{: }Res\left( F,z_{0}\right) =\frac{f\left( z_{0}\right) }{%
g^{\prime }\left( z_{0}\right) }
\end{gather*}

Le ipotesi su $g$ implicano che $z_{0}$ sia un polo di ordine $1$ per $F$.
Infatti $g\left( z\right) =g^{\prime }\left( z_{0}\right) \left(
z-z_{0}\right) +o\left( z-z_{0}\right) $ per $z\rightarrow z_{0}$, quindi $%
F\left( z\right) =\frac{f\left( z\right) }{g^{\prime }\left( z_{0}\right)
\left( z-z_{0}\right) +o\left( z-z_{0}\right) }=\frac{\frac{f\left( z\right) 
}{z-z_{0}}}{g^{\prime }\left( z_{0}\right) +\frac{o\left( z-z_{0}\right) }{%
z-z_{0}}}\sim \frac{f\left( z_{0}\right) }{g^{\prime }\left( z_{0}\right)
\left( z-z_{0}\right) }$ per $z\rightarrow z_{0}$. D'altra parte $F$ \`{e}
sviluppabile in serie di Laurent, per cui sicuramente $F\left( z\right) \sim
\sum_{k=-\infty }^{+\infty }a_{k}\left( z-z_{0}\right) ^{k}$, e per
transitivit\`{a} $\sum_{k=-\infty }^{+\infty }a_{k}\left( z-z_{0}\right)
^{k}\sim \frac{f\left( z_{0}\right) }{g^{\prime }\left( z_{0}\right) \left(
z-z_{0}\right) }$ per $z\rightarrow z_{0}$. Allora non possono esserci altri
termini singolari nello sviluppo di $F$ oltre a $\frac{f\left( z_{0}\right) 
}{g^{\prime }\left( z_{0}\right) \left( z-z_{0}\right) }$, altrimenti non
varrebbe la relazione di asintotico; possono invece esserci i termini
regolari, che tendono a $0$ per $z\rightarrow z_{0}$.

\textbf{Dim }Si applica il teorema 1.25 calcolo dei residui I a $F$ con $m=1$%
: $Res\left( F,z_{0}\right) =\lim_{z\rightarrow z_{0}}F\left( z\right)
\left( z-z_{0}\right) $, e poich\'{e} $F\left( z\right) \sim \frac{f\left(
z_{0}\right) }{g^{\prime }\left( z_{0}\right) \left( z-z_{0}\right) }$ per $%
z\rightarrow z_{0}$, si ottiene $Res\left( F,z_{0}\right) =\frac{f\left(
z_{0}\right) }{g^{\prime }\left( z_{0}\right) }$. $\blacksquare $

\begin{enumerate}
\item Calcolo il residuo di $f\left( z\right) =\frac{e^{\frac{1}{z}}}{2-z}$
in $0$. Vale $e^{\frac{1}{z}}=\sum_{k=0}^{+\infty }\frac{1}{k!z^{k}}$, $%
\frac{1}{2-z}=\frac{1}{2}\frac{1}{1-\frac{z}{2}}=\frac{1}{2}%
\sum_{k=0}^{+\infty }\frac{z^{k}}{2^{k}}$. Il prodotto delle serie \`{e} $%
\frac{1}{2}\sum_{k=0}^{+\infty }\frac{1}{k!z^{k}}\sum_{j=0}^{+\infty }\frac{%
z^{j}}{2^{j}}=\footnote{%
Si ricorda che $\sum_{k=0}^{+\infty }a_{k}\sum_{j=0}^{+\infty
}b_{j}=\sum_{k=0}^{+\infty }\sum_{j=0}^{k}a_{j}b_{k-j}$.}\frac{1}{2}%
\sum_{k=0}^{+\infty }\sum_{j=0}^{k}\frac{1}{j!z^{j}}\frac{z^{k-j}}{2^{k-j}}$%
. $Res\left( f,0\right) $ \`{e} il coefficiente di $\frac{1}{z}$: $\frac{1}{z%
}$ si ottiene quando $k-2j=-1$, cio\`{e} il termine generale della serie
diventa $\frac{1}{j!}\frac{z^{-1}}{2^{j-1}}$. Vale quindi $Res\left(
f,0\right) =\frac{1}{2}\sum_{j=1}^{+\infty }\frac{1}{j!2^{j-1}}%
=\sum_{j=1}^{+\infty }\frac{1}{j!2^{j}}=e^{\frac{1}{2}}-1$.
\end{enumerate}

\subsection{Applicazioni}

Si possono anche calcolare integrali di funzioni reali usando l'analisi
complessa.

\begin{enumerate}
\item Voglio calcolare $\int_{0}^{2\pi }\frac{1}{\cos \theta +2}d\theta $.
Con la sostituzione $z=e^{i\theta }$, dato che $\cos \theta =\frac{%
e^{i\theta }+e^{-i\theta }}{2}$, si ottiene $\int_{\gamma }\frac{1}{\frac{%
e^{i\theta }+e^{-i\theta }}{2}+2}\frac{1}{ie^{i\theta }}dz$, con $\gamma $
parametrizzata da $r\left( t\right) =e^{it}$, $t\in \left[ 0,2\pi \right] $.
Poich\'{e} $\frac{1}{\frac{e^{i\theta }+e^{-i\theta }}{2}+2}\frac{1}{%
e^{i\theta }}=\frac{1}{\frac{e^{2i\theta }+1}{2}+2e^{i\theta }}=\frac{2}{%
z^{2}+4z+1}$, si ottiene $-2i\int_{\gamma }\frac{1}{z^{2}+4z+1}dz$, che pu%
\`{o} essere calcolato con i residui. $f\left( z\right) =\frac{1}{z^{2}+4z+1}
$ ha i due poli $-2\pm \sqrt{3}$: $z_{0}=-2+\sqrt{3}$ \`{e} contenuto in $%
\gamma $ e quindi produce un residuo. Per il teorema sopra, $z_{0}$ \`{e} un
polo di ordine $1$ per $f$ e pu\`{o} quindi essere calcolato come $Res\left(
f,z_{0}\right) =\frac{1}{\left( z+2+\sqrt{3}\right) |_{z_{0}}}=\frac{1}{2%
\sqrt{3}}$. Ne segue che l'integrale originario \`{e} $-2i\left( 2\pi i\frac{%
1}{2\sqrt{3}}\right) =\frac{2\pi }{\sqrt{3}}$.
\end{enumerate}

Ci interessa calcolare in particolare gli integrali coinvolti nei
coefficienti della trasformata di Fourier, che sono del tipo $\int_{-\infty
}^{+\infty }e^{i\omega x}f\left( x\right) dx$, con $\omega >0$, integrale
improprio nel senso di Riemann (pu\`{o} accadere $f\left( x\right) =O\left( 
\frac{1}{x}\right) $). Si calcola l'integrale lungo la frontiera di una
semicirconferenza superiore di raggio $R$, parametrizzata da $\gamma _{R}$,
e se ne valuta il limite per $R\rightarrow +\infty $. La correttezza di
questa tecnica \`{e} assicurata dal seguente lemma.

\textbf{Lemma (Jordan)}%
\begin{gather*}
\text{Hp: }R,\omega >0,\rho :I=\left[ 0,\pi \right] \rightarrow \lbrack
0,+\infty )\text{ }C^{1}\text{ a tratti, }C_{R}\text{ cammino } \\
\text{parametrizzato da }r_{R}:I\rightarrow 
%TCIMACRO{\U{2102} }%
%BeginExpansion
\mathbb{C}
%EndExpansion
,r_{R}\left( t\right) =R\rho \left( t\right) e^{it}\text{, }f\text{ continua
su }\gamma _{R}=\func{Im}r_{R} \\
\text{Ts: }\exists \text{ }c^{\ast }>0\text{, dipendente solo da }\rho \text{%
, tale che }\left\vert \int_{C_{R}}e^{i\omega z}f\left( z\right)
dz\right\vert \leq \frac{c_{\ast }}{\omega }\sup_{z\in \gamma
_{R}}\left\vert f\left( z\right) \right\vert
\end{gather*}

[NB: grasselli dice $c^{\ast }$ dipendente solo da $f$, vs pag. 210 gilardi]

Si ricorda che $C_{R}$ \`{e} una classe di equivalenza di parametrizzazioni, 
$r_{R}$ \`{e} una parametrizzazione e $\gamma _{R}$ \`{e} il sostegno di $%
r_{R}$.

Se $\rho $ \`{e} costante $r_{R}$ parametrizza una circonferenza, come serve
nel nostro caso.

\textbf{Corollario}%
\begin{gather*}
\text{Hp: }f:\left\{ z\in 
%TCIMACRO{\U{2102} }%
%BeginExpansion
\mathbb{C}
%EndExpansion
:\func{Im}z>0\right\} \rightarrow 
%TCIMACRO{\U{2102} }%
%BeginExpansion
\mathbb{C}
%EndExpansion
\text{ \`{e} tale che }\exists \text{ }\alpha >0: \\
\left\vert f\left( z\right) \right\vert \leq \frac{1}{\left\vert
z\right\vert ^{\alpha }}\text{ definitivamente per }\left\vert z\right\vert
\rightarrow +\infty \\
\text{Ts: }\lim_{R\rightarrow +\infty }\left\vert \int_{C_{R}}e^{i\omega
z}f\left( z\right) dz\right\vert =0
\end{gather*}

In tal caso infatti $\sup_{z\in \gamma _{R}}\left\vert f\left( z\right)
\right\vert \leq \sup_{z\in \gamma _{R}}\frac{1}{\left\vert z\right\vert
^{\alpha }}=\frac{1}{\left( \min_{\left[ 0,\pi \right] }\rho \right)
\left\vert R\right\vert ^{\alpha }}$.

Per calcolare $\int_{-\infty }^{+\infty }e^{i\omega x}f\left( x\right) dx$
si procede come segue. Si considera $\int_{C_{R}\cup \left[ -R,R\right]
}e^{i\omega z}f\left( z\right) dz=C$, dove $C_{R}\cup \left[ -R,R\right] $
indica la parametrizzazione della semicirconferenza incollata alla
parametrizzazione del segmento orizzontale. Tale integrale \`{e}
indipendente da $R$, una volta scelto $R$ abbastanza grande in modo che il
circuito contenga le eventuali singolarit\`{a} di $f$, e si pu\`{o}
calcolare con i residui. Per additivit\`{a} $C=\int_{C_{R}}e^{i\omega
z}f\left( z\right) dz+\int_{\left[ -R,R\right] }e^{i\omega z}f\left(
z\right) dz$; il secondo addendo coincide con $\int_{-R}^{R}e^{i\omega
x}f\left( x\right) dx$. Quindi, se si dimostra che $\lim_{R\rightarrow
+\infty }\int_{C_{R}}e^{i\omega z}f\left( z\right) dz=0$, si ha $%
\lim_{R\rightarrow +\infty }\int_{C_{R}\cup \left[ -R,R\right] }e^{i\omega
z}f\left( z\right) dz=C=\int_{-\infty }^{+\infty }e^{i\omega x}f\left(
x\right) dx$. Ma questo \`{e} garantito dal corollario sopra, se $\left\vert
f\left( z\right) \right\vert \leq \frac{1}{\left\vert z\right\vert ^{\alpha }%
}$ definitivamente.

\begin{enumerate}
\item Voglio calcolare $\int_{%
%TCIMACRO{\U{211d} }%
%BeginExpansion
\mathbb{R}
%EndExpansion
}e^{-i\xi x}\frac{1}{x^{2}+1}dx$. Se $\xi <0$ si applica il lemma di Jordan
con $\omega =-\xi $ e si ha $\lim_{R\rightarrow +\infty }\left\vert
\int_{C_{R}}e^{i\omega z}f\left( z\right) dz\right\vert =0$ perch\'{e} $%
f\left( z\right) =\frac{1}{z^{2}+1}$ soddisfa le ipotesi del corollario.
Allora $\int_{%
%TCIMACRO{\U{211d} }%
%BeginExpansion
\mathbb{R}
%EndExpansion
}e^{-i\xi x}\frac{1}{x^{2}+1}dx=\lim_{R\rightarrow +\infty }\int_{C_{R}\cup %
\left[ -R,R\right] }e^{i\omega z}f\left( z\right) dz=2\pi iRes\left(
e^{i\omega z}f\left( z\right) ,i\right) =\pi e^{\xi }$ (perch\'{e} il
circuito \`{e} un semicirconferenza superiore e contiene solo il polo $i$,
non $-i$).

\item Voglio calcolare $\int_{%
%TCIMACRO{\U{211d} }%
%BeginExpansion
\mathbb{R}
%EndExpansion
}e^{-i\left( \xi +5\right) x}\frac{x}{\left( 3x-2i\right) ^{2}}dx$. Per $\xi
<-5$ si applica il lemma di Jordan sulla semicirconferenza goniometrica
superiore e si calcola l'integrale con i residui. Per $\xi >-5$ si considera
invece come circuito la semicirconferenza inferiore per applicare il lemma.
!!
\end{enumerate}

Esiste poi un risultato che permette di calcolare pi\`{u} facilmente gli
integrali di funzioni reali. Si premette il seguente lemma.

\textbf{Lemma (piccolo cerchio locale) }%
\begin{gather*}
\text{Hp: }\varepsilon >0\text{, }f\in \mathcal{H}\left( 
%TCIMACRO{\U{2102} }%
%BeginExpansion
\mathbb{C}
%EndExpansion
\backslash \left\{ z_{0}\right\} \right) \text{ con }z_{0}\in 
%TCIMACRO{\U{2102} }%
%BeginExpansion
\mathbb{C}
%EndExpansion
\text{ polo di ordine }1\text{ per }f\text{; } \\
C_{\varepsilon }\left( \theta _{1},\theta _{2}\right) \text{ cammino
parametrizzato da }r_{\varepsilon }:\left[ \theta _{1},\theta _{2}\right]
\rightarrow 
%TCIMACRO{\U{2102} }%
%BeginExpansion
\mathbb{C}
%EndExpansion
,r_{\varepsilon }\left( t\right) =z_{0}+\varepsilon e^{it} \\
\text{Ts: }\lim_{\varepsilon \rightarrow 0}\int_{C_{\varepsilon }\left(
\theta _{1},\theta _{2}\right) }f\left( z\right) dz=\left( \theta
_{2}-\theta _{1}\right) iRes\left( f,z_{0}\right)
\end{gather*}

Il lemma fornisce dunque un modo per calcolare il limite dell'integrale di $%
f $ su un arco di circonferenza che insiste su un angolo di ampiezza $\theta
_{2}-\theta _{1}$.

\textbf{Dim} Poich\'{e} $z_{0}$ \`{e} un polo semplice, $f$ pu\`{o} essere
sviluppata secondo Laurent come $f\left( z\right) =\frac{a_{-1}}{z-z_{0}}%
+g\left( z\right) $, con $a_{-1}=Res\left( f,z_{0}\right) $ e $g\in \mathcal{%
H}\left( 
%TCIMACRO{\U{2102} }%
%BeginExpansion
\mathbb{C}
%EndExpansion
\right) $. Allora $\int_{C_{\varepsilon }\left( \theta _{1},\theta
_{2}\right) }f\left( z\right) dz=a_{-1}\int_{C_{\varepsilon }\left( \theta
_{1},\theta _{2}\right) }\frac{1}{z-z_{0}}dz+\int_{C_{\varepsilon }\left(
\theta _{1},\theta _{2}\right) }g\left( z\right) dz$. Il secondo addendo 
\`{e} per definizione $\varepsilon \int_{\theta _{1}}^{\theta _{2}}g\left(
z_{0}+\varepsilon e^{it}\right) ie^{it}dt$: poich\'{e} $g$ \`{e} olomorfa e
dunque continua, il secondo fattore \`{e} finito, dunque il prodotto tende a 
$0$ per $\varepsilon \rightarrow 0$. Il primo addendo \`{e} per definizione $%
a_{-1}\int_{\theta _{1}}^{\theta _{2}}\frac{1}{\varepsilon e^{it}}%
i\varepsilon e^{it}dt=Res\left( f,z_{0}\right) i\left( \theta _{2}-\theta
_{1}\right) $ $\forall $ $\varepsilon $, da cui la tesi. $\blacksquare $

\textbf{Teo 1.27 (poli reali)}%
\begin{gather*}
\text{Hp: }\varepsilon >0,f\in \mathcal{H}\left( \left\{ z\in 
%TCIMACRO{\U{2102} }%
%BeginExpansion
\mathbb{C}
%EndExpansion
:\func{Im}z>-\varepsilon \right\} \backslash \left\{
x_{1},...,x_{n},z_{1},...,z_{k}\right\} \right) \text{ con }%
x_{1},...,x_{n}\in 
%TCIMACRO{\U{211d} }%
%BeginExpansion
\mathbb{R}
%EndExpansion
\text{ } \\
\text{e }z_{1},...,z_{k}\in \left\{ z\in 
%TCIMACRO{\U{2102} }%
%BeginExpansion
\mathbb{C}
%EndExpansion
:\func{Im}z>0\right\} \text{ poli di ordine }1\text{ per }f\text{; }C_{R}%
\text{ cammino } \\
\text{parametrizzato da }r\left( t\right) =Re^{it}\text{, }t\in \left[ 0,\pi %
\right] \text{, }\lim_{R\rightarrow +\infty }\int_{C_{R}}f\left( z\right)
dz=0 \\
\text{Ts: }pv\int_{%
%TCIMACRO{\U{211d} }%
%BeginExpansion
\mathbb{R}
%EndExpansion
}f\left( x\right) dx=2\pi i\sum_{j=1}^{k}Res\left( f,z_{j}\right) +\pi
i\sum_{j=1}^{n}Res\left( f,x_{j}\right)
\end{gather*}

Se invece $z_{1},...,z_{k}\in \left\{ z\in 
%TCIMACRO{\U{2102} }%
%BeginExpansion
\mathbb{C}
%EndExpansion
:\func{Im}z<0\right\} $, nella tesi cambia di segno il termine relativo ai
residui nei poli non reali.

\textbf{Dim} Considero un circuito $S_{R,\varepsilon }$, parametrizzato in
senso antiorario, che contenga i poli complessi $z_{1},...,z_{k}$ e sia
semicircolare ad eccezione di $n$ "buchi semicircolari" che lasciano
all'esterno di $S_{R,\varepsilon }$ i poli reali: si indica con $%
S_{\varepsilon }^{i}$ il cammino che parametrizza la semicirconferenza bordo
di ogni buco siffatto, con raggio $\varepsilon $ e $i=1,...,n$. Allora, per
il teorema dei residui, $\int_{S_{R,\varepsilon }}f\left( z\right) dz=2\pi
i\sum_{j=1}^{k}Res\left( f,z_{j}\right) $. Ma $\int_{S_{R,\varepsilon
}}f\left( z\right) dz=\int_{C_{R}}f\left( z\right) dz+\int_{\left[ -R,R%
\right] \backslash \bigcup_{i=1}^{n}\left[ x_{i}-\varepsilon
,x_{i}+\varepsilon \right] }f\left( z\right)
dz+\sum_{i=1}^{n}\int_{S_{\varepsilon }^{i}}f\left( z\right) dz$. Per $%
\varepsilon \rightarrow 0$ il secondo addendo tende a $pv\int_{\left[ -R,R%
\right] }f\left( z\right) dz$, e - per il lemma del piccolo cerchio - il
terzo addendo tende a $\sum_{j=1}^{n}\left( -\pi iRes\left( f,x_{j}\right)
\right) $. Allora, sfruttando l'ipotesi, si ottiene $\lim_{R\rightarrow
+\infty }\lim_{\varepsilon \rightarrow 0}\int_{S_{R,\varepsilon }}f\left(
z\right) dz=pv\int_{%
%TCIMACRO{\U{211d} }%
%BeginExpansion
\mathbb{R}
%EndExpansion
}f\left( z\right) dz-\pi \sum_{j=1}^{n}iRes\left( f,x_{j}\right) =2\pi
i\sum_{j=1}^{k}Res\left( f,z_{j}\right) $, da cui la tesi. $\blacksquare $

Questo teorema permette di calcolare integrali di funzioni reali (che quando
esistono coincidono con il loro valore principale) sfruttando il calcolo dei
residui.

\begin{enumerate}
\item E' noto che $\int_{%
%TCIMACRO{\U{211d} }%
%BeginExpansion
\mathbb{R}
%EndExpansion
}\frac{\sin x}{x}dx<+\infty \footnote{%
Infatti $\int_{%
%TCIMACRO{\U{211d} }%
%BeginExpansion
\mathbb{R}
%EndExpansion
}\frac{\sin x}{x}dx=\sum_{k=0}^{+\infty }\int_{k\pi }^{\left( k+1\right) \pi
}\frac{\sin x}{x}dx=\int_{0}^{\pi }\frac{\sin x}{x}dx+\sum_{k=1}^{+\infty
}\int_{k\pi }^{\left( k+1\right) \pi }\frac{\sin x}{x}dx$ e $\int_{k\pi
}^{\left( k+1\right) \pi }\frac{\sin x}{x}dx=\left[ \frac{-\cos x}{x}\right]
_{k\pi }^{\left( k+1\right) \pi }+\int_{k\pi }^{\left( k+1\right) \pi }\frac{%
\cos x}{x^{2}}dx=$ $\left( \frac{\left( -1\right) ^{k}}{\pi }\left( \frac{1}{%
k+1}+\frac{1}{k}\right) \right) +\int_{k\pi }^{\left( k+1\right) \pi }\frac{%
\cos x}{x^{2}}dx$. La $\sum_{k=1}^{+\infty }$ di ci\`{o} d\`{a} una serie
convergente (per il criterio di Leibniz) pi\`{u} $\int_{\pi }^{+\infty }%
\frac{\cos x}{x^{2}}dx$.}$. Per calcolarne il valore numerico si pu\`{o}
sfruttare il teorema dei poli reali: infatti $\int_{%
%TCIMACRO{\U{211d} }%
%BeginExpansion
\mathbb{R}
%EndExpansion
}\frac{\sin x}{x}dx=\int_{%
%TCIMACRO{\U{211d} }%
%BeginExpansion
\mathbb{R}
%EndExpansion
}\frac{e^{ix}-e^{-ix}}{2ix}dx$, ma $\int_{%
%TCIMACRO{\U{211d} }%
%BeginExpansion
\mathbb{R}
%EndExpansion
}\frac{e^{ix}-e^{-ix}}{2ix}dx=\int_{%
%TCIMACRO{\U{211d} }%
%BeginExpansion
\mathbb{R}
%EndExpansion
}\frac{e^{ix}}{ix}dx$ perch\'{e} l'integrale in cui compare il coseno \`{e}
nullo. Si applica il teorema dei poli reali a $f\left( z\right) =\frac{e^{iz}%
}{iz}$, che ha il solo polo reale $z=0$, da cui $\int_{%
%TCIMACRO{\U{211d} }%
%BeginExpansion
\mathbb{R}
%EndExpansion
}\frac{\sin x}{x}dx=\pi i\frac{1}{i}=\pi $.

$\int_{-\infty }^{+\infty }\frac{\cos x}{x}dx$
\end{enumerate}

\subsection{Funzioni polidrome}

Ci si interroga sulla possibilit\`{a} di invertire alcune funzioni note,
come l'esponenziale in campo complesso.

\textbf{Logaritmo} E' noto che in $%
%TCIMACRO{\U{211d} }%
%BeginExpansion
\mathbb{R}
%EndExpansion
$, se $z>0$, $\exists $ $w\in 
%TCIMACRO{\U{211d} }%
%BeginExpansion
\mathbb{R}
%EndExpansion
:e^{w}=z$. Se invece $z\in 
%TCIMACRO{\U{2102} }%
%BeginExpansion
\mathbb{C}
%EndExpansion
\backslash \left\{ 0\right\} $, esistono infiniti $w\in 
%TCIMACRO{\U{2102} }%
%BeginExpansion
\mathbb{C}
%EndExpansion
:e^{w}=z$ e sono della forma $w=\ln \left\vert z\right\vert +i\theta $, al
variare di $\theta $ nell'insieme degli argomenti di $z$. Dall'infinit\`{a}
di queste soluzioni si ricava che non \`{e} possibile definire un'unica
funzione "inversa" dell'esponenziale, cio\`{e} una funzione logaritmo in $%
%TCIMACRO{\U{2102} }%
%BeginExpansion
\mathbb{C}
%EndExpansion
$: si definiscono allora, stabilendo un certo intervallo di appartenenza per 
$\theta $, dei rami del logaritmo, che per\`{o} non sono olomorfi.

Si definisce funzione logaritmo principale di $z$, e si indica con $Lnz$, $%
w=\ln \left\vert z\right\vert +i\theta :\theta =\arg z$ e $\theta \in \left(
-\pi ,\pi \right) $. Tale funzione \`{e} ben definita e olomorfa in $%
%TCIMACRO{\U{2102} }%
%BeginExpansion
\mathbb{C}
%EndExpansion
\backslash \left\{ z\in 
%TCIMACRO{\U{2102} }%
%BeginExpansion
\mathbb{C}
%EndExpansion
:\func{Re}z\leq 0\text{ e }\func{Im}z=0\right\} $ (il piano complesso
privato della semiretta corrispondente al "taglio" fatto con la scelta $%
\theta \in \left( -\pi ,\pi \right) $).

Analogamente al logaritmo principale, si possono definire infiniti rami
olomorfi del logaritmo con diversi tagli, ad esempio $\left( 0,2\pi \right) $%
. Questi rami non possono essere estesi con olomorfia: e. g. $f\left(
z\right) =\ln \left\vert z\right\vert +i\theta :\theta =\arg z$ e $\theta
\in \left( 0,2\pi \right) $, ben definita e olomorfa in $%
%TCIMACRO{\U{2102} }%
%BeginExpansion
\mathbb{C}
%EndExpansion
\backslash \left\{ z\in 
%TCIMACRO{\U{2102} }%
%BeginExpansion
\mathbb{C}
%EndExpansion
:\func{Re}z\geq 0\text{ e }\func{Im}z=0\right\} $ \`{e} tale che, dato $x\in 
%TCIMACRO{\U{211d} }%
%BeginExpansion
\mathbb{R}
%EndExpansion
$, $\lim_{\substack{ z\rightarrow x  \\ x\geq 0}}f\left( z\right) =\left\{ 
\begin{array}{c}
-\infty \text{ se }x=0 \\ 
\NEG{\exists}\text{ se }x>0%
\end{array}%
\right. $ (infatti, se $x>0$, scegliendo per calcolare il limite i due
cammini verticali $\func{Im}z=y>0$ e $\func{Im}z=y<0$, si ottiene
rispettivamente $\lim_{\substack{ z\rightarrow x  \\ x>0}}f\left( z\right)
=\ln \left\vert x\right\vert $, $\lim_{\substack{ z\rightarrow x  \\ x>0}}%
f\left( z\right) =\ln \left\vert x\right\vert +2\pi i$). Non \`{e} quindi
possibile definire una funzione olomorfa in $%
%TCIMACRO{\U{2102} }%
%BeginExpansion
\mathbb{C}
%EndExpansion
\backslash \left\{ 0\right\} $ che prolunghi $f$: non sarebbe continua in $%
\left\{ z\in 
%TCIMACRO{\U{2102} }%
%BeginExpansion
\mathbb{C}
%EndExpansion
:\func{Re}z>0,\func{Im}z=0\right\} $.

Dunque in generale, nonostante questi rami coincidano in una parte
dell'intersezione dei loro domini, non possono essere "incollati" in
un'unica funzione olomorfa: non esiste una funzione olomorfa che abbia tra
le sue restrizioni tutti i rami del logaritmo. Si dice quindi che il
logaritmo complesso \`{e} una funzione polidroma: non \`{e} una funzione in
senso classico (a ogni $z\neq 0$ associa un unico $w:f\left( w\right) =z$),
ma si dirama in branche che lo sono. In quest'ottica, le funzioni classiche
sono dette monodrome. In realt\`{a} una funzione polidroma potrebbe essere
monodroma se come codominio si considerasse l'insieme delle parti di $%
%TCIMACRO{\U{2102} }%
%BeginExpansion
\mathbb{C}
%EndExpansion
$, ma non \`{e} la scelta migliore per maneggiare tali funzioni.

Si pu\`{o} tuttavia stabilire una corrispondenza biunivoca (un omeomorfismo)
tra i valori assunti da tutti i rami e i punti di una superficie complessa,
detta superficie di Riemann.

\textbf{Funzioni potenza} Sia $z\neq 0,\alpha \in 
%TCIMACRO{\U{2102} }%
%BeginExpansion
\mathbb{C}
%EndExpansion
$. Il tentativo naturale di definizione della funzione potenza \`{e} $%
z^{\alpha }:=e^{\alpha \log z}$. Esplicitando il logaritmo si ottiene $%
e^{\alpha \left( \ln \left\vert z\right\vert +i\theta \right) }=\left\vert
z\right\vert ^{\func{Re}\alpha }e^{i\ln \left\vert z\right\vert \func{Im}%
\alpha }e^{i\alpha \theta }$, $\theta \in \arg z$.

Se $\alpha \in 
%TCIMACRO{\U{2124} }%
%BeginExpansion
\mathbb{Z}
%EndExpansion
$, $z^{\alpha }$ \`{e} una funzione monodroma: $z^{\alpha }=\left\vert
z\right\vert ^{n}e^{in\theta }$, che coincide con l'usuale $z^{n}$, per la
regola di De Moivre. Se $n\geq 0$ $z^{\alpha }$ \`{e} olomorfa, altrimenti
ha un polo in $0$.

Se $\alpha \in 
%TCIMACRO{\U{211a} }%
%BeginExpansion
\mathbb{Q}
%EndExpansion
\backslash 
%TCIMACRO{\U{2124} }%
%BeginExpansion
\mathbb{Z}
%EndExpansion
$, $z^{\alpha }=\left( z^{m}\right) ^{\frac{1}{n}}$ ha un numero finito di
rami (calcolare $f$ richiede infatti di estrarre una radice n-esima).

Se $\alpha \in 
%TCIMACRO{\U{2102} }%
%BeginExpansion
\mathbb{C}
%EndExpansion
\backslash 
%TCIMACRO{\U{211a} }%
%BeginExpansion
\mathbb{Q}
%EndExpansion
$, $z^{\alpha }$ ha infiniti rami.

\begin{enumerate}
\item Se $\alpha =i$, $z^{i}=\left\{ e^{i\ln \left\vert z\right\vert
}e^{-\theta }=e^{-\theta }\left( \cos \ln \left\vert z\right\vert +i\sin \ln
\left\vert z\right\vert \right) :\theta \in \arg z\right\} $. Se in
particolare $z=i$, si trova $i^{i}=e^{-\pi /2}$.
\end{enumerate}

\section{Analisi reale e funzionale}

\subsection{Spazi di Banach}

\textbf{Def} Dato un insieme $X$ non vuoto, una funzione $d:X\times
X\rightarrow \lbrack 0,+\infty )$ si dice distanza o metrica su $X$ se
soddisfa le seguenti propriet\`{a}:

\begin{description}
\item[M1] annullamento: $d\left( x,y\right) =0\Longleftrightarrow x=y$;

\item[M2] simmetria: $\forall $ $x,y\in X,$ $d\left( x,y\right) =d\left(
y,x\right) $;

\item[M3] disuguaglianza triangolare: $\forall $ $x,y,z\in X,$ $d\left(
x,y\right) \leq d\left( x,z\right) +d\left( y,z\right) $.
\end{description}

In tal caso $\left( X,d\right) $ si dice spazio metrico, o si dice che $X$ 
\`{e} uno spazio metrico rispetto alla distanza $d$.

\begin{enumerate}
\item Se $X=%
%TCIMACRO{\U{211d} }%
%BeginExpansion
\mathbb{R}
%EndExpansion
$ e $d\left( x,y\right) =\left\vert x-y\right\vert $, $\left( X,d\right) $ 
\`{e} uno spazio metrico. Pi\`{u} in generale, se $X=%
%TCIMACRO{\U{211d} }%
%BeginExpansion
\mathbb{R}
%EndExpansion
^{n}$ e $d\left( x,y\right) =\left\vert \left\vert x-y\right\vert
\right\vert $, $\left( X,d\right) $ \`{e} uno spazio metrico.

\item Ogni spazio metrico $\left( X,d\right) $ \`{e} uno spazio topologico $%
\left( X,\tau \right) $. Infatti, definito $B_{r}\left( x\right) =\left\{
y\in X:d\left( y,x\right) <r\right\} $ intorno di $x$ di raggio $r$, si pu%
\`{o} dare la definizione di punto interno (dato $A\subseteq X$, si dice che 
$x\in A$ \`{e} interno ad $A$ se $\exists $ $r:B_{r}\left( x\right)
\subseteq A$) e di parte interna (dato $A\subseteq X$, si dice parte interna
di $A$, e si indica con $A$, l'insieme dei punti interni ad $A$): allora si
chiama aperto di $X$ ogni insieme $A\subseteq X$ che coincide con la sua
parte interna, e la collezione $\tau $ degli insiemi aperti in $X$ rende $%
\left( X,\tau \right) $ uno spazio topologico.
\end{enumerate}

Per coniugare struttura metrica e struttura lineare si introduce la seguente
nozione.

\textbf{Def} Dato $X\neq \varnothing $ spazio vettoriale sul campo $%
%TCIMACRO{\U{211d} }%
%BeginExpansion
\mathbb{R}
%EndExpansion
$, si dice norma una funzione $N:X\rightarrow \lbrack 0,+\infty )$ con le
seguenti propriet\`{a}:

\begin{description}
\item[N1] annullamento: $N\left( x\right) =0\Longleftrightarrow x=0$;

\item[N2] omogeneit\`{a}: $N\left( \lambda x\right) =\left\vert \lambda
\right\vert N\left( x\right) $ $\forall $ $x\in X,\forall $ $\lambda \in 
%TCIMACRO{\U{211d} }%
%BeginExpansion
\mathbb{R}
%EndExpansion
$;

\item[N3] disuguaglianza triangolare: $N\left( x+y\right) \leq N\left(
x\right) +N\left( y\right) $ $\forall $ $x,y\in X$.
\end{description}

In tal caso la coppia $\left( X,N\right) $ si dice spazio vettoriale
normato, e $N\left( x\right) $ si indica con $\left\vert \left\vert
x\right\vert \right\vert $ o $\left\vert \left\vert x\right\vert \right\vert
_{X}$.

Si noti che le operazioni che appaiono in N1, N2, N3 sono ben definite
grazie alla struttura lineare.

\begin{enumerate}
\item $%
%TCIMACRO{\U{211d} }%
%BeginExpansion
\mathbb{R}
%EndExpansion
$ \`{e} uno spazio normato con $N\left( x\right) =\left\vert x\right\vert $; 
$%
%TCIMACRO{\U{2102} }%
%BeginExpansion
\mathbb{C}
%EndExpansion
$ lo \`{e} con $N\left( z\right) =\left\vert z\right\vert $.
\end{enumerate}

\textbf{Prop 2.1 (la norma induce una metrica)}%
\begin{gather*}
\text{Hp: }\left( X,\left\vert \left\vert \cdot \right\vert \right\vert
\right) \text{ \`{e} uno spazio vettoriale normato } \\
\text{Ts: }\left( X,d\right) \text{ \`{e} uno spazio metrico con la distanza 
}d\left( x,y\right) =\left\vert \left\vert x-y\right\vert \right\vert
\end{gather*}

Tale metrica si dice indotta dalla norma.

\textbf{Dim} E' evidente che $d\left( x,y\right) \geq 0$ $\forall $ $x,y$; $%
\left\vert \left\vert x-y\right\vert \right\vert =0\Longleftrightarrow x=y$
per N1. Vale $d\left( x,y\right) =d\left( y,x\right) $ per N2 con $\lambda
=-1$. Inoltre $d\left( x,y\right) =\left\vert \left\vert x-y\right\vert
\right\vert =\left\vert \left\vert x-z+z-y\right\vert \right\vert \leq
\left\vert \left\vert x-z\right\vert \right\vert +\left\vert \left\vert
z-y\right\vert \right\vert =d\left( x,z\right) +d\left( y,z\right) $, per
cui vale anche la disuguaglianza triangolare. $\blacksquare $

Le seguenti definizioni, seppure date per uno spazio vettoriale normato,
necessitano solo di una struttura metrica.

\textbf{Def} Dato $\left( X,\left\vert \left\vert \cdot \right\vert
\right\vert \right) $ spazio vettoriale normato, data una successione $%
\left\{ x_{n}\right\} $ a valori in $X$ e $x\in X$, si dice che in $X$ $%
x_{n} $ converge a $x$ per $n\rightarrow +\infty $, e si scrive $%
x_{n}\rightarrow ^{n\rightarrow +\infty }x$, se $\lim_{n\rightarrow +\infty
}\left\vert \left\vert x_{n}-x\right\vert \right\vert =0$.

Tale nozione \`{e} detta convergenza in norma. Il limite \`{e} unico: se $%
\lim_{n\rightarrow +\infty }\left\vert \left\vert x_{n}-x\right\vert
\right\vert =0$ e $\lim_{n\rightarrow +\infty }\left\vert \left\vert
x_{n}-x^{\ast }\right\vert \right\vert =0$, anche $\lim_{n\rightarrow
+\infty }\left( \left\vert \left\vert x_{n}-x\right\vert \right\vert
+\left\vert \left\vert x^{\ast }-x_{n}\right\vert \right\vert \right) =0$,
ma $\left\vert \left\vert x_{n}-x\right\vert \right\vert +\left\vert
\left\vert x^{\ast }-x_{n}\right\vert \right\vert \geq \left\vert \left\vert
x^{\ast }-x\right\vert \right\vert $, quindi $\lim_{n\rightarrow +\infty
}\left\vert \left\vert x^{\ast }-x\right\vert \right\vert =0$, ma essendo
l'argomento costante pu\`{o} essere solo $x^{\ast }=x$.

\textbf{Def} Dato $\left( X,\left\vert \left\vert \cdot \right\vert
\right\vert \right) $ spazio vettoriale normato e $f:X\rightarrow 
%TCIMACRO{\U{2102} }%
%BeginExpansion
\mathbb{C}
%EndExpansion
$, si dice che $f$ \`{e} continua in $x_{0}\in X$ se $\forall $ $\left\{
x_{n}\right\} \subseteq X:x_{n}\rightarrow x$ vale $f\left( x_{n}\right)
\rightarrow f\left( x\right) $.

Si noti che nella definizione sopra figura la convergenza in norma di $X$
(quando si scrive $x_{n}\rightarrow x$) e la convergenza in $%
%TCIMACRO{\U{2102} }%
%BeginExpansion
\mathbb{C}
%EndExpansion
$ (quando si scrive $f\left( x_{n}\right) \rightarrow f\left( x\right) $),
secondo l'usuale norma del modulo.

In uno spazio normato vale $\left\vert \left\vert \left\vert x\right\vert
\right\vert -\left\vert \left\vert y\right\vert \right\vert \right\vert \leq
\left\vert \left\vert x-y\right\vert \right\vert $. Se $x_{n}\rightarrow x$, 
$\lim_{n\rightarrow +\infty }\left\vert \left\vert x_{n}-x\right\vert
\right\vert =0$ e quindi $\lim_{n\rightarrow +\infty }\left\vert \left\vert
\left\vert x_{n}\right\vert \right\vert -\left\vert \left\vert x\right\vert
\right\vert \right\vert =0$, cio\`{e} $\left\vert \left\vert
x_{n}\right\vert \right\vert \rightarrow \left\vert \left\vert x\right\vert
\right\vert $, che \`{e} una convergenza in $%
%TCIMACRO{\U{211d} }%
%BeginExpansion
\mathbb{R}
%EndExpansion
$. Questo significa che la convergenza in norma implica la convergenza delle
norme, cio\`{e} la norma \`{e} una funzione continua secondo la definizione
sopra.

\textbf{Def} Dato $\left( X,\left\vert \left\vert \cdot \right\vert
\right\vert \right) $ spazio vettoriale normato, data una successione $%
\left\{ x_{n}\right\} $ a valori in $X$, si dice che $x_{n}$ \`{e} di Cauchy
se $\forall $ $\varepsilon >0$ $\exists $ $\nu :m,n>\nu \Longrightarrow
\left\vert \left\vert x_{n}-x_{m}\right\vert \right\vert <\varepsilon $.

Intuitivamente, se $\lim_{m,n\rightarrow +\infty }\left\vert \left\vert
x_{n}-x_{m}\right\vert \right\vert =0$. E' facile dimostrare che una
successione convergente \`{e} di Cauchy; il fenomeno contrario \`{e}
descritto dalla seguente definizione.

\textbf{Def} Dato $\left( X,\left\vert \left\vert \cdot \right\vert
\right\vert \right) $ spazio vettoriale normato, $\left( X,\left\vert
\left\vert \cdot \right\vert \right\vert \right) $ si dice spazio di Banach
se ogni successione di Cauchy in $X$ converge a un elemento $x\in X$.

La completezza \`{e} utile perch\'{e} permette di dedurre che una
successione converge anche senza avere idea del valore del limite della
successione, che \`{e} ci\`{o} che accade di solito.

\begin{enumerate}
\item $\left( 
%TCIMACRO{\U{211d} }%
%BeginExpansion
\mathbb{R}
%EndExpansion
,\left\vert \cdot \right\vert \right) $, $\left( 
%TCIMACRO{\U{211d} }%
%BeginExpansion
\mathbb{R}
%EndExpansion
^{n},\left\vert \left\vert \cdot \right\vert \right\vert _{p}\right) $ con $%
\left\vert \left\vert x\right\vert \right\vert _{p}=\sqrt[p]{%
\sum_{i=1}^{N}\left\vert x_{i}\right\vert ^{p}}$ e $\left( 
%TCIMACRO{\U{211d} }%
%BeginExpansion
\mathbb{R}
%EndExpansion
^{n},\left\vert \left\vert \cdot \right\vert \right\vert _{\infty }\right) $
con $\left\vert \left\vert x\right\vert \right\vert _{\infty }=\max \left\{
\left\vert x_{1}\right\vert ,...,\left\vert x_{n}\right\vert \right\} $ sono
tutti spazi di Banach.
\end{enumerate}

\textbf{Teo (funzionali lineari su spazi di Banach)} 
\begin{gather*}
\text{Hp: }\left( X,\left\vert \left\vert \cdot \right\vert \right\vert
\right) \text{ \`{e} di Banach, }F:X\rightarrow 
%TCIMACRO{\U{211d} }%
%BeginExpansion
\mathbb{R}
%EndExpansion
\left( 
%TCIMACRO{\U{2102} }%
%BeginExpansion
\mathbb{C}
%EndExpansion
\right) \text{ \`{e} un funzionale lineare} \\
\text{Ts: sono equivalenti} \\
\text{(i) }F\text{ \`{e} continuo in }0 \\
\text{(ii) }F\text{ \`{e} continuo in }X \\
\text{(iii) }\exists \text{ }k>0:\left\vert F\left( x\right) \right\vert
\leq k\left\vert \left\vert x\right\vert \right\vert \text{ }\forall \text{ }%
x\in X
\end{gather*}

(iii) \`{e} la definizione di funzionale limitato. Se $X=%
%TCIMACRO{\U{211d} }%
%BeginExpansion
\mathbb{R}
%EndExpansion
^{n}$, $k$ in (iii) \`{e} la norma della matrice che rappresenta $F$. In
generale se $X$ ha dimensione finita $F$ lineare \`{e} continuo; questo non 
\`{e} vero se $\dim X=+\infty $.


\textbf{Completezza dello spazio delle funzioni continue} $C^{0}\left( \left[
a,b\right] \right) $ \`{e} uno spazio vettoriale a dimensione infinita, perch%
\'{e} contiene il sottospazio vettoriale dei polinomi algebrici, che ha
dimensione infinita: una base algebrica di tale sottospazio \`{e} $\left\{
1,...,x^{n},...\right\} $, di cardinalit\`{a} infinita.

$C^{0}\left( \left[ a,b\right] \right) $ \`{e} inoltre uno spazio di Banach
con la norma $\left\vert \left\vert f\right\vert \right\vert _{C^{0}\left( %
\left[ a,b\right] \right) }=\max_{x\in \left[ a,b\right] }\left\vert f\left(
x\right) \right\vert $: essa \`{e} ben definita per il teorema di
Weierstrass. La convergenza di una successione $\left\{ f_{n}\right\} _{n\in 
%TCIMACRO{\U{2115} }%
%BeginExpansion
\mathbb{N}
%EndExpansion
}\subseteq C^{0}\left( \left[ a,b\right] \right) $ in tale norma \`{e} la
convergenza uniforme, cio\`{e} $\lim_{n\rightarrow +\infty }\left\vert
\left\vert f_{n}-f\right\vert \right\vert _{C^{0}\left( \left[ a,b\right]
\right) }=0$ se e solo se $f_{n}$ converge uniformemente a $f$.

Si dimostra che $\left( C^{0}\left( \left[ a,b\right] \right) ,\left\vert
\left\vert \cdot \right\vert \right\vert _{C^{0}\left( \left[ a,b\right]
\right) }\right) $ \`{e} uno spazio di Banach, cio\`{e} che ogni successione
di Cauchy \`{e} convergente. Sia $\left\{ f_{n}\right\} _{n\in 
%TCIMACRO{\U{2115} }%
%BeginExpansion
\mathbb{N}
%EndExpansion
}\subseteq C^{0}\left( \left[ a,b\right] \right) $ di Cauchy: significa che $%
\forall $ $\varepsilon >0$ $\exists $ $\nu >0:m,n>\nu \Longrightarrow
\max_{x\in \left[ a,b\right] }\left\vert f_{n}\left( x\right) -f_{m}\left(
x\right) \right\vert <\varepsilon $. Questo implica che se $m,n>\nu $ allora 
$\forall $ $x\in \left[ a,b\right] $ $\left\vert f_{n}\left( x\right)
-f_{m}\left( x\right) \right\vert <\varepsilon $. Ne segue che, fissato $%
x\in \left[ a,b\right] $, la successione numerica $f_{n}\left( x\right) $ 
\`{e} di Cauchy: ma questa \`{e} una successione in $%
%TCIMACRO{\U{211d} }%
%BeginExpansion
\mathbb{R}
%EndExpansion
$, che \`{e} di Banach, quindi essa \`{e} anche convergente. Allora posso
definire punto per punto $f\left( x\right) =\lim_{n\rightarrow +\infty
}f_{n}\left( x\right) $: devo mostrare che $\left\{ f_{n}\right\} $ converge
uniformemente a $f$. Da quanto scritto sopra si deduce che $\nu $ dipende
solo da $\varepsilon $ e non da $x$: cio\`{e} $\forall $ $\varepsilon >0$ $%
\exists $ $\nu =\nu \left( \varepsilon \right) :m,n>\nu \Longrightarrow
\forall $ $x\in \left[ a,b\right] $ $\left\vert f_{n}\left( x\right)
-f_{m}\left( x\right) \right\vert <\varepsilon $. In tal disuguaglianza si pu%
\`{o} allora passare al limite per $n\rightarrow +\infty $, per cui $\forall 
$ $\varepsilon >0$ $\exists $ $\nu :m>\nu \Longrightarrow \forall $ $x\in %
\left[ a,b\right] $ $\left\vert f_{m}\left( x\right) -f\left( x\right)
\right\vert <\varepsilon $: per definizone allora $\left\{ f_{n}\right\} $
converge uniformemente a $f$, e poich\'{e} la convergenza uniforme preserva
la continuit\`{a}, $f$ \`{e} continua e $\left\{ f_{n}\right\} _{n\in 
%TCIMACRO{\U{2115} }%
%BeginExpansion
\mathbb{N}
%EndExpansion
}$ \`{e} convergente in $C^{0}\left( \left[ a,b\right] \right) $.

Si pu\`{o} definire in $C^{0}\left( \left[ a,b\right] \right) $ anche una
norma integrale $\left\vert \left\vert f\right\vert \right\vert
_{1}:=\int_{a}^{b}\left\vert f\left( x\right) \right\vert dx$, ma con tale
norma $C^{0}\left( \left[ a,b\right] \right) $ non \`{e} completo. Infatti,
presi $a=-1,b=1$, sia $f_{n}\left( x\right) =\left\{ 
\begin{array}{c}
-1\text{ se }x\in \left[ -1,-1/n\right] \\ 
nx\text{ se }\left\vert x\right\vert <\frac{1}{n} \\ 
1\text{ se }x\in \left[ \frac{1}{n},1\right]%
\end{array}%
\right. $: si mostra che \`{e} di Cauchy, ma non converge secondo tale norma
integrale. $f_{n}$ converge puntualmente a $f\left( x\right) =\left\{ 
\begin{array}{c}
-1\text{ se }x\in \lbrack -1,0) \\ 
0\text{ se }x=0 \\ 
1\text{ se }x\in (0,1]%
\end{array}%
\right. $ e $\left\vert \left\vert f_{n}-f\right\vert \right\vert
_{1}=\int_{-1}^{1}\left\vert f_{n}\left( x\right) -f\left( x\right)
\right\vert dx=\int_{-1/n}^{0}\left\vert nx+1\right\vert
dx+\int_{0}^{1/n}\left\vert nx-1\right\vert dx=\frac{1}{n}$ (da questo so gi%
\`{a} che $f_{n}\rightarrow f$ in norma integrale, ma $f$ non \`{e}
continua). Allora $\forall $ $\varepsilon >0$ $\exists $ $\nu :n,m>\nu $
implica, separatamente, $\left\vert \left\vert f_{n}-f\right\vert
\right\vert _{1}<\varepsilon $ e $\left\vert \left\vert f_{m}-f\right\vert
\right\vert _{1}<\varepsilon $ (dato che $\frac{1}{n},\frac{1}{m}\rightarrow
0$ per $n,m\rightarrow +\infty $). Ne segue per disuguaglianza triangolare
che $\left\vert \left\vert f_{n}-f_{m}\right\vert \right\vert _{1}\leq
\left\vert \left\vert f_{n}-f\right\vert \right\vert _{1}+\left\vert
\left\vert f_{m}-f\right\vert \right\vert _{1}<2\varepsilon $, dunque $%
\left\{ f_{n}\right\} $ \`{e} di Cauchy secondo la norma integrale.
Tuttavia, $\left\{ f_{n}\right\} $ non converge a una funzione continua in
tale norma. Infatti, si supponga per assurdo che esista $f^{\ast }\in
C^{0}\left( \left[ a,b\right] \right) :\lim_{n\rightarrow +\infty
}\int_{-1}^{1}\left\vert f_{n}\left( x\right) -f^{\ast }\left( x\right)
\right\vert dx=0$. $f_{n},f^{\ast }$ sono continue in un compatto, quindi
certamente limitate, dunque $\left\vert f_{n}\left( x\right) -f^{\ast
}\left( x\right) \right\vert $ \`{e} limitata e per convergenza dominata
vale $\lim_{n\rightarrow +\infty }\int_{-1}^{1}\left\vert f_{n}\left(
x\right) -f^{\ast }\left( x\right) \right\vert dx=\int_{-1}^{1}\left\vert
f\left( x\right) -f^{\ast }\left( x\right) \right\vert dx=0$. Essendo la
funzione integranda nonnegativa, questo implica anche $\int_{0}^{1}\left%
\vert f\left( x\right) -f^{\ast }\left( x\right) \right\vert dx=0$, cio\`{e} 
$f^{\ast }\left( x\right) =1$ $\forall $ $x\in (0,1)$, e $%
\int_{-1}^{0}\left\vert f\left( x\right) -f^{\ast }\left( x\right)
\right\vert dx=0$, cio\`{e} $f^{\ast }\left( x\right) =-1$ $\forall $ $x\in
\left( -1,0\right) $, per cui $f^{\ast }$ non \`{e} continua, che \`{e}
assurdo.

Evidentemente la norma integrale definita (che pi\`{u} avanti prender\`{a}
il nome di norma $L^{1}$) non \`{e} adatta a $C^{0}$, e in effetti non ha
alcun legame con la continuit\`{a}, a differenza della norma del max.

Visto questo esempio, \`{e} naturale chiedersi quale sia invece uno spazio a
cui questa norma integrale \`{e} adatta, e pi\`{u} in gnerale se esistono
spazi di funzioni che sono di Banach secondo norme integrali. Per rispondere
positivamente occorre una nuova nozione di integrale.

\subsection{Integrale di Lebesgue}

\textbf{Def} Dato $R\subseteq 
%TCIMACRO{\U{211d} }%
%BeginExpansion
\mathbb{R}
%EndExpansion
^{n}$, $R$ si dice rettangolo pluridimensionale se $R=\left(
a_{1},b_{1}\right) \times ...\times \left( a_{n},b_{n}\right) $ con $%
a_{j}\leq b_{j}$ $\forall $ $j=1,...,n$.

\textbf{Def} Dato $R=\left( a_{1},b_{1}\right) \times ...\times \left(
a_{n},b_{n}\right) \subseteq 
%TCIMACRO{\U{211d} }%
%BeginExpansion
\mathbb{R}
%EndExpansion
^{n}$ rettangolo a $n$ dimensioni, si dice misura di $R$ in $%
%TCIMACRO{\U{211d} }%
%BeginExpansion
\mathbb{R}
%EndExpansion
^{n}$ il numero reale nonnegativo $\left\vert R\right\vert
_{n}:=\prod_{i=1}^{n}\left( b_{i}-a_{i}\right) $.

\textbf{Def} Dato $A\subseteq 
%TCIMACRO{\U{211d} }%
%BeginExpansion
\mathbb{R}
%EndExpansion
^{n}$, $A$ si dice di misura nulla secondo Lebesgue se $\forall $ $%
\varepsilon >0$ $\exists $ $\left\{ R_{k}\right\} _{k\in 
%TCIMACRO{\U{2115} }%
%BeginExpansion
\mathbb{N}
%EndExpansion
}:\bigcup_{k=1}^{+\infty }R_{k}\supseteq A$ e $\sum_{k=1}^{+\infty
}\left\vert R_{k}\right\vert _{n}<\varepsilon $. In tal caso si pone $%
\left\vert A\right\vert _{n}=0$.

Un insieme \`{e} quindi di misura nulla se pu\`{o} essere ricoperto da una
famiglia numerabile di rettangoli di misura totale arbitrariamente piccola.
Questa definizione si distingue da quella di misura di Peano-Jordan perch%
\'{e} la famiglia di rettangoli considerata in generale \`{e} numerabile,
non necessariamente finita.

\begin{enumerate}
\item Un punto $\left( x,y\right) $ nel piano ha misura nulla secondo
Lebesgue. Infatti $R_{k}=\left( x-\frac{1}{2k}\sqrt{\frac{6\varepsilon }{%
2\pi ^{2}}},x+\frac{1}{2k}\sqrt{\frac{6\varepsilon }{2\pi ^{2}}}\right)
\times \left( y-\frac{1}{2k}\sqrt{\frac{6\varepsilon }{2\pi ^{2}}},y+\frac{1%
}{2k}\sqrt{\frac{6\varepsilon }{2\pi ^{2}}}\right) $ \`{e} tale che $%
\sum_{k=1}^{+\infty }\left\vert R_{k}\right\vert _{n}=\sum_{k=1}^{+\infty }%
\frac{6\varepsilon }{2\pi ^{2}}\frac{1}{k^{2}}=\frac{\varepsilon }{2}%
<\varepsilon $.

\item Una conseguenza della definizione \`{e} che un'unione numerabile di
insiemi di misura nulla ha misura nulla: se $\left\{ A_{k}\right\} _{k\in 
%TCIMACRO{\U{2115} }%
%BeginExpansion
\mathbb{N}
%EndExpansion
}$ \`{e} tale che $\left\vert A_{k}\right\vert _{n}=0$ $\forall $ $k$,
allora $\left\vert \bigcup_{k=1}^{+\infty }A_{k}\right\vert _{n}=0$. Infatti 
$\forall $ $k$ $\exists $ $\left\{ R_{i,k}\right\} _{i\in 
%TCIMACRO{\U{2115} }%
%BeginExpansion
\mathbb{N}
%EndExpansion
}:\bigcup_{i=1}^{+\infty }R_{i,k}\supseteq A_{k}$ e $\sum_{i=1}^{+\infty
}\left\vert R_{i,k}\right\vert _{n}<\varepsilon $. Dato che si pu\`{o}
scegliere $\varepsilon $ arbitrariamente piccolo, $\forall $ $k$ si ha anche
che $\exists $ $\left\{ R_{i,k}\right\} _{i\in 
%TCIMACRO{\U{2115} }%
%BeginExpansion
\mathbb{N}
%EndExpansion
}:\bigcup_{i=1}^{+\infty }R_{i,k}\supseteq A_{k}$ e $\sum_{i=1}^{+\infty
}\left\vert R_{i,k}\right\vert _{n}<\frac{\varepsilon }{2^{k}}$. Allora $%
\bigcup_{k=1}^{+\infty }A_{k}$ \`{e} ricoperta da $\bigcup_{k=1}^{+\infty
}\bigcup_{i=1}^{+\infty }R_{i,k}$, tale che $\sum_{k=1}^{+\infty }\left\vert
\bigcup_{i=1}^{+\infty }R_{i,k}\right\vert _{n}=\sum_{k=1}^{+\infty
}\sum_{i=1}^{+\infty }\left\vert R_{i,k}\right\vert _{n}<\sum_{k=1}^{+\infty
}\frac{\varepsilon }{2^{k}}=\varepsilon $.

E. g. $%
%TCIMACRO{\U{211a} }%
%BeginExpansion
\mathbb{Q}
%EndExpansion
$, unione numerabile di singoletti, in $%
%TCIMACRO{\U{211d} }%
%BeginExpansion
\mathbb{R}
%EndExpansion
$ ha misura nulla. L'insieme degli insiemi numerabili in $%
%TCIMACRO{\U{211d} }%
%BeginExpansion
\mathbb{R}
%EndExpansion
$ \`{e} strettamente contenuto nell'insieme degli insiemi di misura nulla:
e. g. l'insieme di Cantor ha misura nulla ma non \`{e} numerabile.
\end{enumerate}

Vale la propriet\`{a} di completezza: se $\left\vert A\right\vert _{n}=0$, $%
\forall $ $B\subseteq A$ $\left\vert B\right\vert _{n}=0$.

\textbf{Def} Si dice che una propriet\`{a} $p\left( \mathbf{x}\right) $ vale
quasi ovunque in $%
%TCIMACRO{\U{211d} }%
%BeginExpansion
\mathbb{R}
%EndExpansion
^{n}$ se $\left\{ \mathbf{x}\in 
%TCIMACRO{\U{211d} }%
%BeginExpansion
\mathbb{R}
%EndExpansion
^{n}:\text{non }p\left( \mathbf{x}\right) \right\} $ ha misura nulla.

\begin{enumerate}
\item $I_{%
%TCIMACRO{\U{211a} }%
%BeginExpansion
\mathbb{Q}
%EndExpansion
}\left( x\right) =0$ q. o. in $%
%TCIMACRO{\U{211d} }%
%BeginExpansion
\mathbb{R}
%EndExpansion
$.

\item $f\left( x\right) =\frac{1}{x}I_{\left( x\neq 0\right) }$ \`{e}
continua q. o. in $%
%TCIMACRO{\U{211d} }%
%BeginExpansion
\mathbb{R}
%EndExpansion
$.

\item $f_{n}\left( x\right) =\arctan \left( n\left\vert x\right\vert \right) 
$ converge a $\frac{\pi }{2}$ q. o. in $%
%TCIMACRO{\U{211d} }%
%BeginExpansion
\mathbb{R}
%EndExpansion
$ (infatti solo se $x=0$ $f_{n}\left( x\right) \rightarrow ^{n\rightarrow
+\infty }0\neq \frac{\pi }{2}$).

\item $f\left( x,y\right) =\frac{1}{\left\vert x^{2}-y^{2}\right\vert }%
I_{\left( x\neq \pm y\right) }$ \`{e} positiva q. o. in $%
%TCIMACRO{\U{211d} }%
%BeginExpansion
\mathbb{R}
%EndExpansion
^{2}$, ove le rette hanno misura nulla.
\end{enumerate}

In generale, in $%
%TCIMACRO{\U{211d} }%
%BeginExpansion
\mathbb{R}
%EndExpansion
^{n}$ ogni sottospazio vettoriale di dimensione inferiore a $n$ ha misura
nulla.

\textbf{Def} Data $f$ a valori in $%
%TCIMACRO{\U{211d} }%
%BeginExpansion
\mathbb{R}
%EndExpansion
\left( 
%TCIMACRO{\U{2102} }%
%BeginExpansion
\mathbb{C}
%EndExpansion
\right) $, si dice che $f$ \`{e} definita q. o. in $%
%TCIMACRO{\U{211d} }%
%BeginExpansion
\mathbb{R}
%EndExpansion
^{n}$ se il suo dominio pu\`{o} essere scritto come $%
%TCIMACRO{\U{211d} }%
%BeginExpansion
\mathbb{R}
%EndExpansion
^{n}\backslash E$ con $E:\left\vert E\right\vert _{n}=0$.

\textbf{Def} Dati $R_{1},...,R_{m}$ $n$-rettangoli disgiunti, una funzione $%
h:%
%TCIMACRO{\U{211d} }%
%BeginExpansion
\mathbb{R}
%EndExpansion
^{n}\rightarrow 
%TCIMACRO{\U{211d} }%
%BeginExpansion
\mathbb{R}
%EndExpansion
\left( 
%TCIMACRO{\U{2102} }%
%BeginExpansion
\mathbb{C}
%EndExpansion
\right) $ della forma $h\left( \mathbf{x}\right)
=\sum_{j=1}^{m}h_{j}I_{R_{j}}\left( \mathbf{x}\right) $, con $h_{j}\in 
%TCIMACRO{\U{211d} }%
%BeginExpansion
\mathbb{R}
%EndExpansion
\left( 
%TCIMACRO{\U{2102} }%
%BeginExpansion
\mathbb{C}
%EndExpansion
\right) $ $\forall $ $j$, si dice funzione semplice o a scala.

L'insieme delle funzioni semplici \`{e} uno spazio vettoriale su $%
%TCIMACRO{\U{211d} }%
%BeginExpansion
\mathbb{R}
%EndExpansion
\left( 
%TCIMACRO{\U{2102} }%
%BeginExpansion
\mathbb{C}
%EndExpansion
\right) $ rispetto alle operazioni canoniche\footnote{%
da approfondire dim}.

\textbf{Def} $u:%
%TCIMACRO{\U{211d} }%
%BeginExpansion
\mathbb{R}
%EndExpansion
^{n}\rightarrow 
%TCIMACRO{\U{211d} }%
%BeginExpansion
\mathbb{R}
%EndExpansion
\left( 
%TCIMACRO{\U{2102} }%
%BeginExpansion
\mathbb{C}
%EndExpansion
\right) $ si dice funzione misurabile secondo Lebesgue se esiste una
successione di funzioni semplici $\left\{ h_{n}\right\} _{n\in 
%TCIMACRO{\U{2115} }%
%BeginExpansion
\mathbb{N}
%EndExpansion
}:h_{n}\rightarrow ^{n\rightarrow +\infty }u$ q. o. in $%
%TCIMACRO{\U{211d} }%
%BeginExpansion
\mathbb{R}
%EndExpansion
^{n}$.

Questa \`{e} l'unica regolarit\`{a} che si richiede alle funzioni per essere
candidate a essere integrabili.

\textbf{Prop 2.2}%
\begin{eqnarray*}
\text{Hp}\text{: } &&\mathbf{u}:%
%TCIMACRO{\U{211d} }%
%BeginExpansion
\mathbb{R}
%EndExpansion
^{n}\rightarrow I\subseteq 
%TCIMACRO{\U{211d} }%
%BeginExpansion
\mathbb{R}
%EndExpansion
^{m}\text{ \`{e} misurabile, }F:I\rightarrow 
%TCIMACRO{\U{211d} }%
%BeginExpansion
\mathbb{R}
%EndExpansion
\left( 
%TCIMACRO{\U{2102} }%
%BeginExpansion
\mathbb{C}
%EndExpansion
\right) \text{ \`{e} continua} \\
\text{Ts}\text{: } &&F\circ \mathbf{u}\text{ \`{e} misurabile}
\end{eqnarray*}

Ne segue che somma, prodotto, quoziente di funzioni misurabili \`{e}
misurabile, dove $F\left( x,y\right) =x+y,xy,\frac{x}{y}I_{\left( y\neq
0\right) }$ rispettivamente.

\textbf{Prop 2.3}%
\begin{gather*}
\text{Hp}\text{: }\left\{ u_{n}\right\} _{n\in 
%TCIMACRO{\U{2115} }%
%BeginExpansion
\mathbb{N}
%EndExpansion
}\text{ \`{e} una successione di funzioni misurabili, }u_{n}\rightarrow
^{n\rightarrow +\infty }u\text{ q. o. in }%
%TCIMACRO{\U{211d} }%
%BeginExpansion
\mathbb{R}
%EndExpansion
^{n} \\
\text{Ts}\text{: }u\text{ \`{e} misurabile}
\end{gather*}

In generale la misurabilit\`{a} \`{e} una propriet\`{a} estremamente debole: 
\`{e} difficile costruire funzioni non misurabili.

\textbf{Def} Data $h:%
%TCIMACRO{\U{211d} }%
%BeginExpansion
\mathbb{R}
%EndExpansion
^{n}\rightarrow 
%TCIMACRO{\U{211d} }%
%BeginExpansion
\mathbb{R}
%EndExpansion
\left( 
%TCIMACRO{\U{2102} }%
%BeginExpansion
\mathbb{C}
%EndExpansion
\right) ,h\left( \mathbf{x}\right) =\sum_{j=1}^{m}h_{j}I_{R_{j}}\left( 
\mathbf{x}\right) $ funzione semplice, si dice integrale di Lebesgue di $h$,
e si indica con $\int_{%
%TCIMACRO{\U{211d} }%
%BeginExpansion
\mathbb{R}
%EndExpansion
^{n}}h\left( \mathbf{x}\right) d\mathbf{x}$, il numero reale (complesso) $%
\sum_{j=1}^{m}h_{j}\left\vert R_{j}\right\vert _{n}$.

\textbf{Def} Data $u:%
%TCIMACRO{\U{211d} }%
%BeginExpansion
\mathbb{R}
%EndExpansion
^{n}\rightarrow 
%TCIMACRO{\U{211d} }%
%BeginExpansion
\mathbb{R}
%EndExpansion
\left( 
%TCIMACRO{\U{2102} }%
%BeginExpansion
\mathbb{C}
%EndExpansion
\right) $, si dice che $u$ \`{e} integrabile secondo Lebesgue in $%
%TCIMACRO{\U{211d} }%
%BeginExpansion
\mathbb{R}
%EndExpansion
^{n}$ se valgono le seguenti condizioni:

\begin{description}
\item[I1] esiste una successione di funzioni semplici $\left\{ h_{n}\right\}
_{n\in 
%TCIMACRO{\U{2115} }%
%BeginExpansion
\mathbb{N}
%EndExpansion
}:h_{n}\rightarrow ^{n\rightarrow +\infty }u$ q. o. in $%
%TCIMACRO{\U{211d} }%
%BeginExpansion
\mathbb{R}
%EndExpansion
^{n}$

\item[I2] $\forall $ $\varepsilon >0$ $\exists $ $n_{0}=n_{0}\left(
\varepsilon \right) :m,n>n_{0}\Longrightarrow \int_{%
%TCIMACRO{\U{211d} }%
%BeginExpansion
\mathbb{R}
%EndExpansion
^{n}}\left\vert h_{m}\left( \mathbf{x}\right) -h_{n}\left( \mathbf{x}\right)
\right\vert d\mathbf{x}<\varepsilon $.
\end{description}

In tal caso $\lim_{n\rightarrow +\infty }\int_{%
%TCIMACRO{\U{211d} }%
%BeginExpansion
\mathbb{R}
%EndExpansion
^{n}}h_{n}\left( \mathbf{x}\right) d\mathbf{x}$ si dice integrale di
Lebesgue di $u$ in $%
%TCIMACRO{\U{211d} }%
%BeginExpansion
\mathbb{R}
%EndExpansion
^{n}$, e si indica con $\int_{%
%TCIMACRO{\U{211d} }%
%BeginExpansion
\mathbb{R}
%EndExpansion
^{n}}u\left( \mathbf{x}\right) d\mathbf{x}$. Dunque, come si vedr\`{a} pi%
\`{u} avanti, se $u$ \`{e} integrabile esiste una successione di funzioni
semplici che converge a $u$ in $L^{1}$.

I1 \`{e} equivalente a chiedere che $u$ sia misurabile e ci\`{o} fa s\`{\i}
che l'integrale in I2 sia ben posto. I2 implica che la successione $\left\{
\int_{%
%TCIMACRO{\U{211d} }%
%BeginExpansion
\mathbb{R}
%EndExpansion
^{n}}h_{n}\left( \mathbf{x}\right) d\mathbf{x}\right\} _{n\in 
%TCIMACRO{\U{2115} }%
%BeginExpansion
\mathbb{N}
%EndExpansion
}$ sia di Cauchy: infatti, se $\forall $ $\varepsilon >0$ $\exists $ $%
n_{0}:m,n>n_{0}\Longrightarrow \int_{%
%TCIMACRO{\U{211d} }%
%BeginExpansion
\mathbb{R}
%EndExpansion
^{n}}\left\vert h_{m}\left( \mathbf{x}\right) -h_{n}\left( \mathbf{x}\right)
\right\vert d\mathbf{x}<\varepsilon $, allora $\left\vert \int_{%
%TCIMACRO{\U{211d} }%
%BeginExpansion
\mathbb{R}
%EndExpansion
^{n}}h_{n}\left( \mathbf{x}\right) d\mathbf{x-}\int_{%
%TCIMACRO{\U{211d} }%
%BeginExpansion
\mathbb{R}
%EndExpansion
^{n}}h_{m}\left( \mathbf{x}\right) d\mathbf{x}\right\vert \leq \int_{%
%TCIMACRO{\U{211d} }%
%BeginExpansion
\mathbb{R}
%EndExpansion
^{n}}\left\vert h_{m}\left( \mathbf{x}\right) -h_{n}\left( \mathbf{x}\right)
\right\vert d\mathbf{x}<\varepsilon $. In tal caso, la successione $\left\{
\int_{%
%TCIMACRO{\U{211d} }%
%BeginExpansion
\mathbb{R}
%EndExpansion
^{n}}h_{n}\left( \mathbf{x}\right) d\mathbf{x}\right\} _{n\in 
%TCIMACRO{\U{2115} }%
%BeginExpansion
\mathbb{N}
%EndExpansion
}$ converge e $\lim_{n\rightarrow +\infty }\int_{%
%TCIMACRO{\U{211d} }%
%BeginExpansion
\mathbb{R}
%EndExpansion
^{n}}h_{n}\left( \mathbf{x}\right) d\mathbf{x}=\alpha \in 
%TCIMACRO{\U{211d} }%
%BeginExpansion
\mathbb{R}
%EndExpansion
\left( 
%TCIMACRO{\U{2102} }%
%BeginExpansion
\mathbb{C}
%EndExpansion
\right) $: si pu\`{o} dimostrare che $\alpha $ non dipende dalla scelta
della successione $\left\{ h_{n}\right\} _{n\in 
%TCIMACRO{\U{2115} }%
%BeginExpansion
\mathbb{N}
%EndExpansion
}$, dunque la definizione \`{e} ben posta.

\textbf{Prop 2.4} (\textbf{propriet\`{a} dell'integrale di Lebesgue I)}%
\begin{gather*}
\text{Hp: }u:%
%TCIMACRO{\U{211d} }%
%BeginExpansion
\mathbb{R}
%EndExpansion
^{n}\rightarrow 
%TCIMACRO{\U{211d} }%
%BeginExpansion
\mathbb{R}
%EndExpansion
\left( 
%TCIMACRO{\U{2102} }%
%BeginExpansion
\mathbb{C}
%EndExpansion
\right) \text{ \`{e} integrabile secondo Lebesgue} \\
\text{Ts: (i) se }u=v\text{ q. o., allora }v\text{ \`{e} integrabile e }%
\int_{%
%TCIMACRO{\U{211d} }%
%BeginExpansion
\mathbb{R}
%EndExpansion
^{n}}u\left( \mathbf{x}\right) d\mathbf{x=}\int_{%
%TCIMACRO{\U{211d} }%
%BeginExpansion
\mathbb{R}
%EndExpansion
^{n}}v\left( \mathbf{x}\right) d\mathbf{x} \\
\text{(ii) }\left\vert u\right\vert \text{ \`{e} integrabile}
\end{gather*}

(ii) \`{e} una novit\`{a} dell'integrale di Lebesgue e segue da I2; vale
anche l'implicazione inversa.

\begin{enumerate}
\item Siano $f\left( x\right) =I_{%
%TCIMACRO{\U{211a} }%
%BeginExpansion
\mathbb{Q}
%EndExpansion
\cap \left[ -1,1\right] }-I_{\left( 
%TCIMACRO{\U{211d} }%
%BeginExpansion
\mathbb{R}
%EndExpansion
\backslash 
%TCIMACRO{\U{211a} }%
%BeginExpansion
\mathbb{Q}
%EndExpansion
\right) \cap \left[ -1,1\right] }$, $g\left( x\right) =-I_{\left( \left\vert
x\right\vert \leq 1\right) }$: $g$ \`{e} semplice e $\int_{%
%TCIMACRO{\U{211d} }%
%BeginExpansion
\mathbb{R}
%EndExpansion
^{n}}g\left( x\right) dx=-2$, ma $g=f$ q. o., quindi anche $\int_{%
%TCIMACRO{\U{211d} }%
%BeginExpansion
\mathbb{R}
%EndExpansion
^{n}}f\left( x\right) dx=-2$.
\end{enumerate}

Esistono quindi funzioni Riemann-integrabili ma non Lebesgue su intervalli
illimitati, e. g. $u\left( x\right) =\frac{\sin x}{x}$. Infatti $\left\vert
u\right\vert $ non \`{e} Lebesgue-integrabile.

\textbf{Prop 2.5} (\textbf{propriet\`{a} dell'integrale di Lebesgue II)}%
\begin{gather*}
\text{Hp: }u,v:%
%TCIMACRO{\U{211d} }%
%BeginExpansion
\mathbb{R}
%EndExpansion
^{n}\rightarrow 
%TCIMACRO{\U{211d} }%
%BeginExpansion
\mathbb{R}
%EndExpansion
\left( 
%TCIMACRO{\U{2102} }%
%BeginExpansion
\mathbb{C}
%EndExpansion
\right) \text{ } \\
\text{Ts: (i) }u\text{ \`{e} integrabile }\Longleftrightarrow u^{+},u^{-}%
\text{ sono integrabili} \\
\text{(ii) se }u,v\text{ sono integrabili, }\max \left\{ u,v\right\} ,\min
\left\{ u,v\right\} \text{ sono integrabili} \\
\text{(iii) se }u\text{ \`{e} reale, nonnegativa e integrabile, }\int_{%
%TCIMACRO{\U{211d} }%
%BeginExpansion
\mathbb{R}
%EndExpansion
^{n}}u\left( \mathbf{x}\right) d\mathbf{x}\geq 0 \\
\text{(iv) se }u,v\text{ sono integrabili e }u\leq v\text{ q. o., }\int_{%
%TCIMACRO{\U{211d} }%
%BeginExpansion
\mathbb{R}
%EndExpansion
^{n}}u\left( \mathbf{x}\right) d\mathbf{x}\leq \int_{%
%TCIMACRO{\U{211d} }%
%BeginExpansion
\mathbb{R}
%EndExpansion
^{n}}v\left( \mathbf{x}\right) d\mathbf{x} \\
\text{(v) se }u\text{ \`{e} integrabile, }\left\vert \int_{%
%TCIMACRO{\U{211d} }%
%BeginExpansion
\mathbb{R}
%EndExpansion
^{n}}u\left( \mathbf{x}\right) d\mathbf{x}\right\vert \leq \int_{%
%TCIMACRO{\U{211d} }%
%BeginExpansion
\mathbb{R}
%EndExpansion
^{n}}\left\vert u\left( \mathbf{x}\right) \right\vert d\mathbf{x} \\
\text{(vi) se }u,v\text{ sono integrabili, }\alpha u+\beta v\text{ \`{e}
integrabile }\forall \text{ }\alpha ,\beta \in 
%TCIMACRO{\U{211d} }%
%BeginExpansion
\mathbb{R}
%EndExpansion
\left( 
%TCIMACRO{\U{2102} }%
%BeginExpansion
\mathbb{C}
%EndExpansion
\right)
\end{gather*}

(iii) (positivit\`{a} dell'integrale), (iv) (monotonia dell'integrale), (v)
(disuguaglianza triangolare), (vi) (linearit\`{a} dell'integrale) sono
propriet\`{a} ben note anche dell'integrale di Riemann.

\textbf{Teo 2.6}%
\begin{gather*}
\text{Hp: }u:%
%TCIMACRO{\U{211d} }%
%BeginExpansion
\mathbb{R}
%EndExpansion
^{n}\rightarrow 
%TCIMACRO{\U{211d} }%
%BeginExpansion
\mathbb{R}
%EndExpansion
\left( 
%TCIMACRO{\U{2102} }%
%BeginExpansion
\mathbb{C}
%EndExpansion
\right) \text{ \`{e} misurabile, }\exists \text{ }\phi :%
%TCIMACRO{\U{211d} }%
%BeginExpansion
\mathbb{R}
%EndExpansion
^{n}\rightarrow 
%TCIMACRO{\U{211d} }%
%BeginExpansion
\mathbb{R}
%EndExpansion
\left( 
%TCIMACRO{\U{2102} }%
%BeginExpansion
\mathbb{C}
%EndExpansion
\right) :\left\vert u\right\vert \leq \phi \text{ q. o. e }\phi \text{ \`{e}
integrabile} \\
\text{Ts: }u\text{ \`{e} integrabile}
\end{gather*}

Una funzione misurabile dominata quasi ovunque da una funzione integrabile 
\`{e} dunque integrabile.

\textbf{Teo 2.7 (convergenza dominata)} 
\begin{gather*}
\text{Hp: }u_{n}:%
%TCIMACRO{\U{211d} }%
%BeginExpansion
\mathbb{R}
%EndExpansion
^{n}\rightarrow 
%TCIMACRO{\U{211d} }%
%BeginExpansion
\mathbb{R}
%EndExpansion
\text{ }\forall \text{ }n\text{, }\left\{ u_{n}\right\} _{n\in 
%TCIMACRO{\U{2115} }%
%BeginExpansion
\mathbb{N}
%EndExpansion
}\text{ successione di funzioni misurabili,} \\
u_{n}\rightarrow ^{n\rightarrow +\infty }u\text{ q. o., }\phi \text{ \`{e}
integrabile, }\left\vert u_{n}\right\vert \leq \phi \text{ q. o. }\forall 
\text{ }n \\
\text{Ts: }u\text{ \`{e} integrabile, }u_{n}\text{ \`{e} integrabile }%
\forall \text{ }n\text{ e }\lim_{n\rightarrow +\infty }\int_{%
%TCIMACRO{\U{211d} }%
%BeginExpansion
\mathbb{R}
%EndExpansion
^{n}}\left\vert u\left( \mathbf{x}\right) -u_{n}\left( \mathbf{x}\right)
\right\vert d\mathbf{x}=0
\end{gather*}

Le prime due tesi sono una consegunza del teorema sopra. La terza tesi,
ancora conseguenza di I2, implica per disuguaglianza triangolare che $%
\left\vert \int_{%
%TCIMACRO{\U{211d} }%
%BeginExpansion
\mathbb{R}
%EndExpansion
^{n}}\left( u\left( \mathbf{x}\right) -u_{n}\left( \mathbf{x}\right) \right)
d\mathbf{x}\right\vert \rightarrow ^{n\rightarrow +\infty }0$, cio\`{e} $%
\lim_{n\rightarrow +\infty }\int_{%
%TCIMACRO{\U{211d} }%
%BeginExpansion
\mathbb{R}
%EndExpansion
^{n}}u_{n}\left( \mathbf{x}\right) d\mathbf{x=}\int_{%
%TCIMACRO{\U{211d} }%
%BeginExpansion
\mathbb{R}
%EndExpansion
^{n}}u\left( \mathbf{x}\right) d\mathbf{x}$.

\textbf{Teo 2.8 (convergenza monotona)} 
\begin{gather*}
\text{Hp: }u_{n}:%
%TCIMACRO{\U{211d} }%
%BeginExpansion
\mathbb{R}
%EndExpansion
^{n}\rightarrow 
%TCIMACRO{\U{211d} }%
%BeginExpansion
\mathbb{R}
%EndExpansion
\text{ }\forall \text{ }n\text{, }\left\{ u_{n}\right\} _{n\in 
%TCIMACRO{\U{2115} }%
%BeginExpansion
\mathbb{N}
%EndExpansion
}\text{ successione di funzioni integrabili } \\
\text{nonnegative, }u_{n}\rightarrow ^{n\rightarrow +\infty }u\text{ q. o., }%
\lim_{n\rightarrow +\infty }\int_{%
%TCIMACRO{\U{211d} }%
%BeginExpansion
\mathbb{R}
%EndExpansion
^{n}}u_{n}\left( \mathbf{x}\right) d\mathbf{x}\in 
%TCIMACRO{\U{211d} }%
%BeginExpansion
\mathbb{R}
%EndExpansion
\text{, }u_{n}\leq u_{n+1}\text{ q. o. }\forall \text{ }n \\
\text{Ts: }u\text{ \`{e} integrabile e }\lim_{n\rightarrow +\infty }\int_{%
%TCIMACRO{\U{211d} }%
%BeginExpansion
\mathbb{R}
%EndExpansion
^{n}}\left\vert u\left( \mathbf{x}\right) -u_{n}\left( \mathbf{x}\right)
\right\vert d\mathbf{x}=0
\end{gather*}

Si noti che, essendo $\left\{ u_{n}\left( x\right) \right\} $ monotona quasi 
$\forall $ $x$ e $\lim_{n\rightarrow +\infty }\int_{%
%TCIMACRO{\U{211d} }%
%BeginExpansion
\mathbb{R}
%EndExpansion
^{n}}u_{n}\left( \mathbf{x}\right) d\mathbf{x}\in 
%TCIMACRO{\U{211d} }%
%BeginExpansion
\mathbb{R}
%EndExpansion
$, $u$ limite di $u_{n}$ \`{e} ben definita quasi ovunque. Senza l'ipotesi $%
\lim_{n\rightarrow +\infty }\int_{%
%TCIMACRO{\U{211d} }%
%BeginExpansion
\mathbb{R}
%EndExpansion
^{n}}u_{n}\left( \mathbf{x}\right) d\mathbf{x}\in 
%TCIMACRO{\U{211d} }%
%BeginExpansion
\mathbb{R}
%EndExpansion
$ \`{e} possibile che $\lim_{n\rightarrow +\infty }\int_{%
%TCIMACRO{\U{211d} }%
%BeginExpansion
\mathbb{R}
%EndExpansion
^{n}}u_{n}\left( \mathbf{x}\right) d\mathbf{x}=+\infty $: in tal caso $u$
limite puntuale q. o. \`{e} misurabile ma in generale non integrabile; \`{e}
possibile inoltre che $u$ non sia ben definita q. o.

Se $u_{n}:%
%TCIMACRO{\U{211d} }%
%BeginExpansion
\mathbb{R}
%EndExpansion
^{n}\rightarrow \lbrack 0,+\infty )$ \`{e} misurabile ma non integrabile, in
generale si scrive $\int_{%
%TCIMACRO{\U{211d} }%
%BeginExpansion
\mathbb{R}
%EndExpansion
^{n}}u\left( \mathbf{x}\right) d\mathbf{x}=+\infty $.

A questo punto \`{e} naturale voler definire un integrale di Lebesgue anche
su sottinsiemi propri di $%
%TCIMACRO{\U{211d} }%
%BeginExpansion
\mathbb{R}
%EndExpansion
^{n}$.

\textbf{Def} Dato $E\subseteq 
%TCIMACRO{\U{211d} }%
%BeginExpansion
\mathbb{R}
%EndExpansion
^{n}$, si dice che $E$ \`{e} misurabile secondo Lebesgue se la funzione
indicatrice di $E$ \`{e} integrabile in $%
%TCIMACRO{\U{211d} }%
%BeginExpansion
\mathbb{R}
%EndExpansion
^{n}$, cio\`{e} se $\int_{%
%TCIMACRO{\U{211d} }%
%BeginExpansion
\mathbb{R}
%EndExpansion
^{n}}\chi _{E}\left( \mathbf{x}\right) d\mathbf{x}\in 
%TCIMACRO{\U{211d} }%
%BeginExpansion
\mathbb{R}
%EndExpansion
$. In tal caso si definisce misura di Lebesgue di $E$ $\left\vert
E\right\vert _{n}:=\int_{%
%TCIMACRO{\U{211d} }%
%BeginExpansion
\mathbb{R}
%EndExpansion
^{n}}\chi _{E}\left( \mathbf{x}\right) d\mathbf{x}$.

In caso contrario si scrive, con abuso, $\left\vert E\right\vert
_{n}=+\infty $.

Come sono fatti gli insiemi misurabili?

\textbf{Prop 2.9 } 
\begin{eqnarray*}
&&\text{Gli aperti, i chiusi, le intersezioni numerabili e le unioni
numerabili di} \\
&&\text{insiemi aperti o chiusi sono insiemi misurabili in }%
%TCIMACRO{\U{211d} }%
%BeginExpansion
\mathbb{R}
%EndExpansion
^{n}
\end{eqnarray*}

\textbf{Def} Dato $E\subseteq 
%TCIMACRO{\U{211d} }%
%BeginExpansion
\mathbb{R}
%EndExpansion
^{n}$ misurabile e $u:%
%TCIMACRO{\U{211d} }%
%BeginExpansion
\mathbb{R}
%EndExpansion
^{n}\rightarrow 
%TCIMACRO{\U{211d} }%
%BeginExpansion
\mathbb{R}
%EndExpansion
\left( 
%TCIMACRO{\U{2102} }%
%BeginExpansion
\mathbb{C}
%EndExpansion
\right) $, si dice che $u$ \`{e} integrabile in $E$ secondo Lebesgue se la
funzione $\tilde{u}:=u\chi _{E}$ \`{e} integrabile secondo Lebesgue in $%
%TCIMACRO{\U{211d} }%
%BeginExpansion
\mathbb{R}
%EndExpansion
^{n}$. In tal caso si definisce integrale di $u$ in $E$ $\int_{E}u\left( 
\mathbf{x}\right) d\mathbf{x}:=\int_{%
%TCIMACRO{\U{211d} }%
%BeginExpansion
\mathbb{R}
%EndExpansion
^{n}}u\left( \mathbf{x}\right) \chi _{E}\left( \mathbf{x}\right) d\mathbf{x}$%
. Se inoltre $\tilde{E}\subseteq E$ \`{e} misurabile, si dice che $u$ \`{e}
integrabile in $\tilde{E}$ se $u|_{\tilde{E}}$ \`{e} integrabile. Si dice
che $u$ \`{e} misurabile in $E$ se $\tilde{u}$ \`{e} misurabile in $%
%TCIMACRO{\U{211d} }%
%BeginExpansion
\mathbb{R}
%EndExpansion
^{n}$.

Le propriet\`{a} e i teoremi visti per l'integrale di Lebesgue in $%
%TCIMACRO{\U{211d} }%
%BeginExpansion
\mathbb{R}
%EndExpansion
^{n}$ si estendono anche agli integrali su $E\subseteq 
%TCIMACRO{\U{211d} }%
%BeginExpansion
\mathbb{R}
%EndExpansion
^{n}$.

E' naturale voler stabilire un collegamento tra l'integrale di Riemann e
quello di Lebesgue.

\textbf{Teo 2.10 (confronto tra Riemann e Lebesgue)} 
\begin{gather*}
\text{Hp: }f:%
%TCIMACRO{\U{211d} }%
%BeginExpansion
\mathbb{R}
%EndExpansion
^{n}\rightarrow 
%TCIMACRO{\U{211d} }%
%BeginExpansion
\mathbb{R}
%EndExpansion
\left( 
%TCIMACRO{\U{2102} }%
%BeginExpansion
\mathbb{C}
%EndExpansion
\right) \text{, }\exists \text{ }K\subseteq 
%TCIMACRO{\U{211d} }%
%BeginExpansion
\mathbb{R}
%EndExpansion
^{n}\text{ compatto tale che }f\left( \mathbf{x}\right) =0\text{ }\forall 
\text{ }\mathbf{x}\in K^{c}\text{,} \\
f\text{ \`{e} Riemann integrabile in }K \\
\text{Ts: }f\text{ \`{e} Lebesgue integrabile in }K\text{ e i due integrali
coincidono}
\end{gather*}

Grazie a questo teorema, se si vuole calcolare l'integrale di Lebesgue di
una funzione $g$ che differisce per un insieme di misura nulla da una
funzione $f$ che si sa essere integrabile secondo Riemann, si pu\`{o}
calcolare direttamente l'integrale di Riemann di $f$ con le tecniche usuali:
non occorre sviluppare nuove tecniche di integrazione per gli integrali di
Lebesgue.

Si noti che il teorema esclude il caso degli integrali di Riemann impropri: 
\`{e} infatti possibile che $f$ sia integrabile in senso improprio secondo
Riemann, ma non secondo Lebesgue.

\subsection{Spazi $L^{p}$}

Se $\Omega \subseteq 
%TCIMACRO{\U{211d} }%
%BeginExpansion
\mathbb{R}
%EndExpansion
^{n}$ \`{e} misurabile e $\left\vert \Omega \right\vert _{n}>0$, l'insieme
delle funzioni misurabili su $\Omega $ $\mathcal{M}\left( \Omega \right)
:=\left\{ u:\Omega \rightarrow 
%TCIMACRO{\U{211d} }%
%BeginExpansion
\mathbb{R}
%EndExpansion
\left( 
%TCIMACRO{\U{2102} }%
%BeginExpansion
\mathbb{C}
%EndExpansion
\right) \text{ misurabili}\right\} $ \`{e} uno spazio vettoriale su $%
%TCIMACRO{\U{211d} }%
%BeginExpansion
\mathbb{R}
%EndExpansion
\left( 
%TCIMACRO{\U{2102} }%
%BeginExpansion
\mathbb{C}
%EndExpansion
\right) $, avente come $0$ la funzione nulla. Si vuole normare tale spazio $%
X=\mathcal{M}\left( \Omega \right) $ con una norma integrale (che usa
l'integrale di Lebesgue): tuttavia, con queste premesse, non pu\`{o} essere
soddisfatta la propriet\`{a} N1 della norma, perch\'{e} \`{e} possibile che $%
N\left( u\right) =\int_{\Omega }u\left( \mathbf{x}\right) d\mathbf{x}=0$
anche se $u$ non \`{e} lo zero di $\mathcal{M}\left( \Omega \right) $.

Per riuscire a normare $\mathcal{M}\left( \Omega \right) $ si introduce
allora una relazione di equivalenza: si dice che $u$ \`{e} equivalente a $v$%
, e si scrive $u\sim v$, se $u=v$ quasi ovunque. E' allora ben definito
l'insieme $M\left( \Omega \right) :=\frac{\mathcal{M}\left( \Omega \right) }{%
\sim }=\left\{ \left[ u\right] :u\in \mathcal{M}\left( \Omega \right)
\right\} $: \`{e} l'insieme $\mathcal{M}\left( \Omega \right) $ quozientato
rispetto alla relazione di equivalenza introdotta, cio\`{e} l'insieme delle
classi di equivalenza $\left[ u\right] $ indotte in $\mathcal{M}\left(
\Omega \right) $ da $\sim \footnote{%
Se $\left[ u\right] $ \`{e} una classe di equivalenza, $u$ \`{e} detto
rappresentante della classe.}$.

$M\left( \Omega \right) $ \`{e} uno spazio vettoriale sul campo $%
%TCIMACRO{\U{211d} }%
%BeginExpansion
\mathbb{R}
%EndExpansion
$ rispetto alle operazioni $+:M\left( \Omega \right) \times M\left( \Omega
\right) \rightarrow M\left( \Omega \right) ,\cdot :%
%TCIMACRO{\U{211d} }%
%BeginExpansion
\mathbb{R}
%EndExpansion
\times M\left( \Omega \right) \rightarrow M\left( \Omega \right) $ cos\`{\i}
definite: $\left[ u\right] +\left[ v\right] :=\left[ u+v\right] $ $\forall $ 
$\left[ u\right] ,\left[ v\right] \in M\left( \Omega \right) $ e $\lambda %
\left[ u\right] :=\left[ \lambda u\right] $ $\forall $ $\left[ u\right] \in
M\left( \Omega \right) $, $\forall $ $\lambda \in 
%TCIMACRO{\U{211d} }%
%BeginExpansion
\mathbb{R}
%EndExpansion
$. Lo zero di $M\left( \Omega \right) $ \`{e} $0_{M\left( \Omega \right) }:=%
\left[ 0\right] $.

Tali operazioni sono ben definite perch\'{e} $\mathcal{M}\left( \Omega
\right) $ \`{e} uno spazio vettoriale; inoltre si pu\`{o} dimostrare che,
essendo $\left[ u\right] +\left[ v\right] =\left[ u+v\right] $, se $w\in %
\left[ u\right] ,z\in \left[ v\right] $ vale $\left[ w+z\right] =\left[ u+v%
\right] $, cio\`{e} la somma tra classi di equivalenza che si \`{e} definita
non dipende dai rappresentanti scelti.

\begin{enumerate}
\item Si noti che se $v,w\in \left[ u\right] $ allora $\int_{\Omega
}v=\int_{\Omega }w$ per la propriet\`{a} (i) dell'integrale di Lebesgue. Ad
esempio se $\Omega =\left( 0,1\right) $, $u\left( x\right) =\chi _{%
%TCIMACRO{\U{211a} }%
%BeginExpansion
\mathbb{Q}
%EndExpansion
\cap \Omega }\left( x\right) \in \left[ 0\right] $ (funzione di Dirichlet) e
vale $\int_{\Omega }u=0$.

\item Se $\Omega $ \`{e} aperto, allora $\forall $ $\left[ u\right] \in
M\left( \Omega \right) ,\left[ u\right] \neq \left[ 0\right] $, esiste al pi%
\`{u} una funzione $f$ continua tale che $f\in \left[ u\right] $. Infatti,
se esistesse anche $g$ continua tale che $g\in \left[ u\right] $, $f-g$
sarebbe continua e nulla quasi ovunque, ma allora deve essere costantemente
nulla\footnote{%
Se per assurdo esistesse $x$ tale che $f\left( x\right) -g\left( x\right)
=k>0$, per continuit\`{a} esisterebbe un intorno $\mathcal{B}\left( x\right) 
$ in cui $f$ sarebbe nonnegativa, ma dato che la misura di $\mathcal{B}%
\left( x\right) $ non \`{e} nulla non varrebbe $f-g=0$ q.o.}, dunque $f=g$.
\end{enumerate}

Si pu\`{o} ora definire, per $p\geq 1$, $L^{p}\left( \Omega \right)
:=\left\{ \left[ u\right] \in M\left( \Omega \right) :\int_{\Omega
}\left\vert u\left( \mathbf{x}\right) \right\vert ^{p}d\mathbf{x\in 
%TCIMACRO{\U{211d} }%
%BeginExpansion
\mathbb{R}
%EndExpansion
}\right\} $. Si noti che, per come \`{e} definita la relazione di
equivalenza, la scelta del rappresentante \`{e} indifferente per il valore
dell'integrale. $L^{p}\left( \Omega \right) $ \`{e} finalmente lo spazio
adeguato per definire una norma con l'integrale di Lebesgue: le due
disuguaglianze seguenti sono necessarie per mostrare che esso \`{e} uno
spazio vettoriale normato.

\textbf{Teo 2.11} \textbf{(disuguaglianza di Hoelder)}%
\begin{eqnarray*}
\text{Hp}\text{: } &&u,v:\Omega \rightarrow 
%TCIMACRO{\U{211d} }%
%BeginExpansion
\mathbb{R}
%EndExpansion
\left( 
%TCIMACRO{\U{2102} }%
%BeginExpansion
\mathbb{C}
%EndExpansion
\right) \text{, }u\in L^{p}\left( \Omega \right) ,v\in L^{q}\left( \Omega
\right) \text{, }\frac{1}{p}+\frac{1}{q}=1 \\
\text{Ts}\text{: } &&uv\in L^{1}\left( \Omega \right) \text{ e }\int_{\Omega
}\left\vert u\left( \mathbf{x}\right) v\left( \mathbf{x}\right) \right\vert
dx\leq \left( \int_{\Omega }\left\vert u\left( \mathbf{x}\right) \right\vert
^{p}d\mathbf{x}\right) ^{\frac{1}{p}}\left( \int_{\Omega }\left\vert v\left( 
\mathbf{x}\right) \right\vert ^{q}d\mathbf{x}\right) ^{\frac{1}{q}}
\end{eqnarray*}

$p,q$ come nell'ipotesi si dicono esponenti coniugati. Tale teorema mostra
che se $u\in L^{p},v\in L^{q}$ e $\frac{1}{p}+\frac{1}{q}=1$ allora $uv\in
L^{1}$ e $\left\vert \left\vert uv\right\vert \right\vert _{L^{1}}\leq
\left\vert \left\vert u\right\vert \right\vert _{L^{p}}\left\vert \left\vert
v\right\vert \right\vert _{L^{q}}$.

\textbf{Teo 2.12} \textbf{(disuguaglianza di Minkowski)}%
\begin{eqnarray*}
\text{Hp}\text{: } &&u,v:\Omega \rightarrow 
%TCIMACRO{\U{211d} }%
%BeginExpansion
\mathbb{R}
%EndExpansion
\left( 
%TCIMACRO{\U{2102} }%
%BeginExpansion
\mathbb{C}
%EndExpansion
\right) \text{, }u,v\in L^{p}\left( \Omega \right) \\
\text{Ts}\text{: } &&\left( \int_{\Omega }\left\vert u\left( \mathbf{x}%
\right) +v\left( \mathbf{x}\right) \right\vert ^{p}dx\right) ^{\frac{1}{p}%
}\leq \left( \int_{\Omega }\left\vert u\left( \mathbf{x}\right) \right\vert
^{p}d\mathbf{x}\right) ^{\frac{1}{p}}+\left( \int_{\Omega }\left\vert
v\left( \mathbf{x}\right) \right\vert ^{p}d\mathbf{x}\right) ^{\frac{1}{p}}
\end{eqnarray*}

Per $p=1$ la disuguaglianza \`{e} ovvia per disuguaglianza triangolare. La
tesi \`{e} che $\left\vert \left\vert u+v\right\vert \right\vert
_{L^{p}}\leq \left\vert \left\vert u\right\vert \right\vert
_{L^{p}}+\left\vert \left\vert v\right\vert \right\vert _{L^{p}}$.

\textbf{Prop 2.13 (norma }$L^{p}$\textbf{)} 
\begin{eqnarray*}
\text{Hp}\text{: } &&\left\vert \left\vert u\right\vert \right\vert
_{p}:=\left( \int_{\Omega }\left\vert u\left( \mathbf{x}\right) \right\vert
^{p}d\mathbf{x}\right) ^{\frac{1}{p}}\text{, }p\in \lbrack 1,+\infty ) \\
\text{Ts}\text{: } &&\left( L^{p}\left( \Omega \right) ,\left\vert
\left\vert \cdot \right\vert \right\vert _{p}\right) \text{ \`{e} uno spazio
normato }
\end{eqnarray*}

\textbf{Dim} Per la disuguaglianza di Minkowski $L^{p}\left( \Omega \right) $
\`{e} uno spazio vettoriale: se $\left[ u\right] ,\left[ v\right] \in
L^{p}\left( \Omega \right) $, $\left[ u\right] +\left[ v\right] \in
L^{p}\left( \Omega \right) $.

Si verifica che valgono le propriet\`{a} della norma. Evidentemente $%
\left\vert \left\vert u\right\vert \right\vert _{p}\geq 0$. Inoltre $%
\left\vert \left\vert u\right\vert \right\vert _{p}=0\Longleftrightarrow %
\left[ u\right] =\left[ 0\right] $: l'implicazione da destra a sinistra
segue dalla propriet\`{a} (i) dell'integrale di Lebesgue, quella da sinistra
a destra vale perch\'{e} se per assurdo l'insieme in cui $u\neq 0$ non
avesse misura nulla allora l'integrale in $\left\vert \left\vert
u\right\vert \right\vert _{p}$ non sarebbe nullo.

Vale N2 perch\'{e} $\left\vert \left\vert \alpha u\right\vert \right\vert
_{p}=\left( \int_{\Omega }\left\vert \alpha u\left( \mathbf{x}\right)
\right\vert ^{p}d\mathbf{x}\right) ^{\frac{1}{p}}=\left( \left\vert \alpha
\right\vert ^{p}\int_{\Omega }\left\vert u\left( \mathbf{x}\right)
\right\vert ^{p}d\mathbf{x}\right) ^{\frac{1}{p}}=\left\vert \alpha
\right\vert \left( \int_{\Omega }\left\vert u\left( \mathbf{x}\right)
\right\vert ^{p}d\mathbf{x}\right) ^{\frac{1}{p}}=\left\vert a\right\vert
\left\vert \left\vert u\right\vert \right\vert _{p}$. N3 \`{e} una
conseguenza diretta della disuguaglianza di Minkowski: $\left\vert
\left\vert u+v\right\vert \right\vert _{p}\leq \left\vert \left\vert
u\right\vert \right\vert _{p}+\left\vert \left\vert v\right\vert \right\vert
_{p}$. $\blacksquare $

La convergenza in norma $L^{p}$ \`{e} una convergenza in norma integrale. In
generale non c'\`{e} relazione tra la convergenza in norma $L^{p}$ e la
convergenza puntuale quasi ovunque.

Si pu\`{o} definire anche $L^{\infty }\left( \Omega \right) :=\left\{ \left[
u\right] \in M\left( \Omega \right) :\exists \text{ }M\geq 0:\left\vert
u\right\vert \leq M\text{ q.o. in }\Omega \right\} $. Una funzione in $%
L^{\infty }\left( \Omega \right) $ si dice essenzialmente limitata in $%
\Omega $, perch\'{e} \`{e} limitata da $M$ a meno di un insieme di misura
nulla. Si noti che tale propriet\`{a} non dipende dal rappresentante scelto
della classe di equivalenza. $L^{\infty }\left( \Omega \right) $ \`{e}
ovviamente uno spazio vettoriale.

Data $\left[ u\right] \in L^{\infty }\left( \Omega \right) $, $\alpha :=\min
\left\{ M\geq 0:\left\vert u\right\vert \leq M\text{ q.o. in }\Omega
\right\} $ si dice estremo superiore essenziale di $u$ e si indica con $\sup
\operatorname{ess}\left( u\right) $: si pu\`{o} dimostrare che $L^{\infty }\left( \Omega
\right) $ \`{e} uno spazio normato con $\left\vert \left\vert u\right\vert
\right\vert _{\infty }:=\sup \operatorname{ess}\left( u\right) $. La convergenza secondo
tale norma \`{e} una convergenza uniforme quasi ovunque; quindi la
convergenza in noma $L^{\infty }$ implica la convergenza puntuale quasi
ovunque.

\begin{enumerate}
\item $\sup ess\left( \chi _{%
%TCIMACRO{\U{211a} }%
%BeginExpansion
\mathbb{Q}
%EndExpansion
\cap \left( 0,1\right) }\right) =0$.

\item La disuguaglianza di Hoelder si pu\`{o} estendere al caso $%
p=1,q=+\infty $ (che con abuso si possono dire soddisfare $\frac{1}{p}+\frac{%
1}{q}=1$): infatti, se $u\in L^{1}\left( \Omega \right) $ e $v\in L^{\infty
}\left( \Omega \right) $ con $\left\vert \left\vert v\right\vert \right\vert
_{\infty }=M$, allora $\int_{\Omega }\left\vert u\left( \mathbf{x}\right)
v\left( \mathbf{x}\right) \right\vert dx\leq M\int_{\Omega }\left\vert
u\left( \mathbf{x}\right) \right\vert dx=\left\vert \left\vert u\right\vert
\right\vert _{1}\left\vert \left\vert v\right\vert \right\vert _{\infty
}=\left( \int_{\Omega }\left\vert u\left( \mathbf{x}\right) \right\vert ^{p}d%
\mathbf{x}\right) ^{\frac{1}{p}}\left( \int_{\Omega }\left\vert v\left( 
\mathbf{x}\right) \right\vert ^{q}d\mathbf{x}\right) ^{\frac{1}{q}}$.
\end{enumerate}

Si \`{e} visto che $L^{p}\left( \Omega \right) $, al variare di $p\in \left[
1,+\infty \right] $, \`{e} una famiglia di spazi vettoriali normati. Vale il
seguente risultato fondamentale.

\textbf{Teo 2.14 (completezza degli }$L^{p}$\textbf{)} 
\begin{gather*}
\text{Hp: }p\in \left[ 1,+\infty \right] \\
\text{Ts: }\left( L^{p}\left( \Omega \right) ,\left\vert \left\vert \cdot
\right\vert \right\vert _{p}\right) \text{ \`{e} uno spazio di Banach }
\end{gather*}

\subsection{Spazi di Hilbert}

\textbf{Def} Dato $H\neq \varnothing $ spazio vettoriale sul campo $%
%TCIMACRO{\U{211d} }%
%BeginExpansion
\mathbb{R}
%EndExpansion
\left( 
%TCIMACRO{\U{2102} }%
%BeginExpansion
\mathbb{C}
%EndExpansion
\right) $, si dice che $p$ \`{e} un prodotto scalare su $H$ se $p:H\times
H\rightarrow 
%TCIMACRO{\U{211d} }%
%BeginExpansion
\mathbb{R}
%EndExpansion
\left( 
%TCIMACRO{\U{2102} }%
%BeginExpansion
\mathbb{C}
%EndExpansion
\right) $ ha le seguenti propriet\`{a}:

\begin{description}
\item[P1] annullamento e positivit\`{a}: $p\left( x,x\right)
=0\Longleftrightarrow x=0$, $p\left( x,x\right) \geq 0$ $\forall $ $x\in H$

\item[P2] simmetria: $p\left( x,y\right) =\bar{p}\left( y,x\right) $ $%
\forall $ $x,y\in H$

\item[P3] bilinearit\`{a}: $p\left( \alpha x+\beta y,z\right) =\alpha
p\left( x,z\right) +\beta p\left( y,z\right) $ $\forall $ $x,y,z\in H$, $%
\forall $ $\alpha ,\beta \in 
%TCIMACRO{\U{211d} }%
%BeginExpansion
\mathbb{R}
%EndExpansion
\left( 
%TCIMACRO{\U{2102} }%
%BeginExpansion
\mathbb{C}
%EndExpansion
\right) $
\end{description}

In tal caso la coppia $\left( H,p\right) $ si dice spazio prehilbertiano.

La simmetria permette di dedurre la linearit\`{a} anche rispetto al secondo
argomento: quindi il prodotto scalare \`{e} una forma bilineare simmetrica.

\textbf{Prop 2.15 (Cauchy-Schwarz; il prodotto scalare induce una norma;
identit\`{a} del parallelogramma)}%
\begin{gather*}
\text{Hp: }\left( H,p\right) \text{ \`{e} uno spazio prehilbertiano } \\
\text{Ts: (i) }\left\vert p\left( x,y\right) \right\vert \leq \sqrt{p\left(
x,x\right) }\sqrt{p\left( y,y\right) } \\
\text{(ii) }\left( H,\left\vert \left\vert \cdot \right\vert \right\vert
\right) \text{ \`{e} uno spazio normato con la norma }\left\vert \left\vert
x\right\vert \right\vert =\sqrt{p\left( x,x\right) } \\
\text{(iii) }\left\vert \left\vert x-y\right\vert \right\vert
^{2}+\left\vert \left\vert x+y\right\vert \right\vert ^{2}=2\left(
\left\vert \left\vert x\right\vert \right\vert ^{2}+\left\vert \left\vert
y\right\vert \right\vert ^{2}\right) \text{ }\forall \text{ }x,y\in H
\end{gather*}

(i) serve per dimostrare (ii).

\textbf{Dim} (ii) $\sqrt{p\left( x,x\right) }\geq 0$ $\forall $ $u$ e $\sqrt{%
p\left( x,x\right) }=0\Longleftrightarrow p\left( x,x\right)
=0\Longleftrightarrow x=0$. Inoltre $\sqrt{p\left( \alpha x,\alpha x\right) }%
=\sqrt{\alpha ^{2}p\left( x,x\right) }=\left\vert \alpha \right\vert \sqrt{%
p\left( x,x\right) }=\left\vert \alpha \right\vert \left\vert \left\vert
x\right\vert \right\vert $. $\left\vert \left\vert x+y\right\vert
\right\vert =\sqrt{p\left( x+y,x+y\right) }=\sqrt{p\left( x,x\right)
+p\left( y,y\right) +2p\left( x,y\right) }=\sqrt{\left\vert p\left(
x,x\right) +p\left( y,y\right) +2p\left( x,y\right) \right\vert }$, per
bilinearit\`{a}, e usando la disuguaglianza triangolare e Cauchy-Schwarz si
ottiene la maggiorazione $\sqrt{p\left( x,x\right) +p\left( y,y\right) +2%
\sqrt{p\left( x,x\right) }\sqrt{p\left( y,y\right) }}=\sqrt{\left( \sqrt{%
p\left( x,x\right) }+\sqrt{p\left( y,y\right) }\right) ^{2}}=\sqrt{p\left(
x,x\right) }+\sqrt{p\left( y,y\right) }=\left\vert \left\vert x\right\vert
\right\vert +\left\vert \left\vert y\right\vert \right\vert $.

(iii) $\left\langle x-y,x-y\right\rangle +\left\langle x+y,x+y\right\rangle
=\left\vert \left\vert x\right\vert \right\vert ^{2}-2\left\langle
x,y\right\rangle +\left\vert \left\vert y\right\vert \right\vert
^{2}+\left\vert \left\vert x\right\vert \right\vert ^{2}+2\left\langle
x,y\right\rangle +\left\vert \left\vert y\right\vert \right\vert
^{2}=2\left( \left\vert \left\vert x\right\vert \right\vert ^{2}+\left\vert
\left\vert y\right\vert \right\vert ^{2}\right) $. $\blacksquare $

Nel seguito il prodotto scalare $p\left( x,y\right) $ si indica con $%
\left\langle x,y\right\rangle $.

Sia $H$ uno spazio prehilbertiano. E' gi\`{a} noto che la norma \`{e} una
funzione continua, cio\`{e} se $x_{n}\rightarrow x$ in $H$, allora $%
\left\vert \left\vert x_{n}\right\vert \right\vert ^{2}=\left\langle
x_{n},x_{n}\right\rangle \rightarrow \left\vert \left\vert x\right\vert
\right\vert ^{2}=\left\langle x,x\right\rangle $ (con la norma indotta dal
prodotto scalare di $H$). Questo \`{e} un caso particolare del seguente
fatto: data $x_{n}\rightarrow x$ in $H$, se inoltre $y_{n}\rightarrow y$, si
ha che $\left\langle x_{n},y_{n}\right\rangle \rightarrow \left\langle
x,y\right\rangle $ in $%
%TCIMACRO{\U{211d} }%
%BeginExpansion
\mathbb{R}
%EndExpansion
$ (continuit\`{a} del prodotto scalare).

\textbf{Prop 2.16 (continuit\`{a} del prodotto scalare)}%
\begin{eqnarray*}
\text{Hp}\text{: } &&H\text{ \`{e} uno spazio prehilbertiano, }%
x_{n}\rightarrow x,y_{n}\rightarrow y\text{ in }H \\
\text{Ts}\text{: } &&\left\langle x_{n},y_{n}\right\rangle \rightarrow
\left\langle x,y\right\rangle
\end{eqnarray*}

\textbf{Dim} Dimostro che $\left\vert \left\langle x_{n},y_{n}\right\rangle
-\left\langle x,y\right\rangle \right\vert \rightarrow 0$ per $n\rightarrow
+\infty $. Per Cauchy-Schwarz $\left\vert \left\langle
x_{n},y_{n}\right\rangle -\left\langle x,y\right\rangle \right\vert
=\left\vert \left\langle x_{n}-x,y_{n}-y\right\rangle \right\vert \leq
\left\vert \left\vert x_{n}-x\right\vert \right\vert \left\vert \left\vert
y_{n}-y\right\vert \right\vert \rightarrow 0$, quindi per confronto si ha la
tesi. $\blacksquare $

\textbf{Def} Dato uno spazio prehilbertiano $\left( H,\left\langle
\_,\_\right\rangle \right) $, esso si dice spazio di Hilbert se $\left( H,%
\sqrt{\left\langle \_,\_\right\rangle }\right) $ \`{e} uno spazio di Banach.

Uno spazio di Hilbert \`{e} quindi uno spazio prehilbertiano completo
rispetto alla norma indotta dal prodotto scalare.

\textbf{Teo 2.17 (caratterizzazione degli spazi di Hilbert)}%
\begin{eqnarray*}
\text{Hp}\text{: } &&\left( X,\left\vert \left\vert \cdot \right\vert
\right\vert \right) \text{ \`{e} di Banach e }\left\vert \left\vert \cdot
\right\vert \right\vert \text{ \`{e} indotta da un prodotto scalare} \\
\text{ } &\Longleftrightarrow &\text{ }\left\vert \left\vert \cdot
\right\vert \right\vert \text{ soddisfa l'identit\`{a} del parallelogramma}
\end{eqnarray*}

Quindi, per verificare se uno spazio di Banach \`{e} anche di Hilbert, \`{e}
sufficiente verificare che la norma soddisfi l'identit\`{a} del
parallelogramma. L'implicazione da sinistra a destra \`{e} ovvia e si \`{e}
dimostrata sopra.

Si vedono alcuni esempi di spazi di Hilbert: poich\'{e} gli spazi vettoriali
di dimensione finita $n$ sono isomorfi a $%
%TCIMACRO{\U{211d} }%
%BeginExpansion
\mathbb{R}
%EndExpansion
^{n}$ o a $%
%TCIMACRO{\U{2102} }%
%BeginExpansion
\mathbb{C}
%EndExpansion
^{n}$, gli spazi di Hilbert realmente interessanti per l'analisi sono quelli
a dimensione infinita.

\begin{enumerate}
\item $%
%TCIMACRO{\U{211d} }%
%BeginExpansion
\mathbb{R}
%EndExpansion
^{n}$, dotato del prodotto scalare standard $\left\langle \mathbf{x,y}%
\right\rangle =\sum_{i=1}^{n}x_{i}y_{i}$ che induce la norma euclidea $%
\left\vert \left\vert \mathbf{x}\right\vert \right\vert =\sqrt{%
\sum_{i=1}^{n}x_{i}^{2}}$, \`{e} uno spazio di Hilbert di dimensione finita.

\item $%
%TCIMACRO{\U{2102} }%
%BeginExpansion
\mathbb{C}
%EndExpansion
^{n}$, dotato del prodotto scalare $\left\langle u,w\right\rangle
=\sum_{i=1}^{n}u_{i}\bar{w}_{i}$ che induce la norma $\left\vert \left\vert
u\right\vert \right\vert =\sum_{i=1}^{n}\left\vert u_{i}\right\vert ^{2}$, 
\`{e} uno spazio di Hilbert di dimensione finita.

\item In $C^{0}\left( \left[ a,b\right] \right) $ $\left\langle
f,g\right\rangle :=\int_{a}^{b}f\left( x\right) g\left( x\right) dx$ \`{e}
un prodotto scalare. $\left\vert \left\vert f\right\vert \right\vert =\sqrt{%
\int_{a}^{b}f^{2}\left( x\right) dx}$, che \`{e} la norma $L^{2}$. Questo
spazio vettoriale normato, prehilbertiano, non \`{e} completo: esiste una
successione di funzioni continue che in tale norma \`{e} di Cauchy ma
converge a una funzione discontinua (e. g. la successione $f_{n}\left(
x\right) =\left\{ 
\begin{array}{c}
0\text{ se }x\leq 0 \\ 
nx\text{ se }0<x\leq \frac{1}{n} \\ 
1\text{ se }x>\frac{1}{n}%
\end{array}%
\right. $).

\item In $L^{2}\left( \Omega \right) $ il prodotto scalare $\left\langle
u,v\right\rangle =\int_{\Omega }u\bar{v}$ induce la norma $L^{2}$ $%
\left\vert \left\vert u\right\vert \right\vert _{2}=\sqrt{\int_{\Omega
}\left\vert u\right\vert ^{2}}$: $L^{2}$ \`{e} uno spazio di Hilbert (di
dimensione infinita) rispetto a tale prodotto scalare perch\'{e} sappiamo gi%
\`{a} che $L^{2}$ \`{e} uno spazio di Banach. $L^{p}\left( \Omega \right) $,
per $p\neq 2$, \`{e} di Banach ma non di Hilbert.

\item $l^{2}=\left\{ \left\{ x_{n}\right\} \subseteq 
%TCIMACRO{\U{2102} }%
%BeginExpansion
\mathbb{C}
%EndExpansion
:\sum_{n=1}^{+\infty }\left\vert x_{n}\right\vert ^{2}<+\infty \right\} $ 
\`{e} l'insieme delle successioni reali la cui serie dei quadrati converge
(la versione discreta di $L^{2}$). Dati $x,y\in l^{2}$, $\left\langle
x,y\right\rangle :=\sum_{n=1}^{+\infty }x_{n}\bar{y}_{n}$ (si dimostra che
converge): $l^{2}$ con tale prodotto scalare \`{e} di Hilbert. $\left\vert
\left\vert x\right\vert \right\vert _{2}=\sqrt{\sum_{n=1}^{+\infty
}\left\vert x_{n}\right\vert ^{2}}$.
\end{enumerate}

\textbf{Def} Dato $\left( H,\left\langle \_,\_\right\rangle \right) $ spazio
prehilbertiano e $x,y\in H,x\neq 0,y\neq 0$, si dice angolo tra $x$ e $y$ $%
\alpha :=\arccos \frac{\left\langle x,y\right\rangle }{\left\vert \left\vert
x\right\vert \right\vert \left\vert \left\vert y\right\vert \right\vert }$.
Si dice che $x$ e $y$ sono ortogonali in $H$, e si scrive $x\perp y$, se $%
\left\langle x,y\right\rangle =0$.

$\alpha $ \`{e} ben definito per la disuguaglianza di Cauchy-Schwarz; $%
\alpha \in \left[ 0,\pi \right] $. Se $\left\langle x,y\right\rangle =0$ si
ritrova il teorema di Pitagora $\left\vert \left\vert x+y\right\vert
\right\vert ^{2}=\left\vert \left\vert x\right\vert \right\vert
^{2}+\left\vert \left\vert y\right\vert \right\vert ^{2}$.

\textbf{Teo}%
\begin{gather*}
\text{Hp: }\left( H,\left\langle \_,\_\right\rangle \right) \text{ spazio
prehilbertiano, }\left\{ x_{1},...,x_{n}\right\} \text{ sono a due a due
ortogonali in }H \\
\text{Ts: }\left\vert \left\vert x_{1}+...+x_{n}\right\vert \right\vert
^{2}=\left\vert \left\vert x_{1}\right\vert \right\vert ^{2}+...+\left\vert
\left\vert x_{n}\right\vert \right\vert ^{2}
\end{gather*}

\textbf{Dim} $\left\langle
\sum_{i=1}^{n}x_{i},\sum_{j=1}^{n}x_{j}\right\rangle
=\sum_{i,j=1}^{n}\left\langle x_{i},x_{j}\right\rangle
=\sum_{i=1}^{n}\left\langle x_{i},x_{i}\right\rangle $. $\blacksquare $

Si \`{e} gi\`{a} visto ad algebra lineare un teorema sulla proiezione
ortogonale per spazi euclidei a dimensione finita. Per togliere l'ipotesi di
dimensione finita occorre la struttura di spazio di Hilbert. Infatti tutti
gli spazi euclidei sul campo $%
%TCIMACRO{\U{211d} }%
%BeginExpansion
\mathbb{R}
%EndExpansion
$ a dimensione finita sono spazi di Hilbert\footnote{%
Si noti che questo non \`{e} vero se si considera un campo qualsiasi.
Infatti $%
%TCIMACRO{\U{211a} }%
%BeginExpansion
\mathbb{Q}
%EndExpansion
^{n}$ sul campo $%
%TCIMACRO{\U{211a} }%
%BeginExpansion
\mathbb{Q}
%EndExpansion
$ \`{e} uno spazio prehilbertiano con il prodotto scalare standard, ma $%
%TCIMACRO{\U{211a} }%
%BeginExpansion
\mathbb{Q}
%EndExpansion
^{n}$ non \`{e} completo rispetto alla norma euclidea, cio\`{e} non \`{e} di
Banach.}.

\textbf{Teo 2.18 (minima distanza)}%
\begin{gather*}
\text{Hp}\text{: }\left( H,\left\langle \_,\_\right\rangle \right) \text{ 
\`{e} uno spazio di Hilbert, }V\text{ \`{e} un sottospazio chiuso di }H \\
\text{Ts}\text{: }\forall \text{ }x\in H\text{ }\exists \text{ }!\text{ }%
y:V:\left\vert \left\vert y-x\right\vert \right\vert =\inf_{w\in
V}\left\vert \left\vert w-x\right\vert \right\vert
\end{gather*}

In tal caso si definisce distanza tra $x$ e $V$ $d\left( x,V\right)
:=\left\vert \left\vert y-x\right\vert \right\vert $.

Nell'ipotesi si intende che $V$ \`{e} chiuso rispetto alla metrica indotta
dalla norma\footnote{%
Un generico sottinsieme $E$ di uno spazio metrico $\left( X,d\right) $ si
dice chiuso se ogni successione $\left\{ x_{n}\right\} $ a valori in $E$ e
convergente in $X$ converge a un elemento di $E$.
\par
Tale ipotesi non sarebbe necessaria in $%
%TCIMACRO{\U{211d} }%
%BeginExpansion
\mathbb{R}
%EndExpansion
^{n}$, tutti i sottospazi del quale sono chiusi.}. La tesi afferma che per
ogni elemento $x$ di $H$ esiste un unico $y\in V$ che, tra tutti gli
elementi di $V$, \`{e} quello pi\`{u} vicino a $x$ (secondo la metrica
indotta dalla norma).

\textbf{Def} Dato $E$ sottospazio di $H$ e $\left( H,\left\langle
\_,\_\right\rangle \right) $ spazio di Hilbert, si definisce sottospazio
ortogonale a $E$ l'insieme $E^{\perp }:=\left\{ x\in H:\left\langle
x,y\right\rangle =0\text{ }\forall \text{ }y\in E\right\} $.

Dalla continuit\`{a} del prodotto scalare segue che $E^{\perp }$ \`{e} chiuso%
\footnote{%
Si vuole dimostrare che $\forall $ $\left\{ x_{n}\right\} \subseteq E^{\perp
}$ che converge in $H$ a $x$ vale $x\in E^{\perp }$.
\par
Poich\'{e} $\left\{ x_{n}\right\} \subseteq E^{\perp }$, vale $\left\langle
x_{n},y\right\rangle =0$ $\forall $ $n$, $\forall $ $y\in E$. Allora, se $%
x_{n}\rightarrow x$ in $H$, per continuit\`{a} del prodotto scalare vale $%
\left\langle x,y\right\rangle =\lim_{n\rightarrow +\infty }\left\langle
x_{n},y\right\rangle =0$ $\forall ~y\in E$, cio\`{e} $x\in E^{\perp }$.}. Si
pu\`{o} dimostrare che vale inoltre $\left( \bar{E}\right) ^{\perp
}=E^{\perp }$.

\textbf{Def} Dato $\left( H,\left\langle \_,\_\right\rangle \right) $ spazio
di Hilbert e $V,W\subseteq H$, $V$ e $W$ si dicono sottospazi ortogonali, e
si scrive $V\perp W$, se $\left\langle x,y\right\rangle =0$ $\forall $ $x\in
V$, $\forall $ $y\in W$.

Ovviamente $V^{\perp }$ e $V$ sono ortogonali.

\begin{enumerate}
\item Se $V\perp W$, $V\cap W=\left\{ \mathbf{0}\right\} $ per la propriet%
\`{a} di annullamento del prodotto scalare.
\end{enumerate}

\textbf{Def} Dato $\left( H,\left\langle \_,\_\right\rangle \right) $ spazio
di Hilbert e $V,W\subseteq H$ sottospazi ortogonali, si dice somma diretta
di $V$ e $W$ l'insieme $V\oplus W=\left\{ x+y:x\in V,y\in W\right\} $.

Si pu\`{o} dimostrare che $\forall $ $z\in V\oplus W$ $\exists $ $!$ $x\in V$
e $\exists $ $!$ $y\in W:z=x+y$, cio\`{e} ogni $z$ in $V\oplus W$ pu\`{o}
essere decomposto in un unico modo come somma di un elemento di $V$ e uno di 
$W$.

\textbf{Teo 2.20 (proiezioni) }%
\begin{gather*}
\text{Hp: }\left( H,\left\langle \_,\_\right\rangle \right) \text{ spazio di
Hilbert, }V\subseteq H\text{ sottospazio chiuso} \\
\text{Ts: }H=V\oplus V^{\perp }
\end{gather*}

Inoltre per il teorema di minima distanza, essendo sia $V$ che $V^{\perp }$
chiusi, sono ben definiti le seguenti funzioni, dette operatori di
proiezione: $p_{V}:H\rightarrow V,p_{V}\left( x\right) =y=\arg \min_{w\in
V}d\left( x,w\right) $ e $p_{V^{\perp }}:H\rightarrow V^{\perp },p_{V^{\perp
}}\left( x\right) =z=\arg \min_{w\in V^{\perp }}d\left( x,w\right) $, e vale 
$x=p_{V}\left( x\right) +p_{V^{\perp }}\left( x\right) $ $\forall $ $x\in H$.

\subsection{Serie di Fourier astratte}

\textbf{Def} Dato $\left( H,\left\langle \_,\_\right\rangle \right) $ spazio
di Hilbert, $E\subseteq 
%TCIMACRO{\U{2115} }%
%BeginExpansion
\mathbb{N}
%EndExpansion
$ e $\left\{ u_{n}\right\} _{n\in E}\subseteq H$, $\left\{ u_{n}\right\}
_{n\in E}$ si dice sistema ortogonale se $\left\langle
u_{n},u_{m}\right\rangle =0$ $\forall $ $m,n\in E,m\neq n$. Se inoltre $%
\left\langle u_{n},u_{n}\right\rangle =1$ $\forall $ $n\in E$, il sistema si
dice ortonormale.

\textbf{Teo 2.19 }%
\begin{gather*}
\text{Hp: }\left( H,\left\langle \_,\_\right\rangle \right) \text{ spazio di
Hilbert, }E\subseteq 
%TCIMACRO{\U{2115} }%
%BeginExpansion
\mathbb{N}
%EndExpansion
\text{, }\left\{ u_{n}\right\} _{n\in E}\subseteq H\text{ sistema
ortogonale, }\left\{ c_{j}\right\} _{j\in E}\subseteq 
%TCIMACRO{\U{2102} }%
%BeginExpansion
\mathbb{C}
%EndExpansion
\text{ } \\
\text{Ts: }\sum_{n\in E}c_{n}u_{n}\text{ converge }\Longleftrightarrow
\sum_{n\in E}\left\vert c_{n}\right\vert ^{2}<+\infty \text{; in tal caso }%
\exists \text{ }x\in H:x=\sum_{n\in E}c_{n}u_{n}\text{;} \\
\text{se inoltre il sistema \`{e} ortonormale vale }c_{n}=\left\langle
x,u_{n}\right\rangle \text{ }\forall \text{ }n\in E\text{, e se inoltre } \\
y=\sum_{n\in E}d_{n}u_{n}\text{, allora }\left\langle x,y\right\rangle
=\sum_{n\in E}c_{n}\bar{d}_{n}\text{ e in particolare }\left\vert \left\vert
x\right\vert \right\vert ^{2}=\sum_{n\in E}\left\vert c_{n}\right\vert ^{2}%
\text{ (identit\`{a} di Parseval)}
\end{gather*}

Il teorema \`{e} rilevante se $E=%
%TCIMACRO{\U{2115} }%
%BeginExpansion
\mathbb{N}
%EndExpansion
$. Si ricorda che $\sum_{n\in E}c_{n}u_{n}$ \`{e} la successione delle somme
parziali $\left\{ \sum_{j=0}^{n}c_{j}u_{j}\right\} _{n\in 
%TCIMACRO{\U{2115} }%
%BeginExpansion
\mathbb{N}
%EndExpansion
}$, e la convergenza di cui si parla \`{e} la convergenza nella norma
indotta dal prodotto scalare.

Inoltre, per gli $x\in H$ che possono essere scritti come somma di una serie
del tipo $\sum_{n\in E}c_{n}u_{n}$ il calcolo di norme e prodotti scalari si
riduce al calcolare somme di serie numeriche. In particolare, $\left\vert
\left\vert x\right\vert \right\vert ^{2}$ pu\`{o} essere calcolata come la
norma in $l^{2}$ della successione dei suoi coefficienti $\left\{
c_{n}\right\} $ e $\left\langle x,y\right\rangle $ come prodotto scalare in $%
l^{2}$ delle successioni dei coefficienti.

\textbf{Dim} (i) Sia $E=%
%TCIMACRO{\U{2115} }%
%BeginExpansion
\mathbb{N}
%EndExpansion
$. Si dimostra solo che se $x=\sum_{n=1}^{+\infty }c_{n}u_{n}$, allora $%
c_{j}=\left\langle x,u_{j}\right\rangle $. Questo \`{e} vero perch\'{e} $%
\left\langle x,u_{j}\right\rangle =\left\langle \sum_{n=1}^{+\infty
}c_{n}u_{n},u_{j}\right\rangle =\sum_{n=1}^{+\infty }\left\langle
c_{n}u_{n},u_{j}\right\rangle =c_{n}\left\vert \left\vert u_{n}\right\vert
\right\vert ^{2}=c_{n}$; lo scambio tra serie e prodotto scalare \`{e}
lecito per la continuit\`{a} del prodotto scalare. $\blacksquare $

\textbf{Def} Dato $\left( H,\left\langle \_,\_\right\rangle \right) $ spazio
di Hilbert, $E\subseteq 
%TCIMACRO{\U{2115} }%
%BeginExpansion
\mathbb{N}
%EndExpansion
$ e $\left\{ u_{n}\right\} _{n\in E}\subseteq H$ sistema ortogonale
(ortonormale), se $\forall $ $x\in H$ $\exists $ $!$ $\left\{ c_{n}\right\}
_{n\in E}\subseteq 
%TCIMACRO{\U{2102} }%
%BeginExpansion
\mathbb{C}
%EndExpansion
:x=\sum_{n\in E}c_{n}u_{n}$, $\left\{ u_{n}\right\} _{n\in E}$ si dice base
ortogonale (ortonormale) di $H$.

In tal caso tutte le propriet\`{a} enunciate nel teorema sopra sono
soddisfatte. Si pu\`{o} dimostrare grazie all'assioma della scelta che ogni
spazio di Hilbert $H$ separabile (cio\`{e} per il quale $\exists $ $%
S\subseteq H$ denso in $H$ e numerabile) ammette una base ortonormale al pi%
\`{u} numerabile (che si pu\`{o} costruire con l'algoritmo di Gram-Schmidt).
Se $\dim H=+\infty $ spesso non \`{e} affatto banale trovare una base.

\textbf{Def} Dato $\left( H,\left\langle \_,\_\right\rangle \right) $ spazio
di Hilbert, $E\subseteq 
%TCIMACRO{\U{2115} }%
%BeginExpansion
\mathbb{N}
%EndExpansion
$ e $\left\{ u_{n}\right\} _{n\in E}\subseteq H$ sistema ortogonale
(ortonormale), se $\forall $ $x\in H$ $\left\langle x,u_{n}\right\rangle =0$ 
$\forall $ $n\Longrightarrow x=0$, $\left\{ u_{n}\right\} _{n\in E}$ si dice
sistema ortogonale (ortonormale) completo di $H$.


Dato $\left( H,\left\langle \_,\_\right\rangle \right) $ spazio di Hilbert
con base ortonormale $\left\{ u_{n}\right\} _{n\in E}$, per definizione $%
\forall $ $x\in H$ $\exists $ $!$ $\left\{ c_{n}\right\} _{n\in
E}:x=\sum_{n\in E}c_{n}u_{n}$ e $c_{n}=\left\langle x,u_{n}\right\rangle $.
L'uguaglianza $x=\sum_{n\in E}\hat{x}_{n}u_{n}$ significa solo che la
successione delle somme parziali converge a $x$ nella norma indotta dal
prodotto scalare di $H$, quindi a seconda dello spazio in cui si lavora
occorreranno teoremi che permettano di dedurre anche altri tipi di
convergenza.

Se $E=%
%TCIMACRO{\U{2115} }%
%BeginExpansion
\mathbb{N}
%EndExpansion
$, i coefficienti $c_{n}$ si dicono coefficienti di Fourier astratti e si
indicano con $\hat{x}_{n}$, mentre la serie $\sum_{n=0}^{+\infty }\hat{x}%
_{n}u_{n}$ si dice serie di Fourier astratta di $x$. dal teorema segue
dunque che l'operatore $\mathcal{F}:H\rightarrow l^{2}$ che a ogni $x\in H$
associa la successione $\hat{x}_{n}=\left\langle x,u_{n}\right\rangle $ dei
suoi coefficienti di Fourier preserva il prodotto scalare e la norma, cio%
\`{e} \`{e} un'isometria: $\left\langle x,y\right\rangle
=\sum_{n=1}^{+\infty }\hat{x}_{n}\hat{y}_{n}$.

Poich\'{e} per il teorema $\sum_{n\in E}\left\vert \hat{x}_{n}\right\vert
^{2}<+\infty $, allora per la condizione necessaria di convergenza delle
serie reali $\lim_{n\rightarrow +\infty }\left\vert \hat{x}_{n}\right\vert
^{2}=0$, quindi $\hat{x}_{n}\rightarrow ^{n\rightarrow +\infty }0$.

\begin{enumerate}
\item Se $\left\{ u_{n}\right\} _{n\in E}$ \`{e} una base ortonormale di $H$%
, vale $x=\sum_{n\in E}\hat{x}_{n}u_{n}$. D'altra parte, se si fissa $k$ e
si prende $V=span\left\{ u_{1},...,u_{k}\right\} $, vale anche $%
x=p_{V}\left( x\right) +p_{V^{\perp }}\left( x\right) =\sum_{n=1}^{k}\hat{x}%
_{n}u_{n}+\sum_{n=k+1}^{+\infty }\hat{x}_{n}u_{n}$
\end{enumerate}

Si affronta ora l'esempio fondamentale di $H=L^{2}\left( \left[ -\pi ,\pi %
\right] \right) $ spazio vettoriale su $%
%TCIMACRO{\U{211d} }%
%BeginExpansion
\mathbb{R}
%EndExpansion
$, in cui $\left\langle f,g\right\rangle =\int_{-\pi }^{\pi }f\left(
t\right) g\left( t\right) dt$ (l'integrale \`{e} ben definito per la
disuguaglianza di Hoelder). Si pu\`{o} dimostrare che $\left\{ \frac{1}{%
\sqrt{2\pi }},\frac{\sin nt}{\sqrt{\pi }},\frac{\cos nt}{\sqrt{\pi }}%
\right\} _{n\in 
%TCIMACRO{\U{2115} }%
%BeginExpansion
\mathbb{N}
%EndExpansion
}$ \`{e} una base ortonormale per $H$ (\`{e} facile dimostrare che \`{e} un
sistema ortonormale, molto meno che \`{e} una base). Per quanto detto, se $%
f\in H$ la serie di Fourier di $f$ \`{e}%
\begin{equation*}
\sum_{n=0}^{+\infty }\left\langle f,u_{n}\right\rangle u_{n}=\frac{a_{0}}{%
\sqrt{2\pi }}+\sum_{n=1}^{+\infty }a_{n}\frac{\cos nt}{\sqrt{\pi }}%
+\sum_{n=1}^{+\infty }b_{n}\frac{\sin nt}{\sqrt{\pi }}
\end{equation*}

cio\`{e} $f=\left\langle f,\frac{1}{\sqrt{2\pi }}\right\rangle \frac{1}{%
\sqrt{2\pi }}+\sum_{n=1}^{+\infty }\left\langle f,\frac{\cos nt}{\sqrt{\pi }}%
\right\rangle \frac{\cos nt}{\sqrt{\pi }}+\sum_{n=1}^{+\infty }\left\langle
f,\frac{\sin nt}{\sqrt{\pi }}\right\rangle \frac{\sin nt}{\sqrt{\pi }}$. Dal
teorema visto \`{e} noto che la serie di Fourier converge a $f$ in norma $%
L^{2}$ (il che non implica la convergenza quasi ovunque). I termini con il
coseno catturano la parte pari, quelli con il seno la parte dispari di $f$:
infatti se $f$ \`{e} pari $b_{n}=\left\langle f,\frac{\sin nt}{\sqrt{\pi }}%
\right\rangle =0$ $\forall $ $n$, se $f$ \`{e} dispari $a_{n}=0$ $\forall $ $%
n$.

Ogni somma parziale della serie si dice polinomio trigonometrico (in
analogia con la serie di Taylor, per la quale le somme parziali sono
polinomi effettivi).

\begin{enumerate}
\item Considero la serie di Fourier $\sum_{k=1}^{+\infty }b_{k}\sin kx$ e la
sua somma $f$ (quando \`{e} ben definita) in $\left[ 0,\pi \right] $. Se $%
\sum_{k=1}^{+\infty }\left\vert b_{k}\right\vert ^{2}<+\infty $, allora $%
f\in L^{2}$. Infatti sono soddisfatte le condizioni di 2.19, per cui $%
\exists $ $f\in L^{2}:f\left( x\right) =\sum_{k=1}^{+\infty }b_{k}\sin kx$.
\end{enumerate}

\textbf{Def} Data $f:\left[ a,b\right] \rightarrow 
%TCIMACRO{\U{211d} }%
%BeginExpansion
\mathbb{R}
%EndExpansion
$, si dice che $f$ soddisfa la condizione di Dirichlet (D) in $x_{0}\in
\left( a,b\right) $ se (i) $f$ \`{e} derivabile in $x_{0}$ oppure (ii) $f$ 
\`{e} continua in $x_{0}$ ma ha un punto angoloso in $x_{0}$ oppure (iii) $f$
ha una discontinuit\`{a} a salto in $x_{0}$ ma esistono finiti $%
\lim_{x\rightarrow x_{0}^{-}}f\left( x\right) $ e $\lim_{x\rightarrow
x_{0}^{+}}f\left( x\right) $.

Se $f$ soddisfa la condizione di Dirichlet in tutto $\left[ a,b\right] $ allora \`{e} integrabile e $L^{2}$.

Si definisce funzione regolare a tratti una funzione $f:\left[ a,b\right]
\rightarrow 
%TCIMACRO{\U{211d} }%
%BeginExpansion
\mathbb{R}
%EndExpansion
$ derivabile con derivata prima continua in $\left[ a,b\right] $ a eccezione
di un numero finito di punti in cui $f$ pu\`{o} essere non derivabile o
addirittura discontinua, ma in cui esistono finiti i limiti destro e
sinistro sia di $f$ che di $f^{\prime }$.

\textbf{Teo 2.22 (convergenza puntuale della serie di Fourier)}%
\begin{gather*}
\text{Hp: }f:\left[ -\pi ,\pi \right] \rightarrow 
%TCIMACRO{\U{211d} }%
%BeginExpansion
\mathbb{R}
%EndExpansion
\text{ soddisfa (D) in }x_{0}\in \left( -\pi ,\pi \right) \\
\text{Ts: la serie di Fourier di }f\text{ converge in }x_{0}\text{ e la sua
somma vale }\frac{\lim_{x\rightarrow x_{0}^{-}}f\left( x\right)
+\lim_{x\rightarrow x_{0}^{+}}f\left( x\right) }{2}
\end{gather*}

Se vale (i) o (ii) di D $f$ \`{e} continua in $x_{0}$ e la somma \`{e} $%
f\left( x_{0}\right) $.

\textbf{Teo } 
\begin{gather*}
\text{Hp: }f:%
%TCIMACRO{\U{211d} }%
%BeginExpansion
\mathbb{R}
%EndExpansion
\rightarrow 
%TCIMACRO{\U{211d} }%
%BeginExpansion
\mathbb{R}
%EndExpansion
\text{ \`{e} }T\text{-periodica, continua, regolare a tratti in }\left[ 0,T%
\right] \text{; } \\
f\left( 0\right) =f\left( T\right) \text{, }a_{k}=\left\langle f,\frac{\cos
kt}{\sqrt{\pi }}\right\rangle ,b_{k}=\left\langle f,\frac{\sin kt}{\sqrt{\pi 
}}\right\rangle \\
\text{Ts: le serie }\sum_{k=1}^{+\infty }\left\vert a_{k}\right\vert
,\sum_{k=1}^{+\infty }\left\vert b_{k}\right\vert \text{ convergono}
\end{gather*}

\textbf{Prop 2.21 (convergenza assoluta della serie di Fourier)}%
\begin{gather*}
\text{Hp: }a_{n}=\left\langle f,\frac{\cos nt}{\sqrt{\pi }}\right\rangle
,b_{n}=\left\langle f,\frac{\sin nt}{\sqrt{\pi }}\right\rangle \text{, }%
\sum_{n=1}^{+\infty }\left\vert a_{n}\right\vert ,\sum_{n=1}^{+\infty
}\left\vert b_{n}\right\vert \text{ convergono} \\
\text{Ts: la serie di Fourier di }f\text{ converge assolutamente e
uniformemente in }\left[ -\pi ,\pi \right] \text{ a }f
\end{gather*}

La dimostrazione si basa sull'applicazione del criterio di Weierstrass: in
realt\`{a} la convergenza \`{e} totale. Se si applica questa proposizione, $f$ \`{e} continua.

Dalle ipotesi segue che $\sum_{n=1}^{+\infty }\left\vert a_{n}\right\vert
^{2},\sum_{n=1}^{+\infty }\left\vert b_{n}\right\vert ^{2}$ convergono,
quindi $f\in L^{2}$: le ipotesi pi\`{u} forti permettono di concludere di pi%
\`{u} sulla convergenza della serie di Fourier.

\begin{enumerate}
\item Considero la serie di Fourier $\sum_{k=1}^{+\infty }b_{k}\sin kx$ e la
sua somma $f$ (quando \`{e} ben definita) in $\left[ 0,\pi \right] $. Se $%
\sum_{k=1}^{+\infty }\left\vert b_{k}\right\vert $ converge, allora $%
\lim_{x\rightarrow 0^{+}}f\left( x\right) =0$. Infatti in tal caso la
convergenza a $f$ \`{e} uniforme, quindi $\lim_{x\rightarrow 0^{+}}f\left(
x\right) =\sum_{k=1}^{+\infty }\lim_{x\rightarrow 0^{+}}b_{k}\sin kx=0$.

\item Considero la serie di Fourier $\sum_{k=1}^{+\infty }b_{k}\sin kx$ e la
sua somma $f$ (quando \`{e} ben definita) in $\left[ 0,\pi \right] $. Se $%
\sum_{k=1}^{+\infty }k\left\vert b_{k}\right\vert $ converge, allora $f\in
C^{1}$. Infatti in tal caso la convergenza della serie delle derivate \`{e}
uniforme per 2.21, quindi\footnote{%
vedi teo analisi 2 su scambio serie e derivata} la funzione somma \`{e} $%
C^{1}$.
\end{enumerate}

\textbf{Teo 2.23 (convergenza uniforme della serie di Fourier)}%
\begin{gather*}
\text{Hp: }f:\left[ -\pi ,\pi \right] \rightarrow 
%TCIMACRO{\U{211d} }%
%BeginExpansion
\mathbb{R}
%EndExpansion
\text{ \`{e} }C^{1}\text{ tranne al pi\`{u} in un numero finito di punti
angolosi} \\
\text{Ts: la serie di Fourier di }f\text{ converge assolutamente e
uniformemente a }f\text{ in }\left[ -\pi ,\pi \right]
\end{gather*}

L'ipotesi implica che $f$ soddisfi (i) o (ii) di (D) in tutto $\left[ -\pi
,\pi \right] $ e che $f$ sia $L^{2}$. 2.23 \`{e} una conseguenza immediata
del teorema sopra e di 2.21.

Quindi (\textbf{corollario}) se $f\in C^{1}\left( \left[ -\pi ,\pi \right]
\right) $ vale la tesi del teorema: questo risultato \`{e} particolarmente
utile perch\'{e} l'ipotesi si verifica facilmente.

Si noti che se $f\in L^{1}\left( \left[ -\pi ,\pi \right] \right) $, poich%
\'{e} $u_{n}\in L^{\infty }$ $\forall $ $n$, per la disuguaglianza di
Hoelder si possono comunque costruire i coefficienti della serie di Fourier.
Si pu\`{o} dimostrare che esiste una serie di Fourier che diverge q.o. 

Il motivo per cui si considera proprio l'intervallo $\left[ -\pi ,\pi \right]
$ \`{e} che, avendo ottenuto dei risultati in tale intervallo, essi possono
essere estesi con periodicit\`{a} alle funzioni di variabile reale: quindi
le funzioni di variabile reale che si approssimano con la serie di Fourier
sono le funzioni periodiche. Si noti che mentre per scrivere la serie di
Taylor di una funzione occorre che essa sia $C^{\infty }$, la serie di
Fourier pu\`{o} essere scritta anche per funzioni discontinue: per questo i
risultati sulla convergenza della seconda sono molto pi\`{u} variegati.

Se invece $H=L^{2}\left( \left[ -\pi ,\pi \right] \right) $ spazio
vettoriale su $%
%TCIMACRO{\U{2102} }%
%BeginExpansion
\mathbb{C}
%EndExpansion
$, in cui $\left\langle f,g\right\rangle =\int_{-\pi }^{\pi }f\left(
t\right) \bar{g}\left( t\right) dt$, si pu\`{o} dimostrare che $\left\{ 
\frac{e^{int}}{\sqrt{2\pi }}\right\} _{n\in 
%TCIMACRO{\U{2124} }%
%BeginExpansion
\mathbb{Z}
%EndExpansion
}$ \`{e} una base ortonormale per $H$. Per quanto detto, se $f\in H$ la
serie di Fourier di $f$ \`{e}%
\begin{equation*}
\sum_{n\in 
%TCIMACRO{\U{2124} }%
%BeginExpansion
\mathbb{Z}
%EndExpansion
}\left\langle f,u_{n}\right\rangle u_{n}=\sum_{n\in 
%TCIMACRO{\U{2124} }%
%BeginExpansion
\mathbb{Z}
%EndExpansion
}c_{n}\frac{e^{int}}{\sqrt{2\pi }}=\sum_{n\in 
%TCIMACRO{\U{2124} }%
%BeginExpansion
\mathbb{Z}
%EndExpansion
}\left\langle f,\frac{e^{int}}{\sqrt{2\pi }}\right\rangle \frac{e^{int}}{%
\sqrt{2\pi }}
\end{equation*}

Se $f\in C^{1}\left( 
%TCIMACRO{\U{211d} }%
%BeginExpansion
\mathbb{R}
%EndExpansion
;%
%TCIMACRO{\U{2102} }%
%BeginExpansion
\mathbb{C}
%EndExpansion
\right) $ \`{e} periodica di periodo $2\pi $, $f^{\prime }\in C^{0}$ e $%
f^{\prime }\in L^{2}\left( \left[ -\pi ,\pi \right] \right) $ (in quanto
continua in un compatto, \`{e} limitata e quindi appartiene a $L^{p}$ $%
\forall $ $p$). Come gi\`{a} notato, $\left( \hat{f}^{\prime }\right)
_{n}\rightarrow ^{n\rightarrow +\infty }0$; ma poich\'{e} $\left( \hat{f}%
^{\prime }\right) _{n}=\frac{1}{\sqrt{2\pi }}\int_{-\pi }^{\pi }f^{\prime
}\left( t\right) e^{-int}dt=\frac{1}{\sqrt{2\pi }}\left[ f\left( t\right)
e^{-int}\right] _{-\pi }^{\pi }-\frac{-in}{\sqrt{2\pi }}\int_{-\pi }^{\pi
}f\left( t\right) e^{-int}dt=\frac{in}{\sqrt{2\pi }}\int_{-\pi }^{\pi
}f\left( t\right) e^{-int}dt$ (grazie alla formula di integrazione per parti
e alla periodicit\`{a} di $f$), si ottiene $in\hat{f}_{n}\rightarrow
^{n\rightarrow +\infty }0$, il che significa che $\hat{f}_{n}=o\left( \frac{1%
}{n}\right) $ per $n\rightarrow +\infty $. Reiterando questo procedimento si
ricava che se $f\in C^{k}$ allora $\hat{f}_{n}=o\left( \frac{1}{n^{k}}%
\right) $: pi\`{u} $f$ \`{e} regolare, pi\`{u} in fretta i coefficienti di
Fourier di $f$ tendono a $0$. Si osserver\`{a} un fenomeno analogo per la
trasformata di Fourier.

\begin{enumerate}
\item Il ragionamento vale anche in campo reale per funzioni pari. Se ad
esempio $f\left( x\right) =e^{\left\vert x\right\vert }$ con serie di
Fourier $\frac{a_{0}}{\sqrt{2\pi }}+\sum_{n=1}^{+\infty }a_{n}\frac{\cos nt}{%
\sqrt{\pi }}$, $f$ \`{e} $C^{\infty }$ in $\left[ -\pi ,\pi \right] $ (a
meno di un insieme di misura nulla), quindi i suoi coefficienti di Fourier
tendono a $0$ pi\`{u} in fretta di qualsiasi polinomio. Questo implica
peraltro che $\sum_{n=1}^{+\infty }\left\vert a_{n}\right\vert $ converge e
dunque la serie di Fourier converge uniformemente a $f$.

Non si pu\`{o} concludere lo stesso per $f\left( x\right) =\left\{ 
\begin{array}{c}
e^{x}\text{ se }x\geq 0 \\ 
-e^{-x}\text{ se }x<0%
\end{array}%
\right. $, dato che $f\left( \pi \right) \neq f\left( -\pi \right) $.
\end{enumerate}

\subsection{Convoluzione}

Si definisce un'operazione tra funzioni che, come si vedr\`{a} nel seguito,
ha una propriet\`{a} di centrale importanza che permette di risolvere le
equazioni differenziali con la trasformata di Fourier.

\textbf{Def} Date $f,g\in L^{1}\left( 
%TCIMACRO{\U{211d} }%
%BeginExpansion
\mathbb{R}
%EndExpansion
^{n}\right) $, si dice convoluzione di $f$ e $g$ la funzione $\left( f\ast
g\right) \left( x\right) :=\int_{%
%TCIMACRO{\U{211d} }%
%BeginExpansion
\mathbb{R}
%EndExpansion
^{n}}f\left( y\right) g\left( x-y\right) dy$.

La seguente disuguaglianza mostra che la definizione \`{e} ben posta: se $%
f,g\in L^{1}\left( 
%TCIMACRO{\U{211d} }%
%BeginExpansion
\mathbb{R}
%EndExpansion
^{n}\right) $ $\int_{%
%TCIMACRO{\U{211d} }%
%BeginExpansion
\mathbb{R}
%EndExpansion
^{n}}f\left( y\right) g\left( x-y\right) dy$ esiste finito.

\textbf{Teo 2.24 (disuguaglianza di Young)}%
\begin{gather*}
\text{Hp}\text{: }f\in L^{p}\left( 
%TCIMACRO{\U{211d} }%
%BeginExpansion
\mathbb{R}
%EndExpansion
^{n}\right) ,g\in L^{q}\left( 
%TCIMACRO{\U{211d} }%
%BeginExpansion
\mathbb{R}
%EndExpansion
^{n}\right) \text{, }\frac{1}{p}+\frac{1}{q}=1+\frac{1}{r} \\
\text{Ts}\text{: }\exists \text{ }f\ast g\in L^{r}\text{ e }\left\vert
\left\vert f\ast g\right\vert \right\vert _{L^{r}}\leq \left\vert \left\vert
f\right\vert \right\vert _{L^{p}}\left\vert \left\vert g\right\vert
\right\vert _{L^{q}}
\end{gather*}

Si noti che se in particolare $p=q=r=1$ si ha $\left\vert \left\vert f\ast
g\right\vert \right\vert _{L^{1}}\leq \left\vert \left\vert f\right\vert
\right\vert _{L^{1}}\left\vert \left\vert g\right\vert \right\vert _{L^{1}}$.

Il teorema vale anche per $L^{\infty }$ con la convenzione $\frac{1}{\infty }%
=0$.

Si pu\`{o} mostrare che il prodotto di convoluzione \`{e} commutativo,
associativo e distributivo.

Il teorema sopra permette di determinare la norma di vari operatori limitati.

\begin{enumerate}
\item Dati $X=Y=L^{p}\left( 
%TCIMACRO{\U{211d} }%
%BeginExpansion
\mathbb{R}
%EndExpansion
^{n}\right) $, $f\in L^{1}\left( 
%TCIMACRO{\U{211d} }%
%BeginExpansion
\mathbb{R}
%EndExpansion
^{n}\right) $, definisco l'operatore di convoluzione $T_{f}:L^{p}\left( 
%TCIMACRO{\U{211d} }%
%BeginExpansion
\mathbb{R}
%EndExpansion
^{n}\right) \rightarrow L^{p}\left( 
%TCIMACRO{\U{211d} }%
%BeginExpansion
\mathbb{R}
%EndExpansion
^{n}\right) $, $T_{f}\left( g\right) =f\ast g$. Per 2.24, se $g\in
L^{p}\left( 
%TCIMACRO{\U{211d} }%
%BeginExpansion
\mathbb{R}
%EndExpansion
^{n}\right) $ anche $f\ast g\in L^{p}\left( 
%TCIMACRO{\U{211d} }%
%BeginExpansion
\mathbb{R}
%EndExpansion
^{n}\right) $, quindi \`{e} ben definito, e inoltre lineare. Poich\'{e} $%
\left\vert \left\vert f\ast g\right\vert \right\vert _{L^{p}}\leq \left\vert
\left\vert f\right\vert \right\vert _{L^{1}}\left\vert \left\vert
g\right\vert \right\vert _{L^{p}}$, $T_{f}$ \`{e} lineare e continuo con $%
\left\vert \left\vert T\right\vert \right\vert _{\tciLaplace \left(
X,Y\right) }\leq \left\vert \left\vert f\right\vert \right\vert _{L^{1}}$.

\item Considero la trasformata di Fourier su $L^{1}\left( 
%TCIMACRO{\U{211d} }%
%BeginExpansion
\mathbb{R}
%EndExpansion
^{n}\right) $ (che si vedr\`{a} meglio pi\`{u} avanti): $\forall $ $f\in
L^{1}\left( 
%TCIMACRO{\U{211d} }%
%BeginExpansion
\mathbb{R}
%EndExpansion
^{n}\right) $ si definisce trasformata di Fourier di $f$ $\hat{f}\left( \xi
\right) =\int_{%
%TCIMACRO{\U{211d} }%
%BeginExpansion
\mathbb{R}
%EndExpansion
^{n}}f\left( \mathbf{x}\right) e^{-i\left\langle \mathbf{x},\xi
\right\rangle }d\mathbf{x}$, ben definita perch\'{e} l'argomento \`{e}
assolutamente integrabile. Si pu\`{o} dimostrare che se $f\in L^{1}\left( 
%TCIMACRO{\U{211d} }%
%BeginExpansion
\mathbb{R}
%EndExpansion
^{n}\right) $, $\hat{f}\in C^{0}\left( 
%TCIMACRO{\U{211d} }%
%BeginExpansion
\mathbb{R}
%EndExpansion
^{n}\right) $ e $\lim_{\left\vert \xi \right\vert \rightarrow +\infty }\hat{f%
}\left( \xi \right) =0$: quindi $\hat{f}\in C_{\ast }^{0}\left( 
%TCIMACRO{\U{211d} }%
%BeginExpansion
\mathbb{R}
%EndExpansion
^{n}\right) $ e posso usare la norma del massimo. Allora si pu\`{o} vedere
la trasformata come un operatore $\mathcal{F}:L^{1}\left( 
%TCIMACRO{\U{211d} }%
%BeginExpansion
\mathbb{R}
%EndExpansion
^{n}\right) \rightarrow C_{\ast }^{0}\left( 
%TCIMACRO{\U{211d} }%
%BeginExpansion
\mathbb{R}
%EndExpansion
^{n}\right) $. Inoltre $\forall $ $\xi $ $\left\vert \hat{f}\left( \xi
\right) \right\vert \leq \int_{%
%TCIMACRO{\U{211d} }%
%BeginExpansion
\mathbb{R}
%EndExpansion
^{n}}\left\vert f\left( \mathbf{x}\right) \right\vert d\mathbf{x=}\left\vert
\left\vert f\right\vert \right\vert _{L^{1}}$, quindi anche $\max_{\xi \in 
%TCIMACRO{\U{211d} }%
%BeginExpansion
\mathbb{R}
%EndExpansion
^{n}}\left\vert \hat{f}\left( \xi \right) \right\vert =\left\vert \left\vert 
\hat{f}\right\vert \right\vert _{C_{\ast }^{0}\left( 
%TCIMACRO{\U{211d} }%
%BeginExpansion
\mathbb{R}
%EndExpansion
^{n}\right) }\leq \left\vert \left\vert f\right\vert \right\vert _{L^{1}}$:
l'operatore \`{e} lineare e continuo con norma operatoriale $k\leq 1$.
\end{enumerate}

\textbf{Teo 2.25 (regolarit\`{a} della convoluzione)}

\begin{gather*}
\text{(1) Hp}\text{: }f\in L^{p}\left( 
%TCIMACRO{\U{211d} }%
%BeginExpansion
\mathbb{R}
%EndExpansion
^{n}\right) ,\phi \in C^{0}\left( 
%TCIMACRO{\U{211d} }%
%BeginExpansion
\mathbb{R}
%EndExpansion
^{n}\right) \\
\text{Ts: }f\ast \phi \in C^{0}\left( 
%TCIMACRO{\U{211d} }%
%BeginExpansion
\mathbb{R}
%EndExpansion
^{n}\right) \\
\text{(2) Hp}\text{: }f\in L^{p}\left( 
%TCIMACRO{\U{211d} }%
%BeginExpansion
\mathbb{R}
%EndExpansion
^{n}\right) ,\phi \in C^{1}\left( 
%TCIMACRO{\U{211d} }%
%BeginExpansion
\mathbb{R}
%EndExpansion
^{n}\right) \\
\text{Ts: }f\ast \phi \in C^{1}\left( 
%TCIMACRO{\U{211d} }%
%BeginExpansion
\mathbb{R}
%EndExpansion
^{n}\right) \text{ e }\frac{\partial \left( f\ast \phi \right) }{\partial
x_{i}}=f\ast \frac{\partial \phi }{\partial x_{i}}
\end{gather*}

In generale, la convoluzione tra due funzioni eredita la regolarit\`{a}
della funzione pi\`{u} regolare tra le due.

Inoltre se $f\in L^{p}\left( 
%TCIMACRO{\U{211d} }%
%BeginExpansion
\mathbb{R}
%EndExpansion
^{n}\right) ,\phi \in C^{1}\left( 
%TCIMACRO{\U{211d} }%
%BeginExpansion
\mathbb{R}
%EndExpansion
^{n}\right) $, allora $\nabla \left( f\ast \phi \right) =f\ast \nabla \phi $%
. Se $\phi \in C^{\infty }$, $f\ast \phi \in C^{\infty }$.

Se $\phi ,\psi \in L^{1}\left( 
%TCIMACRO{\U{211d} }%
%BeginExpansion
\mathbb{R}
%EndExpansion
\right) $ hanno supporti rispettivamente $I,J$ compatti, anche $\phi \ast
\psi $ \`{e} a supporto compatto. Questo si pu\`{o} giustificare
informalmente con il seguente ragionamento. $\left( \phi \ast \psi \right)
\left( x\right) =\int_{%
%TCIMACRO{\U{211d} }%
%BeginExpansion
\mathbb{R}
%EndExpansion
}\phi \left( y\right) I_{I}\left( y\right) \psi \left( x-y\right)
I_{J}\left( x-y\right) dy=\int_{%
%TCIMACRO{\U{211d} }%
%BeginExpansion
\mathbb{R}
%EndExpansion
}\phi \left( y\right) I_{I}\left( y\right) \psi \left( x-y\right)
I_{J+x}\left( y\right) dy$, dove $J+x$ indica sinteticamente - posto $J=%
\left[ a_{1},b_{1}\right] \cup ...\cup \left[ a_{n},b_{n}\right] $ -
l'insieme $\left[ x-b_{1},x+a_{1}\right] \cup ...\cup \left[ x-b_{n},x+a_{n}%
\right] $. Allora $\left( \phi \ast \psi \right) \left( x\right) =\int_{%
%TCIMACRO{\U{211d} }%
%BeginExpansion
\mathbb{R}
%EndExpansion
}\phi \left( y\right) \psi \left( y\right) I_{I\cap J+x}\left( y\right) dy$
e se $x$ \`{e} abbastanza grande in modulo si ha che $I\cap J+x=\varnothing $%
, cio\`{e} al di fuori di un intervallo del tipo $\left[ -a,a\right] $ per $%
x $ si ha $\left( \phi \ast \psi \right) \left( x\right) =0$.

Un'applicazione della convoluzione \`{e} l'approssimazione di funzioni
irregolari con funzioni regolari definite mediante la convoluzione.

\textbf{Def} Data $\left\{ \rho _{k}\right\} _{k\in 
%TCIMACRO{\U{2115} }%
%BeginExpansion
\mathbb{N}
%EndExpansion
}\subseteq \mathcal{D}\left( \Omega \right) $, essa si dice successione
regolarizzante se $\rho _{k}\left( y\right) \geq 0$ $\forall $ $y,\forall $ $%
k$, se $\int_{%
%TCIMACRO{\U{211d} }%
%BeginExpansion
\mathbb{R}
%EndExpansion
}\rho _{k}\left( y\right) dy=1$ $\forall $ $k$ ed esiste una successione $%
\varepsilon _{k}$ positiva e infinitesima tale che supp$\left( \rho
_{k}\right) \subseteq \mathcal{B}_{\varepsilon _{k}}$ $\forall $ $k$.

Quindi le $\rho _{k}$ hanno supporto sempre pi\`{u} piccolo; di contro, dato
che l'integrale \`{e} sempre uno, assumono in tale supporto valori sempre pi%
\`{u} grandi.

\textbf{Teo 2.26 (regolarizzazione di una funzione }$L^{p}$\textbf{)}%
\begin{gather*}
\text{Hp}\text{: }f\in L^{p}\left( A\right) ,A\subseteq 
%TCIMACRO{\U{211d} }%
%BeginExpansion
\mathbb{R}
%EndExpansion
\\
\text{Ts}\text{: }\exists \text{ }\left\{ f_{k}\right\} _{k\in 
%TCIMACRO{\U{2115} }%
%BeginExpansion
\mathbb{N}
%EndExpansion
}\subseteq \mathcal{D}\left( A\right) :f_{k}\rightarrow ^{L^{p}}f\text{ e }%
f_{k}\rightarrow f\text{ q.o.}
\end{gather*}

Quindi una funzione $L^{p}$ \`{e} limite di una successione di funzioni $%
\mathcal{D}\left( \Omega \right) $, cio\`{e} estremamente regolari.

Uno dei modi per costruire $\left\{ f_{k}\right\} $ \`{e} il seguente.
Fissata $\rho \in \mathcal{D}\left( \Omega \right) $ tale che $\int_{%
%TCIMACRO{\U{211d} }%
%BeginExpansion
\mathbb{R}
%EndExpansion
}\rho \left( y\right) dy=1$ ($\rho $ si dice nucleo mollificatore), si
definisce $\rho _{k}\left( x\right) =k\rho \left( kx\right) $: allora $\rho
_{k}$ \`{e} tale che $\int_{%
%TCIMACRO{\U{211d} }%
%BeginExpansion
\mathbb{R}
%EndExpansion
}\rho _{k}\left( x\right) dx=1$ e, se supp$\left( \rho \right) \subseteq 
\mathcal{B}_{r}$, supp$\left( \rho _{k}\right) \subseteq \mathcal{B}_{\frac{r%
}{k}}$.
\begin{enumerate}
\item $\rho \left( y\right) =e^{\frac{1}{y^{2}-1}}I_{\left( \left\vert
y\right\vert <1\right) }\left( y\right) $ \`{e} un nucleo mollificatore.

\item $\rho \left( y\right) =\frac{1}{2}I_{\left( -1,1\right) }\left(
y\right) $ ha tutte le propriet\`{a} richieste, eccetto l'essere $C^{\infty
} $.
\end{enumerate}

Si definisce $f_{k}\left( x\right) =f\left( x\right) \ast \rho _{k}\left(
x\right) =\int_{%
%TCIMACRO{\U{211d} }%
%BeginExpansion
\mathbb{R}
%EndExpansion
}f\left( y\right) k\rho \left( k\left( x-y\right) \right) dy$: $f_{k}\in
C^{\infty }$ per le propriet\`{a} di regolarit\`{a} della convoluzione.
Allora $f_{k}\rightarrow f$ q.o. per $k\rightarrow +\infty $ e $%
f_{k}\rightarrow ^{L^{p}}f$.

\begin{enumerate}
\item Fissato $A=\left[ -1,1\right] $, ci si pu\`{o} convincere della verit%
\`{a} della prima convergenza considerando il nucleo mollificatore $\rho
\left( y\right) =\frac{1}{2}I_{\left[ -1,1\right] }\left( y\right) $. $\rho
_{1/\varepsilon }\left( x\right) =\frac{1}{2\varepsilon }I_{\left(
-\varepsilon ,\varepsilon \right) }\left( x\right) $: allora $f_{\varepsilon
}\left( x\right) =\int_{%
%TCIMACRO{\U{211d} }%
%BeginExpansion
\mathbb{R}
%EndExpansion
}f\left( y\right) \frac{1}{2\varepsilon }I_{\left( -\varepsilon ,\varepsilon
\right) }\left( x-y\right) dy=\int_{%
%TCIMACRO{\U{211d} }%
%BeginExpansion
\mathbb{R}
%EndExpansion
}f\left( y\right) \frac{1}{2\varepsilon }I_{\left( x-\varepsilon
,x+\varepsilon \right) }\left( y\right) dy$, cio\`{e} $f_{\varepsilon
}\left( x\right) =\frac{1}{2\varepsilon }\int_{x-\varepsilon
}^{x+\varepsilon }f\left( y\right) dy$: per il teorema della media integrale
- che vale se $f$\ \`{e} continua - $\exists $\ $x^{\ast }\in \left(
x-\varepsilon ,x+\varepsilon \right) :f_{\varepsilon }\left( x\right)
=f\left( x^{\ast }\right) $, dunque, fissato $x$, $\lim_{\varepsilon
\rightarrow 0^{+}}f_{\varepsilon }\left( x\right) =f\left( x\right) $.
\end{enumerate}

\section{Teoria delle distribuzioni}

\textbf{Def} Dato $\Omega \subseteq 
%TCIMACRO{\U{211d} }%
%BeginExpansion
\mathbb{R}
%EndExpansion
^{n}$ aperto e $v:\Omega \rightarrow 
%TCIMACRO{\U{211d} }%
%BeginExpansion
\mathbb{R}
%EndExpansion
$ continua, si dice supporto di $v$ supp$\left( v\right) :=clos\left(
\left\{ x\in \Omega :v\left( x\right) \neq 0\right\} \right) $.

La chiusura \`{e} fatta in $%
%TCIMACRO{\U{211d} }%
%BeginExpansion
\mathbb{R}
%EndExpansion
^{n}$.

\begin{enumerate}
\item $f\left( x\right) =x\sin \frac{1}{x}I_{\left( x\neq 0\right) }$ \`{e}
nulla in $\left\{ \frac{1}{k\pi }:k\in 
%TCIMACRO{\U{2124} }%
%BeginExpansion
\mathbb{Z}
%EndExpansion
\right\} \cup \left\{ 0\right\} $, la chiusura del cui complementare \`{e} $%
%TCIMACRO{\U{211d} }%
%BeginExpansion
\mathbb{R}
%EndExpansion
$, dunque supp$\left( f\right) =%
%TCIMACRO{\U{211d} }%
%BeginExpansion
\mathbb{R}
%EndExpansion
$.

\item $g\left( x\right) =e^{-\frac{1}{1-x^{2}}}I_{\left( \left\vert
x\right\vert <1\right) }$ ha supp$\left( g\right) =\left[ -1,1\right] $
\end{enumerate}

\textbf{Def} Dato $\Omega \subseteq 
%TCIMACRO{\U{211d} }%
%BeginExpansion
\mathbb{R}
%EndExpansion
^{n}$ aperto, $C^{\infty }\left( \Omega \right) :=\left\{ v:\Omega
\rightarrow 
%TCIMACRO{\U{211d} }%
%BeginExpansion
\mathbb{R}
%EndExpansion
:D^{\alpha }v\text{ \`{e} continua in }\Omega \text{ }\forall \text{ }\alpha 
\text{ multindice}\right\} $.

Con multiindice si intende un vettore $\alpha =\left( \alpha _{1},...,\alpha
_{n}\right) \in 
%TCIMACRO{\U{2115} }%
%BeginExpansion
\mathbb{N}
%EndExpansion
^{n}$ tale che $D^{\alpha }v=\frac{\partial ^{\left( \alpha _{1}\right) }v}{%
\partial x_{1}^{\left( \alpha _{1}\right) }}...\frac{\partial ^{\left(
\alpha _{n}\right) }v}{\partial x_{n}^{\left( \alpha _{n}\right) }}$: $%
\alpha _{i}$ indica il numero di volte che si deriva $v$ rispetto a $x_{i}$.

\textbf{Def} Dato $\Omega \subseteq 
%TCIMACRO{\U{211d} }%
%BeginExpansion
\mathbb{R}
%EndExpansion
^{n}$ aperto, si dice spazio delle funzioni test $\mathcal{D}\left( \Omega
\right) :=\left\{ v:\Omega \rightarrow 
%TCIMACRO{\U{211d} }%
%BeginExpansion
\mathbb{R}
%EndExpansion
:v\in C^{\infty }\left( \Omega \right) \text{ e supp}\left( v\right) \text{ 
\`{e} compatto in }%
%TCIMACRO{\U{211d} }%
%BeginExpansion
\mathbb{R}
%EndExpansion
^{n}\right\} $.

Se $v\in \mathcal{D}\left( \Omega \right) $, allora $v$ \`{e} limitata (e lo
sono anche tutte le sue derivate), in quanto continua su un compatto: questo
implica che $v\in L^{p}\left( \Omega \right) $ $\forall $ $p$, in quanto
funzione limitata su un insieme di misura finita. Sia $C^{\infty }$ che $%
\mathcal{D}$ sono spazi vettoriali rispetto all'usuale somma tra funzioni,
ma si pu\`{o} dimostrare che non esiste una norma che li renda spazi di
Banach (sono solo spazi topologici localmente convessi). Non c'\`{e} quindi
una scelta naturale della metrica in base a cui definire la convergenza.

\textbf{Def} Data una successione di funzioni test $\left\{ \phi
_{n}\right\} _{n\in 
%TCIMACRO{\U{2115} }%
%BeginExpansion
\mathbb{N}
%EndExpansion
}\subseteq \mathcal{D}\left( \Omega \right) $, si dice che $\phi _{n}$
converge a $\phi $ in $\mathcal{D}\left( \Omega \right) $, e si scrive $\phi
_{n}\rightarrow ^{\mathcal{D}\left( \Omega \right) }\phi $ per $n\rightarrow
+\infty $, se valgono le seguenti condizioni:

\begin{description}
\item[D1] $\exists $ $K\subseteq \Omega ,K\neq \varnothing $ compatto tale
che supp$\left( \phi _{n}\right) \subseteq K$ $\forall $ $n$

\item[D2] $D^{\alpha }\phi _{n}\rightarrow D^{\alpha }\phi $ uniformemente $%
\forall $ $\alpha $ multiindice
\end{description}

La condizione saliente \`{e} D2, la convergenza uniforme di tutte le
derivate di $\phi _{n}$ (e, prendendo $\alpha =\mathbf{0}$, di $\phi _{n}$
stessa). Richiedere la convergenza uniforme \`{e} naturale, essendo $%
\mathcal{D}\left( \Omega \right) $ uno spazio di funzioni $C^{\infty }$: se $%
\phi _{n}\rightarrow \phi $, D2 implica che allora anche $\phi \in C^{\infty
}$, D1 implica che $\phi $ ha supporto compatto.

\textbf{Def} Data una successione di funzioni test $\left\{ \phi
_{n}\right\} _{n\in 
%TCIMACRO{\U{2115} }%
%BeginExpansion
\mathbb{N}
%EndExpansion
}\subseteq \mathcal{D}\left( \Omega \right) $, si dice che $\phi _{n}$ \`{e}
di Cauchy in $\mathcal{D}\left( \Omega \right) $ se valgono le seguenti
condizioni:

\begin{description}
\item[D1] $\exists $ $K\subseteq \Omega ,K\neq \varnothing $ compatto tale
che supp$\left( \phi _{n}\right) \subseteq K$ $\forall $ $n$

\item[D2] $\forall $ $\alpha $ multiindice $D^{\alpha }\phi _{n}$ \`{e} di
Cauchy uniformemente, cio\`{e} $\forall $ $\varepsilon >0$ $\exists $ $%
n_{0}:m,n>n_{0}\Longrightarrow \max_{x\in K}\left\vert D^{\alpha }\phi
_{m}-D^{\alpha }\phi _{n}\right\vert <\varepsilon $.
\end{description}

Si pu\`{o} dimostrare che ogni successione di Cauchy in $\mathcal{D}\left(
\Omega \right) $ \`{e} convergente: segue dal criterio di Cauchy per la
convergenza uniforme applicato a $D^{\alpha }\phi _{n}$.

\textbf{Def} Dato $X$ spazio vettoriale su $%
%TCIMACRO{\U{211d} }%
%BeginExpansion
\mathbb{R}
%EndExpansion
\left( 
%TCIMACRO{\U{2102} }%
%BeginExpansion
\mathbb{C}
%EndExpansion
\right) $, si dice funzionale lineare su $X$ una funzione $L:X\rightarrow 
%TCIMACRO{\U{211d} }%
%BeginExpansion
\mathbb{R}
%EndExpansion
\left( 
%TCIMACRO{\U{2102} }%
%BeginExpansion
\mathbb{C}
%EndExpansion
\right) $ tale che $L\left( \alpha x+\beta y\right) =\alpha L\left( x\right)
+\beta L\left( y\right) $ $\forall $ $x,y\in X,\forall $ $\alpha ,\beta \in 
%TCIMACRO{\U{211d} }%
%BeginExpansion
\mathbb{R}
%EndExpansion
\left( 
%TCIMACRO{\U{2102} }%
%BeginExpansion
\mathbb{C}
%EndExpansion
\right) $. L'insieme dei funzionali lineari su $X$ si dice duale algebrico
di $X$ e si indica con $X^{\ast }$.

Tipicamente si sceglie come $X$ uno spazio di funzioni. Un funzionale \`{e}
una generica funzione $L:X\rightarrow 
%TCIMACRO{\U{211d} }%
%BeginExpansion
\mathbb{R}
%EndExpansion
\left( 
%TCIMACRO{\U{2102} }%
%BeginExpansion
\mathbb{C}
%EndExpansion
\right) $ e non \`{e} un operatore: $L\left( x\right) $ non \`{e} un
elemento di un altro spazio di funzioni, ma uno scalare.

\begin{enumerate}
\item Se $X=%
%TCIMACRO{\U{211d} }%
%BeginExpansion
\mathbb{R}
%EndExpansion
^{n}$, $X^{\prime }$ \`{e} l'insieme delle applicazioni lineari su $%
%TCIMACRO{\U{211d} }%
%BeginExpansion
\mathbb{R}
%EndExpansion
^{n}$.
\end{enumerate}

\textbf{Def} Dato $X$ spazio vettoriale in cui \`{e} definita una
convergenza per successioni e $L:X\rightarrow 
%TCIMACRO{\U{211d} }%
%BeginExpansion
\mathbb{R}
%EndExpansion
\left( 
%TCIMACRO{\U{2102} }%
%BeginExpansion
\mathbb{C}
%EndExpansion
\right) $ funzionale su $X$, $L$ si dice funzionale continuo se $\forall $ $%
x\in X,\forall $ $\left\{ x_{n}\right\} \subseteq X:x_{n}\rightarrow
^{n\rightarrow +\infty }x$ vale $L\left( x_{n}\right) \rightarrow
^{n\rightarrow +\infty }L\left( x\right) $. L'insieme dei funzionali lineari
e continui su $X$ si dice duale continuo di $X$ e si indica con $X^{\prime }$%
.

La convergenza $x_{n}\rightarrow ^{n\rightarrow +\infty }x$ \`{e} quella
definita in $X$ (non necessariamente una convergenza in norma), $L\left(
x_{n}\right) \rightarrow ^{n\rightarrow +\infty }L\left( x\right) $ \`{e} la
consueta convergenza scalare.
[DA QUI IN POI $\ast $ INDICA IL DUALE
CONTINUO]

[Si pu\`{o} dimostrare che negli spazi di Hilbert
vale il teorema di rappresentazione di Riesz: se $\left( H,\left\langle
\_,\_\right\rangle \right) $ \`{e} di Hilbert e $L\in H^{\ast }$, $L$ \`{e}
continuo $\Longleftrightarrow \exists $ $!$ $y\in H:L\left( x\right)
=\left\langle x,y\right\rangle $ $\forall $ $x\in H$. Quindi ogni funzionale
lineare continuo su uno spazio di Hilbert pu\`{o} essere rappresentato come
un prodotto scalare, con un fattore fissato.

La dimostrazione di $\Longleftarrow $ \`{e} immediata: se $x_{n}\rightarrow
x $, allora $\left\vert L\left( x_{n}\right) -L\left( x\right) \right\vert
=\left\vert \left\langle x_{n}-x,y\right\rangle \right\vert \leq \left\vert
\left\vert x_{n}-x\right\vert \right\vert \left\vert \left\vert y\right\vert
\right\vert \rightarrow ^{n\rightarrow +\infty }0$, cio\`{e} $L\left(
x_{n}\right) \rightarrow L\left( x\right) $, quindi $L$ \`{e} continuo.]

\textbf{Def} Dato $u:\mathcal{D}\left( \Omega \right) \rightarrow 
%TCIMACRO{\U{211d} }%
%BeginExpansion
\mathbb{R}
%EndExpansion
\left( 
%TCIMACRO{\U{2102} }%
%BeginExpansion
\mathbb{C}
%EndExpansion
\right) $ funzionale lineare e continuo sullo spazio delle funzioni test, $u$
si dice distribuzione (o funzione generalizzata) su $\Omega $.

Si noti che la convergenza di $\phi _{n}\rightarrow \phi $ a cui si fa
riferimento, necessaria per definire la continuit\`{a} di $L$, \`{e} quella
definita sopra con D1, D2. Per indicare $u\left( \phi \right) $ si usa
spesso la notazione $\left\langle u,\phi \right\rangle $ (che \`{e} un abuso
di notazione perch\'{e} non \`{e} definito alcun prodotto scalare in $%
\mathcal{D}\left( \Omega \right) $), in analogia con la scrittura che si usa
negli spazi di Hilbert, dove - per il teorema di Riesz - l'azione di ogni
funzionale lineare pu\`{o} essere scritta come prodotto scalare.

\textbf{Def} Dato $\mathcal{D}^{\prime }\left( \Omega \right) $ insieme
delle distribuzioni su $\Omega $ e una successione di distribuzioni $\left\{
u_{j}\right\} _{j\in 
%TCIMACRO{\U{2115} }%
%BeginExpansion
\mathbb{N}
%EndExpansion
}\subseteq $ $\mathcal{D}^{\prime }\left( \Omega \right) $, si dice che $%
u_{j}$ converge a $u$, e si scrive $u_{j}\rightarrow ^{\mathcal{D}^{\prime
}\left( \Omega \right) }u$ per $j\rightarrow +\infty $, se $u_{j}\left( \phi
\right) \rightarrow ^{j\rightarrow +\infty }u\left( \phi \right) $ $\forall $
$\phi \in $ $\mathcal{D}\left( \Omega \right) $.

La convergenza definita si dice convergenza successionale ed \`{e} di fatto
una convergenza puntuale. Si pu\`{o} mostrare che se $\lim_{j\rightarrow
+\infty }u_{j}\left( \phi \right) $ \`{e} finito $\forall $ $\phi $, allora
esiste una distribuzione cui $\left\{ u_{j}\right\} $ converge nel senso
delle distribuzioni. 

\begin{enumerate}
\item Se $f_{n}\left( x\right) =n^{2}xe^{-n\left\vert x\right\vert }$, $%
u_{f_{n}}\rightarrow ^{\mathcal{D}^{\prime }\left( \Omega \right) }0$.
Infatti, fissata $\phi $, $u_{f_{n}}\left( \phi \right) =\int_{-\infty
}^{+\infty }n^{2}xe^{-n\left\vert x\right\vert }\phi \left( x\right)
dx=\int_{-\infty }^{0}n^{2}xe^{nx}\phi \left( x\right) dx+\int_{0}^{+\infty
}n^{2}xe^{-nx}\phi \left( x\right) dx$. Il secondo addendo \`{e} $%
\int_{0}^{+\infty }n^{2}xe^{-nx}\phi \left( x\right) dx=\int_{0}^{+\infty
}ye^{-y}\phi \left( \frac{y}{n}\right) dy$, la cui funzione integranda \`{e}
maggiorata da $ye^{-y}\max_{y\in 
%TCIMACRO{\U{211d} }%
%BeginExpansion
\mathbb{R}
%EndExpansion
}\phi \left( y\right) $, che \`{e} integrabile. Dunque per convergenza
dominata \`{e} lecito scambiare limite e integrale e $\lim_{n\rightarrow
+\infty }\int_{0}^{+\infty }n^{2}xe^{-nx}\phi \left( x\right) dx=0$.
Analogamente si conclude per il primo addendo.

\item La convergenza uniforme di $f_{n}\left( x\right) \phi \left( x\right) $
a $f\left( x\right) \phi \left( x\right) $, con $f$ limite puntuale di $%
f_{n} $, \`{e} sufficiente per concludere che $u_{f_{n}}$ converge a $u_{f}$
in distribuzione: infatti \`{e} lecito lo scambio tra limite e integrale.
\end{enumerate}

\textbf{Def} Dato $\mathcal{D}^{\prime }\left( \Omega \right) $ insieme
delle distribuzioni su $\Omega $ e $\left\{ u_{j}\right\} _{j\in 
%TCIMACRO{\U{2115} }%
%BeginExpansion
\mathbb{N}
%EndExpansion
}\subseteq $ $\mathcal{D}^{\prime }\left( \Omega \right) $, si dice che $%
u_{j}$ \`{e} di Cauchy se $\forall $ $\phi \in \mathcal{D}\left( \Omega
\right) $ $\left\{ u_{j}\left( \phi \right) \right\} _{j\in 
%TCIMACRO{\U{2115} }%
%BeginExpansion
\mathbb{N}
%EndExpansion
}$ \`{e} di Cauchy.

Si noti che, fissato $\phi $, $\left\{ u_{j}\left( \phi \right) \right\}
_{j\in 
%TCIMACRO{\U{2115} }%
%BeginExpansion
\mathbb{N}
%EndExpansion
}$ \`{e} una successione di numeri reali. Quindi una successione di
distribuzioni converge (\`{e} di Cauchy) se la successione di numeri reali
che si ottiene fissando $\phi $ converge (\`{e} di Cauchy). Si pu\`{o}
dimostrare che ogni successione di Cauchy in $\mathcal{D}^{\prime }\left(
\Omega \right) $ \`{e} convergente.

E' ora naturale voler trovare qualche esempio di distribuzione, cio\`{e}
mostrare che $\mathcal{D}^{\prime }\left( \Omega \right) \neq \varnothing $.

Si definisce l'insieme delle funzioni localmente integrabili $%
L_{loc}^{1}\left( \Omega \right) :=\left\{ f\in M\left( \Omega \right)
:\forall \text{ }K\subseteq \Omega \text{ compatto }f|_{K}\text{ \`{e} }%
L^{1}\left( K\right) \right\} $, con $\Omega \subseteq 
%TCIMACRO{\U{211d} }%
%BeginExpansion
\mathbb{R}
%EndExpansion
^{n}$. E' uno spazio vettoriale, che non pu\`{o} essere reso di Banach.

\begin{enumerate}
\item Se $f\left( x\right) =\frac{1}{x}$ e $\Omega =\left( 0,1\right) $, $%
f\not\in L^{1}\left( \Omega \right) $, ma $f\in L_{loc}^{1}\left( \Omega
\right) $.

\item Se $f\in L_{loc}^{p}\left( \Omega \right) $ con $p\in \left( 1,+\infty
\right) $, allora $f\in L_{loc}^{1}\left( \Omega \right) $. Infatti, se $%
K\subseteq \Omega $ \`{e} compatto, allora $\int_{K}\left\vert f\right\vert
dx=\int_{K}\left\vert f\right\vert \cdot 1\leq \left\vert \left\vert
f\right\vert \right\vert _{L^{p}\left( K\right) }\left( \int_{K}1d\mathbf{x}%
\right) ^{\frac{1}{q}}=\left( \left\vert K\right\vert _{n}\right) ^{\frac{1}{%
q}}\left\vert \left\vert f\right\vert \right\vert _{L^{p}\left( K\right)
}<+\infty $. Se $p=+\infty $, $\int_{K}\left\vert f\right\vert dx\leq
\left\vert K\right\vert _{n}\sup ess_{K}f<+\infty $.
\end{enumerate}

Si definisce ora $u_{f}:\mathcal{D}\left( \Omega \right) \rightarrow 
%TCIMACRO{\U{211d} }%
%BeginExpansion
\mathbb{R}
%EndExpansion
,u_{f}\left( \phi \right) =\int_{\Omega }f\left( \mathbf{x}\right) \phi
\left( \mathbf{x}\right) d\mathbf{x}$, con $f\in L_{loc}^{1}\left( \Omega
\right) $. E' una distribuzione? L'integrale \`{e} ben definito perch\'{e}
per definizione di $\mathcal{D}\left( \Omega \right) $ $\phi $ \`{e}
limitata ed \`{e} in $L^{\infty }\left( K\right) $ con $K=$ supp$\left( \phi
\right) $: allora, per Hoelder, $\int_{\Omega }f\left( \mathbf{x}\right)
\phi \left( \mathbf{x}\right) d\mathbf{x=}\int_{K}f\left( \mathbf{x}\right)
\phi \left( \mathbf{x}\right) d\mathbf{x}$ \`{e} finito. $u_{f}$ \`{e} un
funzionale lineare per linearit\`{a} dell'integrale di Lebesgue: $%
u_{f}\left( \alpha \phi +\beta \psi \right) =\alpha u_{f}\left( \phi \right)
+\beta u_{f}\left( \psi \right) $, $\forall $ $\phi ,\psi ,\forall $ $\alpha
,\beta $. E' continuo? Occorre mostrare che se $\phi _{n}\rightarrow ^{%
\mathcal{D}\left( \Omega \right) }\phi $ allora $\left\vert u_{f}\left( \phi
_{n}\right) -u_{f}\left( \phi \right) \right\vert \rightarrow ^{n\rightarrow
+\infty }0$. Ma $\left\vert u_{f}\left( \phi _{n}\right) -u_{f}\left( \phi
\right) \right\vert =\left\vert \int_{\Omega }f\phi _{n}-\int_{\Omega }f\phi
\right\vert =\left\vert \int_{\Omega }f\left( \phi _{n}-\phi \right)
\right\vert \leq \int_{\Omega }\left\vert f\left( \phi _{n}-\phi \right)
\right\vert \leq \left\vert \left\vert f\right\vert \right\vert
_{L^{1}}\left\vert \left\vert \phi _{n}-\phi \right\vert \right\vert
_{L^{\infty }}$, per la disuguaglianza di Hoelder. Poich\'{e} la convergenza
in norma $L^{\infty }$ \`{e} una convergenza uniforme q.o. e per ipotesi $%
\phi _{n}\rightarrow ^{\mathcal{D}\left( \Omega \right) }\phi $, si ha in
particolare (per $\alpha =\mathbf{0}$) che $\phi _{n}\rightarrow \phi $
uniformemente, dunque vale $\left\vert \left\vert \phi _{n}-\phi \right\vert
\right\vert _{L^{\infty }}\rightarrow ^{n\rightarrow +\infty }0$. Di
conseguenza $\left\vert u_{f}\left( \phi _{n}\right) -u_{f}\left( \phi
\right) \right\vert \rightarrow ^{n\rightarrow +\infty }$e il funzionale 
\`{e} anche continuo, perci\`{o} \`{e} una distribuzione.

\textbf{Teo 3.1 (annullamento)}%
\begin{eqnarray*}
\text{Hp}\text{: } &&f\in L_{loc}^{1}\left( \Omega \right) \text{, }u_{f}:%
\mathcal{D}\left( \Omega \right) \rightarrow 
%TCIMACRO{\U{211d} }%
%BeginExpansion
\mathbb{R}
%EndExpansion
,u_{f}\left( \phi \right) =\int_{\Omega }f\phi =0\text{ }\forall \text{ }%
\phi \in \mathcal{D}\left( \Omega \right) \\
\text{Ts}\text{: } &&f=0\text{ quasi ovunque}
\end{eqnarray*}

Quindi conoscere $\forall $ $\phi $ il valore della distribuzione associata
a $f$ in questo caso permette di determinare interamente $f$. Il risultato
sarebbe ovvio se $f$ fosse continua.

\textbf{Dim} Preso $K\subseteq \Omega $ compatto qualsiasi, $\int_{K}f\left(
x\right) dx\in 
%TCIMACRO{\U{211d} }%
%BeginExpansion
\mathbb{R}
%EndExpansion
$. Definisco $w\left( x\right) =sgn\left( f\left( x\right) \right)
I_{K}\left( x\right) $: $w\in L^{p}\left( K\right) $ $\forall $ $p\in \left[
1,+\infty \right] $, in quanto funzione limitata. Allora\footnote{%
vedi paragrafo "convoluzione"} $\exists $ $\left\{ w_{k}\right\} _{k\in 
%TCIMACRO{\U{2115} }%
%BeginExpansion
\mathbb{N}
%EndExpansion
}\subseteq \mathcal{D}\left( \Omega \right) :w_{k}\rightarrow w$ in $%
L^{p}\left( \Omega \right) $ e q.o., e per ipotesi $\int_{\Omega }fw_{k}=0$ $%
\forall $ $k$. Ma $w_{k}=w\ast \rho _{k}$, e, poich\'{e} $w$ \`{e} $%
L^{\infty }$ e $\rho _{k}$ \`{e} $L^{1}$, per 2.24 $\left\vert \left\vert
w_{k}\right\vert \right\vert _{L^{\infty }}\leq \int_{%
%TCIMACRO{\U{211d} }%
%BeginExpansion
\mathbb{R}
%EndExpansion
}\left\vert \rho _{k}\left( y-x\right) \right\vert dy=1$. Allora per Hoelder 
$\left\vert \left\vert fw_{k}\right\vert \right\vert _{L^{1}\left( K\right)
}\leq \left\vert \left\vert f\right\vert \right\vert _{L^{1}\left( K\right)
} $ e per convergenza dominata (con funzione dominante $f$: vale $\left\vert
fw_{k}\right\vert \leq f$ $\forall $ $k$ q. o. perch\'{e} $w_{k}$ \`{e}
essenzialmente limitata) $0=\lim_{k\rightarrow +\infty }\int_{\Omega
}fw_{k}=\int_{\Omega }\lim_{k\rightarrow +\infty }fw_{k}=\int_{\Omega
}fw=\int_{K}\left\vert f\right\vert $, il che implica $\left\vert
f\right\vert =0$ q.o. in $K$. Essendo $K$ un compatto arbitrario, si deduce
che $f=0$ q.o. in $\Omega $. $\blacksquare $

A questo punto \`{e} naturale chiedersi quanto l'esempio di distribuzione
fatto sia significativo. L'applicazione $\mathcal{F}:L_{loc}^{1}\left(
\Omega \right) \rightarrow \mathcal{D}^{\prime }\left( \Omega \right) $, che
a ogni funzione localmente integrabile associa la distribuzione $u_{f}$ a
essa associata, \`{e} iniettiva (si dice che $L_{loc}^{1}\left( \Omega
\right) $ \`{e} \textit{immerso} in $\mathcal{D}^{\prime }\left( \Omega
\right) $). Infatti se $f\neq g\footnote{%
Per non appesantire la notazione si \`{e} scritto, ogniqualvolta si \`{e}
trattato di elementi di spazi $L^{p}$ o $L_{loc}^{1}$, $f$ invece che $\left[
f\right] $.}$ allora $u_{f}$ e $u_{g}$ non sono la stessa distribuzione, cio%
\`{e} $\exists $ $\phi :$ $u_{f}\left( \phi \right) \neq u_{g}\left( \phi
\right) $: se per assurdo fosse $\int_{\Omega }f\phi =\int_{\Omega }g\phi $ $%
\forall $ $\phi \in \mathcal{D}\left( \Omega \right) $, si avrebbe $%
\int_{\Omega }\left( f-g\right) \phi =0$ $\forall $ $\phi \in \mathcal{D}%
\left( \Omega \right) $, cio\`{e}, per il teorema di annullamento, $f=g$
q.o., contro l'ipotesi $f\neq g$.

Da ci\`{o} segue che la distribuzione $u_{f}$ caratterizza $f$: data $f$,
esiste un'unica $u_{f}$ a essa associata, ma soprattutto, data $u_{f}$, essa 
\`{e} associata a un'unica funzione $f\in L_{loc}^{1}\left( \Omega \right) 
\footnote{%
Come al solito si parla con abuso di $f\in L_{loc}^{1}\left( \Omega \right) $%
: in termini pi\`{u} precisi, a $u_{f}$ \`{e} associata un'unica classe di
equivalenza in $L_{loc}^{1}\left( \Omega \right) $, cio\`{e} $u_{f}$
individua un'unica $f$ a meno di un insieme di misura nulla.}$. Questo
mostra che la nozione di distribuzione d\`{a} un nuovo modo di vedere le
funzioni $f\in L_{loc}^{1}\left( \Omega \right) $: ad esempio, oltre alle
usuali nozioni di convergenza puntuale quasi ovunque, in norma $L^{p}$ ecc.,
si pu\`{o} dire che $f_{n}\rightarrow f$ in distribuzione quando $%
u_{f_{n}}\rightarrow ^{\mathcal{D}^{\prime }\left( \Omega \right) }u_{f}$.

L'applicazione $\mathcal{F}$ tuttavia non \`{e} suriettiva: non per ogni
distribuzione $u$ esiste $f:u\left( \phi \right) =u_{f}\left( \phi \right)
=\int_{\Omega }f\phi $ $\forall $ $\phi $, come mostra il seguente esempio.

\begin{enumerate}
\item Sia $u\left( \phi \right) =\phi \left( 0\right) $: \`{e} lineare;
inoltre se $\phi _{n}\rightarrow ^{\mathcal{D}\left( \Omega \right) }\phi $
allora in particolare $\phi _{n}\rightarrow \phi $ uniformemente e dunque
puntualmente, in particolare in $0$, per cui $u\left( \phi _{n}\right)
\rightarrow u\left( \phi \right) $. Dunque $u$ \`{e} continua ed \`{e} una
distribuzione. Tale distribuzione \`{e} detta delta di Dirac in $0$ (pi\`{u}
in generale la delta di Dirac in $x_{0}$ \`{e} $\delta _{x_{0}}\left( \phi
\right) :=\phi \left( x_{0}\right) $) e dal punto di vista fisico
rappresenta un impulso centrato in un punto.

Si pu\`{o} dimostrare che non esiste $f\in L_{loc}^{1}\left( 
%TCIMACRO{\U{211d} }%
%BeginExpansion
\mathbb{R}
%EndExpansion
\right) :\phi \left( 0\right) =\int_{%
%TCIMACRO{\U{211d} }%
%BeginExpansion
\mathbb{R}
%EndExpansion
}f\phi $. Questo intuitivamente \`{e} vero perch\'{e} un integrale di
Lebesgue non pu\`{o} dipendere dal valore della funzione integranda in un
certo punto: l'integrale non varia se l'integranda \`{e} alterata in un
insieme di misura nulla.

Analiticamente, se per assurdo esistesse $f\in L_{loc}^{1}\left( 
%TCIMACRO{\U{211d} }%
%BeginExpansion
\mathbb{R}
%EndExpansion
\right) :\phi \left( 0\right) =\int_{%
%TCIMACRO{\U{211d} }%
%BeginExpansion
\mathbb{R}
%EndExpansion
}f\phi $ $\forall $ $\phi \in \mathcal{D}\left( \Omega \right) $, allora in
particolare per le $\phi $ a supporto in $\left( a,b\right) $ con $0<a<b$ si
avrebbe $0=\int_{%
%TCIMACRO{\U{211d} }%
%BeginExpansion
\mathbb{R}
%EndExpansion
}f\phi $ $\forall $ $\phi \in \mathcal{D}\left( \left( a,b\right) \right) $,
dunque per il teorema di annullamento sarebbe $f=0$ q.o. in $\left(
a,b\right) $. Essendo $\left( a,b\right) $ arbitrario si conclude che $f=0$
q.o. in $\left( 0,+\infty \right) $, e con un analogo ragionamento che $f=0$
q.o. in $\left( -\infty ,0\right) $, per cui $f=0$ q.o. in $%
%TCIMACRO{\U{211d} }%
%BeginExpansion
\mathbb{R}
%EndExpansion
$. Questo \`{e} assurdo perch\'{e} si avrebbe $u_{f}\left( \phi \right) =0$ $%
\forall $ $\phi $, anche se $\phi :\phi \left( 0\right) \neq 0$.

\item Si pu\`{o} definire in un altro modo la delta di Dirac. Sia $f_{n}:%
%TCIMACRO{\U{211d} }%
%BeginExpansion
\mathbb{R}
%EndExpansion
^{n}\rightarrow 
%TCIMACRO{\U{211d} }%
%BeginExpansion
\mathbb{R}
%EndExpansion
,f_{n}\left( \mathbf{x}\right) =\frac{1}{\left\vert B_{1/n}\left( \mathbf{0}%
\right) \right\vert _{N}}I_{B_{1/n}\left( \mathbf{0}\right) }\left( \mathbf{x%
}\right) $. $\forall $ $n$ $f_{n}\in L_{loc}^{1}\left( 
%TCIMACRO{\U{211d} }%
%BeginExpansion
\mathbb{R}
%EndExpansion
^{n}\right) $ in quanto funzione limitata. Allora $u_{f_{n}}\left( \phi
\right) =\int_{B_{1/n}\left( \mathbf{0}\right) }\frac{1}{\left\vert
B_{1/n}\left( \mathbf{0}\right) \right\vert _{N}}\phi $: per il teorema
della media integrale esiste $\alpha \in B_{1/n}\left( \mathbf{0}\right)
:\phi \left( \alpha \right) =\frac{\int_{B_{1/n}\left( \mathbf{0}\right)
}\phi }{\left\vert B_{1/n}\left( \mathbf{0}\right) \right\vert _{N}}$,
dunque $u_{f_{n}}\left( \phi \right) =\phi \left( \alpha \right) $ e per $%
n\rightarrow +\infty $ $u_{f_{n}}\left( \phi \right) \rightarrow \phi \left(
0\right) =\delta _{0}$. La delta, che non pu\`{o} essere rappresentata come $%
u_{f}$, \`{e} limite di una successione di distribuzioni del tipo $u_{f_{n}}$%
.

\item Sia $f_{n}\left( x\right) =\arctan \left( nx\right) $. Il limite
puntuale quasi ovunque \`{e} $f\left( x\right) =\left\{ 
\begin{array}{c}
\frac{\pi }{2}\text{ se }x>0 \\ 
0\text{ se }x=0 \\ 
-\frac{\pi }{2}\text{ se }x<0%
\end{array}%
\right. $. Fissata $\phi $, calcolo $\lim_{n\rightarrow +\infty
}u_{f_{n}}\left( \phi \right) =\lim_{n\rightarrow +\infty }\int_{-\infty
}^{+\infty }\arctan \left( nx\right) \phi \left( x\right) dx$. $%
\lim_{n\rightarrow +\infty }\int_{-\infty }^{0}\arctan \left( nx\right) \phi
\left( x\right) dx=\int_{-\infty }^{0}-\frac{\pi }{2}\phi \left( x\right) dx$
per convergenza dominata, e analogamente $\lim_{n\rightarrow +\infty
}\int_{0}^{+\infty }\arctan \left( nx\right) \phi \left( x\right)
dx=\int_{0}^{+\infty }\frac{\pi }{2}\phi \left( x\right) dx$, quindi $%
\lim_{n\rightarrow +\infty }u_{f_{n}}\left( \phi \right) =\int_{-\infty
}^{0}-\frac{\pi }{2}\phi \left( x\right) dx+\int_{0}^{+\infty }\frac{\pi }{2}%
\phi \left( x\right) dx$: il limite in distribuzione coincide con il limite
puntuale.
\end{enumerate}

\textbf{Distribuzioni e valori principali }Esiste un'altra distribuzione che
non \`{e} del tipo $u_{f}$. Sia $f\left( t\right) =\frac{1}{t}$ per $t\neq 0$%
: $f\not\in L_{loc}^{1}\left( 
%TCIMACRO{\U{211d} }%
%BeginExpansion
\mathbb{R}
%EndExpansion
\right) $ perch\'{e} $\frac{1}{t}$ non \`{e} integrabile su $%
%TCIMACRO{\U{211d} }%
%BeginExpansion
\mathbb{R}
%EndExpansion
$ - n\'{e} secondo Riemann n\'{e} secondo Lebesgue - n\'{e} su qualsiasi
compatto contenente $0$, quindi la distribuzione $u_{f}\left( \phi \right)
=\int_{%
%TCIMACRO{\U{211d} }%
%BeginExpansion
\mathbb{R}
%EndExpansion
}f\phi $ non \`{e} ben definita.

Per ovviare a tale problema si ricorre alla nozione di valore principale di
un integrale, che permette di assegnare un valore reale anche a integrali
non ben definiti in base alla teoria degli integrali impropri di Riemann.

Se $f:\left[ a,b\right] \backslash \left\{ c\right\} \rightarrow 
%TCIMACRO{\U{211d} }%
%BeginExpansion
\mathbb{R}
%EndExpansion
$, si dice valore principale di $\int_{a}^{b}f\left( t\right) dt$ (e si
scrive $vp\int_{a}^{b}f\left( t\right) dt$) il valore di 
\begin{equation*}
\lim_{\varepsilon
\rightarrow 0^{+}}\left( \int_{a}^{c-\varepsilon }f\left( t\right)
dt+\int_{c+\varepsilon }^{b}f\left( t\right) dt\right)
\end{equation*}
qualora esso
esista finito. Rispetto all'integrale improprio di Riemann non si
considerano due limiti indipendenti (uno per $\varepsilon _{1}\rightarrow
0^{+}$, l'altro per $\varepsilon _{2}\rightarrow 0^{+}$), ma si introduce il
vincolo che $\varepsilon $ sia lo stesso in entrambi i limiti.

Se invece $f:\left( -\infty ,a\right) \cup \left( b,+\infty \right)
\rightarrow 
%TCIMACRO{\U{211d} }%
%BeginExpansion
\mathbb{R}
%EndExpansion
$, si dice valore principale di $\int_{\left( -\infty ,a\right) \cup \left(
b,+\infty \right) }f\left( t\right) dt$ il valore di 
\begin{equation*}
\lim_{\varepsilon
\rightarrow 0^{+}}\left( \int_{a-\frac{1}{\varepsilon }}^{a}f\left( t\right)
dt+\int_{b}^{b+\frac{1}{\varepsilon }}f\left( t\right) dt\right)
\end{equation*}
qualora
esso esista finito. Di nuovo, si introduce il vincolo che l'estremo mobile
dell'integrale sia lo stesso in entrambi i limiti. Infine, se $f:%
%TCIMACRO{\U{211d} }%
%BeginExpansion
\mathbb{R}
%EndExpansion
\backslash \left\{ c\right\} \rightarrow 
%TCIMACRO{\U{211d} }%
%BeginExpansion
\mathbb{R}
%EndExpansion
$, si definisce valore principale di $\int_{%
%TCIMACRO{\U{211d} }%
%BeginExpansion
\mathbb{R}
%EndExpansion
}f\left( t\right) dt$ il valore di $\lim_{\varepsilon \rightarrow
0^{+}}\left( \int_{c-\frac{1}{\varepsilon }}^{c-\varepsilon }f\left(
t\right) dt+\int_{c+\varepsilon }^{c+\frac{1}{\varepsilon }}f\left( t\right)
dt\right) $, qualora esso esista finito.

\begin{enumerate}
\item Se e. g. si cerca il valore principale di $\int_{%
%TCIMACRO{\U{211d} }%
%BeginExpansion
\mathbb{R}
%EndExpansion
}\frac{1}{t}dt$, si calcola $\lim_{\varepsilon \rightarrow 0^{+}}\left(
\int_{-\frac{1}{\varepsilon }}^{-\varepsilon }\frac{1}{t}dt+\int_{%
\varepsilon }^{\frac{1}{\varepsilon }}\frac{1}{t}dt\right)
=\lim_{\varepsilon \rightarrow 0^{+}}\int_{\varepsilon <\left\vert
t\right\vert <\frac{1}{\varepsilon }}\frac{1}{t}dt$. Poich\'{e} $\forall $ $%
\varepsilon >0$ $\int_{-\frac{1}{\varepsilon }}^{-\varepsilon }\frac{1}{t}%
dt+\int_{\varepsilon }^{\frac{1}{\varepsilon }}\frac{1}{t}dt=0$ (in quanto
integrale di una funzione dispari su un intervallo limitato e simmetrico
attorno all'origine), il valore del limite \`{e} $0$.
\end{enumerate}

Con queste definizioni si pu\`{o} valutare se esiste il valore principale di 
$\int_{%
%TCIMACRO{\U{211d} }%
%BeginExpansion
\mathbb{R}
%EndExpansion
}\frac{1}{t}\phi \left( t\right) dt$ (data $\phi \in \mathcal{D}\left( 
%TCIMACRO{\U{211d} }%
%BeginExpansion
\mathbb{R}
%EndExpansion
\right) $), cos\`{\i} che la distribuzione sia definita. Essendo supp$\left(
\phi \right) $ compatto, $\exists $ $a>0:$ supp$\left( \phi \right)
\subseteq \left[ -a,a\right] $, dunque si cerca di determinare il valore
principale di $\int_{-a}^{a}\frac{\phi \left( t\right) }{t}dt$. Per
definizione esso \`{e} $\lim_{\varepsilon \rightarrow 0^{+}}\left(
\int_{-a}^{-\varepsilon }\frac{\phi \left( t\right) }{t}dt+\int_{\varepsilon
}^{a}\frac{\phi \left( t\right) }{t}dt\right) $, qualora il limite esista
finito. Ma $\lim_{\varepsilon \rightarrow 0^{+}}\left(
\int_{-a}^{-\varepsilon }\frac{\phi \left( t\right) }{t}dt+\int_{\varepsilon
}^{a}\frac{\phi \left( t\right) }{t}dt\right) =\lim_{\varepsilon \rightarrow
0^{+}}\int_{\varepsilon <\left\vert t\right\vert <a}\frac{\phi \left(
t\right) }{t}dt$, che \`{e} uguale a $\lim_{\varepsilon \rightarrow
0^{+}}\left( \int_{\varepsilon <\left\vert t\right\vert <a}\frac{\phi \left(
t\right) -\phi \left( 0\right) }{t}dt+\int_{\varepsilon <\left\vert
t\right\vert <a}\frac{\phi \left( 0\right) }{t}dt\right) $. Il secondo
addendo \`{e} nullo $\forall $ $\varepsilon $; $\lim_{\varepsilon
\rightarrow 0^{+}}\int_{\varepsilon <\left\vert t\right\vert <a}\frac{\phi
\left( t\right) -\phi \left( 0\right) }{t}dt$ esiste finito perch\'{e} per
il teorema di Lagrange applicato a $\phi \in C^{\infty }$ $\exists $ $\xi
\in \left( 0,1\right) :\phi \left( t\right) -\phi \left( 0\right) =\phi
^{\prime }\left( t\right) t$, dunque $\int_{\varepsilon <\left\vert
t\right\vert <a}\frac{\phi \left( t\right) -\phi \left( 0\right) }{t}%
dt=\int_{\varepsilon <\left\vert t\right\vert <a}\phi ^{\prime }\left( \xi
\right) dt$ e la funzione integranda \`{e} limitata $\forall $ $\varepsilon $%
. Allora \`{e} ben definito il v.p. di $\int_{%
%TCIMACRO{\U{211d} }%
%BeginExpansion
\mathbb{R}
%EndExpansion
}\frac{1}{t}\phi \left( t\right) dt$, che coincide con il v.p. di $%
\int_{-a}^{a}\frac{\phi \left( t\right) }{t}dt$ ed \`{e} uguale a $%
\lim_{\varepsilon \rightarrow 0^{+}}\int_{\varepsilon <\left\vert
t\right\vert <a}\frac{\phi \left( t\right) -\phi \left( 0\right) }{t}%
dt=\int_{%
%TCIMACRO{\U{211d} }%
%BeginExpansion
\mathbb{R}
%EndExpansion
}\frac{\phi \left( t\right) -\phi \left( 0\right) }{t}dt$.

Si vuole ora dimostrare che $u\left( \phi \right) =$ $\int_{%
%TCIMACRO{\U{211d} }%
%BeginExpansion
\mathbb{R}
%EndExpansion
}\frac{\phi \left( t\right) -\phi \left( 0\right) }{t}dt$ \`{e} una
distribuzione (che si dir\`{a} \textit{associata al v.p. di }$\frac{1}{t}$).
La linearit\`{a} \`{e} ovvia. E' inoltre continua perch\'{e}, se $\phi
_{n}\rightarrow ^{\mathcal{D}\left( \Omega \right) }\phi $, allora $%
\left\vert \int_{%
%TCIMACRO{\U{211d} }%
%BeginExpansion
\mathbb{R}
%EndExpansion
}\frac{\phi _{n}\left( t\right) -\phi _{n}\left( 0\right) }{t}dt-\int_{%
%TCIMACRO{\U{211d} }%
%BeginExpansion
\mathbb{R}
%EndExpansion
}\frac{\phi \left( t\right) -\phi \left( 0\right) }{t}dt\right\vert
=\left\vert \lim_{\varepsilon \rightarrow 0^{+}}\int_{\varepsilon
<\left\vert t\right\vert <a}\left( \frac{\phi _{n}\left( t\right) -\phi
_{n}\left( 0\right) }{t}-\frac{\phi \left( t\right) -\phi \left( 0\right) }{t%
}\right) dt\right\vert $. $\int_{\varepsilon <\left\vert t\right\vert
<a}\left( \frac{\phi _{n}\left( t\right) -\phi \left( t\right) -\left( \phi
_{n}\left( 0\right) -\phi \left( 0\right) \right) }{t}\right)
dt=\int_{\varepsilon <\left\vert t\right\vert <a}\frac{\left( \phi
_{n}^{\prime }\left( \xi \right) -\phi ^{\prime }\left( \xi \right) \right) t%
}{t}dt$, ma per $n\rightarrow +\infty $ tutte le derivate di $\phi _{n}$
convergono uniformemente a $\phi $, quindi in particolare $\phi _{n}^{\prime
}\left( \xi \right) \rightarrow ^{n\rightarrow +\infty }\phi ^{\prime
}\left( \xi \right) $, l'integranda tende a $0$, dunque l'integrale tende a $%
0$ (lo scambio \`{e} lecito grazie alla convergenza uniforme) e cos\`{\i}
anche il limite.

Inoltre $\NEG{\exists}$ $f\in L_{loc}^{1}\left( 
%TCIMACRO{\U{211d} }%
%BeginExpansion
\mathbb{R}
%EndExpansion
\right) $ che rappresenti questa distribuzione, cio\`{e} tale che $u\left(
\phi \right) =vp\int_{%
%TCIMACRO{\U{211d} }%
%BeginExpansion
\mathbb{R}
%EndExpansion
}\frac{\phi \left( t\right) }{t}dt=\int_{%
%TCIMACRO{\U{211d} }%
%BeginExpansion
\mathbb{R}
%EndExpansion
}\frac{\phi \left( t\right) -\phi \left( 0\right) }{t}dt$ sia uguale a $%
\int_{%
%TCIMACRO{\U{211d} }%
%BeginExpansion
\mathbb{R}
%EndExpansion
}f\phi dt$ $\forall $ $\phi \in \mathcal{D}\left( 
%TCIMACRO{\U{211d} }%
%BeginExpansion
\mathbb{R}
%EndExpansion
\right) $. Infatti, se cos\`{\i} fosse, in particolare si avrebbe $\int_{%
%TCIMACRO{\U{211d} }%
%BeginExpansion
\mathbb{R}
%EndExpansion
}\frac{\phi \left( t\right) -\phi \left( 0\right) }{t}dt=\int_{%
%TCIMACRO{\U{211d} }%
%BeginExpansion
\mathbb{R}
%EndExpansion
}f\phi dt$, cio\`{e} $\int_{%
%TCIMACRO{\U{211d} }%
%BeginExpansion
\mathbb{R}
%EndExpansion
}\left( \frac{1}{t}-f\right) \phi dt=0$ $\forall $ $\phi \in \mathcal{D}%
\left( 
%TCIMACRO{\U{211d} }%
%BeginExpansion
\mathbb{R}
%EndExpansion
\right) :\phi \left( 0\right) =0$, ma allora per il teorema di annullamento
sarebbe $f\left( t\right) =\frac{1}{t}$ q.o., cio\`{e} $f\in \left[ \frac{1}{%
t}\right] $, ma $\left[ \frac{1}{t}\right] $ non pu\`{o} essere in $%
L_{loc}^{1}\left( 
%TCIMACRO{\U{211d} }%
%BeginExpansion
\mathbb{R}
%EndExpansion
\right) $.

Ha senso cercare una caratterizzazione delle distribuzioni che eviti ogni
volta la verifica di linearit\`{a} e continuit\`{a}.

\textbf{Teo 3.2 (caratterizzazione delle distribuzioni)}%
\begin{gather*}
\text{Hp: }u:\mathcal{D}\left( \Omega \right) \rightarrow 
%TCIMACRO{\U{211d} }%
%BeginExpansion
\mathbb{R}
%EndExpansion
\left( 
%TCIMACRO{\U{2102} }%
%BeginExpansion
\mathbb{C}
%EndExpansion
\right) \text{ \`{e} un funzionale lineare} \\
\text{Ts: }u\text{ \`{e} una distribuzione }\Longleftrightarrow \forall 
\text{ }K\subseteq \Omega \text{ compatto }\exists \text{ }c\left( K\right)
>0, \\
m\left( K\right) \in 
%TCIMACRO{\U{2115} }%
%BeginExpansion
\mathbb{N}
%EndExpansion
:u\left( \phi \right) \leq c\sum_{\left\vert \alpha \right\vert \leq
m}\left\vert \left\vert D^{\alpha }\phi \right\vert \right\vert _{\infty }%
\text{ }\forall \text{ }\phi \in \mathcal{D}\left( \Omega \right) :\text{supp%
}\left( \phi \right) \subseteq K
\end{gather*}

Si noti che $\left\vert \left\vert D^{\alpha }\phi \right\vert \right\vert
_{\infty }$ (norma del sup essenziale in $L^{\infty }\left( K\right) $) \`{e}
ben definita perch\'{e} ogni $\phi \in \mathcal{D}\left( \Omega \right) $ 
\`{e} continua in un compatto, dunque \`{e} limitata. La somma \`{e} fatta
su tutti i multiindici che hanno lunghezza\footnote{$\left\vert \alpha
\right\vert :=\sum_{i=1}^{n}\alpha _{i}$} minore o uguale di $m$.

\textbf{Dim} $\Longleftarrow $ Si mostra che $u$ \`{e} continua. Questo
segue facilmente dalla definizione di convergenza nello spazio delle
funzioni test: data $\phi _{n}\rightarrow ^{\mathcal{D}\left( \Omega \right)
}\phi $, si sa che vale $u\left( \phi _{n}-\phi \right) \leq
c\sum_{\left\vert \alpha \right\vert \leq m}\left\vert \left\vert D^{\alpha
}\phi _{n}-D^{\alpha }\phi \right\vert \right\vert _{\infty }$: ma la
convergenza in $\mathcal{D}\left( \Omega \right) $ implica la convergenza
uniforme di tutte le derivate, e poich\'{e} la convergenza uniforme \`{e}
equivalente alla convergenza in norma infinito, il lato destro della
disuguaglianza tende a $0$ per $n\rightarrow +\infty $ e quindi, per
confronto, anche il lato sinistro. $\blacksquare $

Questo teorema deve essere visto come un adattamento al caso delle
distribuzioni del teorema che afferma che un funzionale lineare tra due
spazi normati \`{e} limitato se e solo se \`{e} continuo: in questo caso la
continuit\`{a} del funzionale $u$ \`{e} caratterizzata dalla limitatezza non
solo del suo argomento ma anche di un certo numero di sue derivate.

\begin{enumerate}
\item $u\left( \phi \right) =pv\int_{%
%TCIMACRO{\U{211d} }%
%BeginExpansion
\mathbb{R}
%EndExpansion
}\frac{\phi \left( t\right) -\phi \left( 0\right) }{t}dt$ soddisfa la
condizione del teorema perch\'{e}, preso $K=\left[ -a,a\right] $ e $\phi \in 
\mathcal{D}\left( \Omega \right) :$ supp$\left( \phi \right) \subseteq K$,
per il teorema di Lagrange $\lim_{\varepsilon \rightarrow
0^{+}}\int_{\varepsilon <\left\vert t\right\vert <a}\frac{\phi \left(
t\right) -\phi \left( 0\right) }{t}dt\leq \lim_{\varepsilon \rightarrow
0^{+}}\int_{\varepsilon <\left\vert t\right\vert <a}\frac{\left\vert
\left\vert \phi ^{\prime }\right\vert \right\vert _{L^{\infty }\left(
K\right) }t}{t}dt=\lim_{\varepsilon \rightarrow 0^{+}}\left\vert \left\vert
\phi ^{\prime }\right\vert \right\vert _{L^{\infty }\left( K\right)
}\int_{\varepsilon <\left\vert t\right\vert <a}dt=2a\left\vert \left\vert
\phi ^{\prime }\right\vert \right\vert _{L^{\infty }\left( K\right) }$: $%
c=2a $, $m=1$.
\end{enumerate}

Se $u$ soddisfa il teorema con $m$ indipendente da $K$ e minimale, $m$ si
dice ordine della distribuzione.

\begin{enumerate}
\item Data $u_{f}\left( \phi \right) =\int_{%
%TCIMACRO{\U{211d} }%
%BeginExpansion
\mathbb{R}
%EndExpansion
}f\phi dt$ con $f\in L_{loc}^{1}$, $u$ \`{e} una distribuzione di ordine $0$%
. Infatti $\int_{%
%TCIMACRO{\U{211d} }%
%BeginExpansion
\mathbb{R}
%EndExpansion
}f\phi dt=\int_{K}f\phi dt\leq \left\vert \left\vert \phi \right\vert
\right\vert _{L^{\infty }\left( K\right) }\int_{K}fdt$ (il secondo fattore 
\`{e} un numero reale perch\'{e} $f\in L_{loc}^{1}$): dunque $u$ soddisfa la
condizione del teorema con $c=\left\vert \left\vert f\right\vert \right\vert
_{L^{1}\left( K\right) }$ e $m=0$. $u_{f}$ si dice misura.

\item $u\left( \phi \right) =\phi \left( 0\right) \leq \left\vert \left\vert
\phi \right\vert \right\vert _{L^{\infty }\left( K\right) }$, quindi anche $%
\delta _{0}$ ha ordine $0$.
\end{enumerate}

\subsection{Derivate distribuzionali}

\begin{enumerate}
\item Sia $f\in C^{1}\left( 
%TCIMACRO{\U{211d} }%
%BeginExpansion
\mathbb{R}
%EndExpansion
\right) $: allora $f,f^{\prime }$ sono continue, quindi limitate in ogni
compatto, dunque $f,f^{\prime }\in L_{loc}^{1}\left( 
%TCIMACRO{\U{211d} }%
%BeginExpansion
\mathbb{R}
%EndExpansion
\right) $ e hanno distribuzioni associate $u_{f}\left( \phi \right) =\int_{%
%TCIMACRO{\U{211d} }%
%BeginExpansion
\mathbb{R}
%EndExpansion
}f\phi dt,u_{f^{\prime }}\left( \phi \right) =\int_{%
%TCIMACRO{\U{211d} }%
%BeginExpansion
\mathbb{R}
%EndExpansion
}f^{\prime }\phi dt$. Sarebbe coerente con l'intuizione definire derivata
della distribuzione $u_{f}$ la distribuzione $u_{f^{\prime }}$. Tuttavia
integrando per parti si osserva che vale $\int_{%
%TCIMACRO{\U{211d} }%
%BeginExpansion
\mathbb{R}
%EndExpansion
}f^{\prime }\phi dt=-\int_{%
%TCIMACRO{\U{211d} }%
%BeginExpansion
\mathbb{R}
%EndExpansion
}f\phi ^{\prime }dt$ (basta considerare $K$ che contenga propriamente supp$%
\left( \phi \right) $), quindi \`{e} possibile definire $u_{f^{\prime }}$
anche senza chiedere $f\in C^{1}\left( 
%TCIMACRO{\U{211d} }%
%BeginExpansion
\mathbb{R}
%EndExpansion
\right) $, ma solo $f\in L_{loc}^{1}\left( 
%TCIMACRO{\U{211d} }%
%BeginExpansion
\mathbb{R}
%EndExpansion
\right) $, sfruttando la grande regolarit\`{a} di $\phi $: $v\left( \phi
\right) =-\int_{%
%TCIMACRO{\U{211d} }%
%BeginExpansion
\mathbb{R}
%EndExpansion
}f\phi ^{\prime }dt$.
\end{enumerate}

\textbf{Def} Data $u\in \mathcal{D}^{\prime }\left( 
%TCIMACRO{\U{211d} }%
%BeginExpansion
\mathbb{R}
%EndExpansion
\right) $, si definisce derivata distribuzionale prima di $u$ la
distribuzione $v\left( \phi \right) :=-u\left( \phi ^{\prime }\right) $ e si
indica con $Du$.

E' evidente che $v$ cos\`{\i} definita \`{e} una distribuzione. Se $u$ ha
ordine $m$, $v$ ha ordine $m+1$: infatti $\forall $ $\phi \in \mathcal{D}%
\left( \Omega \right) :$supp$\left( \phi \right) \subseteq K$ vale $v\left(
\phi \right) =-u\left( \phi ^{\prime }\right) \leq -c\sum_{\left\vert \alpha
\right\vert \leq m}\left\vert \left\vert D^{\alpha }\phi ^{\prime
}\right\vert \right\vert _{\infty }=-c\sum_{\left\vert \alpha \right\vert
\leq m+1}\left\vert \left\vert D^{\alpha }\phi \right\vert \right\vert
_{\infty }$.

La definizione data \`{e} coerente con l'esempio sopra: se $f\in C^{1}$, $%
v\left( \phi \right) =Du\left( \phi \right) =-\int_{%
%TCIMACRO{\U{211d} }%
%BeginExpansion
\mathbb{R}
%EndExpansion
}f\phi ^{\prime }dt$ \`{e} la derivata distribuzionale di $u$, cio\`{e} $%
Du_{f}=u_{f^{\prime }}$: la derivata distribuzionale della distribuzione
associata a $f\in C^{1}$ \`{e} la distribuzione associata alla derivata di $%
f $.

La definizione si estende a distribuzioni su $\mathcal{D}\left( \Omega
\right) $ con $\Omega $ generico usando il teorema della divergenza.

\textbf{Def} Data $u\in \mathcal{D}^{\prime }\left( \Omega \right) $, $%
\Omega \subseteq 
%TCIMACRO{\U{211d} }%
%BeginExpansion
\mathbb{R}
%EndExpansion
^{n}$ aperto e $\alpha \in 
%TCIMACRO{\U{2115} }%
%BeginExpansion
\mathbb{N}
%EndExpansion
^{n}$ multiindice, si dice derivata distribuzionale di ordine $\alpha $ di $%
u $ $D^{\alpha }u\left( \phi \right) :=\left( -1\right) ^{\left\vert \alpha
\right\vert }u\left( D^{\alpha }\phi \right) $.

Essa \`{e} ben definita $\forall $ $\alpha $ perch\'{e} $\phi \in C^{\infty
} $. In analogia al caso di $\Omega =%
%TCIMACRO{\U{211d} }%
%BeginExpansion
\mathbb{R}
%EndExpansion
$ e $\left\vert \alpha \right\vert =1$, se $f\in C^{m}\left( 
%TCIMACRO{\U{211d} }%
%BeginExpansion
\mathbb{R}
%EndExpansion
\right) $ (quindi $D^{\alpha }f\in C\left( \Omega \right) $ $\forall $ $%
\alpha :\left\vert \alpha \right\vert \leq m$) vale $D^{\alpha
}u_{f}=u_{D^{\alpha }f}$ $\forall $ $\alpha :\left\vert \alpha \right\vert
\leq m$, quindi la definizione rispetta l'intuizione originaria.

\begin{enumerate}
\item Si verifica ci\`{o} per $m=1$. Se $\phi \in \mathcal{D}\left( 
%TCIMACRO{\U{211d} }%
%BeginExpansion
\mathbb{R}
%EndExpansion
^{n}\right) $, $\exists $ $R_{n}=\left[ a_{1},b_{1}\right] \times ...\times %
\left[ a_{n},b_{n}\right] \supseteq $ supp$\left( \phi \right) $ e si ha -
considerando $\alpha :\left\vert \alpha \right\vert =1$ con $\alpha _{j}=1$
- $u_{D^{\alpha }f}\left( \phi \right) =\int_{%
%TCIMACRO{\U{211d} }%
%BeginExpansion
\mathbb{R}
%EndExpansion
^{n}}\phi D_{x_{j}}f=\int_{a_{1}}^{b_{1}}...\int_{a_{j}}^{b_{j}}\phi
D_{x_{j}}f...\int_{a_{n}}^{b_{n}}d\mathbf{x}=-\int_{a_{1}}^{b_{1}}...%
\int_{a_{j}}^{b_{j}}fD_{x_{j}}\phi ...\int_{a_{n}}^{b_{n}}d\mathbf{x=-}\int_{%
%TCIMACRO{\U{211d} }%
%BeginExpansion
\mathbb{R}
%EndExpansion
^{n}}fD_{x_{j}}\phi =\left( D^{\alpha }u_{f}\right) \left( \phi \right) $,
cio\`{e} la derivata distribuzionale di ordine $\alpha $ di $u_{f}$ \`{e}
effettivamente la distribuzione associata a $D^{\alpha }f$. Se invece fosse
stato $m=2$, si sarebbe considerato $\alpha :\left\vert \alpha \right\vert
\leq 2$ e per $\left\vert \alpha \right\vert =2$ si sarebbe integrato due
volte per parti.

\item Se $u\in \mathcal{D}^{\prime }\left( \Omega \right) $, $%
D_{x_{i}}D_{x_{j}}u=D_{x_{j}}D_{x_{i}}u$. Infatti $D_{x_{i}}D_{x_{j}}u=u%
\left( \frac{\partial ^{2}\phi }{\partial x_{i}\partial x_{j}}\right)
=u\left( \frac{\partial ^{2}\phi }{\partial x_{j}\partial x_{i}}\right)
=D_{x_{j}}D_{x_{i}}u$ per definizione e il teorema di Schwarz applicato a $%
\phi $, che \`{e} $C^{\infty }$.

\item Data $\delta \left( \phi \right) =\phi \left( 0\right) $ e $\alpha =k$%
, $D^{k}\delta =\left( -1\right) ^{k}\left( D^{k}\phi \right) \left(
0\right) $. Quindi $D^{k}\delta $ \`{e} una distribuzione di ordine $k$.
\end{enumerate}

Se $u_{n}\rightarrow ^{\mathcal{D}^{\prime }\left( \Omega \right) }u$,
allora $D^{\alpha }u_{n}\rightarrow ^{\mathcal{D}^{\prime }\left( \Omega
\right) }D^{\alpha }u$ $\forall $ $\alpha \in 
%TCIMACRO{\U{2115} }%
%BeginExpansion
\mathbb{N}
%EndExpansion
^{n}$: per una successione di distribuzioni \`{e} sempre lecito scambiare il 
$\lim_{n\rightarrow +\infty }$ e la derivata distribuzionale. Infatti, se $%
\forall $ $\phi $ vale $u_{n}\left( \phi \right) \rightarrow ^{n\rightarrow
+\infty }u\left( \phi \right) $, si ha anche $\left( -1\right) ^{\left\vert
\alpha \right\vert }u_{n}\left( D^{\alpha }\phi \right) \rightarrow
^{n\rightarrow +\infty }\left( -1\right) ^{\left\vert \alpha \right\vert
}u\left( D^{\alpha }\phi \right) $, perch\'{e} $D^{\alpha }\phi \in \mathcal{%
D}\left( \Omega \right) $. In particolare, se $\sum_{i=1}^{n}u_{i}%
\rightarrow ^{n\rightarrow +\infty ,\mathcal{D}^{\prime }\left( \Omega
\right) }u$, allora $\sum_{i=1}^{n}D^{\alpha }u_{i}\rightarrow
^{n\rightarrow +\infty ,\mathcal{D}^{\prime }\left( \Omega \right)
}D^{\alpha }u$. Le operazioni di passaggio al limite sono quindi molto
semplificate nello spazio delle distribuzioni.

\begin{enumerate}
\item Sia $f\left( t\right) =I_{\left( [0,+\infty )\right) }\left( t\right)
=:H\left( t\right) $ la funzione scalino di Heaviside. $f\in
L_{loc}^{1}\left( 
%TCIMACRO{\U{211d} }%
%BeginExpansion
\mathbb{R}
%EndExpansion
\right) $ e $u_{f}\left( \phi \right) =\int_{0}^{+\infty }\phi \left(
t\right) dt$. La derivata distribuzionale di $u_{f}$ \`{e} $%
Du_{f}=-u_{f}\left( \phi ^{\prime }\right) =-\int_{0}^{+\infty }\phi
^{\prime }\left( t\right) dt=\phi \left( 0\right) =\delta _{0}\left( \phi
\right) $, la distribuzione delta.

Si \`{e} visto che se $f\in C^{1}$, $Du_{f}=u_{f^{\prime }}$: per $H$, che
non \`{e} continua in $0$, questo non vale perch\'{e} $Du_{H}=\delta _{0}$,
mentre $u_{H^{\prime }}$ \`{e} la distribuzione nulla.
\end{enumerate}

In generale la differenza tra $u_{f^{\prime }}$ e $Du_{f}$ si osserva quando 
$f$ ha discontinuit\`{a} a salto.

\textbf{Prop 3.4 (relazione tra derivata distribuzionale e distribuzione
della derivata)}%
\begin{gather*}
\text{Hp: }f\in C^{1}\left( 
%TCIMACRO{\U{211d} }%
%BeginExpansion
\mathbb{R}
%EndExpansion
\backslash \left\{ x_{0}\right\} \right) :\lim_{x\rightarrow
x_{0}^{-}}f\left( x\right) =f\left( x_{0}^{-}\right) ,\lim_{x\rightarrow
x_{0}^{+}}f\left( x\right) =f\left( x_{0}^{+}\right) , \\
Sf\left( x_{0}\right) =f\left( x_{0}^{+}\right) -f\left( x_{0}^{-}\right) \\
\text{Ts: }Du_{f}=Sf\left( x_{0}\right) \delta _{x_{0}}+u_{f^{\prime }}
\end{gather*}

$Sf\left( x_{0}\right) $ \`{e} la misura del salto di $f$ in $x_{0}$. Si
noti che poich\'{e} $f,f^{\prime }$ sono continue a meno di un insieme di
misura nulla, sono essenzialmente limitate in ogni compatto e dunque
appartengono a $L_{loc}^{1}\left( 
%TCIMACRO{\U{211d} }%
%BeginExpansion
\mathbb{R}
%EndExpansion
\right) $: sono ben definite $u_{f},u_{f^{\prime }}$.

Questa proposizione fornisce un metodo per calcolare in fretta la derivata
di una distribuzione associata a una funzione $C^{1}$ con una discontinuit%
\`{a} a salto.

La proposizione pu\`{o} essere facilmente estesa a funzioni $C^{1}\left( 
%TCIMACRO{\U{211d} }%
%BeginExpansion
\mathbb{R}
%EndExpansion
\backslash \left\{ x_{0},...,x_{m}\right\} \right) $ con $m+1$ discontinuit%
\`{a} a salto.

\textbf{Dim} Per definizione $Du_{f}=-u_{f}\left( \phi ^{\prime }\right) $:
si calcola $u_{f}\left( \phi ^{\prime }\right) $ sfruttando un procedimento
di limite. Integrando per parti si ha $\int_{\left\vert x-x_{0}\right\vert
>\varepsilon }f\phi ^{\prime }=\int_{-\infty }^{x_{0}-\varepsilon }f\phi
^{\prime }+\int_{x_{0}+\varepsilon }^{+\infty }f\phi ^{\prime }=$ $\left[
f\phi \right] _{-\infty }^{x_{0}-\varepsilon }-\int_{-\infty
}^{x_{0}-\varepsilon }f^{\prime }\phi +\left[ f\phi \right]
_{x_{0}+\varepsilon }^{+\infty }-\int_{x_{0}+\varepsilon }^{+\infty
}f^{\prime }\phi =$ $f\left( x_{0}-\varepsilon \right) \phi \left(
x_{0}-\varepsilon \right) -f\left( x_{0}+\varepsilon \right) \phi \left(
x_{0}+\varepsilon \right) -\int_{\left\vert x-x_{0}\right\vert >\varepsilon
}f^{\prime }\phi $. Passando al limite $u_{f}\left( \phi ^{\prime }\right)
=\lim_{\varepsilon \rightarrow 0^{+}}\int_{\left\vert x-x_{0}\right\vert
>\varepsilon }f\phi ^{\prime }=\phi \left( x_{0}\right) \left( f\left(
x_{0}^{-}\right) -f\left( x_{0}^{+}\right) \right) -\int_{%
%TCIMACRO{\U{211d} }%
%BeginExpansion
\mathbb{R}
%EndExpansion
}f^{\prime }\phi $, quindi $Du_{f}=\phi \left( x_{0}\right) Sf\left(
x_{0}\right) +u_{f^{\prime }}$. $\blacksquare $

Si \`{e} dunque visto che si possono effettuare operazioni sulle
distribuzioni (ad esempio calcolo di derivate) riconducendosi - seguendo
l'esempio di ci\`{o} che accade per le distribuzioni del tipo $u_{f}$ - a
operazioni sulle funzioni test $\phi $, di cui si sfrutta l'enorme regolarit%
\`{a}, senza alcuna richiesta sulle $f$ eccetto l'appartenenza a $%
L_{loc}^{1} $.

\begin{enumerate}
\item Ad esempio, si vuole definire la traslata di una distribuzione. Se $%
f\in C^{1}\left( 
%TCIMACRO{\U{211d} }%
%BeginExpansion
\mathbb{R}
%EndExpansion
\right) $, \`{e} coerente con l'intuizione definire distribuzione traslata
la distribuzione associata a $f$ traslata, cio\`{e} $v\left( \phi \right)
=\int_{%
%TCIMACRO{\U{211d} }%
%BeginExpansion
\mathbb{R}
%EndExpansion
}f\left( t+a\right) \phi \left( t\right) dt$. Con il cambio di variabile $%
y=t+a$ si ha $\int_{%
%TCIMACRO{\U{211d} }%
%BeginExpansion
\mathbb{R}
%EndExpansion
}f\left( t+a\right) \phi \left( t\right) dt=\int_{%
%TCIMACRO{\U{211d} }%
%BeginExpansion
\mathbb{R}
%EndExpansion
}f\left( y\right) \phi \left( y-a\right) dy$, per cui si pu\`{o} definire,
per $f\in L_{loc}^{1}$, la distribuzione traslata di $u_{f}$ come $%
u_{1,a}\left( \phi \right) =\int_{%
%TCIMACRO{\U{211d} }%
%BeginExpansion
\mathbb{R}
%EndExpansion
}f\left( y\right) \phi \left( y-a\right) dy$.
\end{enumerate}

Pi\`{u} in generale, se $u\in \mathcal{D}^{\prime }\left( 
%TCIMACRO{\U{211d} }%
%BeginExpansion
\mathbb{R}
%EndExpansion
^{n}\right) $, si definisce distribuzione traslata di $u$, con $A\in M_{%
%TCIMACRO{\U{211d} }%
%BeginExpansion
\mathbb{R}
%EndExpansion
}\left( n,n\right) ,\mathbf{b\in 
%TCIMACRO{\U{211d} }%
%BeginExpansion
\mathbb{R}
%EndExpansion
}^{n}$ parametri di traslazione, la distribuzione $u_{A,\mathbf{b}}\left(
\phi \right) =u\left( \phi \left( A^{-1}\left( \mathbf{y-b}\right) \right) 
\frac{1}{\left\vert \det A\right\vert }\right) $.

\begin{enumerate}
\item La distribuzione delta traslata di $x_{0}$ \`{e} $\delta
_{1,-x_{0}}\left( \phi \right) =\delta \left( \phi \left( y+x_{0}\right)
\right) =\phi \left( y+x_{0}\right) |_{y=0}=\phi \left( x_{0}\right) $.
\end{enumerate}

Con la definizione di distribuzione traslata si possono estendere alle
distribuzioni definizioni gi\`{a} note per le funzioni di pi\`{u} variabili
reali.

\textbf{Def} Data $u\in \mathcal{D}^{\prime }\left( \Omega \right) $, $%
\Omega \subseteq 
%TCIMACRO{\U{211d} }%
%BeginExpansion
\mathbb{R}
%EndExpansion
^{n}$ aperto e $\mathbf{v}\in 
%TCIMACRO{\U{211d} }%
%BeginExpansion
\mathbb{R}
%EndExpansion
^{n}:\left\vert \left\vert \mathbf{v}\right\vert \right\vert =1$, si dice
derivata distribuzionale direzionale di $u$ nella direzione $\mathbf{v}$ la
distribuzione $w\left( \phi \right) :=\lim_{h\rightarrow 0}\frac{u_{Id,h%
\mathbf{v}}\left( \phi \right) -u\left( \phi \right) }{h}$, che si indica
con $D_{\mathbf{v}}u$.

$v_{Id,h\mathbf{v}}\left( \phi \right) $ \`{e} la distribuzione $u$ traslata
con $\mathbf{x}+h\mathbf{v}$, per cui $u_{Id,h\mathbf{v}}\left( \phi \right)
=u\left( \phi \left( \mathbf{y-}h\mathbf{v}\right) \right) $. $w$ \`{e} ben
definita perch\'{e} $\mathcal{D}^{\prime }\left( \Omega \right) $ \`{e} uno
spazio vettoriale e inoltre si pu\`{o} dimostrare che $\lim_{h\rightarrow 0}%
\frac{u_{Id,h\mathbf{v}}\left( \phi \right) -u\left( \phi \right) }{h}$ \`{e}
una distribuzione.

\textbf{Def} Data $u\in \mathcal{D}^{\prime }\left( 
%TCIMACRO{\U{211d} }%
%BeginExpansion
\mathbb{R}
%EndExpansion
\right) $, si dice che $u$ \`{e} periodica di periodo $T>0$ se $%
u_{1,T}\left( \phi \right) =u_{1}\left( \phi \right) $ $\forall $ $\phi \in 
\mathcal{D}\left( \Omega \right) $.

Se una funzione periodica definisce una distribuzione, allora tale
distribuzione \`{e} periodica.

In tal caso le due distribuzioni coincidono, cio\`{e} - fissato $y$ - $%
u\left( \phi \left( y-T\right) \right) =u\left( \phi \right) $ $\forall $ $%
\phi \in \mathcal{D}\left( \Omega \right) $.

Si dice che $u$ \`{e} pari se $u_{-1}=u_{1}$; si dice che \`{e} dispari se $%
u_{-1}=-u_{1}$.

\begin{enumerate}
\item Ogni distribuzione associata a una funzione dispari (pari) \`{e}
dispari (pari).

\item $u\left( \phi \right) =vp\int_{%
%TCIMACRO{\U{211d} }%
%BeginExpansion
\mathbb{R}
%EndExpansion
}\frac{\phi \left( t\right) }{t}dt$ \`{e} dispari. Infatti $u\left( \phi
\left( -y\right) \right) =vp\int_{%
%TCIMACRO{\U{211d} }%
%BeginExpansion
\mathbb{R}
%EndExpansion
}\frac{\phi \left( -y\right) }{-y}dy=\int_{%
%TCIMACRO{\U{211d} }%
%BeginExpansion
\mathbb{R}
%EndExpansion
}\frac{\phi \left( t\right) }{t}dt$.

\item Se $f\left( x\right) =\log \left\vert x\right\vert $, $D\left(
u_{f}\right) =vp\int_{%
%TCIMACRO{\U{211d} }%
%BeginExpansion
\mathbb{R}
%EndExpansion
}\frac{1}{x}dx$. Infatti $D\left( u_{f}\right) =-\int_{%
%TCIMACRO{\U{211d} }%
%BeginExpansion
\mathbb{R}
%EndExpansion
}\ln \left\vert x\right\vert \phi ^{\prime }\left( x\right)
dx=-\lim_{\varepsilon \rightarrow 0}\int_{\left\vert x\right\vert
>\varepsilon }\ln \left\vert x\right\vert \phi ^{\prime }\left( x\right) dx$%
. L'obiettivo \`{e} sfruttare l'integrazione per parti: $\int_{\varepsilon
}^{+\infty }\ln \left\vert x\right\vert \phi ^{\prime }\left( x\right)
dx+\int_{-\infty }^{-\varepsilon }\ln \left\vert x\right\vert \phi ^{\prime
}\left( x\right) dx=-\ln \left\vert \varepsilon \right\vert \phi \left(
\varepsilon \right) -\int_{\varepsilon }^{+\infty }\frac{1}{x}\phi \left(
x\right) dx+\ln \left\vert -\varepsilon \right\vert \phi ^{\prime }\left(
-\varepsilon \right) -\int_{-\infty }^{-\varepsilon }\frac{\phi \left(
x\right) }{x}dx$ $=\ln \left\vert \varepsilon \right\vert \left[ \phi \left(
-\varepsilon \right) -\phi \left( \varepsilon \right) \right]
-\int_{\left\vert x\right\vert >\varepsilon }\frac{\phi \left( x\right) }{x}%
dx$, quindi $D\left( u_{f}\right) =\lim_{\varepsilon \rightarrow
0}\int_{\left\vert x\right\vert >\varepsilon }\frac{\phi \left( x\right) }{x}%
dx+\lim_{\varepsilon \rightarrow 0}\ln \left\vert \varepsilon \right\vert %
\left[ \phi \left( \varepsilon \right) -\phi \left( -\varepsilon \right) %
\right] $. Il primo addendo \`{e} $vp\int_{%
%TCIMACRO{\U{211d} }%
%BeginExpansion
\mathbb{R}
%EndExpansion
}\frac{1}{x}dx$, il secondo \`{e} nullo perch\'{e} $\phi \left( \varepsilon
\right) -\phi \left( -\varepsilon \right) =2\phi ^{\prime }\left( 0\right)
\varepsilon +o\left( \varepsilon \right) $ e quindi $\lim_{\varepsilon
\rightarrow 0}\ln \left\vert \varepsilon \right\vert \left( 2\phi ^{\prime
}\left( 0\right) \varepsilon +o\left( \varepsilon \right) \right) =0$.

\item $\sum_{k\in 
%TCIMACRO{\U{2124} }%
%BeginExpansion
\mathbb{Z}
%EndExpansion
}\delta _{x-k}$ converge nel senso delle distribuzioni a una distribuzione
1-periodica (per definizione $\delta _{x-k}\left( \phi \right) =\delta
\left( \phi \left( y+k\right) \right) =\phi \left( k\right) $). Infatti la
successione delle somme parziali \`{e} $\left( \sum_{\left\vert k\right\vert
\leq n,k\in 
%TCIMACRO{\U{2124} }%
%BeginExpansion
\mathbb{Z}
%EndExpansion
}\delta _{x-k}\right) \left( \phi \right) =\sum_{\left\vert k\right\vert
\leq n,k\in 
%TCIMACRO{\U{2124} }%
%BeginExpansion
\mathbb{Z}
%EndExpansion
}\delta _{x-k}\left( \phi \right) =\sum_{\left\vert k\right\vert \leq n,k\in 
%TCIMACRO{\U{2124} }%
%BeginExpansion
\mathbb{Z}
%EndExpansion
}\phi \left( k\right) $, ma $\lim_{n\rightarrow +\infty }\sum_{\left\vert
k\right\vert \leq n,k\in 
%TCIMACRO{\U{2124} }%
%BeginExpansion
\mathbb{Z}
%EndExpansion
}\phi \left( k\right) $ \`{e} di fatto una somma finita perch\'{e} le $\phi $
hanno supporto compatto, quindi la serie converge a una distribuzione $u$.
Inoltre $u$ ha periodo $1$ perch\'{e} $u_{1,1}\left( \phi \right) =u\left(
\phi \left( y-1\right) \right) =\sum_{k\in 
%TCIMACRO{\U{2124} }%
%BeginExpansion
\mathbb{Z}
%EndExpansion
}\phi \left( k-1\right) =u\left( \phi \right) $.

\item La serie $\sum_{n=0}^{+\infty }\delta _{x-\frac{1}{k}}$ converge? Dove?

$\left( \sum_{n=0}^{+\infty }\delta _{x-\frac{1}{k}}\right) \left( \phi
\right) =\sum_{n=0}^{+\infty }\delta _{x-\frac{1}{k}}\left( \phi \right)
=\sum_{n=0}^{+\infty }\phi \left( \frac{1}{k}\right) $. E' evidente che in
generale la serie non converge in $\mathcal{D}^{\prime }\left( 
%TCIMACRO{\U{211d} }%
%BeginExpansion
\mathbb{R}
%EndExpansion
\right) $, perch\'{e} se $\phi \left( 0\right) \neq 0$ non \`{e} neanche
verificata la condizione necessaria di convergenza. Tuttavia converge in $%
\mathcal{D}^{\prime }\left( \left( 0,+\infty \right) \right) $ perch\'{e} le 
$\phi $ hanno supporto compatto e quindi la serie si riduce a una somma
finita.
\end{enumerate}

Si pu\`{o} inoltre dare la definizione di primitiva di una distribuzione.

\textbf{Teo 3.5 (esistenza della primitiva)}%
\begin{gather*}
\text{Hp: }u\in \mathcal{D}^{\prime }\left( 
%TCIMACRO{\U{211d} }%
%BeginExpansion
\mathbb{R}
%EndExpansion
\right) \\
\text{Ts: (i) }\exists \text{ }v\in \mathcal{D}^{\prime }\left( 
%TCIMACRO{\U{211d} }%
%BeginExpansion
\mathbb{R}
%EndExpansion
\right) :Dv=u\text{; in tal caso, se }c\in 
%TCIMACRO{\U{211d} }%
%BeginExpansion
\mathbb{R}
%EndExpansion
\text{, }\forall \text{ }\bar{v}=v+u_{c}\text{ vale }D\bar{v}=u
\end{gather*}

\begin{gather*}
\text{Hp: }v_{1},v_{2}\in \mathcal{D}^{\prime }\left( 
%TCIMACRO{\U{211d} }%
%BeginExpansion
\mathbb{R}
%EndExpansion
\right) :Dv_{1}=Dv_{2} \\
\text{Ts: }\exists \text{ }c\in 
%TCIMACRO{\U{211d} }%
%BeginExpansion
\mathbb{R}
%EndExpansion
:v_{1}=v_{2}+c
\end{gather*}

Quindi tutte le distribuzioni hanno primitive.

E' naturale, una volta definite traslate e derivate di distribuzioni,
pensare anche alla definizione di un prodotto tra distribuzioni. Questo si
rivela essere un proposito estremamente ostico - nei fatti, impossibile: non
si riesce a dare una buona definizione del prodotto di due distribuzioni. Si
pu\`{o} per\`{o} definire il prodotto tra una distribuzione e una funzione
standard.

\textbf{Def} Data $u\in \mathcal{D}^{\prime }\left( \Omega \right) $ e $\psi
\in C^{\infty }\left( \Omega \right) $, si definisce la distribuzione $u\psi
\left( \phi \right) :=u\left( \psi \phi \right) $.

$\phi \psi \in \mathcal{D}\left( \Omega \right) $, quindi tale distribuzione 
\`{e} ben definita $\forall $ $\phi $.

\begin{enumerate}
\item Data $u=\delta _{0}$ e la funzione $\psi \left( x\right) =\frac{1}{%
x^{2}+1}$, si ha $u\psi \left( \phi \right) =u\left( \frac{1}{x^{2}+1}\phi
\right) =\phi \left( 0\right) =\delta _{0}$.
\end{enumerate}

\textbf{Prop 3.6 (regola di Leibniz)}%
\begin{gather*}
\text{Hp: }u\in \mathcal{D}^{\prime }\left( \Omega \right) ,\psi \in
C^{\infty }\left( \Omega \right) \\
\text{Ts: }D\left( \psi u\right) =\psi ^{\prime }u+\psi Du
\end{gather*}

\textbf{Dim} Per definizione $D\left( \psi u\right) =-\psi u\left( \phi
^{\prime }\right) =-u\left( \psi \phi ^{\prime }\right) =u\left( \psi
^{\prime }\phi -\left( \psi \phi \right) ^{\prime }\right) =\left( \psi
^{\prime }u\right) \left( \phi \right) +Du\left( \phi \psi \right) =\left(
\psi ^{\prime }u\right) \left( \phi \right) +\psi Du\left( \phi \right) $. $%
\blacksquare $

\subsection{Distribuzioni temperate}

\textbf{Def} Data $\phi \in C^{\infty }\left( 
%TCIMACRO{\U{211d} }%
%BeginExpansion
\mathbb{R}
%EndExpansion
^{n}\right) $, $\phi $ si dice funzione a decrescenza rapida se $\forall $ $%
\alpha \in 
%TCIMACRO{\U{2115} }%
%BeginExpansion
\mathbb{N}
%EndExpansion
^{n}$ vale $D^{\alpha }\phi =o\left( \frac{1}{\left\vert \left\vert \mathbf{x%
}\right\vert \right\vert ^{k}}\right) $ per $\left\vert \left\vert \mathbf{x}%
\right\vert \right\vert \rightarrow +\infty $, $\forall $ $k\in 
%TCIMACRO{\U{2115} }%
%BeginExpansion
\mathbb{N}
%EndExpansion
$. L'insieme delle funzioni a decrescenza rapida si dice spazio di Schwartz
e si indica con $\mathcal{S}\left( 
%TCIMACRO{\U{211d} }%
%BeginExpansion
\mathbb{R}
%EndExpansion
^{n}\right) $.

$\mathcal{S}\left( 
%TCIMACRO{\U{211d} }%
%BeginExpansion
\mathbb{R}
%EndExpansion
^{n}\right) $ \`{e} uno spazio vettoriale, grazie alla linearit\`{a} della
derivata.

\begin{enumerate}
\item $e^{-x^{2}}\in \mathcal{S}\left( 
%TCIMACRO{\U{211d} }%
%BeginExpansion
\mathbb{R}
%EndExpansion
^{n}\right) $, $e^{x},e^{-x}\not\in S\left( 
%TCIMACRO{\U{211d} }%
%BeginExpansion
\mathbb{R}
%EndExpansion
^{n}\right) $. Nessun polinomio \`{e} in $\mathcal{S}\left( 
%TCIMACRO{\U{211d} }%
%BeginExpansion
\mathbb{R}
%EndExpansion
^{n}\right) $.
\end{enumerate}

Si noti che se $P\left( \mathbf{x}\right) $ \`{e} un polinomio algebrico
nelle variabili $x_{1},...,x_{n}$ e $\phi \in \mathcal{S}\left( 
%TCIMACRO{\U{211d} }%
%BeginExpansion
\mathbb{R}
%EndExpansion
^{n}\right) $, allora $P\cdot D^{\alpha }\phi $ \`{e} anch'essa a
decrescenza rapida per qualsiasi multiindice.

Inoltre $\phi \in \mathcal{S}\left( 
%TCIMACRO{\U{211d} }%
%BeginExpansion
\mathbb{R}
%EndExpansion
^{n}\right) $ \`{e} anche limitata e in $L^{p}\left( 
%TCIMACRO{\U{211d} }%
%BeginExpansion
\mathbb{R}
%EndExpansion
^{n}\right) $ $\forall $ $p$, perch\'{e}, fissato un qualsiasi compatto, $%
\phi $ \`{e} ivi limitata, e fuori dal compatto qualsiasi sua potenza $p$%
-esima \`{e} limitata e integrabile perch\'{e} $\phi =o\left( \frac{1}{%
\left\vert \left\vert \mathbf{x}\right\vert \right\vert ^{k}}\right) $ per $%
\left\vert \left\vert \mathbf{x}\right\vert \right\vert \rightarrow +\infty $%
.

\textbf{Def} Data $\left\{ \phi _{n}\right\} _{n\in 
%TCIMACRO{\U{2115} }%
%BeginExpansion
\mathbb{N}
%EndExpansion
}\subseteq \mathcal{S}\left( 
%TCIMACRO{\U{211d} }%
%BeginExpansion
\mathbb{R}
%EndExpansion
^{n}\right) $, si dice che $\phi _{n}$ converge a $\phi $ in $\mathcal{S}%
\left( 
%TCIMACRO{\U{211d} }%
%BeginExpansion
\mathbb{R}
%EndExpansion
^{n}\right) $ se $\forall $ $\alpha \in 
%TCIMACRO{\U{2115} }%
%BeginExpansion
\mathbb{N}
%EndExpansion
^{n}$, $\forall $ $P$ polinomio $PD^{\alpha }\phi _{n}$ converge
uniformemente in $%
%TCIMACRO{\U{211d} }%
%BeginExpansion
\mathbb{R}
%EndExpansion
^{n}$ a $PD^{\alpha }\phi $.

Si pu\`{o} mostrare che le successioni di Cauchy in $\mathcal{S}\left( 
%TCIMACRO{\U{211d} }%
%BeginExpansion
\mathbb{R}
%EndExpansion
^{n}\right) $ convergono.

$\mathcal{D}\left( 
%TCIMACRO{\U{211d} }%
%BeginExpansion
\mathbb{R}
%EndExpansion
^{n}\right) \subseteq \mathcal{S}\left( 
%TCIMACRO{\U{211d} }%
%BeginExpansion
\mathbb{R}
%EndExpansion
^{n}\right) $ perch\'{e} se $\phi $ ha supporto compatto \`{e}
definitivamente nulla per $\left\vert \left\vert \mathbf{x}\right\vert
\right\vert \rightarrow +\infty $, dunque \`{e} certamente un o piccolo di $%
\frac{1}{\left\vert \left\vert \mathbf{x}\right\vert \right\vert ^{k}}$ $%
\forall $ $k$. L'inclusione \`{e} propria perch\'{e} $e^{-x^{2}}\not\in 
\mathcal{D}\left( 
%TCIMACRO{\U{211d} }%
%BeginExpansion
\mathbb{R}
%EndExpansion
^{n}\right) $. Se una successione converge in $\mathcal{D}\left( 
%TCIMACRO{\U{211d} }%
%BeginExpansion
\mathbb{R}
%EndExpansion
^{n}\right) $, converge anche in $\mathcal{S}\left( 
%TCIMACRO{\U{211d} }%
%BeginExpansion
\mathbb{R}
%EndExpansion
^{n}\right) \footnote{%
in generale il prodotto di due successioni di funzioni uniformemente
convergenti converge uniformemente al prodotto delle funzioni limite se
queste sono entrambe limitate, cosa vera nel nostro caso (essendo le $\phi $
a supporto compatto, ai fini del prodotto anche $P$ pu\`{o} essere
considerato a supporto compatto)}$.

Allora l'insieme dei funzionali lineari continui su $\mathcal{S}\left( 
%TCIMACRO{\U{211d} }%
%BeginExpansion
\mathbb{R}
%EndExpansion
^{n}\right) $ sar\`{a} pi\`{u} piccolo di quelli definiti su $\mathcal{D}%
\left( 
%TCIMACRO{\U{211d} }%
%BeginExpansion
\mathbb{R}
%EndExpansion
^{n}\right) $: un funzionale definito su $\mathcal{S}\left( 
%TCIMACRO{\U{211d} }%
%BeginExpansion
\mathbb{R}
%EndExpansion
^{n}\right) $ \`{e} anche definito su $\mathcal{D}\left( 
%TCIMACRO{\U{211d} }%
%BeginExpansion
\mathbb{R}
%EndExpansion
^{n}\right) $, mentre il viceversa in generale non \`{e} vero perch\'{e} un
funzionale su $\mathcal{D}\left( 
%TCIMACRO{\U{211d} }%
%BeginExpansion
\mathbb{R}
%EndExpansion
^{n}\right) $ potrebbe non poter essere esteso a $\mathcal{S}\left( 
%TCIMACRO{\U{211d} }%
%BeginExpansion
\mathbb{R}
%EndExpansion
^{n}\right) $. In generale, pi\`{u} un insieme \`{e} "grande", pi\`{u} il
suo duale \`{e} "piccolo".

\textbf{Def} Un funzionale $u:\mathcal{S}\left( 
%TCIMACRO{\U{211d} }%
%BeginExpansion
\mathbb{R}
%EndExpansion
^{n}\right) \rightarrow 
%TCIMACRO{\U{211d} }%
%BeginExpansion
\mathbb{R}
%EndExpansion
\left( 
%TCIMACRO{\U{2102} }%
%BeginExpansion
\mathbb{C}
%EndExpansion
\right) $ lineare e continuo si dice distribuzione temperata. L'insieme
delle distribuzioni temperate si indica con $\mathcal{S}^{\prime }\left( 
%TCIMACRO{\U{211d} }%
%BeginExpansion
\mathbb{R}
%EndExpansion
^{n}\right) $.

La continuit\`{a} \`{e} definita come al solito: $\forall $ $\phi \in 
\mathcal{S}\left( 
%TCIMACRO{\U{211d} }%
%BeginExpansion
\mathbb{R}
%EndExpansion
^{n}\right) ,\forall $ $\left\{ \phi _{n}\right\} \subseteq \mathcal{S}%
\left( 
%TCIMACRO{\U{211d} }%
%BeginExpansion
\mathbb{R}
%EndExpansion
^{n}\right) :\phi _{n}\rightarrow ^{\mathcal{S}\left( 
%TCIMACRO{\U{211d} }%
%BeginExpansion
\mathbb{R}
%EndExpansion
^{n}\right) }\phi $ vale $u\left( \phi _{n}\right) \rightarrow u\left( \phi
\right) $. $\mathcal{S}^{\prime }\left( 
%TCIMACRO{\U{211d} }%
%BeginExpansion
\mathbb{R}
%EndExpansion
^{n}\right) $ \`{e} il duale continuo di $\mathcal{S}\left( 
%TCIMACRO{\U{211d} }%
%BeginExpansion
\mathbb{R}
%EndExpansion
^{n}\right) $ ed \`{e} uno spazio vettoriale.

\textbf{Def} Data una successione di distribuzioni temperate $\left\{
u_{j}\right\} _{j\in 
%TCIMACRO{\U{2115} }%
%BeginExpansion
\mathbb{N}
%EndExpansion
}\subseteq $ $\mathcal{S}^{\prime }\left( 
%TCIMACRO{\U{211d} }%
%BeginExpansion
\mathbb{R}
%EndExpansion
^{n}\right) $, si dice che $u_{j}$ converge a $u$, e si scrive $%
u_{j}\rightarrow ^{\mathcal{S}^{\prime }\left( 
%TCIMACRO{\U{211d} }%
%BeginExpansion
\mathbb{R}
%EndExpansion
^{n}\right) }u$ per $j\rightarrow +\infty $, se $u_{j}\left( \phi \right)
\rightarrow ^{j\rightarrow +\infty }u\left( \phi \right) $ $\forall $ $\phi
\in $ $\mathcal{S}\left( 
%TCIMACRO{\U{211d} }%
%BeginExpansion
\mathbb{R}
%EndExpansion
^{n}\right) $.

Le successioni di Cauchy in $\mathcal{S}^{\prime }\left( 
%TCIMACRO{\U{211d} }%
%BeginExpansion
\mathbb{R}
%EndExpansion
^{n}\right) $ convergono. La convergnza puntuale in $\mathcal{S}^{\prime
}\left( 
%TCIMACRO{\U{211d} }%
%BeginExpansion
\mathbb{R}
%EndExpansion
^{n}\right) $ ovviamente implica la convergenza puntuale in $\mathcal{D}%
^{\prime }\left( 
%TCIMACRO{\U{211d} }%
%BeginExpansion
\mathbb{R}
%EndExpansion
^{n}\right) $.

\begin{enumerate}
\item Se $f\left( x\right) =e^{x}$, $u_{f}\in \mathcal{D}^{\prime }\left( 
%TCIMACRO{\U{211d} }%
%BeginExpansion
\mathbb{R}
%EndExpansion
\right) $, ma non \`{e} possibile estendere $u_{f}$ in modo che sia una
distribuzione definita su $\mathcal{S}^{\prime }\left( 
%TCIMACRO{\U{211d} }%
%BeginExpansion
\mathbb{R}
%EndExpansion
\right) $. Intuitivamente \`{e} ovvio: non \`{e} possibile che $\int_{%
%TCIMACRO{\U{211d} }%
%BeginExpansion
\mathbb{R}
%EndExpansion
^{n}}f\phi $ sia ben definito $\forall $ $\phi \in \mathcal{S}\left( 
%TCIMACRO{\U{211d} }%
%BeginExpansion
\mathbb{R}
%EndExpansion
\right) $, perch\'{e} tali $\phi $ tendono a zero pi\`{u} in fretta dei
reciproci di polinomi, non necessariamente di $e^{x}$.

Analiticamente, se per assurdo $u_{f}\in \mathcal{S}^{\prime }\left( 
%TCIMACRO{\U{211d} }%
%BeginExpansion
\mathbb{R}
%EndExpansion
\right) $, in particolare dovrebbe essere finito $u_{f}\left( e^{-\sqrt{%
1+\left\vert x\right\vert ^{2}}}\right) =\int_{%
%TCIMACRO{\U{211d} }%
%BeginExpansion
\mathbb{R}
%EndExpansion
}e^{x}e^{-\sqrt{1+\left\vert x\right\vert ^{2}}}dx$, ma la funzione
integranda \`{e} asintotica a $1$ per $x\rightarrow +\infty $, quindi
l'integrale \`{e} infinito.

\item Se $P\left( \mathbf{x}\right) $ \`{e} un polinomio algebrico, $%
u_{P}\left( \phi \right) =\int_{%
%TCIMACRO{\U{211d} }%
%BeginExpansion
\mathbb{R}
%EndExpansion
^{n}}P\left( \mathbf{x}\right) \phi \left( \mathbf{x}\right) d\mathbf{x}$ 
\`{e} una distribuzione temperata: l'integrale \`{e} ben definito perch\'{e} 
$P\phi $ appartiene anch'essa a $\mathcal{S}\left( 
%TCIMACRO{\U{211d} }%
%BeginExpansion
\mathbb{R}
%EndExpansion
\right) $, dunque \`{e} in $L^{1}$, e si mostrano facilmente linearit\`{a} e
continuit\`{a}. In particolare ogni costante definisce una distribuzione
temperata.

\item Se $P\left( \mathbf{x}\right) $ \`{e} un polinomio algebrico e $%
w\left( \mathbf{x}\right) \in L^{1}$, $u_{Pw}\left( \phi \right) =\int_{%
%TCIMACRO{\U{211d} }%
%BeginExpansion
\mathbb{R}
%EndExpansion
^{n}}P\left( \mathbf{x}\right) w\left( \mathbf{x}\right) \phi \left( \mathbf{%
x}\right) d\mathbf{x}$ \`{e} una distribuzione temperata: l'integrale \`{e}
ben definito per Hoelder (perch\'{e} $P\phi $ \`{e} limitata e $w$ \`{e} $%
L^{1}$) e si mostrano facilmente linearit\`{a} e continuit\`{a}. $Pw$ si
dice funzione a crescita lenta.

Quindi ad esempio $f\left( x\right) =\frac{\cos x}{x^{1/3}}$ definisce una
distribuzione temperata perch\'{e} $f=Pw$ con $w\left( x\right) =\cos x\frac{%
x^{-\frac{1}{3}}}{x^{2}+1}$ e $P\left( x\right) =x^{2}+1$.

\item Se $f\in L^{p}\left( 
%TCIMACRO{\U{211d} }%
%BeginExpansion
\mathbb{R}
%EndExpansion
^{n}\right) $ per $p\in \left[ 1,+\infty \right] $ qualsiasi, $u_{f}\left(
\phi \right) =\int_{%
%TCIMACRO{\U{211d} }%
%BeginExpansion
\mathbb{R}
%EndExpansion
^{n}}f\left( \mathbf{x}\right) \phi \left( \mathbf{x}\right) d\mathbf{x}$ 
\`{e} una distribuzione temperata: l'integrale \`{e} ben definito per
Hoelder (perch\'{e} $\phi \in L^{q}$ se $q$ \`{e} l'esponente coniugato di $%
p $) e si mostrano facilmente linearit\`{a} e continuit\`{a}.

Se $f$ \`{e} solamente $L_{loc}^{1}$ (e. g. $e^{x}$) in generale non
definisce una distribuzione temperata.

\item Visto l'esempio sopra, ha senso chiedersi se anche per le
distribuzioni standard $f\in L_{loc}^{p}\left( \Omega \right) $ ($\Omega
\subseteq 
%TCIMACRO{\U{211d} }%
%BeginExpansion
\mathbb{R}
%EndExpansion
^{n}$ aperto) definisce una distribuzione, essendo gi\`{a} noto il risultato
per $p=1$. Ma questo \`{e} vero perch\'{e} $\int_{\Omega }f\phi =\int_{\text{%
supp}\left( \phi \right) }f\phi $ e $\int_{\text{supp}\left( \phi \right)
}\left\vert f\right\vert <+\infty $ perch\'{e} $f\in L_{loc}^{p}\left(
\Omega \right) $ (quindi \`{e} $L^{p}$ e dunque $L^{1}$ sul compatto supp$%
\left( \phi \right) $): essendo $\phi $ limitata, $f\phi $ \`{e} integrabile
e l'integrale \`{e} ben definito. Si mostrano facilmente linearit\`{a} e
continuit\`{a}, dunque $u_{f}$ \`{e} una distribuzione.

\item Se in particolare $f\in L^{1}\left( 
%TCIMACRO{\U{211d} }%
%BeginExpansion
\mathbb{R}
%EndExpansion
^{n}\right) $, \`{e} ben definita $u_{f}$ sia come distribuzione temperata
che standard. Si hanno allora quattro diverse nozioni di convergenza:
puntuale quasi ovunque, in norma $L^{p}$, in distribuzione standard, in
distribuzione temperata. In tutti i casi, il limite - se esiste - \`{e} lo
stesso. Perci\`{o}, se si vuole valutare la convergenza in distribuzione o
in norma, \`{e} naturale calcolare prima il limite puntuale: se esiste, \`{e}
l'unico candidato per la convergenza in distribuzione o in norma.

\item $\sum_{k=0}^{+\infty }c_{k}\delta _{x-k}$, data $\left\{ c_{k}\right\}
_{k}$ qualsiasi, in generale converge in $\mathcal{D}^{\prime }\left( 
%TCIMACRO{\U{211d} }%
%BeginExpansion
\mathbb{R}
%EndExpansion
\right) $ ma non in $\mathcal{S}^{\prime }\left( 
%TCIMACRO{\U{211d} }%
%BeginExpansion
\mathbb{R}
%EndExpansion
\right) $. Infatti, se $\phi $ ha supporto compatto, $\sum_{k=0}^{+\infty
}c_{k}\phi \left( k\right) $ si riduce a una somma finita. Invece, se $\phi
\left( x\right) =e^{-\sqrt{1+x^{2}}}$ e $c_{k}=e^{k}$, $\sum_{k=0}^{+\infty
}c_{k}\phi \left( k\right) $ ha termine generale asintotico a $1$ e quindi
non converge.
\end{enumerate}

Rafforzando le ipotesi su $f\in L_{loc}^{1}$ si pu\`{o} vedere quando essa
definisce una distribuzione temperata.

\textbf{Prop 3.7 }%
\begin{eqnarray*}
\text{Hp}\text{: } &&f\in L_{loc}^{1}\left( 
%TCIMACRO{\U{211d} }%
%BeginExpansion
\mathbb{R}
%EndExpansion
\right) ,\exists \text{ }N\in 
%TCIMACRO{\U{2115} }%
%BeginExpansion
\mathbb{N}
%EndExpansion
:\int_{%
%TCIMACRO{\U{211d} }%
%BeginExpansion
\mathbb{R}
%EndExpansion
}\frac{1}{\left( 1+\left\vert x\right\vert \right) ^{N}}f\left( x\right)
dx<+\infty \\
\text{Ts}\text{: } &&u_{f}\in \mathcal{S}^{\prime }\left( 
%TCIMACRO{\U{211d} }%
%BeginExpansion
\mathbb{R}
%EndExpansion
\right)
\end{eqnarray*}

Si noti che queste ipotesi non servono ad altro che ad assicurare che $f$
sia a crescita lenta: se valgono, si ha che $f$ si pu\`{o} scrivere come
prodotto di $\frac{1}{\left( 1+\left\vert x\right\vert \right) ^{N}}f\left(
x\right) $ funzione integrabile e $\left( 1+\left\vert x\right\vert \right)
^{N}$ polinomio.

\textbf{Dim} $\int_{%
%TCIMACRO{\U{211d} }%
%BeginExpansion
\mathbb{R}
%EndExpansion
}f\phi =\int_{%
%TCIMACRO{\U{211d} }%
%BeginExpansion
\mathbb{R}
%EndExpansion
}\frac{f\left( x\right) }{\left( 1+\left\vert x\right\vert \right) ^{N}}%
\left( 1+\left\vert x\right\vert \right) ^{N}\phi \left( x\right) dx$. Il
primo fattore dell'integranda \`{e} integrabile per ipotesi e $\left(
1+\left\vert x\right\vert \right) ^{N}\phi \left( x\right) \in \mathcal{S}%
\left( 
%TCIMACRO{\U{211d} }%
%BeginExpansion
\mathbb{R}
%EndExpansion
\right) $, dunque per Hoelder l'integrale \`{e} ben definito.

$u_{f}$ \`{e} ovviamente lineare. E' inoltre continuo: se $\phi
_{j}\rightarrow ^{\mathcal{S}\left( 
%TCIMACRO{\U{211d} }%
%BeginExpansion
\mathbb{R}
%EndExpansion
\right) }\phi $, allora $\lim_{j\rightarrow +\infty }u_{f}\left( \phi
_{j}\right) =\lim_{j\rightarrow +\infty }\int_{%
%TCIMACRO{\U{211d} }%
%BeginExpansion
\mathbb{R}
%EndExpansion
}\frac{f\left( x\right) }{\left( 1+\left\vert x\right\vert \right) ^{N}}%
\left( 1+\left\vert x\right\vert \right) ^{N}\phi _{j}\left( x\right)
dx=\int_{%
%TCIMACRO{\U{211d} }%
%BeginExpansion
\mathbb{R}
%EndExpansion
}\frac{f\left( x\right) }{\left( 1+\left\vert x\right\vert \right) ^{N}}%
\left( 1+\left\vert x\right\vert \right) ^{N}\phi \left( x\right) dx$: lo
scambio \`{e} lecito grazie alla convergenza uniforme di $\phi
_{j}\rightarrow \phi $. Allora si ha $\lim_{j\rightarrow +\infty
}u_{f}\left( \phi _{j}\right) =u_{f}\left( \phi \right) $. $\blacksquare $

\begin{enumerate}
\item Sia $n=1$. Se $f\in C^{1}\left( 
%TCIMACRO{\U{211d} }%
%BeginExpansion
\mathbb{R}
%EndExpansion
\right) \cap L^{\infty }\left( 
%TCIMACRO{\U{211d} }%
%BeginExpansion
\mathbb{R}
%EndExpansion
\right) $ e $\phi \in \mathcal{S}\left( 
%TCIMACRO{\U{211d} }%
%BeginExpansion
\mathbb{R}
%EndExpansion
\right) $, \`{e} naturale - come per le distribuzioni standard - voler
definire la derivata della distribuzione associata a $f$ come distribuzione
associata a $f^{\prime }$ $\int_{%
%TCIMACRO{\U{211d} }%
%BeginExpansion
\mathbb{R}
%EndExpansion
}f^{\prime }\phi =\lim_{M\rightarrow +\infty }\left( \left( f\phi \right)
\left( M\right) -\left( f\phi \right) \left( -M\right) \right) -\int_{%
%TCIMACRO{\U{211d} }%
%BeginExpansion
\mathbb{R}
%EndExpansion
}f\phi ^{\prime }=-\int_{%
%TCIMACRO{\U{211d} }%
%BeginExpansion
\mathbb{R}
%EndExpansion
}f\phi ^{\prime }$ perch\'{e} $f$ \`{e} limitata e $\phi $ \`{e}
infinitesima. Questo esempio suggerisce la seguente definizione.
\end{enumerate}

\textbf{Def} Data $u\in \mathcal{S}^{\prime }\left( 
%TCIMACRO{\U{211d} }%
%BeginExpansion
\mathbb{R}
%EndExpansion
\right) $, si definisce distribuzione derivata prima di $u$ $Du\left( \phi
\right) :=-u\left( \phi ^{\prime }\right) $. Dato $\alpha \in 
%TCIMACRO{\U{2115} }%
%BeginExpansion
\mathbb{N}
%EndExpansion
^{n}$, si definisce derivata di ordine $\alpha $ di $u$ $D^{\alpha }u\left(
\phi \right) :=\left( -1\right) ^{\left\vert \alpha \right\vert }u\left(
D^{\alpha }\phi \right) $.

$D^{\alpha }u\in \mathcal{S}^{\prime }\left( 
%TCIMACRO{\U{211d} }%
%BeginExpansion
\mathbb{R}
%EndExpansion
\right) $. Si pu\`{o} mostrare che se $u_{n}\rightarrow ^{\mathcal{S}%
^{\prime }\left( 
%TCIMACRO{\U{211d} }%
%BeginExpansion
\mathbb{R}
%EndExpansion
\right) }u$, allora $PD^{\alpha }u_{n}\rightarrow PD^{\alpha }u$ per ogni
polinomio algebrico, $\forall $ $\alpha \in 
%TCIMACRO{\U{2115} }%
%BeginExpansion
\mathbb{N}
%EndExpansion
^{n}$.

Come per le distribuzioni standard, si pu\`{o} definire il prodotto tra una
distribuzione e una funzione.

\textbf{Def} Data $u\in \mathcal{S}^{\prime }\left( 
%TCIMACRO{\U{211d} }%
%BeginExpansion
\mathbb{R}
%EndExpansion
^{n}\right) $ e $\psi \in \mathcal{S}\left( 
%TCIMACRO{\U{211d} }%
%BeginExpansion
\mathbb{R}
%EndExpansion
^{n}\right) $, si definisce la distribuzione $u\psi \left( \phi \right)
:=u\left( \psi \phi \right) $.

$\phi \psi \in \mathcal{S}\left( 
%TCIMACRO{\U{211d} }%
%BeginExpansion
\mathbb{R}
%EndExpansion
^{n}\right) $, quindi tale distribuzione \`{e} ben definita $\forall $ $\phi 
$.

\textbf{Prop 3.8 (regola di Leibniz)}%
\begin{gather*}
\text{Hp: }u\in \mathcal{S}^{\prime }\left( 
%TCIMACRO{\U{211d} }%
%BeginExpansion
\mathbb{R}
%EndExpansion
\right) ,\psi \in \mathcal{S}\left( 
%TCIMACRO{\U{211d} }%
%BeginExpansion
\mathbb{R}
%EndExpansion
\right) \\
\text{Ts: }D\left( \psi u\right) =\psi ^{\prime }u+\psi Du
\end{gather*}

Se $u\in \mathcal{S}^{\prime }\left( \Omega \right) $ e $P$ \`{e} un
polinomio algebrico allora $Pu$ \`{e} ben definita ed \`{e} in $\mathcal{S}%
^{\prime }\left( \Omega \right) $; continua a valere la regola di Leibniz.

\textbf{Teo 3.9 (densit\`{a} di }$\mathcal{D}\left( \Omega \right) $\textbf{%
\ in }$\mathcal{S}\left( \Omega \right) $\textbf{)}%
\begin{gather*}
\text{Hp}\text{: }\psi \in \mathcal{S}\left( 
%TCIMACRO{\U{211d} }%
%BeginExpansion
\mathbb{R}
%EndExpansion
^{n}\right) \\
\text{Ts}\text{: }\exists \text{ }\left\{ \phi _{n}\right\} _{n\in 
%TCIMACRO{\U{2115} }%
%BeginExpansion
\mathbb{N}
%EndExpansion
}\subseteq \mathcal{D}\left( 
%TCIMACRO{\U{211d} }%
%BeginExpansion
\mathbb{R}
%EndExpansion
^{n}\right) :\phi _{n}\rightarrow ^{\mathcal{S}\left( \Omega \right) }\psi
\end{gather*}

E' ben definita la convergenza in $\mathcal{S}\left( \Omega \right) $ per $%
\phi _{n}$ perch\'{e} tutte le $\phi _{n}$ sono in particolare elementi di $%
\mathcal{S}\left( \Omega \right) $. Il teorema afferma che ogni funzione a
decrescita rapida pu\`{o} essere approssimata con una successione di
funzioni $C^{\infty }$ a supporto compatto, cio\`{e} $\mathcal{D}\left(
\Omega \right) $ \`{e} \textit{denso} in $\mathcal{S}\left( \Omega \right) $.

Data $u\in \mathcal{D}^{\prime }\left( 
%TCIMACRO{\U{211d} }%
%BeginExpansion
\mathbb{R}
%EndExpansion
^{n}\right) $, essa pu\`{o} essere estesa a $v\in \mathcal{S}^{\prime
}\left( 
%TCIMACRO{\U{211d} }%
%BeginExpansion
\mathbb{R}
%EndExpansion
^{n}\right) $ se e solo se $u$ \`{e} continua in $\mathcal{S}\left( 
%TCIMACRO{\U{211d} }%
%BeginExpansion
\mathbb{R}
%EndExpansion
^{n}\right) \backslash \mathcal{D}\left( 
%TCIMACRO{\U{211d} }%
%BeginExpansion
\mathbb{R}
%EndExpansion
^{n}\right) $, che \`{e} equivalente alla continuit\`{a} nella funzione
nulla, cio\`{e} $u\left( v_{n}\right) \rightarrow 0$ $\forall $ $\left\{
v_{n}\right\} \subseteq \mathcal{D}\left( \Omega \right) :v_{n}\rightarrow ^{%
\mathcal{S}\left( 
%TCIMACRO{\U{211d} }%
%BeginExpansion
\mathbb{R}
%EndExpansion
^{n}\right) }0$.\footnote{%
vedi inizio paragrafo sulle derivate distribuzionali
\par
per\`{o} il teo riguarda spazi di banach e lo spazio di schwarz non \`{e} di
banach, quindi c'\`{e} qualcosa da aggiustare} Da questo, per densit\`{a},
si deduce che la propriet\`{a} vale $\forall $ $\left\{ v_{n}\right\}
\subseteq \mathcal{S}\left( \Omega \right) :v_{n}\rightarrow ^{\mathcal{S}%
\left( 
%TCIMACRO{\U{211d} }%
%BeginExpansion
\mathbb{R}
%EndExpansion
^{n}\right) }0$.

\begin{enumerate}
\item $\delta \in \mathcal{D}^{\prime }\left( 
%TCIMACRO{\U{211d} }%
%BeginExpansion
\mathbb{R}
%EndExpansion
^{n}\right) $ pu\`{o} essere estesa a $\mathcal{S}^{\prime }\left( 
%TCIMACRO{\U{211d} }%
%BeginExpansion
\mathbb{R}
%EndExpansion
^{n}\right) $. Infatti se $\left\{ \phi _{n}\right\} \subseteq \mathcal{D}%
\left( \Omega \right) :\phi _{n}\rightarrow ^{\mathcal{S}\left( 
%TCIMACRO{\U{211d} }%
%BeginExpansion
\mathbb{R}
%EndExpansion
^{n}\right) }0$, allora c'\`{e} anche convergenza puntuale e $\phi
_{n}\left( 0\right) \rightarrow 0$.
\end{enumerate}

\textbf{Problemi di divisione} Come si \`{e} definito il prodotto, si pu\`{o}
cercare di definire il rapporto tra due distribuzioni.

Se $u\in \mathcal{S}^{\prime }\left( 
%TCIMACRO{\U{211d} }%
%BeginExpansion
\mathbb{R}
%EndExpansion
\right) $ e $f\in C^{\infty }\left( 
%TCIMACRO{\U{211d} }%
%BeginExpansion
\mathbb{R}
%EndExpansion
\right) $, $\exists $ $t\in \mathcal{D}^{\prime }\left( 
%TCIMACRO{\U{211d} }%
%BeginExpansion
\mathbb{R}
%EndExpansion
\right) :ft=u$ in $\mathcal{D}^{\prime }\left( 
%TCIMACRO{\U{211d} }%
%BeginExpansion
\mathbb{R}
%EndExpansion
\right) $? Una soluzione ovvia \`{e} $t\left( \phi \right) =\frac{1}{f}%
u\left( \phi \right) $, ma essa non \`{e} ben definita quando $f$ \`{e}
nulla.

\begin{enumerate}
\item Siano $f\left( x\right) =x,g\left( x\right) =1$. $\exists $ $t\in 
\mathcal{D}^{\prime }\left( 
%TCIMACRO{\U{211d} }%
%BeginExpansion
\mathbb{R}
%EndExpansion
\right) :xt=u_{g}$? $\frac{1}{x}u_{g}$ non \`{e} ben definita.

Si procede allora come per le equazioni differenziali ordinarie: si cerca
una soluzione $T$ dell'equazione omogenea $xt=0_{D}$ con $h\left( x\right)
=0 $ ($0_{D}$ indica la distribuzione nulla, elemento neutro della somma in $%
\mathcal{S}^{\prime }\left( 
%TCIMACRO{\U{211d} }%
%BeginExpansion
\mathbb{R}
%EndExpansion
\right) $) e le si somma una soluzione particolare. $t\left( \phi \right)
=\alpha \delta _{0}\left( \phi \right) $ $\forall $ $\alpha \in 
%TCIMACRO{\U{211d} }%
%BeginExpansion
\mathbb{R}
%EndExpansion
$ \`{e} una soluzione dell'equazione omogenea. Vale il seguente risultato pi%
\`{u} generale.
\end{enumerate}

\textbf{Prop 3.10 (soluzione di equazioni distribuzionali omogenee)}%
\begin{gather*}
\text{Hp}\text{: }t\in \mathcal{D}^{\prime }\left( 
%TCIMACRO{\U{211d} }%
%BeginExpansion
\mathbb{R}
%EndExpansion
\right) \text{; il polinomio }P\text{ ha gli zeri }x_{1},...,x_{m}\text{, di
molteplicit\`{a} algebrica rispettivamente }a_{1},...,a_{m} \\
\text{Ts}\text{: }t=\sum_{i=1}^{m}\sum_{j=0}^{a_{i}-1}c_{j}\delta
_{x_{i}}^{\left( j\right) }\text{ \`{e} tale che }Pt=0_{D}
\end{gather*}

Infatti $P\left( x\right) $ \`{e}, a meno di una costante, $\left(
x-x_{1}\right) ^{a_{1}}...\left( x-x_{m}\right) ^{a_{m}}$, quindi $\left(
P\left( x\right) t\right) \left( \phi \right) =t\left( P\left( x\right) \phi
\right) =\sum_{i=1}^{m}\sum_{j=0}^{a_{i}-1}c_{j}\delta _{x_{i}}^{\left(
j\right) }\left( P\left( x\right) \phi \left( x\right) \right)
=\sum_{i=1}^{m}\sum_{j=0}^{a_{i}-1}c_{j}\left( -1\right) ^{j}\left( \frac{%
d^{j}}{dx^{j}}\left( P\left( x\right) \phi \left( x\right) \right) \right)
|_{x=x_{i}}=0$ perch\'{e} $\left( \frac{d^{j}}{dx^{j}}\left( P\left(
x\right) \phi \left( x\right) \right) \right) |_{x=x_{i}}=0$ $\forall $ $%
j=0,...,a_{i}-1$, dato che $a_{i}$ \`{e} l'ordine dello zero $x_{i}$.

Nel caso particolare di $m=1$ si ha che $xt=0_{D}$ \`{e} risolta da $%
t=c_{0}\delta _{0}$. Si vedr\`{a} con la trasformata di Fourier che nel
dominio della trasformata la ricerca di primitive si trasforma proprio in un
problema di divisione.

\textbf{Prop 3.11 (soluzione di equazioni distribuzionali)}%
\begin{gather*}
\text{Hp}\text{: }u\in \mathcal{S}^{\prime }\left( 
%TCIMACRO{\U{211d} }%
%BeginExpansion
\mathbb{R}
%EndExpansion
\right) ,f\in C^{\infty }\left( 
%TCIMACRO{\U{211d} }%
%BeginExpansion
\mathbb{R}
%EndExpansion
\right) \\
\text{Ts}\text{: }ft=u\text{ ha come soluzione generale }t=t_{0}+t_{p}\text{%
, } \\
\text{con }t_{0}:ft_{0}=0_{D}\text{ e }T_{p}:ft_{p}=u
\end{gather*}

\begin{enumerate}
\item Si cerca una soluzione particolare dell'equazione sopra. $T\left( \phi
\right) =\frac{1}{x}u_{g}\left( \phi \right) $ non \`{e} ben definita, ma
invece di usare $\frac{1}{x}$ si pu\`{o} ricorrere al $\left( vp\int_{%
%TCIMACRO{\U{211d} }%
%BeginExpansion
\mathbb{R}
%EndExpansion
}\frac{1}{x}dx\right) \left( \phi \right) =\lim_{\varepsilon \rightarrow
0}\int_{\left\vert x\right\vert >\varepsilon }\frac{\phi \left( x\right) }{x}%
dx$. Infatti $\left( fT\right) \left( \phi \right) =T\left( f\phi \right)
=\lim_{\varepsilon \rightarrow 0}\int_{\left\vert x\right\vert >\varepsilon
}\phi \left( x\right) dx=\int_{%
%TCIMACRO{\U{211d} }%
%BeginExpansion
\mathbb{R}
%EndExpansion
}\phi \left( x\right) dx=u_{g}\left( \phi \right) $.
\end{enumerate}

\textbf{Def} Data una funzione $w\in \mathcal{S}\left( 
%TCIMACRO{\U{211d} }%
%BeginExpansion
\mathbb{R}
%EndExpansion
^{n}\right) $ e una distribuzione $u\in \mathcal{S}^{\prime }\left( 
%TCIMACRO{\U{211d} }%
%BeginExpansion
\mathbb{R}
%EndExpansion
^{n}\right) $, si definisce prodotto di convoluzione di $u$ e $w=w\left( 
\mathbf{y}\right) $ la funzione $\left( u\ast w\right) \left( \mathbf{x}%
\right) =u\left( w\left( \mathbf{x-y}\right) \right) $.

Si pu\`{o} mostrare che $u\ast w$ cos\`{\i} definito \`{e} una funzione $%
C^{\infty }$ e che definisce una distribuzione temperata.

\begin{enumerate}
\item Nel caso particolare in cui $u_{f}$ \`{e} una distribuzione temperata
associata a $f$, $\left( u_{f}\ast w\right) \left( x\right) =\int_{%
%TCIMACRO{\U{211d} }%
%BeginExpansion
\mathbb{R}
%EndExpansion
}f\left( y\right) w\left( x-y\right) dy=\left( f\ast w\right) \left(
x\right) $: il prodotto di convoluzione tra $w$ e $u_{f}$ coincide con $%
w\ast f$.

\item Se $u=\delta _{0}$ e $w\in \mathcal{S}\left( 
%TCIMACRO{\U{211d} }%
%BeginExpansion
\mathbb{R}
%EndExpansion
^{n}\right) $ \`{e} qualsiasi, $\left( u\ast w\right) \left( x\right)
=u\left( w\left( x-y\right) \right) =w\left( x-y\right) |_{y=0}=w\left(
x\right) $: la delta \`{e} l'elemento neutro del prodotto di convoluzione.

\item Se $u=\delta _{0}^{\prime }$, $\left( u\ast w\right) \left( x\right)
=u\left( w\left( x-y\right) \right) =-\left[ \frac{d}{dy}w\left( x-y\right) %
\right] |_{y=0}=-\left[ -\frac{d}{dy}w\left( x-y\right) \right] |_{y=0}$.
\end{enumerate}

\section{Trasformata di Fourier}

\textbf{Def} Data $f\in L^{1}\left( 
%TCIMACRO{\U{211d} }%
%BeginExpansion
\mathbb{R}
%EndExpansion
^{n}\right) $, si dice trasformata di Fourier di $f$ la funzione $\hat{f}:%
%TCIMACRO{\U{211d} }%
%BeginExpansion
\mathbb{R}
%EndExpansion
^{n}\rightarrow 
%TCIMACRO{\U{2102} }%
%BeginExpansion
\mathbb{C}
%EndExpansion
,\hat{f}\left( \xi \right) :=\int_{%
%TCIMACRO{\U{211d} }%
%BeginExpansion
\mathbb{R}
%EndExpansion
^{n}}e^{-i\left\langle \mathbf{x,\xi }\right\rangle }f\left( \mathbf{x}%
\right) d\mathbf{x}$.

La seguente proposizione mostra che $\hat{f}$ \`{e} ben definita, e non
solo: \`{e} molto regolare.

\textbf{Prop 4.1 (regolarit\`{a} della trasformata)}%
\begin{eqnarray*}
\text{Hp}\text{: } &&f\in L^{1}\left( 
%TCIMACRO{\U{211d} }%
%BeginExpansion
\mathbb{R}
%EndExpansion
^{n}\right) \\
\text{Ts}\text{: } &&\hat{f}\in L^{\infty }\left( 
%TCIMACRO{\U{211d} }%
%BeginExpansion
\mathbb{R}
%EndExpansion
^{n}\right) \cap C^{0}\left( 
%TCIMACRO{\U{211d} }%
%BeginExpansion
\mathbb{R}
%EndExpansion
^{n}\right)
\end{eqnarray*}

\textbf{Dim} Per definizione $\forall $ $\xi $ $\left\vert \hat{f}\left( \xi
\right) \right\vert \leq \int_{%
%TCIMACRO{\U{211d} }%
%BeginExpansion
\mathbb{R}
%EndExpansion
^{n}}\left\vert e^{-i\left\langle \mathbf{x,\xi }\right\rangle }f\left( 
\mathbf{x}\right) \right\vert d\mathbf{x\leq }\int_{%
%TCIMACRO{\U{211d} }%
%BeginExpansion
\mathbb{R}
%EndExpansion
^{n}}\left\vert f\left( \mathbf{x}\right) \right\vert d\mathbf{x\in 
%TCIMACRO{\U{211d} }%
%BeginExpansion
\mathbb{R}
%EndExpansion
}$ q.o. in $%
%TCIMACRO{\U{211d} }%
%BeginExpansion
\mathbb{R}
%EndExpansion
^{n}$ per ipotesi, dunque $\hat{f}$ \`{e} limitata e $\left\vert \left\vert 
\hat{f}\right\vert \right\vert _{L^{\infty }}\leq \left\vert \left\vert
f\right\vert \right\vert _{L^{1}}$.

Inoltre, se $\left\{ \xi _{n}\right\} _{n\in 
%TCIMACRO{\U{2115} }%
%BeginExpansion
\mathbb{N}
%EndExpansion
}\subseteq 
%TCIMACRO{\U{211d} }%
%BeginExpansion
\mathbb{R}
%EndExpansion
^{n}$ e $\xi _{n}\rightarrow \xi \in 
%TCIMACRO{\U{211d} }%
%BeginExpansion
\mathbb{R}
%EndExpansion
^{n}$ (secondo una qualsiasi norma di $%
%TCIMACRO{\U{211d} }%
%BeginExpansion
\mathbb{R}
%EndExpansion
^{n}$), allora per continuit\`{a} dell'esponenziale complesso $%
e^{-i\left\langle \mathbf{x,\xi }_{n}\right\rangle }f\left( \mathbf{x}%
\right) \rightarrow e^{-i\left\langle \mathbf{x,\xi }\right\rangle }f\left( 
\mathbf{x}\right) $. Quindi, per il teorema di convergenza dominata (essendo 
$\psi _{n}\left( \mathbf{x}\right) =\left( e^{-i\left\langle \mathbf{x,\xi }%
_{n}\right\rangle }-e^{-i\left\langle \mathbf{x,\xi }\right\rangle }\right)
f\left( \mathbf{x}\right) $ tale che $\left\vert \psi _{n}\left( \mathbf{x}%
\right) \right\vert \leq 2\left\vert f\left( \mathbf{x}\right) \right\vert $ 
$\forall $ $\mathbf{x},\forall $ $n$), $\lim_{n\rightarrow +\infty }\int_{%
%TCIMACRO{\U{211d} }%
%BeginExpansion
\mathbb{R}
%EndExpansion
^{n}}\left\vert \left( e^{-i\left\langle \mathbf{x,\xi }_{n}\right\rangle
}-e^{-i\left\langle \mathbf{x,\xi }\right\rangle }\right) f\left( \mathbf{x}%
\right) \right\vert d\mathbf{x=}\int_{%
%TCIMACRO{\U{211d} }%
%BeginExpansion
\mathbb{R}
%EndExpansion
^{n}}\lim_{n\rightarrow +\infty }\left( e^{-i\left\langle \mathbf{x,\xi }%
_{n}\right\rangle }-e^{-i\left\langle \mathbf{x,\xi }\right\rangle }\right)
f\left( \mathbf{x}\right) d\mathbf{x}=0$, quindi $\hat{f}\left( \xi
_{n}\right) \rightarrow ^{n\rightarrow +\infty }\hat{f}\left( \xi \right) $. 
$\blacksquare $

\textbf{Lemma 4.2 (Riemann-Lebesgue)}%
\begin{eqnarray*}
\text{Hp}\text{: } &&f\in L^{1}\left( 
%TCIMACRO{\U{211d} }%
%BeginExpansion
\mathbb{R}
%EndExpansion
^{n}\right) \\
\text{Ts}\text{: } &&\lim_{\left\vert \left\vert \xi \right\vert \right\vert
\rightarrow +\infty }\hat{f}\left( \mathbf{\xi }\right) =0
\end{eqnarray*}

E' l'analogo del risultato gi\`{a} visto per i coefficienti della serie di
Fourier.

\textbf{Dim} Sia $n=1$ per semplicit\`{a}. Dati $a<b$ in $%
%TCIMACRO{\U{211d} }%
%BeginExpansion
\mathbb{R}
%EndExpansion
$, vale $\chi _{\left[ a,b\right] }\in L^{1}\left( 
%TCIMACRO{\U{211d} }%
%BeginExpansion
\mathbb{R}
%EndExpansion
\right) $, e inoltre $\hat{\chi}_{\left[ a,b\right] }\left( \xi \right)
=\int_{%
%TCIMACRO{\U{211d} }%
%BeginExpansion
\mathbb{R}
%EndExpansion
}e^{-ix\xi }\chi _{\left[ a,b\right] }dx=\int_{a}^{b}e^{-ix\xi }dx=\frac{%
e^{-ia\xi }-e^{-ib\xi }}{i\xi }$. Poich\'{e} $\left\vert \hat{\chi}_{\left[
a,b\right] }\left( \xi \right) \right\vert \leq \frac{2}{i\xi }\rightarrow
^{\left\vert \xi \right\vert \rightarrow +\infty }0$, la tesi vale per
funzioni del tipo $\chi _{\left[ a,b\right] }$, e dunque anche per qualsiasi
funzione semplice grazie alla linearit\`{a} dell'integrale.

Sia ora $\varepsilon >0$ e $f\in L^{1}\left( 
%TCIMACRO{\U{211d} }%
%BeginExpansion
\mathbb{R}
%EndExpansion
\right) $ qualsiasi. Allora, per definizione di funzione integrabile secondo
Lebesgue, $\exists $ $s$ funzione semplice tale che $\left\vert \left\vert
f-s\right\vert \right\vert _{L^{1}\left( 
%TCIMACRO{\U{211d} }%
%BeginExpansion
\mathbb{R}
%EndExpansion
\right) }<\varepsilon $. Dunque $\left\vert \hat{f}\left( \xi \right)
\right\vert =\left\vert \hat{f}\left( \xi \right) -\hat{s}\left( \xi \right)
+\hat{s}\left( \xi \right) \right\vert \leq \left\vert \hat{f}\left( \xi
\right) -\hat{s}\left( \xi \right) \right\vert +\left\vert \hat{s}\left( \xi
\right) \right\vert <\varepsilon +\left\vert \hat{s}\left( \xi \right)
\right\vert $. Ma $s$ \`{e} una funzione semplice, quindi $\exists $ $%
M>0:\left\vert \left\vert \xi \right\vert \right\vert >M$ implica $%
\left\vert \hat{s}\left( \xi \right) \right\vert <\varepsilon $, dunque $%
\left\vert \hat{f}\left( \xi \right) \right\vert <2\varepsilon $ e si ha la
tesi. $\blacksquare $

Pi\`{u} in astratto, l'applicazione $\mathcal{F}:L^{1}\left( 
%TCIMACRO{\U{211d} }%
%BeginExpansion
\mathbb{R}
%EndExpansion
^{n}\right) \rightarrow L^{\infty }\left( 
%TCIMACRO{\U{211d} }%
%BeginExpansion
\mathbb{R}
%EndExpansion
^{n}\right) \cap C^{0}\left( 
%TCIMACRO{\U{211d} }%
%BeginExpansion
\mathbb{R}
%EndExpansion
^{n}\right) ,\mathcal{F}\left( f\right) =\int_{%
%TCIMACRO{\U{211d} }%
%BeginExpansion
\mathbb{R}
%EndExpansion
^{n}}e^{-i\left\langle \mathbf{x,\xi }\right\rangle }f\left( \mathbf{x}%
\right) d\mathbf{x}$ \`{e} lineare e vale $\forall $ $f,g\in L^{1}$ $%
\left\vert \left\vert \mathcal{F}\left( f\right) -\mathcal{F}\left( g\right)
\right\vert \right\vert _{L^{\infty }\left( 
%TCIMACRO{\U{211d} }%
%BeginExpansion
\mathbb{R}
%EndExpansion
^{n}\right) }\leq \left\vert \left\vert f-g\right\vert \right\vert
_{L^{1}\left( 
%TCIMACRO{\U{211d} }%
%BeginExpansion
\mathbb{R}
%EndExpansion
^{n}\right) }$, dunque \`{e} anche continua secondo le convergenze in norma
di dominio e codominio. Inoltre $\func{Im}\mathcal{F}=\left\{ h:%
%TCIMACRO{\U{211d} }%
%BeginExpansion
\mathbb{R}
%EndExpansion
^{n}\rightarrow 
%TCIMACRO{\U{211d} }%
%BeginExpansion
\mathbb{R}
%EndExpansion
\text{ continue}:h\left( \xi \right) \rightarrow 0\text{ per }\left\vert
\left\vert \xi \right\vert \right\vert \rightarrow +\infty \right\} $.
Talvolta come codominio dell'operatore trasformata di usa $C_{\ast
}^{0}\left( 
%TCIMACRO{\U{211d} }%
%BeginExpansion
\mathbb{R}
%EndExpansion
\right) :=\left\{ f:%
%TCIMACRO{\U{211d} }%
%BeginExpansion
\mathbb{R}
%EndExpansion
\rightarrow 
%TCIMACRO{\U{211d} }%
%BeginExpansion
\mathbb{R}
%EndExpansion
\text{ continue tali che }\lim_{\left\vert x\right\vert \rightarrow +\infty
}f\left( x\right) =0\right\} $ (l'appartenenza al quale implica la
limitatezza).

E' ora naturale chiedersi quale sia l'effetto della trasformata di Fourier
su una funzione soggetta a operazioni consuete (la traslazione, la
derivazione,...).

\textbf{Teo 4.3} (\textbf{propriet\`{a} algebriche della trasformata di
Fourier)}%
\begin{gather*}
\text{Hp: }f\in L^{1}\left( 
%TCIMACRO{\U{211d} }%
%BeginExpansion
\mathbb{R}
%EndExpansion
^{n}\right) ,a\in 
%TCIMACRO{\U{211d} }%
%BeginExpansion
\mathbb{R}
%EndExpansion
_{0},\mathbf{x}_{0}\in 
%TCIMACRO{\U{211d} }%
%BeginExpansion
\mathbb{R}
%EndExpansion
^{n},A\in M_{%
%TCIMACRO{\U{211d} }%
%BeginExpansion
\mathbb{R}
%EndExpansion
}\left( m,n\right) ,\det A\neq 0 \\
\text{Ts: (i) }\mathcal{F}\left( f\left( a\mathbf{x}\right) \right) =\frac{1%
}{\left\vert a\right\vert }\hat{f}\left( \frac{\mathbf{\xi }}{a}\right) \\
\text{(ii) }\mathcal{F}\left( f\left( \mathbf{x-x}_{0}\right) \right)
=e^{-i\left\langle \mathbf{x}_{0}\mathbf{,\xi }\right\rangle }\mathcal{F}%
\left( f\right) \\
\text{(iii) }\mathcal{F}\left( e^{i\left\langle \mathbf{x,x}%
_{0}\right\rangle }f\left( \mathbf{x}\right) \right) =\hat{f}\left( \mathbf{%
\xi -x}_{0}\right) \\
\text{(iv) }\mathcal{F}\left( f\left( A^{-1}\mathbf{x}\right) \right)
=\left\vert \det A\right\vert \hat{f}\left( A^{t}\mathbf{\xi }\right) \\
\text{(v) }\mathcal{F}\left( \bar{f}\left( \mathbf{x}\right) \right) =\left( 
\hat{f}\left( -\mathbf{\xi }\right) \right) ^{\ast }
\end{gather*}

Dunque $\mathcal{F}$ trasforma una traslazione di $f$ in un prodotto per un
esponenziale, e viceversa; trasforma un cambiamento lineare di coordinate in
un altro cambiamento lineare di coordinate (a meno di una costante). Il
coniugio \`{e} lasciato inalterato da $\mathcal{F}$, a meno del segno
dell'argomento della trasformata.

\textbf{Dim }(ii) $\mathcal{F}\left( f\left( \mathbf{x-x}_{0}\right) \right)
=\int_{%
%TCIMACRO{\U{211d} }%
%BeginExpansion
\mathbb{R}
%EndExpansion
^{n}}e^{-i\left\langle \mathbf{x,\xi }\right\rangle }f\left( \mathbf{x-x}%
_{0}\right) d\mathbf{x}=\int_{%
%TCIMACRO{\U{211d} }%
%BeginExpansion
\mathbb{R}
%EndExpansion
^{n}}e^{-i\left\langle \mathbf{y+x}_{0}\mathbf{,\xi }\right\rangle }f\left( 
\mathbf{y}\right) d\mathbf{y}=e^{-i\left\langle \mathbf{x}_{0}\mathbf{,\xi }%
\right\rangle }\int_{%
%TCIMACRO{\U{211d} }%
%BeginExpansion
\mathbb{R}
%EndExpansion
^{n}}e^{-i\left\langle \mathbf{y,\xi }\right\rangle }f\left( \mathbf{y}%
\right) d\mathbf{y}$, grazie alla sostituzione $\mathbf{y=x-x}_{0}$.

(iii) $\mathcal{F}\left( e^{i\left\langle \mathbf{x,x}_{0}\right\rangle
}f\left( \mathbf{x}\right) \right) =\int_{%
%TCIMACRO{\U{211d} }%
%BeginExpansion
\mathbb{R}
%EndExpansion
^{n}}e^{-i\left\langle \mathbf{x,\xi }\right\rangle }e^{i\left\langle 
\mathbf{x,x}_{0}\right\rangle }f\left( \mathbf{x}\right) d\mathbf{x=}\int_{%
%TCIMACRO{\U{211d} }%
%BeginExpansion
\mathbb{R}
%EndExpansion
^{n}}e^{-i\left\langle \mathbf{x,\xi -x}_{0}\right\rangle }f\left( \mathbf{x}%
\right) d\mathbf{x=}\hat{f}\left( \mathbf{\xi -x}_{0}\right) $.

(iv) $\mathcal{F}\left( f\left( A^{-1}\mathbf{x}\right) \right) =\int_{%
%TCIMACRO{\U{211d} }%
%BeginExpansion
\mathbb{R}
%EndExpansion
^{n}}e^{-i\left\langle \mathbf{x,\xi }\right\rangle }f\left( A^{-1}\mathbf{x}%
\right) d\mathbf{x=}\int_{%
%TCIMACRO{\U{211d} }%
%BeginExpansion
\mathbb{R}
%EndExpansion
^{n}}\left\vert \det A\right\vert e^{-i\left\langle A\mathbf{y,\xi }%
\right\rangle }f\left( \mathbf{y}\right) d\mathbf{y}$, grazie alla
sostituzione $\mathbf{y}=A^{-1}\mathbf{x}$ e al teorema di cambio di
variabile negli integrali. Si ha la tesi perch\'{e} $\left\langle A\mathbf{%
y,\xi }\right\rangle =\left\langle \mathbf{y,}A^{t}\mathbf{\xi }%
\right\rangle $.

(i) \`{e} una conseguenza di (iv) nel caso particolare $A=\left[ \frac{1}{a}%
\right] $.

(v) Posto $f=f_{1}+if_{2}$, il lato destro \`{e} $\left( \hat{f}\left( -%
\mathbf{\xi }\right) \right) ^{\ast }=\left( \int_{%
%TCIMACRO{\U{211d} }%
%BeginExpansion
\mathbb{R}
%EndExpansion
^{n}}e^{i\left\langle \mathbf{x,\xi }\right\rangle }\left( f_{1}\left( 
\mathbf{x}\right) +if_{2}\left( \mathbf{x}\right) \right) d\mathbf{x}\right)
^{\ast }$: spezzando l'esponenziale complesso e dividendo l'integrale in
parte reale e immaginaria si ottiene $\left( \hat{f}\left( -\mathbf{\xi }%
\right) \right) ^{\ast }=\left( \int_{%
%TCIMACRO{\U{211d} }%
%BeginExpansion
\mathbb{R}
%EndExpansion
^{n}}\left( \cos \left\langle \mathbf{x,\xi }\right\rangle f_{1}\left( 
\mathbf{x}\right) -\sin \left\langle \mathbf{x,\xi }\right\rangle
f_{2}\left( \mathbf{x}\right) \right) d\mathbf{x+}i\int_{%
%TCIMACRO{\U{211d} }%
%BeginExpansion
\mathbb{R}
%EndExpansion
^{n}}\left( \cos \left\langle \mathbf{x,\xi }\right\rangle f_{2}\left( 
\mathbf{x}\right) +\sin \left\langle \mathbf{x,\xi }\right\rangle
f_{1}\left( \mathbf{x}\right) \right) d\mathbf{x}\right) ^{\ast }=$ $\int_{%
%TCIMACRO{\U{211d} }%
%BeginExpansion
\mathbb{R}
%EndExpansion
^{n}}\left( \cos \left\langle \mathbf{x,\xi }\right\rangle f_{1}\left( 
\mathbf{x}\right) -\sin \left\langle \mathbf{x,\xi }\right\rangle
f_{2}\left( \mathbf{x}\right) \right) d\mathbf{x-}i\int_{%
%TCIMACRO{\U{211d} }%
%BeginExpansion
\mathbb{R}
%EndExpansion
^{n}}\left( \cos \left\langle \mathbf{x,\xi }\right\rangle f_{2}\left( 
\mathbf{x}\right) +\sin \left\langle \mathbf{x,\xi }\right\rangle
f_{1}\left( \mathbf{x}\right) \right) d\mathbf{x=}\int_{%
%TCIMACRO{\U{211d} }%
%BeginExpansion
\mathbb{R}
%EndExpansion
^{n}}e^{-i\left\langle \mathbf{x,\xi }\right\rangle }\left( f_{1}\left( 
\mathbf{x}\right) -if_{2}\left( \mathbf{x}\right) \right) d\mathbf{x}$, che 
\`{e} proprio $\mathcal{F}\left( \bar{f}\right) $. $\blacksquare $

\textbf{Teo 4.4} (\textbf{simmetria della trasformata di Fourier)}
\begin{gather*}
\text{Hp: }f\in L^{1}\left( 
%TCIMACRO{\U{211d} }%
%BeginExpansion
\mathbb{R}
%EndExpansion
^{n}\right) \\
\text{Ts: (i) se }f\text{ \`{e} radiale, }\hat{f}\text{ \`{e} radiale} \\
\text{(ii) se }f\text{ \`{e} pari (dispari), }\hat{f}\text{ \`{e} pari
(dispari)} \\
\text{(iii) se }f\text{ \`{e} reale pari, }\hat{f}\text{ \`{e} reale pari} \\
\text{(iv) se }f\text{ \`{e} reale dispari, }\hat{f}\text{ \`{e} immaginaria
pura dispari}
\end{gather*}

\textbf{Dim} (i) Poich\'{e} $f$ \`{e} radiale, per ogni matrice ortogonale $%
R $ vale $f\left( R\mathbf{x}\right) =f\left( \mathbf{x}\right) $. Allora la
sua trasformata ha la stessa propriet\`{a}: $\hat{f}\left( R\mathbf{\xi }%
\right) =\int_{%
%TCIMACRO{\U{211d} }%
%BeginExpansion
\mathbb{R}
%EndExpansion
^{n}}e^{-i\left\langle \mathbf{x,}R\mathbf{\xi }\right\rangle }f\left( 
\mathbf{x}\right) d\mathbf{x=}\int_{%
%TCIMACRO{\U{211d} }%
%BeginExpansion
\mathbb{R}
%EndExpansion
^{n}}e^{-i\left\langle R\mathbf{y,}R\mathbf{\xi }\right\rangle }f\left( R%
\mathbf{y}\right) d\mathbf{y=}\int_{%
%TCIMACRO{\U{211d} }%
%BeginExpansion
\mathbb{R}
%EndExpansion
^{n}}e^{-i\left\langle \mathbf{y,\xi }\right\rangle }f\left( \mathbf{y}%
\right) d\mathbf{y=}\hat{f}\left( \mathbf{\xi }\right) $, grazie alla
sostituzione $\mathbf{x}=R\mathbf{y}$ e alla propriet\`{a} delle matrici
ortogonali per cui $R^{t}R=Id$. $\blacksquare $

\textbf{Teo 4.5 }(\textbf{trasformata della convoluzione)}

\begin{gather*}
\text{Hp: }f,g\in L^{1}\left( 
%TCIMACRO{\U{211d} }%
%BeginExpansion
\mathbb{R}
%EndExpansion
^{n}\right) \\
\text{Ts: }\mathcal{F}\left( f\ast g\right) =\mathcal{F}\left( f\right) 
\mathcal{F}\left( g\right)
\end{gather*}

Inoltre $f\ast g\in L^{1}$.

\textbf{Dim} Per definizione $\mathcal{F}\left( f\ast g\right) =\int_{%
%TCIMACRO{\U{211d} }%
%BeginExpansion
\mathbb{R}
%EndExpansion
^{n}}e^{-i\left\langle \mathbf{x,\xi }\right\rangle }\left( f\ast g\right)
\left( \mathbf{x}\right) d\mathbf{x=}\int_{%
%TCIMACRO{\U{211d} }%
%BeginExpansion
\mathbb{R}
%EndExpansion
^{n}}e^{-i\left\langle \mathbf{x,\xi }\right\rangle }\left( \int_{%
%TCIMACRO{\U{211d} }%
%BeginExpansion
\mathbb{R}
%EndExpansion
^{n}}f\left( \mathbf{x-y}\right) g\left( \mathbf{y}\right) d\mathbf{y}%
\right) d\mathbf{x=}\int_{%
%TCIMACRO{\U{211d} }%
%BeginExpansion
\mathbb{R}
%EndExpansion
^{n}}e^{-i\left\langle \mathbf{x-y,\xi }\right\rangle }e^{-i\left\langle 
\mathbf{y,\xi }\right\rangle }\left( \int_{%
%TCIMACRO{\U{211d} }%
%BeginExpansion
\mathbb{R}
%EndExpansion
^{n}}f\left( \mathbf{x-y}\right) g\left( \mathbf{y}\right) d\mathbf{y}%
\right) d\mathbf{x}$. Per il teorema di Fubini-Tonelli si pu\`{o} invertire
l'ordine di integrazione: $\mathcal{F}\left( f\ast g\right) =\int_{%
%TCIMACRO{\U{211d} }%
%BeginExpansion
\mathbb{R}
%EndExpansion
^{n}}e^{-i\left\langle \mathbf{y,\xi }\right\rangle }\left( \int_{%
%TCIMACRO{\U{211d} }%
%BeginExpansion
\mathbb{R}
%EndExpansion
^{n}}e^{-i\left\langle \mathbf{x-y,\xi }\right\rangle }f\left( \mathbf{x-y}%
\right) d\mathbf{x}\right) g\left( \mathbf{y}\right) d\mathbf{y}$, cio\`{e},
con la sostituzione $\mathbf{x-y=u}$, $\int_{%
%TCIMACRO{\U{211d} }%
%BeginExpansion
\mathbb{R}
%EndExpansion
^{n}}e^{-i\left\langle \mathbf{y,\xi }\right\rangle }\left( \int_{%
%TCIMACRO{\U{211d} }%
%BeginExpansion
\mathbb{R}
%EndExpansion
^{n}}e^{-i\left\langle \mathbf{u,\xi }\right\rangle }f\left( \mathbf{u}%
\right) d\mathbf{x}\right) g\left( \mathbf{y}\right) d\mathbf{y=}\mathcal{F}%
\left( f\right) \int_{%
%TCIMACRO{\U{211d} }%
%BeginExpansion
\mathbb{R}
%EndExpansion
^{n}}e^{-i\left\langle \mathbf{y,\xi }\right\rangle }g\left( \mathbf{y}%
\right) d\mathbf{y}$. $\blacksquare $

La propriet\`{a} pi\`{u} rilevante della trasformata di Fourier riguarda per%
\`{o} come essa agisce sulla derivata di una funzione.

\textbf{Teo} \textbf{4.6 }(\textbf{trasformata della derivata)}

\begin{gather*}
\text{Hp: }u\in C^{1}\left( 
%TCIMACRO{\U{211d} }%
%BeginExpansion
\mathbb{R}
%EndExpansion
^{n}\right) \cap L^{1}\left( 
%TCIMACRO{\U{211d} }%
%BeginExpansion
\mathbb{R}
%EndExpansion
^{n}\right) ,\frac{\partial u}{\partial x_{j}}\in L^{1}\left( 
%TCIMACRO{\U{211d} }%
%BeginExpansion
\mathbb{R}
%EndExpansion
^{n}\right) \\
\text{Ts: (i) }\mathcal{F}\left( \frac{\partial u}{\partial x_{j}}\right)
=i\xi _{j}\mathcal{F}\left( u\right) \\
\text{(ii) }\mathcal{F}\left( u\right) =o\left( \frac{1}{\xi _{j}}\right) 
\text{ per }\xi _{j}\rightarrow +\infty
\end{gather*}

Quindi la trasformata della derivata (ben definita perch\'{e} $\frac{%
\partial u}{\partial x_{j}}\in L^{1}\left( 
%TCIMACRO{\U{211d} }%
%BeginExpansion
\mathbb{R}
%EndExpansion
^{n}\right) $) \`{e} un opportuno monomio per la trasformata della funzione:
la derivazione diventa un prodotto. Si noti che ogni funzione nello spazio
di Schwartz soddisfa le ipotesi di questo teorema.

La dimostrazione usa l'integrazione per parti.

Se $n=1$, dal teorema segue che $\mathcal{F}\left( u\right) =o\left( \frac{1%
}{\left\vert \xi \right\vert }\right) $ per $\left\vert \xi \right\vert
\rightarrow +\infty $: infatti, poich\'{e} per il lemma di Riemann-Lebesgue
ogni trasformata tende a $0$ se $\left\vert \xi \right\vert \rightarrow
+\infty $, $i\xi \mathcal{F}\left( u\right) \rightarrow 0$ e allora $%
\mathcal{F}\left( u\right) $ deve tendere a $0$ pi\`{u} in fretta di $\frac{1%
}{\xi }$.

Se $\frac{\partial ^{2}u}{\partial x^{2}}\in L^{1}$, applicando il teorema
alla funzione $v=\frac{\partial u}{\partial x}$ si trova che $\mathcal{F}%
\left( \frac{\partial ^{2}u}{\partial x^{2}}\right) =i^{2}\xi ^{2}\mathcal{F}%
\left( u\right) $, per cui $\mathcal{F}\left( u\right) =o\left( \frac{1}{\xi
^{2}}\right) $; iterando il procedimento - sotto l'ipotesi di $\frac{%
\partial ^{k}u}{\partial x^{k}}\in L^{1}\left( 
%TCIMACRO{\U{211d} }%
%BeginExpansion
\mathbb{R}
%EndExpansion
\right) $ $\forall $ $k\in 
%TCIMACRO{\U{2115} }%
%BeginExpansion
\mathbb{N}
%EndExpansion
$ - si ha che $\mathcal{F}\left( u\right) =o\left( \frac{1}{\left\vert \xi
\right\vert ^{k}}\right) $ $\forall $ $k$ per $\left\vert \xi \right\vert
\rightarrow +\infty $. All'appartenenza a $L^{1}$ di un numero maggiore di
derivate di $u$ corrisponde dunque un pi\`{u} veloce annullamento di $%
\mathcal{F}\left( u\right) $.

Questo risultato si generalizza facilmente al caso di $n$ qualsiasi.%
\begin{gather*}
\text{Hp: }D^{\alpha }u\in L^{1}\left( 
%TCIMACRO{\U{211d} }%
%BeginExpansion
\mathbb{R}
%EndExpansion
^{n}\right) \text{ }\forall \text{ }\alpha \in 
%TCIMACRO{\U{2115} }%
%BeginExpansion
\mathbb{N}
%EndExpansion
^{n} \\
\text{Ts: (i) }\mathcal{F}\left( D^{\alpha }u\right) =i^{\left\vert \alpha
\right\vert }\mathbf{\xi }^{\alpha }\mathcal{F}\left( u\right) \\
\text{(ii) }\mathcal{F}\left( u\right) =o\left( \frac{1}{\left\vert
\left\vert \xi \right\vert \right\vert ^{k}}\right) \text{ per }\left\vert
\left\vert \xi \right\vert \right\vert \rightarrow +\infty ,\forall k
\end{gather*}

Con $\mathbf{\xi }^{\alpha }$ si indica, con abuso di notazione, $%
\prod_{j=1}^{n}\xi _{j}^{\alpha _{j}}$.

\textbf{Teo 4.7} (\textbf{derivata della trasformata)}
\begin{gather*}
\text{Hp: }u\in L^{1}\left( 
%TCIMACRO{\U{211d} }%
%BeginExpansion
\mathbb{R}
%EndExpansion
^{n}\right) ,x_{j}u\in L^{1}\left( 
%TCIMACRO{\U{211d} }%
%BeginExpansion
\mathbb{R}
%EndExpansion
^{n}\right) \\
\text{Ts: }\frac{\partial \mathcal{F}\left( u\right) }{\partial \xi _{j}}=-i%
\mathcal{F}\left( x_{j}u\right)
\end{gather*}

La derivata della trasformata \`{e}, a meno di una costante, la trasformata
della funzione moltiplicata per un monomio.

Dal teorema segue che $\frac{\partial \mathcal{F}}{\partial \xi _{j}}\in
C^{0}\left( 
%TCIMACRO{\U{211d} }%
%BeginExpansion
\mathbb{R}
%EndExpansion
^{n}\right) $ perch\'{e} ogni trasformata \`{e} continua; applicando il
teorema alla funzione $x_{j}u:x_{j}^{2}u\in L^{1}$ si ha che $\frac{\partial 
\mathcal{F}\left( x_{j}u\right) }{\partial \xi _{j}}=-i\mathcal{F}\left(
x_{j}^{2}u\right) $, ma $\mathcal{F}\left( x_{j}u\right) =i\frac{\partial 
\mathcal{F}\left( u\right) }{\partial \xi _{j}}$ e sostituendo si ha $i\frac{%
\partial ^{2}\mathcal{F}\left( u\right) }{\partial \xi _{j}^{2}}=-i\mathcal{F%
}\left( x_{j}^{2}u\right) $: dunque anche la derivata seconda di $\mathcal{F}%
\left( u\right) $ \`{e} continua. Iterando il procedimento - sotto l'ipotesi 
$u:Pu\in L^{1}\left( 
%TCIMACRO{\U{211d} }%
%BeginExpansion
\mathbb{R}
%EndExpansion
^{n}\right) $ per ogni polinomio algebrico $P$, in modo che siano ben
definite $\mathcal{F}\left( x_{j}u\right) ,\mathcal{F}\left(
x_{j}^{2}u\right) ,...$) - si ottiene che $\mathcal{F}\left( u\right) \in
C^{\infty }\left( 
%TCIMACRO{\U{211d} }%
%BeginExpansion
\mathbb{R}
%EndExpansion
^{n}\right) $.

Da questo stesso ragionamento si conclude che tutte le derivate di $\mathcal{%
F}\left( u\right) $ sono uguali, a meno di una costante, a una trasformata,
e perci\`{o} tendono a $0$ per $\left\vert \left\vert \xi \right\vert
\right\vert \rightarrow +\infty $.

Questo risultato si generalizza facilmente.%
\begin{gather*}
\text{Hp: }Pu\in L^{1}\left( 
%TCIMACRO{\U{211d} }%
%BeginExpansion
\mathbb{R}
%EndExpansion
^{n}\right) \text{ per ogni polinomio algebrico }P \\
\text{Ts: (i) }D^{\alpha }\mathcal{F}\left( u\right) =\left( -i\right)
^{\left\vert \alpha \right\vert }\mathcal{F}\left( \mathbf{x}^{\alpha
}u\right) \\
\text{(ii) }\mathcal{F}\left( u\right) \in C^{\infty }\left( 
%TCIMACRO{\U{211d} }%
%BeginExpansion
\mathbb{R}
%EndExpansion
^{n}\right)
\end{gather*}

L'operatore trasformata di Fourier ha una propriet\`{a} sgradevole: $%
\mathcal{F}\left( L^{1}\left( 
%TCIMACRO{\U{211d} }%
%BeginExpansion
\mathbb{R}
%EndExpansion
^{n}\right) \right) \not\subseteq L^{1}\left( 
%TCIMACRO{\U{211d} }%
%BeginExpansion
\mathbb{R}
%EndExpansion
^{n}\right) $. Questo rende difficile invertire l'operatore, cio\`{e}, data $%
\mathcal{F}\left( f\right) $, risalire a $f$.

\begin{enumerate}
\item Data $f\left( x\right) =\chi _{\left[ -1,1\right] }\left( x\right) $, $%
\mathcal{F}\left( f\right) =\int_{-1}^{1}e^{-i\xi x}dx=\left\{ 
\begin{array}{c}
2\frac{\sin \xi }{\xi }I_{\left( \xi \neq 0\right) }\left( \xi \right) \text{
se }\xi \neq 0 \\ 
2\text{ se }\xi =0%
\end{array}%
\right. $, che non \`{e} una funzione integrabile (pur essendo analitica).
\end{enumerate}

\textbf{Teo 4.8 (formula di inversione)} 
\begin{eqnarray*}
\text{Hp}\text{: } &&f\in L^{1}\left( 
%TCIMACRO{\U{211d} }%
%BeginExpansion
\mathbb{R}
%EndExpansion
^{n}\right) \cap C^{0}\left( 
%TCIMACRO{\U{211d} }%
%BeginExpansion
\mathbb{R}
%EndExpansion
^{n}\right) \text{, }\lim_{\left\vert \left\vert \mathbf{x}\right\vert
\right\vert \rightarrow +\infty }f\left( \mathbf{x}\right) =0\text{, }\hat{f}%
\in L^{1}\left( 
%TCIMACRO{\U{211d} }%
%BeginExpansion
\mathbb{R}
%EndExpansion
^{n}\right) \\
\text{Ts} &\text{:}&\text{ }f\left( \mathbf{x}\right) =\frac{1}{\left( 2\pi
\right) ^{n}}\int_{%
%TCIMACRO{\U{211d} }%
%BeginExpansion
\mathbb{R}
%EndExpansion
^{n}}e^{i\left\langle \mathbf{\xi ,x}\right\rangle }\hat{f}\left( \mathbf{%
\xi }\right) d\mathbf{\xi }\text{ }\forall \text{ }\mathbf{x}\in 
%TCIMACRO{\U{211d} }%
%BeginExpansion
\mathbb{R}
%EndExpansion
^{n}
\end{eqnarray*}

In tal caso la funzione definita dal lato destro della tesi si dice
antitrasformata di $\hat{f}$, e l'operatore che agisce in tal modo su $\hat{f%
}$ si indica con $\mathcal{F}^{-1}$: $\mathcal{F}^{-1}\left( g\right) =\frac{%
1}{\left( 2\pi \right) ^{n}}\int_{%
%TCIMACRO{\U{211d} }%
%BeginExpansion
\mathbb{R}
%EndExpansion
^{n}}e^{i\left\langle \mathbf{\xi ,x}\right\rangle }g\left( \mathbf{\xi }%
\right) d\mathbf{\xi }$. Si noti che tale operatore \`{e} ben definito se $%
g\in L^{1}$, quindi per scrivere che $f=\mathcal{F}^{-1}\left( \mathcal{F}%
\left( f\right) \right) $ \`{e} necessario sapere che $\mathcal{F}\left(
f\right) \in L^{1}$ (che \`{e} un'ipotesi restrittiva, come si \`{e} visto
nell'esempio sopra): quindi la trasformata di Fourier in $L^{1}$ \`{e} un
operatore in generale non invertibile.

E' importante notare che, se $\hat{f}=g\in L^{1}$, $f\left( \mathbf{x}%
\right) =\frac{1}{\left( 2\pi \right) ^{n}}\int_{%
%TCIMACRO{\U{211d} }%
%BeginExpansion
\mathbb{R}
%EndExpansion
^{n}}e^{i\left\langle \mathbf{\xi ,x}\right\rangle }\hat{f}\left( \mathbf{%
\xi }\right) d\mathbf{\xi }$ \`{e} legata alla trasformata di $\hat{f}$: si
ha che $\hat{g}\left( \mathbf{y}\right) =\int_{%
%TCIMACRO{\U{211d} }%
%BeginExpansion
\mathbb{R}
%EndExpansion
^{n}}e^{-i\left\langle \mathbf{\xi ,y}\right\rangle }g\left( \mathbf{\xi }%
\right) d\mathbf{\xi }=\left( 2\pi \right) ^{n}f\left( -\mathbf{y}\right) $,
cio\`{e}%
\begin{equation*}
\mathcal{F}\left( \mathcal{F}\left( f\right) \right) =\left( 2\pi \right)
^{n}f\left( -\mathbf{y}\right)
\end{equation*}

\begin{enumerate}
\item $f\left( x\right) =\frac{1}{x^{2}+1}\in L^{1}\cap C^{\infty }$ e $%
f^{\left( j\right) }\in L^{1}$ $\forall $ $j$. Allora per la conseguenza del
teorema 4.6 dev'essere $\mathcal{F}\left( f\right) =o\left( \frac{1}{%
\left\vert \xi \right\vert ^{k}}\right) $ per $\left\vert \xi \right\vert
\rightarrow +\infty $ $\forall $ $k$ (dal che segue $\mathcal{F}\left(
f\right) \in L^{1}$). Non vale invece $Pf\in L^{1}$ $\forall $ $P$ (neanche
per $P\left( x\right) =x$), quindi non si applica la conseguenza del teorema
4.7: non si pu\`{o} dire che $\mathcal{F}\left( f\right) \in C^{\infty }$.

$\mathcal{F}\left( f\right) =\int_{%
%TCIMACRO{\U{211d} }%
%BeginExpansion
\mathbb{R}
%EndExpansion
}e^{-ix\xi }\frac{1}{x^{2}+1}dx=\pi e^{-\left\vert \xi \right\vert }$ (con
il teorema dei residui): c'\`{e} un punto angoloso in $0$, quindi $f\not\in
C^{1}$.

$f$ soddisfa le ipotesi del teorema di inversione: allora $\frac{1}{x^{2}+1}=%
\frac{1}{2\pi }\int_{%
%TCIMACRO{\U{211d} }%
%BeginExpansion
\mathbb{R}
%EndExpansion
}e^{ix\xi }\pi e^{-\left\vert \xi \right\vert }d\xi =\frac{1}{2\pi }\int_{%
%TCIMACRO{\U{211d} }%
%BeginExpansion
\mathbb{R}
%EndExpansion
}e^{-ix\xi }\pi e^{-\left\vert \xi \right\vert }d\xi $ (l'uguaglianza vale
perch\'{e} $\int_{%
%TCIMACRO{\U{211d} }%
%BeginExpansion
\mathbb{R}
%EndExpansion
}e^{ix\xi }\pi e^{-\left\vert \xi \right\vert }d\xi =\int_{%
%TCIMACRO{\U{211d} }%
%BeginExpansion
\mathbb{R}
%EndExpansion
}\cos \left( x\xi \right) \pi e^{-\left\vert \xi \right\vert }d\xi +i\int_{%
%TCIMACRO{\U{211d} }%
%BeginExpansion
\mathbb{R}
%EndExpansion
}\sin \left( x\xi \right) \pi e^{-\left\vert \xi \right\vert }d\xi =\int_{%
%TCIMACRO{\U{211d} }%
%BeginExpansion
\mathbb{R}
%EndExpansion
}\cos \left( x\xi \right) \pi e^{-\left\vert \xi \right\vert }d\xi $ e la
funzione integranda \`{e} pari). Quindi $\frac{1}{x^{2}+1}=\frac{1}{2\pi }%
\mathcal{F}\left( \pi e^{-\left\vert \xi \right\vert }\right) $: $\mathcal{F}%
\left( e^{-\left\vert \xi \right\vert }\right) =\frac{2}{x^{2}+1}$.

Inoltre $\mathcal{F}\left( \mathcal{F}\left( e^{-\left\vert \xi \right\vert
}\right) \right) =\mathcal{F}\left( \frac{2}{x^{2}+1}\right) =2\pi
e^{-\left\vert \xi \right\vert }$.

\item I risultati visti possono essere combinati per ricavare informazioni a
priori su $u$ quando \`{e} nota la sua trasformata, e viceversa.

In particolare, supponiamo che $\hat{u}\left( \xi \right) =\frac{1}{\left(
2\xi ^{2}+1\right) ^{2}}$. Poich\'{e} $\hat{u}$ \`{e} $L^{1}$,
l'antitrasformata \`{e} ben definita e $u$ pu\`{o} essere vista come
trasformata di $\hat{u}$, a meno di una costante: $u$ ha dunque tutte le
buone propriet\`{a} delle trasformate. $\hat{u}$ \`{e} pari e reale, dunque $%
u$ \`{e} pari e reale; $\hat{u}\in L^{1}\cap L^{2}$, dunque $u\in L^{2}\cap
C^{0}\cap L^{\infty }$ e $\lim_{\left\vert x\right\vert \rightarrow +\infty
}u\left( x\right) =0$. Poich\'{e} $\xi \hat{u}\in L^{1}\cap L^{2}$, $u\in
C^{1}$; poich\'{e} $\xi ^{2}\hat{u}\in L^{1}\cap L^{2}$, $u\in C^{2}$.

Noto inoltre che $\hat{u}$ \`{e} abbastanza regolare da avere la seguente
propriet\`{a}: $\forall $ $k\in 
%TCIMACRO{\U{2115} }%
%BeginExpansion
\mathbb{N}
%EndExpansion
$ $\exists $ $h\in 
%TCIMACRO{\U{2115} }%
%BeginExpansion
\mathbb{N}
%EndExpansion
:\xi ^{k}\hat{u}^{\left( h\right) }\in L^{2}\cap L^{1}$ (infatti derivando
la regolarit\`{a} aumenta). Allora $\frac{d^{k}}{dx^{k}}x^{h}u\in L^{2}\cap
L^{\infty }\cap C^{0}$: infatti $\mathcal{F}\left( \frac{d^{k}}{dx^{k}}%
x^{h}u\right) =i\xi ^{k}\mathcal{F}\left( x^{h}u\right) =i\xi ^{k}\frac{d^{h}%
\hat{u}}{dx^{h}}\frac{1}{\left( -i\right) ^{h}}$, quindi si pu\`{o}
antitrasformare. Poich\'{e} $\forall $ $k\in 
%TCIMACRO{\U{2115} }%
%BeginExpansion
\mathbb{N}
%EndExpansion
$ $\exists $ $h\in 
%TCIMACRO{\U{2115} }%
%BeginExpansion
\mathbb{N}
%EndExpansion
:\frac{d^{k}}{dx^{k}}x^{h}u\in L^{2}\cap L^{\infty }\cap C^{0}$, si ha $%
x^{h}u\in C^{k}$ e in particolare $u\in C^{k}\left( 
%TCIMACRO{\U{211d} }%
%BeginExpansion
\mathbb{R}
%EndExpansion
\backslash \left\{ 0\right\} \right) $. Se ne conclude che $u\in C^{\infty
}\left( 
%TCIMACRO{\U{211d} }%
%BeginExpansion
\mathbb{R}
%EndExpansion
\backslash \left\{ 0\right\} \right) $.
\end{enumerate}

La situazione migliora se si considera l'operatore trasformata definito
sullo spazio di Schwartz.

\textbf{Prop 4.9} 
\begin{eqnarray*}
\text{Hp}\text{: } &&\phi \in \mathcal{S}\left( 
%TCIMACRO{\U{211d} }%
%BeginExpansion
\mathbb{R}
%EndExpansion
^{n}\right) \\
\text{Ts}\text{: } &&\mathcal{F}\left( \phi \right) \in \mathcal{S}\left( 
%TCIMACRO{\U{211d} }%
%BeginExpansion
\mathbb{R}
%EndExpansion
^{n}\right)
\end{eqnarray*}

\textbf{Dim} Mostro che $\mathcal{F}\left( \phi \right) \in C^{\infty
}\left( 
%TCIMACRO{\U{211d} }%
%BeginExpansion
\mathbb{R}
%EndExpansion
^{n}\right) $ e che $\forall $ $\alpha \in 
%TCIMACRO{\U{2115} }%
%BeginExpansion
\mathbb{N}
%EndExpansion
^{n}$ vale $D^{\alpha }\left( \mathcal{F}\left( \phi \right) \right)
=o\left( \frac{1}{\left\vert \left\vert \mathbf{\xi }\right\vert \right\vert
^{k}}\right) $ per $\left\vert \left\vert \mathbf{\xi }\right\vert
\right\vert \rightarrow +\infty $, $\forall $ $k\in 
%TCIMACRO{\U{2115} }%
%BeginExpansion
\mathbb{N}
%EndExpansion
$.

E' noto che $P\phi \in L^{1}\left( 
%TCIMACRO{\U{211d} }%
%BeginExpansion
\mathbb{R}
%EndExpansion
^{n}\right) $ per ogni polinomio algebrico $P$: allora per la conseguenza
vista del teorema 4.7 vale $\mathcal{F}\left( \phi \right) \in C^{\infty
}\left( 
%TCIMACRO{\U{211d} }%
%BeginExpansion
\mathbb{R}
%EndExpansion
^{n}\right) $.

Dalle ipotesi segue che $P\phi \in L^{1}\left( 
%TCIMACRO{\U{211d} }%
%BeginExpansion
\mathbb{R}
%EndExpansion
^{n}\right) $ per ogni polinomio algebrico $P$: per quanto scritto sopra
allora $D^{\alpha }\mathcal{F}\left( \phi \right) =\left( -i\right)
^{\left\vert \alpha \right\vert }\mathcal{F}\left( \mathbf{x}^{\alpha }\phi
\right) $. Ma dalla conseguenza del teorema 4.6 si sa che se $D^{\alpha
}u\in L^{1}\left( 
%TCIMACRO{\U{211d} }%
%BeginExpansion
\mathbb{R}
%EndExpansion
^{n}\right) $ $\forall $ $\alpha \in 
%TCIMACRO{\U{2115} }%
%BeginExpansion
\mathbb{N}
%EndExpansion
^{n}$, allora $\mathcal{F}\left( u\right) =o\left( \frac{1}{\left\vert
\left\vert \xi \right\vert \right\vert ^{k}}\right) $ $\forall $ $k$ per $%
\left\vert \left\vert \xi \right\vert \right\vert \rightarrow +\infty $:
prendendo $u=\mathbf{x}^{\alpha }\phi $, poich\'{e} anche $u$ \`{e} in $%
\mathcal{S}\left( 
%TCIMACRO{\U{211d} }%
%BeginExpansion
\mathbb{R}
%EndExpansion
^{n}\right) $, si ha che $\mathcal{F}\left( \mathbf{x}^{\alpha }\phi \right)
=o\left( \frac{1}{\left\vert \left\vert \xi \right\vert \right\vert ^{k}}%
\right) $ $\forall $ $k$ per $\left\vert \left\vert \xi \right\vert
\right\vert \rightarrow +\infty $ e anche $D^{\alpha }\mathcal{F}\left( \phi
\right) =o\left( \frac{1}{\left\vert \left\vert \xi \right\vert \right\vert
^{k}}\right) $ $\forall $ $k$. $\blacksquare $

[ Fissato $\alpha \in 
%TCIMACRO{\U{2115} }%
%BeginExpansion
\mathbb{N}
%EndExpansion
^{n}$, considero l'operatore differenziale lineare $\sum_{j=1}^{n}c_{j}D^{%
\beta _{j}}$, con $c_{j}\in 
%TCIMACRO{\U{2102} }%
%BeginExpansion
\mathbb{C}
%EndExpansion
$ e $\beta _{j}\in 
%TCIMACRO{\U{2115} }%
%BeginExpansion
\mathbb{N}
%EndExpansion
^{n}$ multiindice. Calcolando la trasformata di tale operatore applicato a $%
D^{\alpha }\phi $ si ottiene $\mathcal{F}\left( \sum_{j=1}^{n}c_{j}D^{\beta
_{j}}\left( D^{\alpha }\phi \right) \right) =\sum_{j=1}^{n}c_{j}\mathcal{F}%
\left( D^{\beta _{j}}\left( D^{\alpha }\phi \right) \right) $. Per il
teorema 1 $\mathcal{F}\left( D^{\beta _{j}}\left( D^{\alpha }\phi \right)
\right) =i^{\left\vert \beta _{j}\right\vert }\mathbf{\xi }^{\left\vert
\beta _{j}\right\vert }\mathcal{F}\left( D^{\alpha }\phi \right) $: dunque $%
\mathcal{F}\left( D^{\alpha }\phi \right) $ tende a zero pi\`{u} in fretta
di $\mathbf{\xi }^{\left\vert \beta _{j}\right\vert }$ per $\left\vert
\left\vert \mathbf{\xi }\right\vert \right\vert \rightarrow +\infty $. Ma
poich\'{e} i $\beta _{j}$ sono tutti arbitrari, se ne ricava che $\mathcal{F}%
\left( D^{\alpha }\phi \right) =o\left( \frac{1}{\left\vert \left\vert 
\mathbf{\xi }\right\vert \right\vert ^{k}}\right) $ $\forall $ $k$. ]

Si pu\`{o} mostrare che l'operatore $\mathcal{F}:\mathcal{S}\left( 
%TCIMACRO{\U{211d} }%
%BeginExpansion
\mathbb{R}
%EndExpansion
^{n}\right) \rightarrow \mathcal{S}\left( 
%TCIMACRO{\U{211d} }%
%BeginExpansion
\mathbb{R}
%EndExpansion
^{n}\right) ,\mathcal{F}\left( \phi \right) =\hat{\phi}$ \`{e} biunivoco e
bicontinuo\footnote{$\phi _{n}\rightarrow ^{\mathcal{S}\left( 
%TCIMACRO{\U{211d} }%
%BeginExpansion
\mathbb{R}
%EndExpansion
^{n}\right) }\phi \Longleftrightarrow \hat{\phi}_{n}\rightarrow ^{\mathcal{S}%
\left( 
%TCIMACRO{\U{211d} }%
%BeginExpansion
\mathbb{R}
%EndExpansion
^{n}\right) }\hat{\phi}$}. Inoltre, poich\'{e} $\phi \in \mathcal{S}\left( 
%TCIMACRO{\U{211d} }%
%BeginExpansion
\mathbb{R}
%EndExpansion
^{n}\right) $ \`{e} in $L^{1}\left( 
%TCIMACRO{\U{211d} }%
%BeginExpansion
\mathbb{R}
%EndExpansion
^{n}\right) \cap C^{0}\left( 
%TCIMACRO{\U{211d} }%
%BeginExpansion
\mathbb{R}
%EndExpansion
^{n}\right) $ e $\lim_{\left\vert \left\vert \mathbf{x}\right\vert
\right\vert \rightarrow +\infty }\phi \left( \mathbf{x}\right) =0$, vale il
teorema di inversione con $\phi \left( \mathbf{x}\right) =\frac{1}{\left(
2\pi \right) ^{n}}\int_{%
%TCIMACRO{\U{211d} }%
%BeginExpansion
\mathbb{R}
%EndExpansion
^{n}}e^{i\left\langle \mathbf{\xi ,x}\right\rangle }\hat{\phi}\left( \mathbf{%
\xi }\right) d\mathbf{\xi }$ e $\mathcal{F}^{-1}\left( \mathcal{F}\left(
\phi \right) \right) =\phi $. Lo spazio di Schwartz \`{e} dunque il dominio
ideale per la trasformata di Fourier.

\subsection{Trasformata di Fourier di distribuzioni temperate}

\textbf{Lemma 4.10}%
\begin{eqnarray*}
\text{Hp}\text{: } &&f,g\in L^{1}\left( 
%TCIMACRO{\U{211d} }%
%BeginExpansion
\mathbb{R}
%EndExpansion
^{n}\right) \\
\text{Ts}\text{: } &&\int_{%
%TCIMACRO{\U{211d} }%
%BeginExpansion
\mathbb{R}
%EndExpansion
^{n}}\hat{f}g=\int_{%
%TCIMACRO{\U{211d} }%
%BeginExpansion
\mathbb{R}
%EndExpansion
^{n}}f\hat{g}
\end{eqnarray*}

\textbf{Dim} $\int_{%
%TCIMACRO{\U{211d} }%
%BeginExpansion
\mathbb{R}
%EndExpansion
^{n}}\hat{f}\left( \mathbf{\xi }\right) g\left( \mathbf{\xi }\right) d%
\mathbf{\xi }=\int_{%
%TCIMACRO{\U{211d} }%
%BeginExpansion
\mathbb{R}
%EndExpansion
^{n}}\left( \int_{%
%TCIMACRO{\U{211d} }%
%BeginExpansion
\mathbb{R}
%EndExpansion
^{n}}e^{-i\left\langle \mathbf{\xi ,x}\right\rangle }f\left( \mathbf{x}%
\right) d\mathbf{x}\right) g\left( \mathbf{\xi }\right) d\mathbf{\xi =}\int_{%
%TCIMACRO{\U{211d} }%
%BeginExpansion
\mathbb{R}
%EndExpansion
^{n}}\left( \int_{%
%TCIMACRO{\U{211d} }%
%BeginExpansion
\mathbb{R}
%EndExpansion
^{n}}e^{-i\left\langle \mathbf{\xi ,x}\right\rangle }g\left( \mathbf{\xi }%
\right) d\mathbf{\xi }\right) f\left( \mathbf{x}\right) d\mathbf{x=}\int_{%
%TCIMACRO{\U{211d} }%
%BeginExpansion
\mathbb{R}
%EndExpansion
^{n}}\hat{g}\left( \mathbf{x}\right) f\left( \mathbf{x}\right) d\mathbf{x}$,
scambiando l'ordine di integrazione grazie al teorema di Fubini-Tonelli. $%
\blacksquare $

Allora, se si vuole definire la trasformata di Fourier di una distribuzione,
per il caso particolare di distribuzioni del tipo $u_{f}$ la candidata
naturale \`{e} la distribuzione $\int_{%
%TCIMACRO{\U{211d} }%
%BeginExpansion
\mathbb{R}
%EndExpansion
^{n}}f\hat{g}$: come al solito, l'operazione \`{e} fatta sulla funzione
test, in modo che si possa facilmente generalizzare questa definizione a
distribuzioni non associate a una funzione.

Il seguente lemma mostra che tale definizione \`{e} ben posta.

\textbf{Lemma 4.11}%
\begin{gather*}
\text{Hp}\text{: }u\in \mathcal{S}^{\prime }\left( 
%TCIMACRO{\U{211d} }%
%BeginExpansion
\mathbb{R}
%EndExpansion
^{n}\right) \\
\text{Ts}\text{: }L:\mathcal{S}\left( 
%TCIMACRO{\U{211d} }%
%BeginExpansion
\mathbb{R}
%EndExpansion
^{n}\right) \rightarrow 
%TCIMACRO{\U{211d} }%
%BeginExpansion
\mathbb{R}
%EndExpansion
,L\left( \phi \right) =u\left( \hat{\phi}\right) \text{ \`{e} una
distribuzione temperata}
\end{gather*}

\textbf{Dim} $L$ \`{e} ovviamente lineare. Se inoltre $\phi _{n}\rightarrow
^{\mathcal{D}\left( 
%TCIMACRO{\U{211d} }%
%BeginExpansion
\mathbb{R}
%EndExpansion
^{n}\right) }\phi $, allora (si pu\`{o} mostrare che) $\hat{\phi}%
_{n}\rightarrow ^{\mathcal{D}\left( 
%TCIMACRO{\U{211d} }%
%BeginExpansion
\mathbb{R}
%EndExpansion
^{n}\right) }\hat{\phi}$, dunque $L$ \`{e} una distribuzione standard. E'
anche temperata perch\'{e} \`{e} verificata la condizione di estensione: se $%
\phi _{n}\rightarrow ^{\mathcal{S}\left( 
%TCIMACRO{\U{211d} }%
%BeginExpansion
\mathbb{R}
%EndExpansion
^{n}\right) }0$, allora $u\left( \hat{\phi}_{n}\right) \rightarrow 0$ perch%
\'{e} anche $\hat{\phi}_{n}\rightarrow ^{\mathcal{S}\left( 
%TCIMACRO{\U{211d} }%
%BeginExpansion
\mathbb{R}
%EndExpansion
^{n}\right) }0$ e $u$ \`{e} una distribuzione temperata in $\psi =\hat{\phi}$%
. $\blacksquare $

\textbf{Def} Data $u\in \mathcal{S}^{\prime }\left( 
%TCIMACRO{\U{211d} }%
%BeginExpansion
\mathbb{R}
%EndExpansion
^{n}\right) $, si dice trasformata di Fourier di $u$, e si indica con $\hat{u%
}$, la distribuzione temperata $u\left( \hat{\phi}\right) $.

\begin{enumerate}
\item Se $f,\hat{f}\in L^{1}\left( 
%TCIMACRO{\U{211d} }%
%BeginExpansion
\mathbb{R}
%EndExpansion
^{n}\right) $, sono ben definite le distribuzioni temperate $u_{f},u_{\hat{f}%
}$. In tal caso la trasformata di Fourier di $u_{f}$ e la distribuzione
temperata associata alla trasformata di Fourier coincidono per il lemma
visto all'inizio del paragrafo: $\hat{u}_{f}=u_{\hat{f}}$.

[Dim, ma ridondante: per definizione $\hat{u}_{f}\left( \phi \right) =\int_{%
%TCIMACRO{\U{211d} }%
%BeginExpansion
\mathbb{R}
%EndExpansion
^{n}}f\left( \mathbf{\xi }\right) \hat{\phi}\left( \mathbf{\xi }\right) d%
\mathbf{\xi }=\int_{%
%TCIMACRO{\U{211d} }%
%BeginExpansion
\mathbb{R}
%EndExpansion
^{n}}f\left( \mathbf{\xi }\right) \left( \int_{%
%TCIMACRO{\U{211d} }%
%BeginExpansion
\mathbb{R}
%EndExpansion
^{n}}e^{-i\left\langle \mathbf{x,\xi }\right\rangle }\phi \left( \mathbf{x}%
\right) d\mathbf{x}\right) d\mathbf{\xi =}\int_{%
%TCIMACRO{\U{211d} }%
%BeginExpansion
\mathbb{R}
%EndExpansion
^{n}}\left( \int_{%
%TCIMACRO{\U{211d} }%
%BeginExpansion
\mathbb{R}
%EndExpansion
^{n}}e^{-i\left\langle \mathbf{x,\xi }\right\rangle }f\left( \mathbf{\xi }%
\right) \phi \left( \mathbf{x}\right) d\mathbf{x}\right) d\mathbf{\xi }$.
Supponendo che tale distribuzione sia associata a una funzione $g$, si ha $%
\hat{u}_{f}\left( \phi \right) =u_{g}\left( \phi \right) =\int_{%
%TCIMACRO{\U{211d} }%
%BeginExpansion
\mathbb{R}
%EndExpansion
^{n}}g\left( \mathbf{x}\right) \phi \left( \mathbf{x}\right) d\mathbf{x}$.
Dunque dev'essere $\int_{%
%TCIMACRO{\U{211d} }%
%BeginExpansion
\mathbb{R}
%EndExpansion
^{n}}g\left( \mathbf{x}\right) \phi \left( \mathbf{x}\right) d\mathbf{x=}%
\int_{%
%TCIMACRO{\U{211d} }%
%BeginExpansion
\mathbb{R}
%EndExpansion
^{n}}\left( \int_{%
%TCIMACRO{\U{211d} }%
%BeginExpansion
\mathbb{R}
%EndExpansion
^{n}}e^{-i\left\langle \mathbf{x,\xi }\right\rangle }f\left( \mathbf{\xi }%
\right) d\mathbf{\xi }\right) \phi \left( \mathbf{x}\right) d\mathbf{x}$, da
cui, per il lemma di annullamento, $g\left( \mathbf{x}\right) =\int_{%
%TCIMACRO{\U{211d} }%
%BeginExpansion
\mathbb{R}
%EndExpansion
^{n}}e^{-i\left\langle \mathbf{x,\xi }\right\rangle }f\left( \mathbf{\xi }%
\right) d\mathbf{\xi }$ e $g=\hat{f}$.]

\item Data $\delta \in \mathcal{S}^{\prime }\left( 
%TCIMACRO{\U{211d} }%
%BeginExpansion
\mathbb{R}
%EndExpansion
^{n}\right) $, $\hat{\delta}\left( \phi \right) :=\delta \left( \hat{\phi}%
\right) =\hat{\phi}\left( 0\right) =\int_{%
%TCIMACRO{\U{211d} }%
%BeginExpansion
\mathbb{R}
%EndExpansion
^{n}}\phi \left( \mathbf{x}\right) d\mathbf{x}$, che \`{e} la distribuzione
associata alla funzione $f\left( \mathbf{x}\right) =1$.
\end{enumerate}

Tutte le propriet\`{a} della trasformata di Fourier integrale si estendono
alla trasformata distribuzionale, con l'accortezza di sostituire alla
derivata classica la derivata distribuzionale. [RISCRIVERE LE PROPRIETA']%
\begin{equation*}
\mathcal{F}\left( Du\right) =i\xi \mathcal{F}\left( u\right)
\end{equation*}

\textbf{Dim} Per semplicit\`{a} si suppone $n=1$. Usando le definizioni e le
propriet\`{a} della trasformata di Fourier integrale, $\mathcal{F}\left(
Du\right) \left( \phi \right) =Du\left( \mathcal{F}\left( \phi \right)
\right) =-u\left( \frac{d\mathcal{F}\left( \phi \right) }{d\xi }\right)
=-u\left( -i\mathcal{F}\left( x\phi \right) \right) =\mathcal{F}\left(
u\right) \left( i\xi \phi \right) $, che \`{e} la distribuzione $i\xi 
\mathcal{F}\left( u\right) $ calcolata in $\phi $. $\blacksquare $

Questa propriet\`{a} \`{e} molto utile per trovare $\mathcal{F}\left(
u\right) $ quando la trasformata di $Du$ \`{e} nota.

\begin{enumerate}
\item Sia $f\left( x\right) =\left( \left\vert x\right\vert -2\right) I_{%
\left[ -2,2\right] }\left( x\right) $. Si vogliono calcolare $%
Du_{f},D^{2}u_{f}$ e $\mathcal{F}\left( f\right) ,\mathcal{F}\left(
f^{\prime \prime }\right) ,\mathcal{F}\left( u_{f}\right) ,\mathcal{F}\left(
u_{f^{\prime \prime }}\right) $.

Poich\'{e} $f$ \`{e} $C^{1}$ a meno di un insieme di misura nulla e non ha
discontinuit\`{a} a salto, vale $Du_{f}=u_{f^{\prime }}$, con $f^{\prime
}\left( x\right) =\left\{ 
\begin{array}{c}
0\text{ se }x\not\in \left[ -2,2\right] \\ 
1\text{ se }x\in \left( 0,2\right) \\ 
-1\text{ se }x\in \left( -2,0\right)%
\end{array}%
\right. $ (il fatto che non sia ben definita in alcuni punti \`{e}
irrilevante, dato che $\left[ f^{\prime }\right] $ \`{e} determinata). Poich%
\'{e} $f^{\prime }$ \`{e} $C^{1}$ a meno di un insieme di misura nulla, vale 
$D^{2}u_{f}=u_{f^{\prime \prime }}-\delta _{-2}+2\delta _{0}-\delta
_{2}=-\delta _{-2}+2\delta _{0}-\delta _{2}$. Dunque $\mathcal{F}\left(
D^{2}u_{f}\right) =\left( -e^{-2i\xi }+2-e^{2i\xi }\right) u_{1}=\left(
2-2\cos 2\xi \right) u_{1}$. Allora $\mathcal{F}\left( D^{2}u_{f}\right)
=-\xi ^{2}\mathcal{F}\left( u_{f}\right) $, da cui $\mathcal{F}\left(
u_{f}\right) =\left( -\frac{2-2\cos 2\xi }{\xi ^{2}}\right) u_{1}=\left( -%
\frac{4\sin ^{2}\xi }{\xi ^{2}}\right) u_{1}$. Da quest'ultima scrittura si
intuisce che

$\mathcal{F}\left( f\right) =-\frac{4\sin ^{2}\xi }{\xi ^{2}}$ \`{e}, a meno
di un segno, la trasformata di un prodotto di convoluzione: infatti $%
\mathcal{F}\left( I_{\left[ -1,1\right] }\left( x\right) \right) =2\frac{%
\sin \xi }{\xi }$ (da estendere per continuit\`{a}), dunque $f=-I_{\left[
-1,1\right] }\ast I_{\left[ -1,1\right] }$.

Per ricavare $\mathcal{F}\left( f\right) \left( 0\right) $ si pu\`{o}
sfruttare la continuit\`{a}, dalla quale \`{e} ovvio che $\mathcal{F}\left(
f\right) \left( 0\right) =-4$, oppure si usa la definizione.
\end{enumerate}

\textbf{Def} Data $u\in \mathcal{S}^{\prime }\left( 
%TCIMACRO{\U{211d} }%
%BeginExpansion
\mathbb{R}
%EndExpansion
^{n}\right) $, si dice antitrasformata di Fourier di $u$, e si indica con $%
\check{u}$, la distribuzione temperata $u\left( \mathcal{F}^{-1}\left( \phi
\right) \right) $.

L'antitrasformata di una distribuzione \`{e} dunque la distribuzione
calcolata nell'antitrasformata integrale di una $\phi $. $\check{u}$ \`{e}
una distribuzione temperata grazie alla biunivocit\`{a} e bicontinuit\`{a}
di $\mathcal{F}^{-1}$.

Di nuovo, se $f,\hat{f}\in L^{1}\left( 
%TCIMACRO{\U{211d} }%
%BeginExpansion
\mathbb{R}
%EndExpansion
^{n}\right) ,f\in C^{0}\left( 
%TCIMACRO{\U{211d} }%
%BeginExpansion
\mathbb{R}
%EndExpansion
^{n}\right) $ e $f\rightarrow 0$ per $\left\vert \left\vert \mathbf{x}%
\right\vert \right\vert \rightarrow +\infty $ (cio\`{e} $f$ soddisfa le
ipotesi del teorema di inversione), l'antitrasformata di $u_{f}$, cio\`{e} $%
\check{u}_{f}$, coincide con $u_{\check{f}}$ (distribuzione associata
all'antitrasformata integrale di $f$).

Poich\'{e} si lavora nello spazio di Schwartz, l'antitrasformata della
trasformata di Fourier di $u$ coincide con $u$.

\begin{enumerate}
\item E' noto che la delta \`{e} una distribuzione temperata: $\delta \in 
\mathcal{S}^{\prime }\left( 
%TCIMACRO{\U{211d} }%
%BeginExpansion
\mathbb{R}
%EndExpansion
^{n}\right) $. Qual \`{e} la sua antitrasformata? Per definizione $\check{%
\delta}\left( \phi \right) =\delta \left( \check{\phi}\right) =\delta \left( 
\frac{1}{2\pi }\int_{%
%TCIMACRO{\U{211d} }%
%BeginExpansion
\mathbb{R}
%EndExpansion
}e^{i\left\langle \mathbf{\xi ,x}\right\rangle }\phi \left( \xi \right) d\xi
\right) =\frac{1}{2\pi }\int_{%
%TCIMACRO{\U{211d} }%
%BeginExpansion
\mathbb{R}
%EndExpansion
}\phi \left( \xi \right) d\xi =u_{1/2\pi }\left( \phi \right) $. Se ne
ricava anche che $\mathcal{F}\left( 2\pi \check{\delta}\right) =\mathcal{F}%
\left( u_{1}\right) $, cio\`{e} $\mathcal{F}\left( u_{1}\right) =2\pi \delta 
$.

\item Sfruttando alcune trasformate note e le propriet\`{a} delle
trasformate \`{e} facile ricavare le trasformate di altre funzioni.

$f\left( x\right) =1$ \`{e} un polinomio e dunque definisce la distribuzione
temperata $u_{f}$. Per il teorema sulla derivata della trasformata vale $D%
\mathcal{F}\left( u_{1}\right) =-i\mathcal{F}\left( xu_{1}\right) =-i%
\mathcal{F}\left( u_{x}\right) $, da cui $2\pi iD\delta =\mathcal{F}\left(
u_{x}\right) $.

\item Si pu\`{o} mostrare che la distribuzione $T$ associata a $vp\int_{%
%TCIMACRO{\U{211d} }%
%BeginExpansion
\mathbb{R}
%EndExpansion
}\frac{1}{t}dt$ \`{e} una distribuzione temperata. Si \`{e} visto che tale
distribuzione risolve il problema di divisione $xT=u_{1}$: allora vale $%
\mathcal{F}\left( xT\right) =2\pi \delta $, cio\`{e} - per la propriet\`{a}
gi\`{a} usata sopra - $D\mathcal{F}\left( T\right) =2\frac{\pi }{i}\delta $.
Si pu\`{o} calcolare la primitiva distribuzionale del lato destro: si trova $%
\mathcal{F}\left( T\right) =\frac{\pi }{i}u_{sign\left( \xi \right) }+u_{c}%
\footnote{%
la funzione segno ha come derivata distribuzionale $2\delta _{0}$; ma
dovrebbe andare bene come primitiva anche la funzione gradino per 2}$, dove $%
c$ \`{e} una costante il cui valore \`{e} $0$ perch\'{e} la trasformata deve
essere dispari.
\end{enumerate}

Si pu\`{o} dunque dimostrare che l'operatore trasformata di Fourier
distribuzionale $\mathcal{F}:\mathcal{S}^{\prime }\left( 
%TCIMACRO{\U{211d} }%
%BeginExpansion
\mathbb{R}
%EndExpansion
^{n}\right) \rightarrow \mathcal{S}^{\prime }\left( 
%TCIMACRO{\U{211d} }%
%BeginExpansion
\mathbb{R}
%EndExpansion
^{n}\right) $ \`{e} biunivoco e bicontinuo.

\subsection{Trasformata di Fourier in $L^{2}$}

C'\`{e} un altro insieme in cui si pu\`{o} studiare la trasformata di
Fourier: $L^{2}$. Poich\'{e} ogni funzione $f\in L^{2}\left( 
%TCIMACRO{\U{211d} }%
%BeginExpansion
\mathbb{R}
%EndExpansion
\right) $ definisce una distribuzione temperata $u_{f}$, la trasformata di
Fourier di $u_{f}$ \`{e} ben definita e la funzione cui essa \`{e} associata
si dir\`{a} trasformata di $f$.


E' gi\`{a} noto che qualora $f$ sia anche in $L^{1}$ la trasformata
integrale coincide con quella distribuzionale, quindi non c'\`{e} incoerenza
tra le definizioni.

\textbf{Teo 4.12 (Plancherel)}%
\begin{eqnarray*}
\text{Hp}\text{: } &&f\in L^{2}\left( 
%TCIMACRO{\U{211d} }%
%BeginExpansion
\mathbb{R}
%EndExpansion
^{n}\right) \\
\text{Ts}\text{: } &&\hat{f}\in L^{2}\left( 
%TCIMACRO{\U{211d} }%
%BeginExpansion
\mathbb{R}
%EndExpansion
^{n}\right) \text{ e }\left\vert \left\vert \hat{f}\right\vert \right\vert
_{L^{2}}=\left( 2\pi \right) ^{\frac{n}{2}}\left\vert \left\vert
f\right\vert \right\vert _{L^{2}}
\end{eqnarray*}

Questo teorema (che \`{e} la versione continua del teorema gi\`{a} visto
sulle serie di Fourier) mostra che l'operatore $\mathcal{F}:L^{2}\left( 
%TCIMACRO{\U{211d} }%
%BeginExpansion
\mathbb{R}
%EndExpansion
^{n}\right) \rightarrow L^{2}\left( 
%TCIMACRO{\U{211d} }%
%BeginExpansion
\mathbb{R}
%EndExpansion
^{n}\right) $ \`{e} continuo secondo la norma $L^{2}$. Si pu\`{o} mostrare
che \`{e} addirittura bicontinuo e biunivoco.

\begin{enumerate}
\item $f\left( x\right) =\ln \left( \frac{x+1}{x-1}\right) $ \`{e} in $L^{2}$%
, dunque la trasformata di $u_{f}$ \`{e} ben definita e mi aspetto che sia
(cio\`{e} sia associata a una funzione) in $L^{2}$. Poich\'{e} $f$ \`{e}
dispari, $u_{f}$ \`{e} dispari e anche la trasformata sar\`{a} dispari. Si
vuole calcolare $\hat{u}_{f}$.

Poich\'{e} $f\left( x\right) =\ln \left( x+1\right) -\ln \left( x-1\right) $
e si \`{e} gi\`{a} visto che la derivata della distribuzione associata al
logaritmo \`{e} $vp\int_{%
%TCIMACRO{\U{211d} }%
%BeginExpansion
\mathbb{R}
%EndExpansion
}\frac{1}{x}dx$, si ha che $Du_{f}=vp\int_{%
%TCIMACRO{\U{211d} }%
%BeginExpansion
\mathbb{R}
%EndExpansion
}\frac{1}{x+1}dx-vp\int_{%
%TCIMACRO{\U{211d} }%
%BeginExpansion
\mathbb{R}
%EndExpansion
}\frac{1}{x-1}dx$. Poich\'{e} $\mathcal{F}\left( Du_{f}\right) =i\xi 
\mathcal{F}\left( u\right) $ ed \`{e} noto che la trasformata del valor
principale \`{e} $\mathcal{F}\left( T\right) =\frac{\pi }{i}u_{sign\left(
\xi \right) }$, la trasformata delle distribuzioni traslate \`{e} $\frac{\pi 
}{i}e^{i\xi }u_{sign\left( \xi \right) },\frac{\pi }{i}e^{-i\xi
}u_{sign\left( \xi \right) }$ e si ottiene quindi $i\xi \mathcal{F}\left(
u\right) =\frac{\pi }{i}\left( e^{i\xi }-e^{-i\xi }\right) u_{sign\left( \xi
\right) }=2\pi \sin \xi u_{sign\left( \xi \right) }$. Si deve allora
risolvere il problema di divisione $\xi \mathcal{F}\left( u\right) =\frac{%
2\pi }{i}\sin \xi u_{sign\left( \xi \right) }$. Le soluzioni dell'equazione
omogenea sono del tipo $c\delta $. Essendo poi $sign\left( \xi \right) =%
\frac{\xi }{\left\vert \xi \right\vert }$, \`{e} facile intuire che una
soluzione particolare \`{e} $\frac{2\pi }{i}\sin \xi u_{\frac{1}{\left\vert
\xi \right\vert }}$. Allora $\mathcal{F}\left( u\right) =\frac{2\pi }{i}\sin
\xi u_{\frac{1}{\left\vert \xi \right\vert }}+c\delta $.
\end{enumerate}

Si noti che in generale $L^{2}\left( 
%TCIMACRO{\U{211d} }%
%BeginExpansion
\mathbb{R}
%EndExpansion
\right) \not\subseteq L^{1}\left( 
%TCIMACRO{\U{211d} }%
%BeginExpansion
\mathbb{R}
%EndExpansion
\right) $, quindi non si pu\`{o} usare la definizione di trasformata
integrale per $f\in L^{2}\left( 
%TCIMACRO{\U{211d} }%
%BeginExpansion
\mathbb{R}
%EndExpansion
\right) $. E' comunque ragionevole cercare una rappresentazione integrale
della trasformata di $f$: la seguente costruzione risponde a tale esigenza.

Data $f\in L^{2}\left( 
%TCIMACRO{\U{211d} }%
%BeginExpansion
\mathbb{R}
%EndExpansion
^{n}\right) $, sia $\left\{ C_{k}\right\} _{k\in 
%TCIMACRO{\U{2115} }%
%BeginExpansion
\mathbb{N}
%EndExpansion
}$ una successione di insiemi compatti tale che $C_{k}\subseteq C_{k+1}$ $%
\forall $ $k$ e $\bigcup_{k\in 
%TCIMACRO{\U{2115} }%
%BeginExpansion
\mathbb{N}
%EndExpansion
}C_{k}=%
%TCIMACRO{\U{211d} }%
%BeginExpansion
\mathbb{R}
%EndExpansion
^{n}$ e $\left\{ f_{k}\left( \mathbf{x}\right) \right\} _{k\in 
%TCIMACRO{\U{2115} }%
%BeginExpansion
\mathbb{N}
%EndExpansion
}$ una successione di funzioni di termine generale $f_{k}=fI_{C_{k}}$.
Allora la successione di funzioni $\left\{ \hat{f}_{k}\left( \mathbf{x}%
\right) \right\} _{k\in 
%TCIMACRO{\U{2115} }%
%BeginExpansion
\mathbb{N}
%EndExpansion
}$ di termine generale $\hat{f}_{k}\left( \mathbf{x}\right) =\int_{%
%TCIMACRO{\U{211d} }%
%BeginExpansion
\mathbb{R}
%EndExpansion
^{n}}e^{-i\left\langle \mathbf{\xi ,x}\right\rangle }f_{k}\left( \mathbf{x}%
\right) d\mathbf{x}$ \`{e} ben definita: $I_{C_{k}}$ \`{e} in $L^{p}\left( 
%TCIMACRO{\U{211d} }%
%BeginExpansion
\mathbb{R}
%EndExpansion
^{n}\right) $ $\forall $ $p$ e in particolare \`{e} in $L^{2}$, dunque $%
fI_{C_{k}}$ \`{e} in $L^{1}\footnote{%
In realt\`{a} \`{e} anche in $L^{2}$, perch\'{e} $f_{k}^{2}=f^{2}I_{C_{k}}%
\leq f^{2}$, che \`{e} integrabile.}$ e $\hat{f}_{k}$ \`{e} ben definita.
Vale inoltre $f_{k}\rightarrow ^{L^{2}}f$: infatti $\left\vert \left\vert
f_{k}-f\right\vert \right\vert _{L^{2}}^{2}=\int_{%
%TCIMACRO{\U{211d} }%
%BeginExpansion
\mathbb{R}
%EndExpansion
}\left( f_{k}\left( \mathbf{x}\right) -f\left( \mathbf{x}\right) \right)
^{2}d\mathbf{x}=\int_{%
%TCIMACRO{\U{211d} }%
%BeginExpansion
\mathbb{R}
%EndExpansion
}\left( f_{k}^{2}\left( \mathbf{x}\right) +f^{2}\left( \mathbf{x}\right)
-2f_{k}\left( \mathbf{x}\right) f\left( \mathbf{x}\right) \right) d\mathbf{x}
$ e $f_{k}^{2}\leq f^{2}$, $f_{k}f\leq f^{2}$, quindi si applica il teorema
di convergenza dominata e $\lim_{k\rightarrow +\infty }\left\vert \left\vert
f_{k}-f\right\vert \right\vert _{L^{2}}^{2}=\int_{%
%TCIMACRO{\U{211d} }%
%BeginExpansion
\mathbb{R}
%EndExpansion
}\lim_{k\rightarrow +\infty }\left( f_{k}\left( \mathbf{x}\right) -f\left( 
\mathbf{x}\right) \right) ^{2}d\mathbf{x}=0$ perch\'{e} $f_{k}\rightarrow f$
q.o. Allora, per la continuit\`{a} sopra menzionata di $\mathcal{F}%
:L^{2}\left( 
%TCIMACRO{\U{211d} }%
%BeginExpansion
\mathbb{R}
%EndExpansion
^{n}\right) \rightarrow L^{2}\left( 
%TCIMACRO{\U{211d} }%
%BeginExpansion
\mathbb{R}
%EndExpansion
^{n}\right) $, vale anche che $\hat{f}_{k}\rightarrow ^{L^{2}}\hat{f}$.

Si ricorda che $\hat{f}_{k}\rightarrow ^{L^{2}}\hat{f}$ implica che esista
una sottosuccessione $\left\{ k_{m}\right\} :\hat{f}_{k_{m}}\rightarrow \hat{%
f}$ q.o.\footnote{%
cfr. teorema "rivincita della convergenza in probabilit\`{a}" della
giuseppina}

\textbf{Teo 4.13 (trasformata e convoluzione)}%
\begin{gather*}
\text{Hp: (i) }u,v\in L^{1}\left( 
%TCIMACRO{\U{211d} }%
%BeginExpansion
\mathbb{R}
%EndExpansion
^{n}\right) \\
\text{(ii) }u\in L^{1}\left( 
%TCIMACRO{\U{211d} }%
%BeginExpansion
\mathbb{R}
%EndExpansion
^{n}\right) ,v\in L^{2}\left( 
%TCIMACRO{\U{211d} }%
%BeginExpansion
\mathbb{R}
%EndExpansion
^{n}\right) \\
\text{(iii) }u,v\in L^{2}\left( 
%TCIMACRO{\U{211d} }%
%BeginExpansion
\mathbb{R}
%EndExpansion
^{n}\right) \\
\text{(iv) }v\in \mathcal{S}\left( 
%TCIMACRO{\U{211d} }%
%BeginExpansion
\mathbb{R}
%EndExpansion
^{n}\right) ,u\in \mathcal{S}^{\prime }\left( 
%TCIMACRO{\U{211d} }%
%BeginExpansion
\mathbb{R}
%EndExpansion
^{n}\right) \\
\text{Ts: se vale una qualsiasi tra i, ii, iii, iv allora }\mathcal{F}\left(
u\ast v\right) =\mathcal{F}\left( u\right) \mathcal{F}\left( v\right)
\end{gather*}

La validit\`{a} della tesi nel caso (i) si era gi\`{a} vista.

Si noti che nel caso (ii) $\mathcal{F}\left( u\right) \in L^{\infty }$ e $%
\mathcal{F}\left( v\right) \in L^{2}$, quindi il prodotto \`{e} in $L^{2}$ e
l'uguaglianza con $\mathcal{F}\left( u\ast v\right) \in L^{2}$ ha senso.

Nel caso (iii) $u\ast v\in L^{\infty }$, mentre $\mathcal{F}\left( u\right) 
\mathcal{F}\left( v\right) \in L^{1}$ per Hoelder, quindi $\mathcal{F}\left(
u\ast v\right) \in L^{2}\cap L^{1}$ e l'antitrasformata \`{e} ben definita.

\subsection{Applicazioni}

L'idea \`{e} sfruttare la trasformata di Fourier per risolvere equazioni
differenziali lineari. Infatti la trasformata di Fourier, grazie a 4.6,
trasforma equazioni differenziali in equazioni algebriche, semplificando
notevolmente il processo di risoluzione.

\begin{enumerate}
\item Data $f\in L^{1}$, considero l'equazione differenziale%
\begin{equation*}
-u^{\prime \prime }+u=f
\end{equation*}

Cerco soluzioni $u\in L^{1}$ (altrimenti le operazioni di trasformata nel
seguito non sarebbero ben definite). Trasformando secondo Fourier entrambi i lati si ottiene 
$\left( 1+\xi ^{2}\right) \mathcal{F}\left( u\right) =\mathcal{F}\left(
f\right) $, cio\`{e} $\mathcal{F}\left( u\right) =\frac{\mathcal{F}\left(
f\right) }{1+\xi ^{2}}$ (in questo caso non ci sono problemi di divisione
perch\'{e} $\xi ^{2}+1$ non si annulla mai). Sapendo che la trasformata di
un prodotto di convoluzione \`{e} il prodotto delle trasformate e che $%
\mathcal{F}\left( e^{-\left\vert x\right\vert }\right) =\frac{2}{1+\xi ^{2}}%
\footnote{%
Si \`{e} gi\`{a} calcolato che $\mathcal{F}\left( \frac{1}{x^{2}+1}\right)
=\pi e^{-\left\vert \xi \right\vert }$. Applicando la formula $\mathcal{F}%
\left( \mathcal{F}\left( f\right) \right) =2\pi f\left( -y\right) $, si
ottiene che $\mathcal{F}\left( \pi e^{-\left\vert \zeta \right\vert }\right)
=2\pi \frac{1}{x^{2}+1}$, cio\`{e} $\mathcal{F}\left( e^{-\left\vert \zeta
\right\vert }\right) =\frac{2}{x^{2}+1}$.}$, si ha che $u=\frac{1}{2}f\ast
e^{-\left\vert x\right\vert }$, per cui $u\left( x\right) =\frac{1}{2}%
\int_{-\infty }^{+\infty }e^{-\left\vert x-y\right\vert }f\left( y\right) dy$
(poich\'{e} $e^{-\left\vert x\right\vert }$ \`{e} $L^{\infty }$ e $f$ \`{e} $%
L^{1}$, il loro prodotto di convoluzione \`{e} $L^{\infty }$). Ma $u$ non 
\`{e} derivabile in $%
%TCIMACRO{\U{211d} }%
%BeginExpansion
\mathbb{R}
%EndExpansion
$, a causa del valore assoluto ad esponente: dunque $u$ non risolve
l'equazione in senso classico, ma nel senso delle distribuzioni. Infatti i
passaggi svolti sono corretti anche se si vede il problema come segue:
supponendo che $u$ sia una distribuzione temperata, si risolve $-u^{\prime
\prime }+u=u_{f}$, cio\`{e} si cerca una distribuzione $u$ tale che - per la
formula della derivata seconda di una distribuzione e per linearit\`{a} - $%
u\left( -\phi ^{\prime \prime }+\phi \right) =u_{f}\left( \phi \right) $ $%
\forall $ $\phi $. La formulazione distribuzionale dell'equazione \`{e}
dunque pi\`{u} debole di quella classica, perch\'{e} richiede meno regolarit%
\`{a} alle soluzioni.

Si noti che la soluzione $u\in L^{1}$ \`{e} unica: se ne esistesse un'altra
la loro differenza risolverebbe l'omogenea, e trasformando si troverebbe che
la loro differenza \`{e} nulla.

Se e. g. $f\in C^{2}\left( 
%TCIMACRO{\U{211d} }%
%BeginExpansion
\mathbb{R}
%EndExpansion
\right) $ e $f^{\prime },f^{\prime \prime }\in L^{1}$, allora $u=\frac{1}{2}%
f\ast e^{-\left\vert x\right\vert }$ \`{e} $C^{2}$ e risolve l'equazione in
senso classico.

Si noti che inoltre l'ipotesi $u\in L^{1}$ \`{e} restrittiva, perch\'{e} si
escludono le funzioni non Fourier-trasformabili: ad esempio le soluzioni
dell'equazione omogenea, che sono del tipo $c_{1}e^{x}+c_{2}e^{-x}$.

\item Data $f\in L^{1}$, considero l'equazione differenziale%
\begin{equation*}
u^{\prime \prime }+u=f
\end{equation*}

Si interpreta subito l'equazione in senso distribuzionale: si risolve $%
u^{\prime \prime }+u=u_{f}$ con $u$ distribuzione temperata. Trasformando si
ottiene $\left( 1-\xi ^{2}\right) \mathcal{F}\left( u\right) =\mathcal{F}%
\left( u_{f}\right) $. Risolvendo il problema di divisione si ottiene $%
\mathcal{F}\left( u\right) =\left( vp\int \frac{1}{1-\xi ^{2}}\right) 
\mathcal{F}\left( u_{f}\right) =\frac{1}{2}\left( vp\int \frac{1}{1-\xi }%
+vp\int \frac{1}{1+\xi }\right) \mathcal{F}\left( u_{f}\right) $. Occorre
quindi antitrasformare la somma dei valori principali. Poich\'{e} $\mathcal{F%
}\left( u_{signx}\right) =\frac{2}{i}vp\int \frac{1}{\xi }\footnote{%
si ottiene analogamente a quanto visto per la trasformata di $e^{-\left\vert
x\right\vert }$}$, per la regola di traslazione si ottiene $\mathcal{F}%
\left( e^{ix}u_{signx}\right) =-\frac{2}{i}vp\int \frac{1}{1-\xi }$, mentre $%
\mathcal{F}\left( e^{-ix}u_{signx}\right) =\frac{2}{i}vp\int \frac{1}{\xi +1}
$. Allora $\mathcal{F}\left( \frac{i}{4}\left( e^{ix}-e^{-ix}\right)
u_{signx}\right) =vp\int \frac{1}{1-\xi ^{2}}$, dove $\frac{i}{4}\left(
e^{ix}-e^{-ix}\right) u_{signx}=\frac{1}{2}\sin xu_{signx}$. Ne segue che $u$
\`{e} il prodotto di convoluzione di $u_{f}$ e $\frac{1}{2}\sin xsignx$:
risolve l'equazione nel senso delle distribuzioni.

Se $f$ \`{e} abbastanza regolare, come sopra $u$ \`{e} anche una soluzione
classica.

Occorre recuperare le soluzioni dell'equazioni omogenea, che si sono escluse
nel risolvere il problema di divisione: si risolve $\left( 1-\xi ^{2}\right) 
\mathcal{F}\left( u\right) =0_{D}$. Per il teorema visto, avendo $1-\xi ^{2}$
due zeri semplici, $\mathcal{F}\left( u\right) =c_{1}\delta _{\xi
+1}+c_{2}\delta _{\xi -1}$. Poich\'{e} $\delta _{\xi -1}$ ($\delta _{\xi +1}$%
) \`{e} a meno di una costante la trasformata della distribuzione associata
a $e^{ix}$ ($e^{-ix}$), antitrasformando si ha $%
u=c_{2}u_{e^{ix}}+c_{1}u_{e^{-ix}}$. Se si impone che le distribuzioni
abbiano valori reali si ottiene $u=\alpha _{1}u_{\sin x}+\alpha _{2}u_{\cos
x}$. Quindi tutte e sole le soluzioni distribuzionali dell'equazione sono
del tipo $u=\alpha _{1}u_{\sin x}+\alpha _{2}u_{\cos x}+u_{f}\ast \frac{1}{2}%
\sin xu_{signx}$.
\end{enumerate}

\end{document}
